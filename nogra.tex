\documentclass[numbers=enddot,12pt,final,onecolumn,notitlepage]{scrartcl}%
\usepackage[headsepline,footsepline,manualmark]{scrlayer-scrpage}
\usepackage[all,cmtip]{xy}
\usepackage{amssymb}
\usepackage{amsmath}
\usepackage{amsthm}
\usepackage{framed}
\usepackage{comment}
\usepackage{color}
\usepackage{hyperref}
\usepackage{datetime}
\usepackage[sc]{mathpazo}
\usepackage[T1]{fontenc}
\usepackage{needspace}
\usepackage{tabls}
%TCIDATA{OutputFilter=latex2.dll}
%TCIDATA{Version=5.50.0.2960}
%TCIDATA{LastRevised=Tuesday, December 13, 2016 21:43:59}
%TCIDATA{SuppressPackageManagement}
%TCIDATA{<META NAME="GraphicsSave" CONTENT="32">}
%TCIDATA{<META NAME="SaveForMode" CONTENT="1">}
%TCIDATA{BibliographyScheme=Manual}
%TCIDATA{Language=American English}
%BeginMSIPreambleData
\providecommand{\U}[1]{\protect\rule{.1in}{.1in}}
%EndMSIPreambleData
\theoremstyle{definition}
\newtheorem{theo}{Theorem}[subsection]
\newenvironment{theorem}[1][]
{\begin{theo}[#1]\begin{leftbar}}
{\end{leftbar}\end{theo}}
\providecommand*\theoremautorefname{Theorem}
\newtheorem{lem}[theo]{Lemma}
\newenvironment{lemma}[1][]
{\begin{lem}[#1]\begin{leftbar}}
{\end{leftbar}\end{lem}}
\providecommand*\lemmaautorefname{Lemma}
\newtheorem{prop}[theo]{Proposition}
\newenvironment{proposition}[1][]
{\begin{prop}[#1]\begin{leftbar}}
{\end{leftbar}\end{prop}}
\providecommand*\propositionautorefname{Proposition}
\newtheorem{defi}[theo]{Definition}
\newenvironment{definition}[1][]
{\begin{defi}[#1]\begin{leftbar}}
{\end{leftbar}\end{defi}}
\providecommand*\definitionautorefname{Definition}
\newtheorem{remk}[theo]{Remark}
\newenvironment{remark}[1][]
{\begin{remk}[#1]\begin{leftbar}}
{\end{leftbar}\end{remk}}
\providecommand*\remarkautorefname{Remark}
\newtheorem{coro}[theo]{Corollary}
\newenvironment{corollary}[1][]
{\begin{coro}[#1]\begin{leftbar}}
{\end{leftbar}\end{coro}}
\providecommand*\corollaryautorefname{Corollary}
\newtheorem{conv}[theo]{Convention}
\newenvironment{condition}[1][]
{\begin{conv}[#1]\begin{leftbar}}
{\end{leftbar}\end{conv}}
\providecommand*\conditionautorefname{Convention}
\newtheorem{warn}[theo]{Warning}
\newenvironment{conclusion}[1][]
{\begin{warn}[#1]\begin{leftbar}}
{\end{leftbar}\end{warn}}
\providecommand*\conclusionautorefname{Warning}
\newtheorem{conj}[theo]{Conjecture}
\newenvironment{conjecture}[1][]
{\begin{conj}[#1]\begin{leftbar}}
{\end{leftbar}\end{conj}}
\providecommand*\conjectureautorefname{Conjecture}
\newtheorem{exam}[theo]{Example}
\newenvironment{example}[1][]
{\begin{exam}[#1]\begin{leftbar}}
{\end{leftbar}\end{exam}}
\providecommand*\exampleautorefname{Example}
\newtheorem{exmp}[theo]{Exercise}
\newenvironment{exercise}[1][]
{\begin{exmp}[#1]\begin{leftbar}}
{\end{leftbar}\end{exmp}}
\providecommand*\exerciseautorefname{Exercise}
\newenvironment{statement}{\begin{quote}}{\end{quote}}
\newtheorem{quest}[theo]{TODO}
\newenvironment{todo}[1][]
{\begin{quest}[#1]\begin{leftbar}}
{\end{leftbar}\end{quest}}
\iffalse
\newenvironment{proof}[1][Proof]{\noindent\textbf{#1.} }{\ \rule{0.5em}{0.5em}}
\fi
\let\sumnonlimits\sum
\let\prodnonlimits\prod
\let\cupnonlimits\bigcup
\let\capnonlimits\bigcap
\renewcommand{\sum}{\sumnonlimits\limits}
\renewcommand{\prod}{\prodnonlimits\limits}
\renewcommand{\bigcup}{\cupnonlimits\limits}
\renewcommand{\bigcap}{\capnonlimits\limits}
\newcommand\arxiv[1]{\href{http://www.arxiv.org/abs/#1}{\texttt{arXiv:#1}}}
% A command for citing arXiv preprints.
% Example syntax: \arxiv{1009.4134v2} and \arxiv{math/0602634v4}.

\setlength\tablinesep{3pt}
\setlength\arraylinesep{3pt}
\setlength\extrarulesep{3pt}
\voffset=0cm
\hoffset=-0.7cm
\setlength\textheight{22.5cm}
\setlength\textwidth{15.5cm}
\newenvironment{verlong}{}{}
\newenvironment{vershort}{}{}
\newenvironment{noncompile}{}{}
\excludecomment{verlong}
\includecomment{vershort}
\excludecomment{noncompile}
\newcommand{\id}{\operatorname{id}}
\newcommand{\NN}{\mathbb{N}}
\newcommand{\ZZ}{\mathbb{Z}}
\newcommand{\QQ}{\mathbb{Q}}
\newcommand{\RR}{\mathbb{R}}
\newcommand{\powset}[2][]{\ifthenelse{\equal{#2}{}}{\mathcal{P}\left(#1\right)}{\mathcal{P}_{#1}\left(#2\right)}}
% $\powset[k]{S}$ stands for the set of all $k$-element subsets of
% $S$. The argument $k$ is optional, and if not provided, the result
% is the whole powerset of $S$.
\newcommand{\set}[1]{\left\{ #1 \right\}}
% $\set{...}$ yields $\left\{ ... \right\}$.
\newcommand{\abs}[1]{\left| #1 \right|}
% $\abs{...}$ yields $\left| ... \right|$.
\newcommand{\tup}[1]{\left( #1 \right)}
% $\tup{...}$ yields $\left( ... \right)$.
\newcommand{\ive}[1]{\left[ #1 \right]}
% $\ive{...}$ yields $\left[ ... \right]$.
\newcommand{\verts}[1]{\operatorname{V}\left( #1 \right)}
% $\verts{...}$ yields $\operatorname{V}\left( ... \right)$.
\newcommand{\edges}[1]{\operatorname{E}\left( #1 \right)}
% $\edges{...}$ yields $\operatorname{E}\left( ... \right)$.
\newcommand{\arcs}[1]{\operatorname{A}\left( #1 \right)}
% $\arcs{...}$ yields $\operatorname{A}\left( ... \right)$.
\newcommand{\E}{\operatorname{E}}
\newcommand{\A}{\operatorname{A}}
\ihead{Notes on graph theory (\today, \currenttime)}
\ohead{page \thepage}
\cfoot{}
\begin{document}

\title{Notes on graph theory}
\author{Darij Grinberg}
\date{
%TCIMACRO{\TeXButton{today}{\today} }%
%BeginExpansion
\today\ 
%EndExpansion
at \currenttime\ 
(unfinished draft!)}
\maketitle
\tableofcontents

\section{Preface}

These are lecture notes on graph theory -- the part of mathematics
involved with graphs. They are currently work in
progress; once finished, they should contain a semester's worth of
material. I have tried to keep the presentation as self-contained and
elementary as possible; the reader is nevertheless assumed to possess
some mathematical maturity (in particular, to know how to read
combinatorial proofs, filling in simple details)\footnote{I believe
that the MIT lecture notes \cite{LeLeMe16} are a good source for
achieving this mathematical maturity. (Actually, there is some
intersection between \cite[Chapter 12]{LeLeMe16} and our notes, but
\cite{LeLeMe16} mostly keeps to the basics of graph theory.)
Another resource to familiarize oneself with proofs is
\cite{Day-proofs}. Generally, most good books about ``reading and
writing mathematicals'' or ``introductions to abstract mathematics''
should convey these skills, although the extent to which they actually
do so may differ.}
and know how to work
with modular arithmetic (i.e., congruences between integers modulo a
positive integer $n$) and summation signs (such as $\sum_{i=1}^n$ and
$\sum_{a \in A}$, where $n$ is an integer and $A$ is a finite
set)\footnote{See \cite[\S 1.3]{detnotes} for a list of important
properties of summation signs.}.
In some chapters, familiarity with matrices (and their products),
permutations (and their signs)\footnote{A summary of the most
fundamental results about signs of permutations can be found in
\cite[\S 8.1]{LaNaSc16}. These results appear with proofs in
\cite[Chapter 6.B]{Day-proofs}. For an even more detailed treatment
(also including proofs),
see \cite[\S 4.1--4.3]{detnotes}. Another treatment can be found in
\cite{Conrad-sign}, but this requires some familiarity with group
theory.}
and polynomials will be required.
I hope that the proofs are readable; feel free to contact me (at
\texttt{dgrinber@umn.edu}) for any clarifications.

The choice of material surveyed in these notes is idiosyncratic
(sometimes even purposefully trying to wander some seldom trodden
paths). Some standard material (Eulerian walks, Hamiltonian paths,
trees, bipartite matching theory, network flows) is present (at least
in the eventual final form of these notes), whereas
other popular topics (planar graphs, random graphs, adjacency matrices
and spectral graph theory) are missing. Some of these omissions have
specific reasons (e.g., many of the omitted topics would make it much
harder to keep the notes self-contained), whereas others are merely
due to my tastes and lack of time.
% [The following is probably far too optimistic;]
% I am trying to give an elementary
% introduction into the (rather new) theory of sandpiles (also known as
% chip-firing) as well as two applications of combinatorics to linear
% algebra (viz., Gessel's proof of the Vandermonde determinant
% \cite{Gessel-Vand}, and some of the applications of matchings to
% Pfaffians); I also intend to include some properties of Hamiltonian
% paths not commonly exposed in textbooks.

These notes are accompanying
\href{http://www.math.umn.edu/~dgrinber/5707s17/}{a
class on graph theory (Math 5707) I am giving at the University of
Minneapolis in Spring 2017}. They contain both the
material of the class (although with no promise of timeliness!) and the
homework exercises (and possibly some additional exercises).
Sections marked with an asterisk (*) are not part of the Math 5707
course.

Various other texts on graph theory are \cite{Bollob79},
\cite{Bollob98},
\cite{BonMur76}, \cite{Ore62}, \cite{BehCha71}, \cite{BeChZh15},
\cite{BonMur08}, \cite{Ruohon13}, \cite{Dieste16}, \cite{Ore90},
\cite{HaHiMo08}, \cite{Berge91}, \cite{ChaLes15}, \cite{Griffi15},
\cite{Wilson96}.
(This is a haphazard list; I have barely touched most of these texts.)
Also, texts on combinatorics and on discrete mathematics (such as
\cite{BenWil12}, \cite{KelTro15}, \cite{PoTaWo83}, \cite{Bona11},
or the introductory \cite{LoPeVe03}) often contain
sections on graph theory, since graph theory is considered to be part
of both.
Material on graph theory can also be found in large quantities on
mathematical contests for students (such as the International
Mathematical Olympiad) and, consequently, in collections of problems
from these contests, such as the AoPS collection of IMO Shortlist
problems \cite{AoPS-ISL}.
Finally, some elementary results in graph theory double as puzzles
(or are related to puzzles), which often has the consequence that they
appear on puzzle websites such as Cut-the-Knot \cite{cut-the-knot}.

The notes you are reading are under construction, and will remain so for at
least the whole Spring of 2017. Please let me know of any errors and
unclarities you encounter (my email address is \texttt{dgrinber@umn.edu}%
)\footnote{The sourcecode of the notes is also publicly available at
\url{https://github.com/darijgr/nogra} .}. Thank you!

\subsection{Acknowledgments}

Thanks to Victor Reiner and Travis Scrimshaw for helpful
conversations.

[Your name could be in here!]

\section{\label{chp.intro}Introduction}

In this chapter, we shall define a first notion of graphs (``first''
because there are several others to follow) and various basic notions
related to it, and prove some elementary properties thereof. This
chapter is meant to give a taste of the whole theory, although (not
unexpectedly) it is not a representative sample.

\subsection{\label{sect.intro.notations}Notations and conventions}

Before we get to anything interesting, let me get some technicalities
out of the way. Namely, I shall be using the following notations:

\begin{itemize}
\item In the following, we use the symbol $\NN$ to denote the set
$\left\{ 0, 1, 2, \ldots \right\}$. (Be warned that some other authors
use this symbol for $\left\{ 1, 2, 3, \ldots \right\}$ instead.)

\item We let $\QQ$ denote the set of all rational numbers; we let
$\RR$ be the set of all real numbers.

\item If $X$ and $Y$ are two sets, then we shall use the notation
``$X\rightarrow Y,\ x\mapsto E$'' (where $x$
is some symbol which has no specific meaning in the current context,
and where $E$ is some expression which usually involves $x$) for ``the
map from $X$ to $Y$ that sends every $x\in X$ to $E$''. For
example, ``$\NN\rightarrow\NN,\ x\mapsto x^{2}+x+6$'' means the map
from $\NN$ to $\NN$ that sends every $x\in\NN$ to $x^{2}+x+6$. For
another example, ``$\NN\rightarrow\QQ,\ x\mapsto\dfrac{x}{1+x}$''
denotes the map from $\NN$ to $\QQ$ that
sends every $x\in\NN$ to $\dfrac{x}{1+x}$.\ \ \ \ \footnote{A word of
warning: Of course, the notation ``$X\rightarrow Y,\ x\mapsto E$''
does not always make sense; indeed, the map that it
stands for might sometimes not exist. For instance, the notation
``$\NN\rightarrow\QQ,\ x\mapsto\dfrac{x}{1-x}$'' does not actually
define a map, because the map that it
is supposed to define (i.e., the map from $\NN$ to $\QQ$ that
sends every $x\in\NN$ to $\dfrac{x}{1-x}$) does not exist (since
$\dfrac{x}{1-x}$ is not defined for $x=1$). For another example, the
notation ``$\NN\rightarrow\ZZ,\ x\mapsto\dfrac{x}{1+x}$'' does not
define a map, because the map that it is
supposed to define (i.e., the map from $\NN$ to $\ZZ$ that
sends every $x\in\NN$ to $\dfrac{x}{1+x}$) does not exist (for $x=2$,
we have $\dfrac{x}{1+x}=\dfrac{2}{1+2}\notin\ZZ$, which shows that a
map from $\NN$ to $\ZZ$ cannot send this $x$ to this $\dfrac
{x}{1+x}$). Thus, when defining a map from $X$ to $Y$ (using whatever
notation), do not forget to check that it is well-defined (i.e., that
your definition specifies precisely one image for each $x\in X$, and
that these images all lie in $Y$). In many cases, this is obvious or
very easy to check (I will usually not even mention this check), but
in some cases, this is a difficult task.}

\item If $S$ is a set, then the \textit{powerset} of $S$ means the set
of all subsets of $S$. This powerset will be denoted by
$\powset{S}$. % This expands to $\mathcal{P} \left( S \right)$.
For example, the powerset of $\left\{  1,2\right\}  $ is
$\powset{ \left\{  1,2\right\} }
=\left\{  \varnothing,\left\{
1\right\}  ,\left\{  2\right\}  ,\left\{  1,2\right\}  \right\}  $.

Furthermore, if $S$ is a set and $k$ is an integer, then
$\powset[k]{S}$ % This expands to $\mathcal{P}_k\left(S\right)$.
shall mean the set of all $k$-element sets of $S$. (This is empty if
$k < 0$.)
\end{itemize}

\subsection{\label{sect.intro.simple}Simple graphs}

As already hinted above, there is not one single concept of a
``graph''. Instead, there are several mutually related (but not
equivalent) concepts of ``graph'', which are often kept apart by
adorning them with adjectives (e.g., ``simple graph'', ``directed
graph'', ``loopless graph'', ``loopless weighted undirected
graph'', ``infinite graph'', etc.) or prefixes (``digraph'',
``multigraph'', etc.). Let me first define the simplest one of these:

\begin{definition} \label{def.intro.simple.sg}
A \textit{simple graph} is a pair $\tup{V, E}$, where $V$ is a
finite set, and where $E$ is a subset of $\powset[2]{V}$.
\end{definition}

Let us unpack this definition first. The word ``simple'' in
``simple graph'' (roughly speaking) has the meaning of ``with no
bells and whistles''; i.e., it says that the notion of
``simple graph'' is one of the crudest, most primitive notions of
a graph known.
% i.e., no additional structure.
It does not mean
that everything that can be said about simple graphs is simple (this
is far from the case, as we will see below). The condition
``$E$ is a subset of $\powset[2]{V}$'' in
Definition~\ref{def.intro.simple.sg} can be rewritten as ``$E$ is a
set of $2$-element subsets of $V$'' (since $\powset[2]{V}$ is the set
of all $2$-element subsets of $V$). Thus, a simple graph is a pair
consisting of a finite set $V$, and a set of $2$-element subsets of
$V$.
For example,
$\left(\set{1,2,3}, \set{\set{1,3}, \set{3,2}} \right)$ and
$\left(\set{2,5}, \set{\set{2,5}}\right)$ and
$\left(\varnothing, \varnothing\right)$ are three simple graphs.

\begin{conclusion}
\textbf{(a)}
Our Definition~\ref{def.intro.simple.sg} differs from the definition
of a ``simple graph'' in many sources, in that we are requiring $V$ to
be finite.

\textbf{(b)}
Simple graphs are often just called ``graphs''. But then again, some
other concepts of graphs (such as multigraphs, which we will
encounter further below) are also often just called ``graphs''. Thus,
the precise meaning of the word ``graph'' depends on the context in
which it appears. For example, Bollob\'as (in \cite{Bollob79}) uses
the word ``graph'' for ``simple graph'', whereas Bondy and Murty
(in \cite{BonMur08}) use it for ``multidigraph'' (a concept we will
define further below). When reading literature, always check the
definitions (and, if these are missing, try to take an educated guess,
ruling out options that make some of the claims false).
\end{conclusion}

So far, we have not explained how we should intuitively think of
simple graphs, and why they are interesting. We will spend a
significant part of these notes answering the latter question; but let
us first comment on the former.

Simple graphs can be used to model symmetric relations between
different objects. For example, if you have $n$ integers
(for some $n \in \NN$), then you can define a graph
$\tup{V, E}$ for which $V$ is the set of these $n$ integers,
and $E$ is the set of all $2$-element subsets
$\set{u, v}$ of $V$ for which $\abs{u-v} \leq 3$. (Notice that
$\set{u, u}$ does not count as a $2$-element subset.)
For a non-mathematical example, consider an (idealized) group $P$ of
(finitely many) people, each two of which are either mutual friends or
not\footnote{We assume that if a person $u$ is a friend of a person
$v$, then $v$ is a friend of $u$. We also assume that no person $u$ is
a friend of $u$ itself (or, at least, we don't count this as
friendship).}. Then, you can define a graph $\tup{P, E}$, where $E$
is the set of all $2$-element subsets $\set{u, v}$ of $P$ for which
$u$ and $v$ are mutual friends. This graph then models the friendships
between the people in the group $P$; in a sense, it is a social
network (similar to the ones widespread on the Internet, but much more
rudimentary, since it only knows who is a friend of
whom).\footnote{This kind of ``social graphs'' has been used for many
years as a language for stating theorems about graphs without saying
the word ``graph'' (and without using mathematical notation): Just
speak of people and their mutual friendships. This language was in use
long before the Internet and actual social networks came about.}

The following notations provide a quick way to reference the elements
of $V$ and $E$ when given a graph $\tup{V, E}$:

\begin{definition} \label{def.intro.simple.VE}
Let $G = \tup{V, E}$ be a simple graph.

\textbf{(a)} The set $V$ is called the \textit{vertex set} of $G$;
it is denoted by $\verts{G}$. (Notice that the letter
``$\operatorname{V}$'' in ``$\verts{G}$'' is upright, as opposed to
the letter ``$V$'' in ``$\tup{V, E}$'', which is italic.
These are two different symbols, and have different meanings: The
letter $V$ stands for the specific set $V$ which is the first
component of the pair $G$, whereas the letter
$\operatorname{V}$ is part of the notation $\verts{G}$ for the
vertex set of any graph. Thus, if $H = \left(W, F\right)$ is another
graph, then $\verts{H}$ is $W$, not $V$.)

The elements of $V$ are called the \textit{vertices} (or the
\textit{nodes}) of $G$.

\textbf{(b)} The set $E$ is called the \textit{edge set} of $G$; it
is denoted by $\edges{G}$. (Again, the letter ``$\operatorname{E}$''
in ``$\edges{G}$'' is upright, and stands for a different thing than
the ``$E$''.)

The elements of $E$ are called the \textit{edges} of $G$. When $u$ and
$v$ are two elements of $V$, we shall often use the notation $uv$ for
$\set{u, v}$; thus, each edge of $G$ has the form $uv$ for two
distinct elements $u$ and $v$ of $V$. Of course, we always have
$uv = vu$.

Notice that each graph $G$ satisfies $G = \tup{\verts{G}, \edges{G}}$.

\textbf{(c)} Two vertices $u$ and $v$ of $G$ are said to be
\textit{adjacent} (or \textit{connected by an edge}) if $uv \in E$
(that is, if $uv$ is an edge of $G$). In this case, the edge $uv$ is
said to \textit{connect} $u$ and $v$; the vertices $u$ and $v$ are
called the \textit{endpoints} of this edge.

Two vertices $u$ and $v$ of $G$
are said to be \textit{non-adjacent} if they are not adjacent (i.e.,
if $uv \notin E$).

We say that a vertex $u$ of $G$ is \textit{adjacent to} a vertex $v$
of $G$ if the vertices $u$ and $v$ are adjacent (i.e., if $uv \in E$).
Similarly, we say that a vertex $u$ of $G$ is \textit{non-adjacent to}
a vertex $v$ of $G$ if the vertices $u$ and $v$ are non-adjacent
(i.e., if $uv \notin E$).

\textbf{(d)} Let $v$ be a vertex of $G$ (that is, $v \in V$). Then,
the \textit{neighbors} of $v$ are the vertices $u$ of $G$ that
satisfy $vu \in E$. (Of course, these neighbors depend on both $v$ and
$G$. When $G$ is not clear from the context, we shall call them the
``neighbors of $v$ in $G$'' instead of just ``neighbors of $v$''.)
\end{definition}

Of course, the relation of adjacency is symmetric\footnote{This means
the following: Given two vertices $u$ and $v$ of a simple graph $G$,
the vertex $u$ is adjacent to $v$ if and only if $v$ is adjacent to
$u$.}. Same holds for the relation of non-adjacency.

\begin{example} \label{exa.intro.simple.VE}
Let $U$ be the $5$-element set $\set{1,2,3,4,5}$. Let $E$ be the
subset $\set{\set{u,v} \in \powset[2]{U} \ \mid \ u + v \geq 5 }$
of $\powset[2]{U}$. This set $E$ is well-defined, because the sum
$u + v$ of two integers $u$ and $v$ depends only on the set
$\set{u,v}$ and not on how this set is written (since
$u + v = v + u$). (This is important, because if we had used the
condition $u - v \geq 3$ instead of $u + v \geq 5$, then the set $E$
would not be well-defined, because it would not be clear whether
$\set{1, 5}$ should be inside it or not -- indeed, if we write
$\set{1, 5}$ as $\set{u, v}$ with $u = 5$ and $v = 1$, then
$u - v \geq 3$ is satisfied, but if we write $\set{1, 5}$ as
$\set{u, v}$ with $u = 1$ and $v = 5$, then $u - v \geq 3$ is not
satisfied.)

Let $G$ be the graph $\tup{U, E}$. Then, $\verts{G} = U
= \set{1,2,3,4,5}$ and
\begin{align}
\edges{G} &= E
= \set{\set{u,v} \in \powset[2]{U} \ \mid \ u + v \geq 5 }
\nonumber \\
&= \set{\set{1,4}, \set{1,5},
        \set{2,3}, \set{2,4}, \set{2,5},
        \set{3,4}, \set{3,5},
        \set{4,5}} .
\label{eq.exa.intro.simple.VE.edgesG=}
\end{align}
Thus, $G$ has $\abs{\verts{G}} = \abs{U} = 5$ vertices and
$\abs{\edges{G}} = \abs{E} = 8$ edges.

Using the shorthand notation
$uv$ for $\set{u, v}$ (introduced in
Definition~\ref{def.intro.simple.VE} \textbf{(b)}), the equality
\eqref{eq.exa.intro.simple.VE.edgesG=} rewrites as
\[
\edges{G}
= \set{14, 15, 23, 24, 25, 34, 35, 45} .
\]

The vertices $2$ and $4$ of $G$ are adjacent (since $24 \in E$).
In other words, $4$ is a neighbor of $2$. Equivalently, $2$ is a
neighbor of $4$. On the other hand, the vertices $1$ and $3$ of $G$
are not adjacent (since $13 \notin E$); thus, $1$ is not a neighbor
of $3$. The neighbors of $1$ are $4$ and $5$.
\end{example}

\subsection{\label{sect.intro.draw}Drawing graphs}

There is a common method to represent graphs visually: Namely, a graph
can be drawn as a set of points in the plane and a set of curves
connecting some of these points with each other.
\begin{todo}
Write this section.
\end{todo}

\subsection{\label{sect.intro.R33}A first fact: The Ramsey number
$R\tup{3,3} = 6$}

After these definitions, it might be time for a first result. The
following classical fact (which is actually the beginning of a deep
theory -- the so-called \textit{Ramsey theory}) should neatly
illustrate the concepts introduced above:

\begin{proposition} \label{prop.simple.R33}
Let $G$ be a simple graph with $\abs{\verts{G}} \geq 6$ (that is,
$G$ has at least $6$ vertices). Then, at least one of the following
two statements holds:

\begin{itemize}
\item \textit{Statement 1:} There exist three distinct vertices $a$,
$b$ and $c$ of $G$ such that $ab$, $bc$ and $ca$ are edges of $G$.

\item \textit{Statement 2:} There exist three distinct vertices $a$,
$b$ and $c$ of $G$ such that none of $ab$, $bc$ and $ca$ is an edge of
$G$.
\end{itemize}
\end{proposition}

In other words, Proposition~\ref{prop.simple.R33} says that if a graph
$G$ has at least $6$ vertices, then we can either find three distinct
vertices that are mutually adjacent\footnote{by which we mean (of
course) that any two \textbf{distinct} ones among these three vertices
are adjacent} or find three distinct vertices that are mutually
non-adjacent (i.e., no two of them are adjacent), or both.

Proposition~\ref{prop.simple.R33} can be stated more tersely as
follows: ``In any graph containing at least six vertices, you can
always find three vertices that are mutually adjacent, or three
vertices that are mutually non-adjacent''. It is also often restated
as follows:
``In any group of at least six people, you can always find three that
are (pairwise) friends to each other, or three no two of whom are
friends'' (provided that friendship is a symmetric relation). This
follows the old paradigm of restating facts about graphs in terms of
people and friendship.

\begin{example} \label{exa.simple.R33}
Let us show three examples of graphs $G$ to which
Proposition~\ref{prop.simple.R33} applies, as well as an example to
which it does not (because it fails to satisfy the condition
$\abs{\verts{G}} \geq 6$):

\begin{enumerate}

\item[\textbf{(a)}] Let $G$ be the graph $\tup{V, E}$, where
\begin{align*}
V &= \set{1, 2, 3, 4, 5, 6} \qquad \text{and} \\
E &= \set{\set{1,2}, \set{2,3}, \set{3,4}, \set{4,5}, \set{5,6},
          \set{6,1}} .
\end{align*}
(This graph can be drawn in such a way as to look like a hexagon.)
This graph satisfies Statement 2 of Proposition~\ref{prop.simple.R33}
for $a=1$, $b=3$ and $c=5$,
because its three vertices $1$, $3$ and $5$ are mutually
non-adjacent (i.e., none of $13$, $35$ and $51$ is an edge of $G$).
It also satisfies Statement 2 of Proposition~\ref{prop.simple.R33}
for $a=2$, $b=4$ and $c=6$.
So in this situation, we witness something slightly stronger than what
Proposition~\ref{prop.simple.R33} says: There are at least
\textit{two} choices of $a$, $b$ and $c$ making one of the Statements
1 and 2 valid (and these two choices are not the same up to order).
See Exercise~\ref{exa.simple.R33.two} \textbf{(a)} below.

\item[\textbf{(b)}] Let $G$ be the graph $\tup{V, E}$, where
\begin{align*}
V &= \set{-2, -1, 0, 1, 2, 3} \qquad \text{and} \\
E &= \set{\set{u,v} \in \powset[2]{V} \mid u \not\equiv v \mod 3} .
\end{align*}
This graph satisfies Statement 1 of Proposition~\ref{prop.simple.R33}
for $a=1$, $b=2$ and $c=3$. (Again, other choices of $a$, $b$ and $c$
are also possible.)

\item[\textbf{(c)}] Let $G$ be the graph $\tup{V, E}$, where
\begin{align*}
V &= \set{1, 2, 3, 4, 5, 6} \qquad \text{and} \\
E &= \set{\set{1,2}, \set{2,3}, \set{3,4}, \set{4,5}, \set{5,6},
          \set{6,1}, \set{1,3}} .
\end{align*}
(This graph can be drawn in such a way as to look like a hexagon with
one extra diagonal.)
This graph satisfies Statement 1 of Proposition~\ref{prop.simple.R33}
for $a=1$, $b=2$ and $c=3$. It also satisfies Statement 2 of
Proposition~\ref{prop.simple.R33} for $a=2$, $b=4$ and $c=6$.

\item[\textbf{(d)}] Let $G$ be the graph $\tup{V, E}$, where
\begin{align*}
V &= \set{1, 2, 3, 4, 5} \qquad \text{and} \\
E &= \set{\set{1,2}, \set{2,3}, \set{3,4}, \set{4,5}, \set{5,1}} .
\end{align*}
(This graph can be drawn to look like a pentagon.)
Proposition~\ref{prop.simple.R33} says nothing about this graph,
since this graph does not satisfy the assumption of
Proposition~\ref{prop.simple.R33} (in fact, its number of vertices
$\abs{\verts{G}}$ fails to be $\geq 6$).
By itself, this does not yield that the claim of
Proposition~\ref{prop.simple.R33} is false for this graph. However,
it is easy to check that the claim actually \textbf{is} false for this
graph: Neither Statement 1 nor Statement 2 hold.
\end{enumerate}

\end{example}

\begin{todo}
Drawings for these examples!
\end{todo}

\begin{proof}[Proof of Proposition~\ref{prop.simple.R33}.]We need to
prove that either Statement 1 holds or Statement 2 holds (or both).

Choose any vertex $u \in \verts{G}$. (This is clearly possible,
since $\abs{\verts{G}} \geq 6 \geq 1$.) Then,
$\abs{\verts{G} \setminus \set{u}} = \abs{\verts{G}} - 1 \geq 5$
(since $\abs{\verts{G}} \geq 6$). We are in one of the following two
cases:

\textit{Case 1:} At least $3$ vertices in
$\verts{G} \setminus \set{u}$ are adjacent to $u$.

\textit{Case 2:} At most $2$ vertices in
$\verts{G} \setminus \set{u}$ are adjacent to $u$.

Let us consider Case 1 first. In this case, at least $3$ vertices in
$\verts{G} \setminus \set{u}$ are adjacent to $u$. Hence, we can find
three distinct vertices $p$, $q$ and $r$ in
$\verts{G} \setminus \set{u}$ that are adjacent to $u$. Consider
these $p$, $q$ and $r$. If none of $pq$, $qr$ and $rp$ is an edge of
$G$, then Statement 2 holds (in fact, we can just take $a = p$,
$b = q$ and $c = r$). Thus, if none of $pq$, $qr$ and $rp$ is an edge
of $G$, then our proof is complete\footnote{because our goal in this
proof is to show that either Statement 1 holds or Statement 2 holds
(or both)}. Thus, we WLOG\footnote{The word ``WLOG'' means
``without loss of generality''. See, e.g., the corresponding Wikipedia
page \url{https://en.wikipedia.org/wiki/Without_loss_of_generality}
for its meaning.} assume that at least one $pq$, $qr$ and $rp$ is an
edge of $G$. In other words, we can pick two distinct elements $g$
and $h$ of $\set{p, q, r}$ such that $gh$ is an edge of $G$. Consider
these $g$ and $h$.

The vertex $g$ is one of $p$, $q$ and $r$
(since $g \in \set{p, q, r}$).
The vertices $p$, $q$ and $r$ are adjacent to $u$. Hence, the vertex
$g$ is adjacent to $u$ (since the vertex $g$ is one of $p$, $q$ and
$r$). In other words, $ug$ is an edge of $G$. Similarly, $uh$ is an
edge of $G$. In other words, $hu$ is an edge of $G$ (since $hu = uh$).

We have $g \in \set{p, q, r} \subseteq \verts{G} \setminus \set{u}$
(since $p$, $q$ and $r$ belong to $\verts{G} \setminus \set{u}$).
Hence, $g \neq u$. In other words, $u \neq g$.
% Clearly, $ug$ is a $2$-element set (since $ug$ is an edge of $G$, but
% each edge of $G$ is a $2$-element set). Thus, $u \neq g$.
Similarly,
$u \neq h$. Hence, $h \neq u$. Finally, $g \neq h$ (since $g$ and $h$
are distinct). Now, we know that the three vertices $u$, $g$ and $h$
are distinct (since $u \neq g$, $g \neq h$ and $h \neq u$), and have
the property that $ug$, $gh$ and $hu$ are edges of $G$. Therefore,
Statement 1 holds (in fact, we can just take $a = u$, $b = g$ and
$c = h$). Hence, the proof is complete in Case 1.

Let us now consider Case 2. In this case, at most $2$ vertices in
$\verts{G} \setminus \set{u}$ are adjacent to $u$. Thus, at least $3$
vertices in $\verts{G} \setminus \set{u}$ are non-adjacent to $u$
\ \ \ \ \footnote{\textit{Proof.} Let $k$ be the number of vertices
in $\verts{G} \setminus \set{u}$ that are adjacent to $u$. Let $\ell$
be the number of vertices in $\verts{G} \setminus \set{u}$ that are
non-adjacent to $u$. Then,
$k + \ell = \abs{\verts{G} \setminus \set{u}}$ (since each vertex
in $\verts{G} \setminus \set{u}$ is either adjacent to $u$ or
non-adjacent to $u$, but not both at the same time). But $k \leq 2$
(since at most $2$ vertices in $\verts{G} \setminus \set{u}$ are
adjacent to $u$). Hence, $k + \ell \leq 2 + \ell = \ell + 2$, so that
$\ell + 2 \geq k + \ell = \abs{\verts{G} \setminus \set{u}} \geq 5$
and thus $\ell \geq 3$. In other words, at least $3$
vertices in $\verts{G} \setminus \set{u}$ are non-adjacent to $u$.
Qed.}. Hence, we can find
three distinct vertices $p$, $q$ and $r$ in
$\verts{G} \setminus \set{u}$ that are non-adjacent to $u$. Consider
these $p$, $q$ and $r$. If all of $pq$, $qr$ and $rp$ are edges of
$G$, then Statement 1 holds (in fact, we can just take $a = p$,
$b = q$ and $c = r$). Thus, if all of $pq$, $qr$ and $rp$ are edges
of $G$, then our proof is complete. Thus, we WLOG
assume that not all of $pq$, $qr$ and $rp$ are edges of $G$. In other
words, at least one of $pq$, $qr$ and $rp$ is \textbf{not} an edge of
$G$. In other words, we can pick two distinct elements $g$ and $h$ of
$\set{p, q, r}$ such that $gh$ is not an edge of $G$. Consider
these $g$ and $h$.

The vertex $g$ is one of $p$, $q$ and $r$
(since $g \in \set{p, q, r}$).
The vertices $p$, $q$ and $r$ are non-adjacent to $u$. Hence, the
vertex $g$ is non-adjacent to $u$ (since the vertex $g$ is one of $p$,
$q$ and $r$). In other words, $ug$ is not an edge of $G$. Similarly,
$uh$ is not an edge of $G$. In other words, $hu$ is not an edge of $G$
(since $hu = uh$).

We have $g \in \set{p, q, r} \subseteq \verts{G} \setminus \set{u}$
(since $p$, $q$ and $r$ belong to $\verts{G} \setminus \set{u}$).
Thus, $g \neq u$. In other words, $u \neq g$. Similarly, $u \neq h$.
Finally, $g \neq h$ (since $g$ and $h$
are distinct). Now, we know that the three vertices $u$, $g$ and $h$
are distinct (since $u \neq g$, $g \neq h$ and $h \neq u$), and have
the property that none of $ug$, $gh$ and $hu$ is an edge of $G$ (since
$ug$ is not an edge of $G$, since $gh$ is not an edge of $G$, and
since $hu$ is not an edge of $G$). Therefore,
Statement 2 holds (in fact, we can just take $a = u$, $b = g$ and
$c = h$). Hence, the proof is complete in Case 2.

We have now proven Proposition~\ref{prop.simple.R33} in each of the
two Cases 1 and 2. Thus, the proof of
Proposition~\ref{prop.simple.R33} is complete.
\end{proof}

\begin{remark}
I have written the above proof in much detail, since it is the first
proof in these notes. I could have easily made it much shorter if I
had relied on the reader to fill in some details (and in fact, I
\textbf{will} rely on the reader in similar situations further below).
The proof can further be shortened by noticing that part of the
argument for Case 2 was a ``mirror version'' of the argument for
Case 1, with the only difference that ``adjacent'' is replaced by
``non-adjacent'' (and vice versa), and ``is an edge'' is replaced by
``is not an edge'' (and vice versa).
\end{remark}

\begin{remark}
Let me observe that Proposition~\ref{prop.simple.R33} could be proven
by brute force as well (using a computer). Indeed, here is how such a
proof would proceed: Let $x_1, x_2, x_3, x_4, x_5, x_6$ be six
distinct vertices of $G$. (Such six vertices exist, since
$\abs{\verts{G}} \geq 6$.) Let
$X = \set{x_1, x_2, x_3, x_4, x_5, x_6}$ be the set of these six
vertics. Notice that the set $X$ has $6$ elements, and thus the set
$\powset[2]{X}$ has $\dbinom{6}{2} = 15$ elements.
Let $F$ be the set of all edges $uv$ of $G$ for which both
$u$ and $v$ belong to $X$. (In other words, $F$ is the set of all
edges of $G$ having the form $x_i x_j$.)
Clearly, it suffices to prove Proposition~\ref{prop.simple.R33} for
the graph $\tup{X, F}$ instead of $G$ (because if we have found, for
example, three distinct vertices $a$, $b$ and $c$ of $\tup{X, F}$ such
that $ab$, $bc$ and $ca$ are edges of $\tup{X, F}$, then these $a$,
$b$ and $c$ are obviously also three vertices of $G$ such that
$ab$, $bc$ and $ca$ are edges of $G$). However,
$F$ is a subset of $\powset[2]{X}$. Since there
are only finitely many subsets of $\powset[2]{X}$ (in fact, there are
$2^{15}$ such subsets, since $\powset[2]{X}$ has $15$ elements), we
thus see that there are only finitely many choices for $F$ (when $X$
is being regarded as fixed). We can
check, for each of these choices, whether the graph $\tup{X, F}$
satisfies Proposition~\ref{prop.simple.R33}. (Just try each
possible choice of three distinct vertices $a$, $b$ and $c$ of this
graph $\tup{X, F}$, and check that at least one of these choices
satisfies either Statement 1 or Statement 2.) After a huge but
finite amount of checking (which you can automate), you will see that
Proposition~\ref{prop.simple.R33} holds for $\tup{X, F}$. Thus, as we
have already mentioned, Proposition~\ref{prop.simple.R33} also holds
for the original graph $G$.
\end{remark}

Proposition~\ref{prop.simple.R33} is the first result in a field
of graph theory known as \textit{Ramsey theory}. I shall not dwell on
this field in these notes, but let me make a few more remarks.
The first step beyond Proposition~\ref{prop.simple.R33} is the
following generalization:

\begin{proposition} \label{prop.simple.Rrs}
Let $r$ and $s$ be two positive integers.
Let $G$ be a simple graph with
$\abs{\verts{G}} \geq \dbinom{r+s-2}{r-1}$.
Then, at least one of the following two statements holds:

\begin{itemize}
\item \textit{Statement 1:} There exist $r$ distinct vertices
of $G$ that are mutually adjacent (i.e., each two distinct ones among
these $r$ vertices are adjacent).

\item \textit{Statement 2:} There exist $s$ distinct vertices
of $G$ that are mutually non-adjacent (i.e., no two distinct ones
among these $s$ vertices are adjacent).
\end{itemize}
\end{proposition}

Applying Proposition~\ref{prop.simple.Rrs} to $r=3$ and $s=3$, we can
recover Proposition~\ref{prop.simple.R33}.

One might wonder whether the number $\dbinom{r+s-2}{r-1}$ in
Proposition~\ref{prop.simple.Rrs} can be improved -- i.e., whether we
can replace it by a smaller number without making
Proposition~\ref{prop.simple.Rrs} false. In the case of $r=3$ and
$s=3$, this is impossible, because the number $6$ in
Proposition~\ref{prop.simple.R33} cannot be made
smaller\footnote{Indeed, there is a graph with $5$ vertices (namely,
the graph $G$ constructed in
Example~\ref{exa.simple.R33} \textbf{(d)}) that
satisfies neither Statement 1 nor Statement 2.}. However, for some
other values of $r$ and $s$, the value $\dbinom{r+s-2}{r-1}$ can be
improved. (For example, for $r=4$ and $s=4$, the best possible value
is $18$ rather than $\dbinom{4+4-2}{4-1}=20$.) The smallest possible
value is called the \textit{Ramsey number} $R\tup{r,s}$, and
finding it is a hard computational challenge; for small values of $r$
and $s$ it has been found using a computer, but $R\tup{5,5}$ is still
unknown.

Proposition~\ref{prop.simple.Rrs} can be further generalized to a
result called \textit{Ramsey's theorem}. The idea behind the
generalization is to slightly change the point of view, and replace
the simple graph $G$ by a complete graph (i.e., a simple graph in
which every two distinct vertices are adjacent) whose edges are
colored in two colors (say, blue and red). This is a completely
equivalent concept, because the concepts of ``adjacent'' and
``non-adjacent'' in $G$ can be identified with the concepts of
``adjacent through a blue edge'' (i.e., the edge connecting them is
colored blue) and ``adjacent through a red edge'', respectively.
Statements 1 and 2 then turn into ``there exist $r$ distinct vertices
that are mutually adjacent through blue edges'' and ``there exist $s$
distinct vertices that are mutually adjacent through red edges'',
respectively. From this point of view, it is only logical to
generalize Proposition~\ref{prop.simple.Rrs} further to the case when
the edges of a complete graph are colored in $k$ (rather than two)
colors. The corresponding generalization is known as Ramsey's theorem.
We refer to the well-written Wikipedia page
\url{https://en.wikipedia.org/wiki/Ramsey's_theorem} for a treatment
of this generalization with proof, as well as a table of known Ramsey
numbers $R\tup{r,s}$ and a self-contained (if somewhat terse) proof of
Proposition~\ref{prop.simple.Rrs}. Ramsey's theorem can be generalized
further, and many variations of it can be defined, which are usually
subsumed under the label ``Ramsey theory''\footnote{See
\cite{RaWi-Ramsey} for a popular view on the philosophy of Ramsey
theory (in the wide sense of this word). It should probably be said
that mathematicians usually define the word ``Ramsey theory'' somewhat
more restrictively, and not every result of the form ``you can find a
pattern in any sufficiently large structure'' belongs to Ramsey
theory; but the rough idea is correct.} There are many papers and at
least one textbook \cite{GrRoSp90} available on Ramsey theory. For
elementary introductions, see the Cut-the-knot page
\url{http://www.cut-the-knot.org/Curriculum/Combinatorics/ThreeOrThree.shtml}
in \cite{cut-the-knot}, the above-mentioned Wikipedia article, as well
as \cite[Chapter VI]{Bollob79} and \cite[Section 8.3]{West01}.

\begin{exercise} \label{exa.simple.R33.two}
Let $G$ be a simple graph. A \textit{triangle} in $G$ means a set
$\set{a, b, c}$ of three distinct vertices $a$, $b$ and $c$ of $G$
such that $ab$, $bc$ and $ca$ are edges of $G$. An
\textit{anti-triangle} in $G$ means a set $\set{a, b, c}$ of three
distinct vertices $a$, $b$ and $c$ of $G$ such that none of $ab$, $bc$
and $ca$ is an edge of $G$. A \textit{triangle-or-anti-triangle} in
$G$ is a set that is either a triangle or an anti-triangle.
(Of the three words I have just introduced, only ``triangle'' is
standard.) 

\textbf{(a)} Assume that $\abs{\verts{G}}\geq 6$.
Prove that $G$ has at least two
triangle-or-anti-triangles. (For comparison:
Proposition~\ref{prop.simple.R33} shows that $G$ has at least one
triangle-or-anti-triangle.)

\textbf{(b)} Assume that $\abs{\verts{G}} = m+6$ for some
$m \in \NN$. Prove that $G$ has at least $m+1$
triangle-or-anti-triangles.
\end{exercise}

\subsection{\label{sect.intro.deg}Degrees}

Next, we introduce the notion of the \textit{degree} of a vertex of a
graph. This is simply the number of neighbors of the vertex:

\begin{definition} \label{def.intro.deg}
Let $G = \tup{V, E}$ be a simple graph. Let $v \in V$ be a vertex of
$G$. Then, the \textit{degree} of $v$ (with respect to $G$) is defined
as the number of all neighbors of $v$ in $G$. This degree is a
nonnegative integer, and is denoted by $\deg v$ (or by
$\deg_G v$, when the graph $G$ is not clear from the context).
Thus,
\begin{align}
\deg v &= \deg_G v =
\left(\text{the number of all neighbors of } v\right)
\label{eq.def.intro.deg.1} \\
&= \abs{\set{u \in V \ \mid \ u \text{ is a neighbor of } v }}
\label{eq.def.intro.deg.2} \\
&= \abs{\set{u \in V \ \mid \ u \text{ is adjacent to } v }}
\label{eq.def.intro.deg.3} \\
&= \abs{\set{u \in V \ \mid \ uv \text{ is an edge of } G }}
\label{eq.def.intro.deg.4} \\
&= \abs{\set{u \in V \ \mid \ uv \in E }} .
\label{eq.def.intro.deg.5}
\end{align}
\end{definition}

\begin{todo}
Examples of degrees.
\end{todo}

\begin{remark}
Different sources use different notations for the degree of a
vertex $v$ of a simple graph $G$. We call it $\deg v$ (and so do
Ore's \cite{Ore90} and the introductory notes
\cite{LeLeMe16}); Ore's \cite{Ore62} calls it $\rho\left(v\right)$;
Bollob\'as's \cite{Bollob79} and Bondy's and Murty's
\cite{BonMur08} and \cite{BonMur76} call it $d\left(v\right)$.
% Not sure if anyone uses $\delta\left(v\right)$.
\end{remark}

At this point, we can state a few simple facts about degrees:

\begin{proposition} \label{prop.intro.deg.in-set}
Let $G$ be a simple graph. Let $n = \abs{\verts{G}}$. Let $v$ be a
vertex of $G$. Then, $\deg v \in \set{0, 1, \ldots, n-1}$.
\end{proposition}

\begin{proof}[Proof of Proposition~\ref{prop.intro.deg.in-set}.]
Let $U$ be the set of all neighbors of $v$. Then, $\abs{U}$ is
the number of all neighbors of $v$, and thus equals $\deg v$ (since
$\deg v$ was defined as the number of all neighbors of $v$).
In other words, $\abs{U} = \deg v$.

But $U \subseteq \verts{G} \setminus \set{v}$
\ \ \ \ \footnote{\textit{Proof.} Let $u \in U$. Thus, $u$ is
a neighbor of $v$ (by the definition of $U$). In other words, $u$
is a vertex of $G$ such that $vu \in \edges{G}$ (by the definition
of ``neighbor''). Thus, $vu \in \edges{G} \subseteq
\powset[2]{\verts{G}}$ (since each edge of $G$ is a $2$-element
subset of $\verts{G}$). In particular, $vu$ is a $2$-element set.
Hence, $u \neq v$. But $u \in \verts{G}$ (since $u$ is a vertex of
$G$). Combining this with $u \neq v$, we obtain
$u \in \verts{G} \setminus \set{v}$.

Now, forget that we fixed $u$. We thus have shown that
$u \in \verts{G} \setminus \set{v}$ for each $u \in U$. In other
words, $U \subseteq \verts{G} \setminus \set{v}$.}. Hence,
$\abs{U} \leq \abs{\verts{G} \setminus \set{v}}
= \abs{\verts{G}} - 1$ (since $v \in \verts{G}$). Since
$\abs{\verts{G}} = n$, this inequality can be rewritten as
$\abs{U} \leq n - 1$. Since $\abs{U}$ is a nonnegative integer, we
thus have $\abs{U} \in \set{0, 1, \ldots, n-1}$. Since
$\abs{U} = \deg v$, we can rewrite this as
$\deg v \in \set{0, 1, \ldots, n-1}$. This proves
Proposition~\ref{prop.intro.deg.in-set}.
\end{proof}

\begin{proposition} \label{prop.intro.2n}
Let $G$ be a simple graph.
The sum of the degrees of all vertices of $G$ equals
twice the number of edges of $G$. In other words,
$\sum_{v \in \verts{G}} \deg v = 2 \abs{\edges{G}}$.
\end{proposition}

\begin{todo}
Proof.
\end{todo}

\begin{proposition} \label{prop.intro.even-odd}
Let $G$ be a simple graph.
Then, the number of vertices $v$ of $G$ whose degree $\deg v$ is odd
is even.
\end{proposition}

\begin{todo}
Proof.
\end{todo}

As usual, Proposition~\ref{prop.intro.even-odd} can be restated in
terms of friendships among people; this restatement takes the
following form: ``In a group of (finitely many) people, the number of
people having an odd number of friends is even''. (Once again, this
assumes that friendship is a symmetric relation, and that nobody
counts as a friend of his own.)

\begin{proposition} \label{prop.intro.pigeonhole}
Let $G$ be a simple graph with at least two vertices.
Then, there exist two distinct vertices $v$ and $w$ of $G$ having the
same degree (that is, having $\deg v = \deg w$).
\end{proposition}

\begin{todo}
Proof.
\end{todo}

Degrees of vertices can be used to prove various facts about graphs.
For an example, let us show \textit{Mantel's theorem}:

\begin{theorem} \label{thm.intro.mantel}
Let $G$ be a simple graph. Let $n = \abs{\verts{G}}$ be the number of
vertices of $G$. Assume that $\abs{\edges{G}} > n^2 / 4$. (In other
words, assume that $G$ has more than $n^2 / 4$ edges.) Then, there
exist three distinct vertices $a$,
$b$ and $c$ of $G$ such that $ab$, $bc$ and $ca$ are edges of $G$.
\end{theorem}

\begin{todo}
Example.
\end{todo}

For proofs of Theorem~\ref{thm.intro.mantel}, see (for example) the
rather well-explained \cite{Choo16} (note that the fourth proof is
incomplete, as the existence of the maximum $S^*$ needs to be proven).

\begin{todo}
Proof.
\end{todo}

\begin{remark}
Let us contrast Proposition~\ref{prop.simple.R33} with
Theorem~\ref{thm.intro.mantel}. The former guarantees that a graph
$G$ with sufficiently many vertices (namely, at least $6$ vertices)
must have a triangle or an anti-triangle (where we are using the
terminology from Exercise~\ref{exa.simple.R33.two}).
Theorem~\ref{thm.intro.mantel} says that any graph $G$ with
sufficiently many edges (namely, more than $n^2 / 4$ edges, where
$n = \abs{\verts{G}}$) must have a triangle. These two facts are
similar and yet different in nature (e.g., the number $6$ in
Proposition~\ref{prop.simple.R33} is a constant, whereas the $n^2 / 4$
in Theorem~\ref{thm.intro.mantel} is not). We can also wonder how many
edges a graph must have in order to be guaranteed an anti-triangle;
this is answered by Exercise~\ref{exe.intro.mantel-co} below.
\end{remark}

\begin{exercise} \label{exe.intro.mantel-co}
Let $G$ be a simple graph. Let $n = \abs{\verts{G}}$ be the number of
vertices of $G$. Assume that $\abs{\edges{G}} < n\tup{n-2} / 4$. (In
other words, assume that $G$ has less than $n\tup{n-2} / 4$ edges.)
Prove that there exist three distinct vertices $a$, $b$ and $c$ of $G$
such that none of $ab$, $bc$ and $ca$ are edges of $G$.
\end{exercise}

Just as Proposition~\ref{prop.simple.R33} is merely the tip of a deep
iceberg called Ramsey theory, Theorem~\ref{thm.intro.mantel} is the
beginning of a longer story. The most famous piece of this story is
the following fact, known as \textit{Tur\'an's theorem}:

\begin{theorem} \label{thm.intro.turan}
Let $r$ be a positive integer.
Let $G$ be a simple graph. Let $n = \abs{\verts{G}}$ be the number of
vertices of $G$. Assume that
$\abs{\edges{G}} > \dfrac{r-1}{r} \cdot \dfrac{n^2}{2}$. Then, there
exist $r + 1$ distinct vertices
of $G$ that are mutually adjacent (i.e., each two distinct ones among
these $r + 1$ vertices are adjacent).
\end{theorem}

Theorem~\ref{thm.intro.mantel} is the particular case of
Theorem~\ref{thm.intro.turan} for $r = 2$. See
\cite[Chapter 4, Theorem 4.8]{Jukna11} for a proof of
Theorem~\ref{thm.intro.turan}.

Results like Theorem~\ref{thm.intro.mantel} (and sometimes also like
Proposition~\ref{prop.simple.Rrs}) are commonly regarded as part of
a subject called \textit{extremal graph theory}. (The word
``extremal'' refers to the appearance of bounds, such as the
$\dfrac{r-1}{r} \cdot \dfrac{n^2}{2}$.) There are textbooks on this
subject, such as \cite{Jukna11}.

\begin{todo}
Turan's theorem: at least state it, and cite a proof.
Apparently the WP
\url{https://en.wikipedia.org/wiki/Tur%C3%A1n's_theorem} again?
\end{todo}

\subsection{\label{sect.intro.iso}Graph isomorphisms}

Two graphs can be distinct and yet ``the same up to the names of their
vertices''. For instance, the two graphs
$\tup{\set{1,2,3},\set{\set{1,2}}}$ and
$\tup{\set{1,2,3},\set{\set{1,3}}}$ are distinct (since $12$ is an
edge of the former graph but not of the latter), but if we rename the
vertices $2$ and $3$ of the former graph as $3$ and $2$, respectively,
then it becomes the latter graph. This kind of relation between two
graphs is weaker than (literal) equality, but still strong enough to
ensure that (roughly speaking) the two graphs have the same properties
(as long as the properties don't refer to specific vertices). Thus,
it is worth giving this relation a rigorous definition and a name:

\begin{definition} \label{def.intro.iso}
Let $G$ and $H$ be two simple graphs.

\textbf{(a)} A \textit{graph isomorphism} from $G$ to $H$ means a
bijection $\phi : \verts{G} \to \verts{H}$ such that for every two
vertices $u$ and $v$ of $G$, the following logical equivalence holds:
\begin{equation}
\left( uv \in \edges{G} \right)
\Longleftrightarrow
\left( \phi\tup{u}\phi\tup{v} \in \edges{H} \right) .
\label{eq.def.intro.iso.a.eq}
\end{equation}
(At this point, let me remind you that $uv$ is shorthand for
$\set{u,v}$, and similarly $\phi\tup{u}\phi\tup{v}$ is shorthand for
$\set{\phi\tup{u},\phi\tup{v}}$.)

When it is clear what we mean, we shall abbreviate
``graph isomorphism'' as ``\textit{isomorphism}''. We also often write
``graph isomorphism $G \to H$'' for ``graph isomorphism from $G$ to
$H$''.

\textbf{(b)} If there exists a graph isomorphism from $G$ to $H$, then
we say the graphs $G$ and $H$ are \textit{isomorphic}, and
we write $G \cong H$. Sometimes, the relation $G \cong H$ itself will
be called an ``isomorphism'' (although it is not a map).
\end{definition}

Given this definition, we can now rigorously state the relation
between the two graphs $\tup{\set{1,2,3},\set{\set{1,2}}}$ and
$\tup{\set{1,2,3},\set{\set{1,3}}}$ we observed above. Namely, if we
denote these two graphs by $G$ and $H$, respectively, then the map
$\set{1,2,3} \to \set{1,2,3}$ that sends $1,2,3$ to $1,3,2$
(respectively) is a graph isomorphism from $G$ to $H$. Thus, these two
graphs are isomorphic. Note that we did not have to pick two graphs
with the same vertex set; isomorphisms are often observed between
graphs with completely different vertex sets and different
definitions. Also, it sometimes happens that there are several
isomorphisms between two graphs.

\begin{example} \label{exa.intro.iso.pentagon}
Consider the graph $G = \tup{V, E}$, where
\begin{align*}
V &= \set{1, 2, 3, 4, 5} \qquad \text{and} \\
E &= \set{\set{1,2}, \set{2,3}, \set{3,4}, \set{4,5}, \set{5,1}} .
\end{align*}
(This graph has already been introduced in
Example~\ref{exa.simple.R33} \textbf{(d)}. It can be drawn to look
like a pentagon.)

Consider furthermore the graph $H = \tup{V, F}$, where $V$ is as
before, and where
\begin{align*}
F &= \set{\set{1,3}, \set{2,4}, \set{3,5}, \set{4,1}, \set{5,2}} .
\end{align*}
(If we draw the vertices $1,2,3,4,5 \in V$ as the five vertices of a
pentagon, then the graph $H$ is the pentagram formed by the diagonals
of this pentagon.)

The graphs $G$ and $H$ are distinct, yet isomorphic. One isomorphism
$G \to H$ is the bijection $V \to V$ sending $1,2,3,4,5$ to
$1,3,5,2,4$, respectively. Another is the bijection $V \to V$ sending
$1,2,3,4,5$ to $4,2,5,3,1$, respectively. There are $10$ isomorphisms
in total.
\end{example}

Any isomorphism between two graphs is invertible (since, by its
definition, it is a bijection). Its inverse, too, is an isomorphism:

\begin{proposition} \label{prop.intro.iso.inverse}
Let $G$ and $H$ be two simple graphs. Let $\phi$ be a graph
isomorphism from $G$ to $H$. Then, its inverse $\phi^{-1}$ is a graph
isomorphism from $H$ to $G$.
\end{proposition}

\begin{proof}[Proof of Proposition~\ref{prop.intro.iso.inverse}.]
The map $\phi$ is a graph isomorphism from $G$ to $H$. By the
definition of ``graph isomorphism'', this means that $\phi$ is a
bijection $\verts{G} \to \verts{H}$ such that for every two
vertices $u$ and $v$ of $G$, the following logical equivalence holds:
\begin{equation}
\left( uv \in \edges{G} \right)
\Longleftrightarrow
\left( \phi\tup{u}\phi\tup{v} \in \edges{H} \right) .
\label{pf.prop.intro.iso.inverse.1}
\end{equation}
Now, for every two vertices $u$ and $v$ of $H$, the following
equivalence holds:
\begin{equation}
\left( uv \in \edges{H} \right)
\Longleftrightarrow
\left( \phi^{-1}\tup{u}\phi^{-1}\tup{v} \in \edges{G} \right)
\label{pf.prop.intro.iso.inverse.2}
\end{equation}
(because we have the following chain of equivalences:
\begin{align*}
\left( \phi^{-1}\tup{u}\phi^{-1}\tup{v} \in \edges{G} \right)
& \Longleftrightarrow
\left( \underbrace{\phi{\phi^{-1}\tup{u}}}_{=u}
   \underbrace{\phi{\phi^{-1}tup{v}}}_{=v} \in \edges{H} \right) \\
&\qquad
\left( \text{by \eqref{pf.prop.intro.iso.inverse.1}, applied to }
\phi^{-1}\tup{u} \text{ and } \phi^{-1}\tup{v} \text{ instead of } u
\text{ and } v \right) \\
& \Longleftrightarrow
\left( uv \in \edges{H} \right)
\end{align*}
). Thus, $\phi^{-1}$ is a bijection $\verts{H} \to \verts{G}$ such
that for every two vertices $u$ and $v$ of $H$, the equivalence
\eqref{pf.prop.intro.iso.inverse.2} holds. By the definition of
``graph isomorphism'', this means precisely that $\phi^{-1}$ is a
graph isomorphism from $H$ to $G$. This proves
Proposition~\ref{prop.intro.iso.inverse}.
\end{proof}

The following fact is similarly easy to check:

\begin{proposition} \label{prop.intro.iso.comp}
Let $G$, $H$ and $I$ be three simple graphs. Let $\phi$ be a graph
isomorphism from $G$ to $H$. Let $\psi$ be a graph isomorphism from
$H$ to $I$. Then, the composition $\psi \circ \phi$ is a graph
isomorphism from $G$ to $I$.
\end{proposition}

As we have already alluded to, isomorphic graphs are ``equal in all
but names'', and thus share all properties that do not depend on the
names of vertices. For example, the following holds:

\begin{proposition} \label{prop.intro.iso.degrees}
Let $G$ and $H$ be two simple graphs. Let $\phi$ be a graph
isomorphism from $G$ to $H$.

\textbf{(a)} For every $v \in \verts{G}$, we have
$\deg_G v = \deg_H \tup{\phi\tup{v}}$.

\textbf{(b)} We have
$\abs{\edges{H}} = \abs{\edges{G}}$.
\end{proposition}

\begin{proof}[Proof of Proposition \ref{prop.intro.iso.degrees}.]
The map $\phi$ is a graph isomorphism from $G$ to $H$. By the
definition of ``graph isomorphism'', this means that $\phi$ is a
bijection $\verts{G} \to \verts{H}$ such that for every two
vertices $u$ and $v$ of $G$, the following logical equivalence holds:
\begin{equation}
\left( uv \in \edges{G} \right)
\Longleftrightarrow
\left( \phi\tup{u}\phi\tup{v} \in \edges{H} \right) .
\label{pf.prop.intro.iso.degrees.1}
\end{equation}

\textbf{(a)} Fix $v \in \verts{G}$. Then, \eqref{eq.def.intro.deg.5}
(applied to $\verts{G}$ and $\edges{G}$ instead of $V$ and $E$)
yields
$\deg_G v = \abs{\set{u \in \verts{G} \ \mid \ uv \in \edges{G}}}$
(since $G = \tup{\verts{G}, \edges{G}}$).
Similarly,
\eqref{eq.def.intro.deg.5} (applied to $H$, $\verts{H}$, $\edges{H}$
and $\phi\tup{v}$ instead of $G$, $V$, $E$ and $v$) yields
\begin{align}
\deg_H \tup{\phi\tup{v}}
&=
\abs{\set{u \in \verts{H} \ \mid \ u \phi\tup{v} \in \edges{H}}}
\nonumber \\
&=
\abs{\set{w \in \verts{H} \ \mid \ w \phi\tup{v} \in \edges{H}}}
\label{pf.prop.intro.iso.degrees.a.1}
\end{align}
(here, we have renamed the index $u$ as $w$).
But recall that $\phi : \verts{G} \to \verts{H}$ is a bijection. Thus,
each $w \in \verts{H}$ can be written uniquely in the form
$\phi\tup{u}$ for some $u \in \verts{G}$. Thus, we can substitute
$\phi\tup{u}$ for $w$ in
$\set{w \in \verts{H} \ \mid \ w \phi\tup{v} \in \edges{H}}$.
We thus find
\[
\set{w \in \verts{H} \ \mid \ w \phi\tup{v} \in \edges{H}}
= \set{\phi\tup{u} \ \mid \ u \in \verts{G} \text{ and }
\phi\tup{u}\phi\tup{v} \in \edges{H}} .
\]
Hence,
\eqref{pf.prop.intro.iso.degrees.a.1} becomes
\begin{align*}
\deg_H \tup{\phi\tup{v}}
&=
\abs{
\underbrace{\set{w \in \verts{H} \ \mid \ w \phi\tup{v}
             \in \edges{H}}}
            _{= \set{\phi\tup{u} \ \mid \ u \in \verts{G}
              \text{ and } \phi\tup{u} \phi\tup{v} \in \edges{H}}}}
\\
&=
\abs{\set{\phi\tup{u} \ \mid \ u \in \verts{G}
          \text{ and } \phi\tup{u} \phi\tup{v} \in \edges{H}}} \\
&=
\abs{\set{u \in \verts{G} \ \mid
          \ \underbrace{\phi\tup{u} \phi\tup{v} \in \edges{H}}
                 _{\substack{\Longleftrightarrow
                     \left( uv \in \edges{G} \right) \\
                     \text{(by \eqref{pf.prop.intro.iso.degrees.1})}}}
          }}
\qquad \left(\text{since } \phi \text{ is a bijection}\right) \\
&= \abs{\set{u \in \verts{G} \ \mid \ uv \in \edges{G}}}
= \deg_G v .
\end{align*}
This proves Proposition~\ref{prop.intro.iso.degrees} \textbf{(a)}.

\textbf{(b)}
Define a map $\phi' : \edges{G} \to \edges{H}$ as follows:
Let $e \in \edges{G}$. Write the edge $e$ in the form $uv$ for two
distinct vertices $u$ and $v$. Then, $uv = e \in \edges{G}$. Thus,
\eqref{pf.prop.intro.iso.degrees.1} shows that
$\phi\tup{u}\phi\tup{v} \in \edges{H}$. Moreover, this new edge
$\phi\tup{u}\phi\tup{v}$ is independent of the choice of $u$ and $v$
(as long as $e$ is fixed)\footnote{There are two ways of choosing $u$
and $v$, since the elements of a $2$-element set (like $e$) can be
listed in two different orders. We are claiming that the edge
$\phi\tup{u}\phi\tup{v}$ does not depend on the choice of order. The
slickest way of proving this is to rewrite this edge in a form that
makes no reference to the choice of $u$ and $v$ to begin with.
Namely: The edge $\phi\tup{u}\phi\tup{v}$ can be rewritten as
$\set{\phi\tup{x} \mid x \in e}$ (because
\begin{align*}
\set{\phi\tup{x} \mid x \in e}
&= \set{\phi\tup{x} \mid x \in \set{u,v}}
\qquad \left(\text{since } e = uv = \set{u,v}\right) \\
&= \set{\phi\tup{u},\phi\tup{v}} = \phi\tup{u}\phi\tup{v}
\end{align*}
), and this is clearly independent of the choice of $u$ and $v$.}.
Hence, we can set $\phi'\tup{e} = \phi\tup{u}\phi\tup{v}$. Doing so,
we obtain a map $\phi' : \edges{G} \to \edges{H}$. This map $\phi'$ is
a bijection\footnote{Indeed, the inverse map is constructed in the
same way as $\phi'$, except that $H$, $G$ and $\phi^{-1}$ take the
roles of $G$, $H$ and $\phi$.}. Thus, we have found a bijection from
$\edges{G}$ to $\edges{H}$. Consequently,
$\abs{\edges{H}} = \abs{\edges{G}}$.
This proves Proposition~\ref{prop.intro.iso.degrees} \textbf{(b)}.
\end{proof}

Results similar to Proposition~\ref{prop.intro.iso.degrees} (saying
that isomorphic graphs share the same properties) can be stated easily
for any property of graphs you can imagine (as long as the property
does not depend on the names of the vertices); the proofs are always
straightforward like the one given above (the only idea being to apply
bijectivity of $\phi$ and the equivalence
\eqref{pf.prop.intro.iso.degrees.1} over and over in order to
``transfer knowledge'' from one graph to the other). I shall not
state such results explicitly; instead, I shall merely refer to the
overarching idea that isomorphic graphs share the same properties
whenever it becomes useful.

One use of graph isomorphisms is to ``relabel vertices'' in a proof:
Instead of proving some property of a given graph $G$, we can just as
well prove the same property for a graph isomorphic to $G$ (since
isomorphic graphs share the same properties), i.e., a graph obtained
from $G$ by renaming its vertices. The vertices can be renamed as we
wish; a popular choice is to rename them as $1,2,\ldots,n$ (where
$n = \abs{\verts{G}}$). Formally speaking, this renaming is possible
because of the following fact:

\begin{proposition} \label{prop.intro.iso.rename}
Let $G$ be a simple graph. Let $S$ be a finite set such that
$\abs{S} = \abs{\verts{G}}$. Then, there exists a simple graph $H$
that is isomorphic to $G$ and has vertex set $\verts{H} = S$.
\end{proposition}

\begin{proof}[Proof of Proposition~\ref{prop.intro.iso.rename}.]
We have $\abs{S} = \abs{\verts{G}}$. Thus, there exists a bijection
$\phi : \verts{G} \to S$. Fix such a $\phi$. Define a set
$F$ by
\[
F = \set{ \phi\tup{e} \ \mid \ e \in \edges{G} }
= \set{ \phi\tup{u} \phi\tup{v} \ \mid \ uv \in \edges{G} }
\]
(where $\phi\tup{e}$, as usual, denotes the subset
$\set{\phi\tup{x} \ \mid \ x \in e}$ of $S$). Now, let $H$ be the
simple graph $\tup{S, F}$. Then, $\verts{H} = S$. Hence, $\phi$ is a
bijection $\verts{G} \to \verts{H}$. It is straightforward to check
that $\phi$ is a graph isomorphism from $G$ to $H$
(indeed, the definition of $F$ was tailored precisely to make the
equivalence \eqref{eq.def.intro.iso.a.eq} hold). Thus, $H$ is
isomorphic to $G$. As we have already seen, $H$ has vertex set
$\verts{H} = S$. The proof of Proposition~\ref{prop.intro.iso.rename}
is thus complete.
\end{proof}

There is more to say about graph isomorphisms. For example, we can ask
how to check whether two given graphs $G$ and $H$ are isomorphic.
This is a famous problem in computation, known as the
\textit{graph isomorphism problem}, and has seen recent
progress\footnote{See
\url{https://en.wikipedia.org/wiki/Graph_isomorphism_problem} for a
general overview. On the recent progress, see L\'aszl\`o Babai's
preprint \arxiv{1512.03547}
% Add this to list of references once it's finalized; citing a broken
% proof seems stupid.
for the work and \cite{Klarre17} for an amusing popularization.

In theory, the problem can be solved by brute force (just check all
possible bijections $\verts{G} \to \verts{H}$ for being graph
isomorphisms); but this is highly inefficient (the number of
such bijections becomes forbiddingly large very quickly). In many
cases, two non-isomorphic graphs $G$ and $H$ can be ``told apart'' by
some property that holds for one and not the other (e.g., if their
numbers of edges differ). But in general, a simple and fast algorithm
is not known. The problem is in complexity class NP, and Babai's work
claims to prove a quasipolynomial-time (but very complicated)
algorithm. In practice, software written for solving the graph
isomorphism problem makes tradeoffs between simplicity and
efficiency.}.

A related problem is to count, for a given $n \in \NN$,
the isomorphism classes of graphs with $n$ vertices. In other words:
If we pretend that isomorphic graphs are equal, then how many graphs
are there with $n$ vertices? The number is finite, since each graph
with $n$ vertices is isomorphic to a graph with vertex set
$\set{1,2,\ldots, n}$ (this follows from
Proposition~\ref{prop.intro.iso.rename}, applied to
$S = \set{1,2,\ldots, n}$), and the number of the latter graphs is
clearly finite. Yet, computing this number exactly is hard, and there
does not seem to be a closed-form formula. (There is a whole book
\cite{HarPal73} written about this. See also
\url{http://oeis.org/A000088} for references and small values.)

\subsection{\label{sect.intro.paths}Examples of graphs}

\begin{todo}
Define complete graphs; empty graphs; path graphs; cycle graphs; a
bunch of basic graphs; mod-3 congruence graph (just as an example);
2-element subsets with common elements.
\end{todo}

\subsection{\label{sect.intro.walks}Walks and paths}

\begin{definition} \label{def.intro.walks}
Let $G$ be a simple graph.

\textbf{(a)} A \textit{walk} (in $G$) means a sequence
$\tup{v_0, v_1, \ldots, v_k}$ of vertices of $G$ (with $k \geq 0$)
such that all of
$v_0 v_1, v_1 v_2, \ldots, v_{k-1} v_k$ are edges of $G$. (We allow
$k$ to be $0$, in which case the condition that
``$v_0 v_1, v_1 v_2, \ldots, v_{k-1} v_k$ are edges of $G$'' is
vacuously true.)

\textbf{(b)} If $\mathbf{w} = \tup{v_0, v_1, \ldots, v_k}$ is a walk
in $G$, then:

\begin{itemize}
\item The \textit{vertices of $\mathbf{w}$} are defined to be
the vertices $v_0, v_1, \ldots, v_k$.
\item The
\textit{edges of $\mathbf{w}$} are defined to be the edges
$v_0 v_1, v_1 v_2, \ldots, v_{k-1} v_k$ of $G$;
\item The nonnegative integer $k$ is called the
\textit{length} of $\mathbf{w}$. (This integer is the number of all
edges of $G$, counted with multiplicity. It is $1$ smaller than the
number of all vertices of $G$, counted with multiplicity.)
\item The vertex $v_0$ is said to be the \textit{starting point} of
$\mathbf{w}$.
\item The vertex $v_k$ is said to be the \textit{ending point} of
$\mathbf{w}$.
\end{itemize}

\textbf{(c)} A \textit{path} (in $G$) means a walk (in $G$) whose
vertices are distinct. (In other words, a \textit{path} in $G$ means
a walk $\tup{v_0, v_1, \ldots, v_k}$ in $G$ such that
$v_0, v_1, \ldots, v_k$ are distinct.)

\textbf{(e)} Let $p$ and $q$ be two vertices of $G$. A
\textit{walk from $p$ to $q$} (in $G$) means a walk (in $G$) whose
starting point is $p$ and whose ending point is $q$. A
\textit{path from $p$ to $q$} (in $G$) means a path (in $G$) whose
starting point is $p$ and whose ending point is $q$.
\end{definition}

\begin{todo}
Examples!
\end{todo}

\begin{exercise} \label{exe.intro.path.edges-dist}
Let $G$ be a simple graph. Let $\mathbf{w}$ be a path in $G$.
Prove that the edges of $\mathbf{w}$ are distinct. (This may look
obvious when you can point to a picture; but we ask you to give a
rigorous proof!)
\end{exercise}

\begin{proposition} \label{prop.intro.paths-and-walks}
Let $G$ be a simple graph. Let $p$ and $q$ be two vertices of $G$.
The following six statements are equivalent:

\begin{itemize}
\item \textit{Statement 1:} There exists a walk from $u$ to $v$.

\item \textit{Statement 2:} There exists a walk from $v$ to $u$.

\item \textit{Statement 3:} There exists a walk from one of the two
vertices $u$ and $v$ to the other.

\item \textit{Statement 4:} There exists a path from $u$ to $v$.

\item \textit{Statement 5:} There exists a path from $v$ to $u$.

\item \textit{Statement 6:} There exists a path from one of the two
vertices $u$ and $v$ to the other.
\end{itemize}
\end{proposition}

\begin{todo}
Proof.
\end{todo}

\begin{todo}
Dijkstra's algorithm?
\end{todo}

\begin{todo}
Cycles and circuits.
\end{todo}

\subsection{\label{sect.intro.teasers}Questions to ask about graphs}

\begin{todo}
Connectedness.
\end{todo}

\begin{todo}
Hamiltonian paths and cycles:
Do they exist? How to find them? (We will have partial results, mainly sufficient conditions.)
\end{todo}

\begin{todo}
Eulerian walks and circuits.
(Theory is really nice here, with easy necessary-and-sufficient conditions, but we'll wait for a better notion of graph.)
\end{todo}

\begin{todo}
Matchings:
how large, how many, structure? (Nice and large theory, still under research; we will see a lot of it but still barely scratch the surface.)
\end{todo}

\section{\label{sect.dominating}Dominating sets}

I will next digress to discuss the notion of \textit{dominating sets}.
The reason why I am doing this at this point is not that dominating
sets are of any particular fundamental importance (there are arguably
more crucial notions in graph theory left to consider), but rather
that they neatly illustrate the concepts we have seen so far and
provide some experience with graph-theoretical proofs, all that while
not requiring any complex theory or advanced techniques.

\subsection{\label{subsect.dominating.defs}Definition}

\begin{definition} \label{def.dominating}
Let $G$ be a simple graph. A subset $U$ of $\verts{G}$ is said to be
\textit{dominating} (for $G$) if it has the following property: For
every vertex $v \in \verts{G} \setminus U$, at least one neighbor of
$v$ belongs to $U$.

A \textit{dominating set} of $G$ means a dominating subset of
$\verts{G}$.
\end{definition}

\begin{example}
For this example, let us consider the graph $G = \tup{V, E}$, where
\begin{align*}
V &= \set{1, 2, 3, 4, 5} \qquad \text{and} \\
E &= \set{\set{1,2}, \set{2,3}, \set{3,4}, \set{4,5}, \set{5,1}} .
\end{align*}
(This graph has already been introduced in
Example~\ref{exa.simple.R33} \textbf{(d)}. It can be drawn to look
like a pentagon.)

The subset $\set{1, 3}$ of $V$ is dominating (for $G$). (This can be
checked directly: We must show that
for every vertex $v \in \verts{G} \setminus \set{1, 3}$, at least
one neighbor of $v$ belongs to $\set{1, 3}$. There are three vertices
$v \in \verts{G} \setminus \set{1, 3}$, namely $2$, $4$ and $5$. For
$v = 2$, the neighbor $1$ of $v$ belongs to $\set{1, 3}$ (and so does
the neighbor $3$). For $v = 4$, the neighbor $3$ of $v$ belongs to
$\set{1, 3}$. For $v = 5$, the neighbor $1$ of $v$ belongs to
$\set{1, 3}$.)

The subset $\set{1, 2}$ of $V$ is \textbf{not} dominating (for $G$).
(Indeed, the vertex $4 \in \verts{G} \setminus \set{1, 2}$ does
\textbf{not} have the property that at least one neighbor of $4$
belongs to $\set{1, 2}$.)

Every subset of $V$ having at least $3$ elements is dominating,
whereas no subset of $V$ having at most $1$ element is dominating.
A $2$-element subset can be either dominating or not.

(Of course, more complicated graphs exhibit more complex behavior.)
\end{example}

\begin{exercise}
Let $n \in \NN$. What is the smallest possible size of a dominating
set of the cycle graph $C_{3n}$ ?
\end{exercise}

Clearly, if $G$ is a simple graph, then its vertex set $\verts{G}$ is
a dominating set.
One natural question to ask is how small a dominating set of a graph
can be. When the graph $G$ is empty, only the vertex set $\verts{G}$
itself is dominating. On the other hand, when $G$ is a complete graph
on $n \geq 1$ vertices, every nonempty subset of $\verts{G}$ is
dominating. Clearly, the more edges a simple graph has, the more
dominating sets it has (in the sense that if we add a new edge, then
all sets that are dominating remain dominating, and possibly new
dominating sets appear). It is furthermore clear that if a vertex of
a simple graph $G$ has no neighbors, then it must belong to each
dominating set of $G$ (because otherwise, at least one neighbor of
this vertex would need to lie in the dominating set; but this is
impossible, since it has no neighbors). Such vertices are said to be
\textit{isolated}.

\begin{definition} \label{def.intro.isolated}
Let $G$ be a simple graph. A vertex $v$ of $G$ is said to be
\textit{isolated} if it has no neighbors. (In other words, a vertex
$v$ of $G$ is said to be \textit{isolated} if its degree $\deg v$
equals $0$.)
\end{definition}

\begin{proposition} \label{prop.dominating.|V|/2}
Let $G = \tup{V, E}$ be a simple graph that has no isolated vertices.

\textbf{(a)} There exist two disjoint dominating subsets $A$ and $B$
of $V$ such that $A \cup B = V$.

\textbf{(b)} There exists a dominating subset of $V$ having size
$\leq \abs{V}/2$.
\end{proposition}

We will see a proof of this proposition later, when we have defined
the notion of the \textit{distance} between two vertices in a graph.

Again, Proposition~\ref{prop.dominating.|V|/2} can be neatly restated
in terms of people and friendships\footnote{Namely:

\textit{Restatement of
Proposition~\ref{prop.dominating.|V|/2} \textbf{(a)}:} Given a group
of people, each of whom has at least one friend (among the others),
it is always possible to subdivide the group into two teams such that
each person has a friend in the opposite team.

\textit{Restatement of
Proposition~\ref{prop.dominating.|V|/2} \textbf{(b)}:} Given a group
of (finitely many) people each of whom has at least one friend (among
the others),
it is always possible to choose at most $\abs{V}/{2}$ people from this
group such that everyone who is not chosen has at least one of the
chosen ones among his friends.

(As usual, we assume that the group of people is finite, and that the
relation of friendship is symmetric.)}.

\begin{todo}
Bound in Proposition~\ref{prop.dominating.|V|/2} \textbf{(b)}
may and may not be sharp: examples.
\end{todo}

\subsection{\label{subsect.dominating.odd}Brouwer's theorem and the
Heinrich-Tittmann formula}

Next, we state a surprising recent result by Brouwer (\cite{Brouwe09},
from 2009) about the number of dominating sets of a graph:

\begin{theorem} \label{thm.dominating.brouwer}
Let $G$ be a simple graph. Then, the number of dominating sets of $G$
is odd.
\end{theorem}

Brouwer (in \cite{Brouwe09}) gives three proofs of this theorem. We
are going to give another. Better yet, we shall prove a more
precise result which is even more recent (a preprint \cite{HeiTit17}
from 2017), due to Heinrich and Tittmann:

\begin{theorem} \label{thm.dominating.heinrich}
Let $G = \tup{V, E}$ be a simple graph. Let $n = \abs{V}$. Assume that
$n > 0$.

A \textit{detached pair} shall mean a pair $\tup{A, B}$ of two
disjoint subsets $A$ and $B$ of $V$ having the property that
there exists no edge $ab \in E$ satisfying
$a \in A$ and $b \in B$.

Let $\alpha$ be the number of all detached pairs $\tup{A, B}$ for
which both numbers $\abs{A}$ and $\abs{B}$ are even and positive.

Let $\beta$ be the number of all detached pairs $\tup{A, B}$ for
which both numbers $\abs{A}$ and $\abs{B}$ are odd.

Then:

\textbf{(a)} The numbers $\alpha$ and $\beta$ are even.

\textbf{(b)} The number of dominating sets of $G$ is
$2^n - 1 + \alpha - \beta$.
\end{theorem}

At this point, let me stress that the word ``pair'' always means an
ordered pair throughout these notes. In particular, in
Theorem~\ref{thm.dominating.heinrich}, a detached pair $\tup{A, B}$
should be distinguished from $\tup{B, A}$, unless they actually are
equal (which only happens when both $A$ and $B$ are the empty set).

Theorem~\ref{thm.dominating.heinrich} is a restatement of
\cite[Theorem 8]{HeiTit17}. The proof we shall give below is
shorter than the proof in \cite{HeiTit17}, but does not lead us
through as many interesting intermediate results.

Let us first see how Theorem~\ref{thm.dominating.brouwer} can be
derived from Theorem~\ref{thm.dominating.heinrich}:

\begin{proof}[Proof of Theorem~\ref{thm.dominating.brouwer} using
Theorem~\ref{thm.dominating.heinrich}.]

Write the graph $G$ in the form $G = \tup{V, E}$. If $\abs{V} = 0$,
then Theorem~\ref{thm.dominating.brouwer}
holds\footnote{\textit{Proof.} Assume that $\abs{V} = 0$. Hence,
the set $V$ is empty. Thus, the only subset of $V$ is $\varnothing$.
This subset $\varnothing$ is dominating (because it is the whole
set $V$). Thus, there exists exactly $1$ dominating set of $G$
(namely, $\varnothing$). In other words, the number of dominating
sets of $G$ is $1$. Therefore, this number is odd. Hence,
Theorem~\ref{thm.dominating.brouwer} is proven (under the assumption
that $\abs{V} = 0$.}. Hence, for the rest of this proof, we
WLOG asssume that $\abs{V} = 0$ does not hold.

Let us use the notations of Theorem~\ref{thm.dominating.heinrich}.
From Theorem~\ref{thm.dominating.heinrich} \textbf{(a)}, we know
that $\alpha$ and $\beta$ are even. In other words,
$\alpha \equiv 0 \mod 2$ and $\beta \equiv 0 \mod 2$.
Furthermore, $n = \abs{V} \neq 0$ (since $\abs{V} = 0$ does not
hold), so that $n$ is a positive integer. Thus, $2^n$ is even.
In other words, $2^n \equiv 0 \mod 2$. Now,
Theorem~\ref{thm.dominating.heinrich} \textbf{(b)} shows that
the number of dominating sets of $G$ is
\[
\underbrace{2^n}_{\equiv 0 \mod 2} - 1
  + \underbrace{\alpha}_{\equiv 0 \mod 2}
  - \underbrace{\beta}_{\equiv 0 \mod 2}
\equiv 0 - 1 + 0 - 0 = -1 \equiv 1 \mod 2.
\]
In other words, the number of dominating sets of $G$ is odd.
This proves Theorem~\ref{thm.dominating.brouwer}.
\end{proof}

Thus, it remains to prove Theorem~\ref{thm.dominating.heinrich}.

\subsection{\label{subsect.dominating.iverson}The Iverson bracket}

Our proof of Theorem~\ref{thm.dominating.heinrich} will rely on
a few lemmas. But first, let us introduce a very basic notation, which
has nothing to do with graphs specifically but is useful
throughout mathematics (particularly combinatorics):

\begin{definition} \label{def.intro.iverson}
Let $\mathcal{A}$ be a logical statement. (It is not required to be
true.) Then, a number
$\ive{\mathcal{A}} \in \set{0, 1}$ is defined as follows:
We set $\ive{\mathcal{A}} =
\begin{cases}
1, & \text{if }\mathcal{A}\text{ is true};\\
0, & \text{if }\mathcal{A}\text{ is false}
\end{cases}$.
This number $\ive{\mathcal{A}}$ is called the \textit{truth value}
of $\mathcal{A}$.
(For example, $\ive{1+1 = 2} = 1$ and $\ive{1+1 = 3} = 0$. For
another example,
$\ive{\text{Proposition~\ref{prop.simple.R33} holds}}
= 1$, because we have proven Proposition~\ref{prop.simple.R33}.)
The notation $\ive{\mathcal{A}}$ for the truth value of $\mathcal{A}$
is known as the \textit{Iverson bracket notation}.
\end{definition}

Truth values satisfy certain simple rules:

\begin{proposition} \label{prop.intro.iverson.rules}
\textbf{(a)} If $\mathcal{A}$ and $\mathcal{B}$ are two equivalent
logical statements, then
$\ive{\mathcal{A}} = \ive{\mathcal{B}}$.

\textbf{(b)} If $\mathcal{A}$ is any logical statement, then
$\ive{\text{not } \mathcal{A}} = 1 - \ive{\mathcal{A}}$.

\textbf{(c)} If $\mathcal{A}$ and $\mathcal{B}$ are two logical
statements, then
$\ive{\mathcal{A} \wedge \mathcal{B}}
= \ive{\mathcal{A}} \ive{\mathcal{B}}$.

\textbf{(d)} If $\mathcal{A}$ and $\mathcal{B}$ are two logical
statements, then
$\ive{\mathcal{A} \vee \mathcal{B}}
= \ive{\mathcal{A}} + \ive{\mathcal{B}}
  - \ive{\mathcal{A}} \ive{\mathcal{B}}$.
\end{proposition}

\begin{proposition} \label{prop.intro.iverson.sums}
Let $P$ be a finite set. Let $Q$ be a subset of $P$.

\textbf{(a)} Then, $\abs{Q} = \sum_{p \in P} \ive{p \in Q}$.

\textbf{(b)} For each $p \in P$, let $a_p$ be a number (for example,
a real number). Then, $\sum_{p \in P} \ive{p \in Q} a_p
= \sum_{p \in Q} a_p$.

\textbf{(c)} For each $p \in P$, let $a_p$ be a number (for example,
a real number). Let $q \in P$. Then,
$\sum_{p \in P} \ive{p = q} a_p = a_q$.
\end{proposition}

\begin{exercise} \label{exe.intro.iverson}
\textbf{(a)} Prove Proposition~\ref{prop.intro.iverson.rules}.

\textbf{(b)} Prove Proposition~\ref{prop.intro.iverson.sums}.

Now, let $G$ be a graph.

\textbf{(c)} Prove that
$\deg v = \sum_{u \in \verts{G}} \ive{uv \in \edges{G}}$
for each vertex $v$ of $G$.

\textbf{(d)} Prove that
$2 \abs{\edges{G}}
= \sum_{u \in \verts{G}} \sum_{v \in \verts{G}}
  \ive{uv \in \edges{G}}$.
\end{exercise}

The following lemma is fundamental to much of combinatorics (if not
to say much of mathematics):

\begin{lemma} \label{lem.dominating.heinrich-lemma1}
Let $P$ be a finite set. Then,
\[
\sum_{\substack{A \subseteq P}} \tup{-1}^{\abs{A}}
= \ive{P = \varnothing} .
\]
(The symbol ``$\sum_{\substack{A \subseteq P}}$'' means ``sum over
all subsets $A$ of $P$''. In other words, it means
``$\sum_{\substack{A \in \powset{P}}}$''.)
\end{lemma}

\begin{proof}[Proof of Lemma~\ref{lem.dominating.heinrich-lemma1}.]
If $P = \varnothing$, then Lemma~\ref{lem.dominating.heinrich-lemma1}
holds\footnote{\textit{Proof.} Assume that $P = \varnothing$. Then,
there exists only one subset of $P$, namely $\varnothing$. Hence,
the sum $\sum_{\substack{A \subseteq P}} \tup{-1}^{\abs{A}}$
has only one addend, namely the addend for $A = \varnothing$.
Therefore, this sum simplifies to
\[
\sum_{\substack{A \subseteq P}} \tup{-1}^{\abs{A}}
= \tup{-1}^{\abs{\varnothing}} = 1
\]
(since $\abs{\varnothing} = 0$). Comparing this with
$\ive{P = \varnothing} = 1$ (which holds, since $P = \varnothing$ is
true), we obtain
$\sum_{\substack{A \subseteq P}} \tup{-1}^{\abs{A}}
= \ive{P = \varnothing}$. Hence, we have shown that
Lemma~\ref{lem.dominating.heinrich-lemma1} holds if
$P = \varnothing$.}. Hence, for the rest of this proof, we WLOG
assume that $P \neq \varnothing$. Thus, there exists at least one
element $p$ of $P$. Pick such a $p$.

There are two kinds of subsets of $P$: the ones that contain $p$,
and the ones that do not. Hence, the sum
$\sum_{\substack{A \subseteq P}} \tup{-1}^{\abs{A}}$ can be
decomposed as follows:
\begin{equation}
\sum_{\substack{A \subseteq P}} \tup{-1}^{\abs{A}}
= \sum_{\substack{A \subseteq P; \\ p \in A}} \tup{-1}^{\abs{A}}
  + \sum_{\substack{A \subseteq P; \\ p \notin A}} \tup{-1}^{\abs{A}} .
\label{pf.lem.dominating.heinrich-lemma1.1}
\end{equation}

But every subset $A$ of $P$ that contains $p$ has the form
$B \cup \set{p}$ for a \textbf{unique} subset $B$ of $P$ that does
not contain $p$ (namely, for $B = A \setminus \set{p}$). Conversely,
of course, if $B$ is a subset of $P$ that does not contain $p$,
then $B \cup \set{p}$ will always be a subset of $P$ that contains
$p$. Hence, there exists a bijection (i.e., a bijective map)
from the set of all subsets of $P$ that do not contain $p$
  to the set of all subsets of $P$ that do contain $p$;
namely, this bijection sends each subset $B$ of $P$ that does not
contain $p$ to $B \cup \set{p}$. Using this bijection, we can rewrite
the sum
$\sum_{\substack{A \subseteq P; \\ p \in A}} \tup{-1}^{\abs{A}}$ as
follows:
\begin{align*}
\sum_{\substack{A \subseteq P; \\ p \in A}} \tup{-1}^{\abs{A}}
&= \sum_{\substack{B \subseteq P; \\ p \notin B}}
   \underbrace{\tup{-1}^{\abs{B \cup \set{p}}}}_{\substack{
     = \tup{-1}^{\abs{B}+1} \\
     \text{(since } \abs{B \cup \set{p}} = \abs{B} + 1 \\
     \text{(because } p \notin B \text{))}
   }}
= \sum_{\substack{B \subseteq P; \\ p \notin B}}
  \underbrace{\tup{-1}^{\abs{B}+1}}_{= - \tup{-1}^{\abs{B}}}
= \sum_{\substack{B \subseteq P; \\ p \notin B}}
  \tup{- \tup{-1}^{\abs{B}}} \\
&= - \sum_{\substack{B \subseteq P; \\ p \notin B}} \tup{-1}^{\abs{B}}
= - \sum_{\substack{A \subseteq P; \\ p \notin A}} \tup{-1}^{\abs{A}}
\end{align*}
(here, we have renamed the summation index $B$ as $A$).
Thus, \eqref{pf.lem.dominating.heinrich-lemma1.1} becomes
\[
\sum_{\substack{A \subseteq P}} \tup{-1}^{\abs{A}}
= \underbrace{\sum_{\substack{A \subseteq P; \\ p \in A}}
       \tup{-1}^{\abs{A}}}_{
     = - \sum_{\substack{A \subseteq P; \\ p \notin A}}
         \tup{-1}^{\abs{A}}}
  + \sum_{\substack{A \subseteq P; \\ p \notin A}} \tup{-1}^{\abs{A}}
= - \sum_{\substack{A \subseteq P; \\ p \notin A}} \tup{-1}^{\abs{A}}
  + \sum_{\substack{A \subseteq P; \\ p \notin A}} \tup{-1}^{\abs{A}}
= 0.
\]
Comparing this with $\ive{P = \varnothing} = 0$ (which holds, since
$P = \varnothing$ is false (because we assumed $P \neq \varnothing$)),
we obtain
$\sum_{\substack{A \subseteq P}} \tup{-1}^{\abs{A}}
= \ive{P = \varnothing}$. Hence,
Lemma~\ref{lem.dominating.heinrich-lemma1} is proven.
\end{proof}

A different proof of Lemma~\ref{lem.dominating.heinrich-lemma1} can
be obtained using the binomial formula and the combinatorial
interpretation of binomial coefficients. The main idea of this latter
proof is to observe that for each $k \in \set{0,1,\ldots,\abs{P}}$,
exactly $\dbinom{\abs{P}}{k}$ among the subsets of $P$ have size $k$,
and therefore
$\sum_{\substack{A \subseteq P}} \tup{-1}^{\abs{A}}$ can be rewritten
as $\sum_{k=0}^{\abs{P}} \dbinom{\abs{P}}{k} \tup{-1}^k$, which in
turn can be simplified (using the binomial formula) to
$\tup{1-1}^{\abs{P}} = 0^{\abs{P}} = \ive{\abs{P} = 0}
= \ive{P = \varnothing}$. We leave the details of this alternative
proof to the interested reader.

The following exercise demonstrates an application of
Lemma~\ref{lem.dominating.heinrich-lemma1} to (elementary) number
theory:

\begin{exercise} \label{exe.dominating.heinrich-lemma-moeb-NT}
Let $\NN_+$ denote the set $\set{1, 2, 3, \ldots}$. An integer $n$ is
said to be \textit{squarefree} if it is not divisible by any perfect
square apart from $1$ (or, equivalently, if it is a product of
\textbf{distinct} prime numbers). We define a map
$\mu : \NN_+ \to \set{-1, 0, 1}$ by setting
\[
\mu\tup{n}
= \begin{cases}
\tup{-1}^{\omega\tup{n}}, & \text{ if } n \text{ is squarefree;} \\
0,                        & \text{ if not }
\end{cases}
\qquad \text{ for each } n \in \NN_+ ,
\]
where $\omega\tup{n}$ denotes the number of distinct prime divisors
of $n$. (For example, $\mu\tup{6} = \tup{-1}^2 = 1$,
$\mu\tup{30} = \tup{-1}^3 = -1$,
$\mu\tup{1} = \tup{-1}^0 = 1$, and $\mu\tup{12} = 0$. The map $\mu$
is called the \textit{(number-theoretical) M\"obius function}.)

Prove that each $n \in \NN_+$ satisfies
\[
\sum_{d\mid n} \mu\tup{d} = \ive{n = 1},
\]
where the sum on the left hand side should be understood as a sum over
all positive divisors of $n$.

[\textbf{Hint:} Apply Lemma~\ref{lem.dominating.heinrich-lemma1} with
$P$ being the set of all prime factors of $n$. Relate the subsets of
$P$ to the squarefree divisors of $n$.]
\end{exercise}

\begin{exercise} \label{exe.dominating.heinrich-lemma-sum}
Let $A$ be a finite set. Let $A_1, A_2, \ldots, A_m$ be some subsets
of $A$. Prove that
\begin{align*}
& \abs{\set{U \subseteq A \ \mid \ U \cap A_i \neq \varnothing
          \text{ for all } i \in \set{1, 2, \ldots, m}}} \\
&\equiv
\abs{\set{J \subseteq \set{1, 2, \ldots, m} \ \mid
          \ \bigcup_{j \in J} A_j = A }}
\mod 2.
\end{align*}

[\textbf{Hint:} Set $I = \set{1, 2, \ldots, m}$. Compute the double
sum
\[
\sum_{U \subseteq A} \sum_{J \subseteq I}
\tup{-1}^{\abs{J}} \tup{-1}^{\abs{U}}
\ive{U \subseteq A \setminus \bigcup_{j \in J} A_j}
\]
in two different ways (once by interchanging the summations, and once
again by rewriting
$\ive{U \subseteq A \setminus \bigcup_{j \in J} A_j}$ as
$\ive{J \subseteq \set{i \in I \mid U \cap A_i = \varnothing}}$).
Notice that powers of $-1$ can be discarded when working
modulo $2$.]
\end{exercise}

\subsection{\label{subsect.dominating.lemmas}Proving
the Heinrich-Tittmann formula}

In this section, we shall finally prove
Theorem~\ref{thm.dominating.heinrich}. Instead of presenting the
proof as a monolithic piece of work, I shall distribute most of it
into a series of easy lemmas (some of which are of independent
interest).

Throughout this
section, we consider a simple graph $G = \tup{V, E}$. The notion of a
``detached pair'' is to be understood as in
Theorem~\ref{thm.dominating.heinrich}.

\begin{lemma} \label{lem.dominating.heinrich-lemma2}
Let $B$ be a subset of $V$. Then,
\[
\sum_{\substack{A \subseteq V; \\ \tup{A, B}
\text{ is a detached pair}}} \tup{-1}^{\abs{A}}
= \ive{B \text{ is dominating}}.
\]
\end{lemma}

\begin{proof}[Proof of Lemma~\ref{lem.dominating.heinrich-lemma2}.]
Let $B'$ be the set of all vertices $v \in V \setminus B$ such
that at least one neighbor of $v$ belongs to $B$. Then, for each
subset $A$ of $V$, the following equivalence holds:
\begin{equation}
\left( \tup{A, B} \text{ is a detached pair} \right)
\Longleftrightarrow
\left( A \subseteq V \setminus \tup{B \cup B'} \right)
\label{pf.lem.dominating.heinrich-lemma2.1}
\end{equation}
\footnote{\textit{Proof of
\eqref{pf.lem.dominating.heinrich-lemma2.1}.} We shall prove the
$\Longrightarrow$ and $\Longleftarrow$ directions of the equivalence
\eqref{pf.lem.dominating.heinrich-lemma2.1} separately:

$\Longrightarrow$: Assume that $\tup{A, B}$ is a detached pair. We
must show that $A \subseteq V \setminus \tup{B \cup B'}$.

We know that $\tup{A, B}$ is a detached pair. By the definition of a
``detached pair'', this means that $A$ and $B$ are two disjoint
subsets of $V$ having the property that there exists no edge
$ab \in E$ satisfying $a \in A$ and $b \in B$.

We have $A \subseteq V \setminus B$ (since $A$ and $B$ are
disjoint).

We claim that $A \cap B' = \varnothing$. Indeed, assume the contrary.
Thus, the set $A \cap B'$ is nonempty. Hence, there exists some
$v \in A \cap B'$. Fix such a $v$.
From $v \in A \cap B' \subseteq B'$, we conclude that $v$ is a vertex
in $V \setminus B$ such that at least one neighbor of $v$
belongs to $B$ (by the definition of $B'$). In particular, at least
one neighbor of $v$ belongs to $B$. Let $w$ be such a neighbor. Then,
$vw \in E$ (since $w$ is a neighbor of $v$). Notice also that
$v \in A \cap B' \subseteq A$ and $w \in B$.
Now, recall that there exists no edge $ab \in E$ satisfying $a \in A$
and $b \in B$. This contradicts the fact that $vw$ is such an edge
(since $vw \in E$, $v \in A$ and $w \in B$). This contradiction shows
that our assumption was wrong. Hence, $A \cap B' = \varnothing$.
Thus, $A \subseteq V \setminus B'$.

Combining $A \subseteq V \setminus B$ with
$A \subseteq V \setminus B'$, we find
\[
A \subseteq \tup{V \setminus B} \cap \tup{V \setminus B'}
  = V \setminus \tup{B \cup B'} .
\]
Hence, the $\Longrightarrow$ direction of the equivalence
\eqref{pf.lem.dominating.heinrich-lemma2.1} is proven.

$\Longleftarrow$: Assume that
$A \subseteq V \setminus \tup{B \cup B'}$. We must then show
that $\tup{A, B}$ is a detached pair.

First of all, we have
$A \subseteq V \setminus \tup{B \cup B'}
\subseteq V \setminus B$ (since $B \cup B' \supseteq B$).
Thus, the sets $A$ and $B$ are disjoint.

Next, I claim that there exists no edge
$ab \in E$ satisfying $a \in A$ and $b \in B$. Indeed, assume the
contrary. Thus, there exists an edge $ab \in E$ satisfying $a \in A$
and $b \in B$. Consider such an edge. Then, $b$ is a neighbor of $a$
(since $ab \in E$) and belongs to $B$. Hence, at least one neighbor
of $a$ belongs to $B$ (namely, the neighbor $b$).
The element $a$ is a vertex in
$V \setminus B$ (since $a \in A \subseteq V \setminus B$) such that
at least one neighbor of $a$ belongs to $B$. In other words, $a$
belongs to $B'$ (by the definition of $B'$). Thus, $a \in B'
\subseteq B \cup B'$. But from
$a \in A \subseteq V \setminus \tup{B \cup B'}$, we obtain
$a \notin B \cup B'$. This contradicts $a \in B \cup B'$. This
contradiction shows that our assumption was wrong. Hence, we have
shown that there exists no edge
$ab \in E$ satisfying $a \in A$ and $b \in B$. Since the subsets $A$
and $B$ of $V$ are disjoint, this shows that $\tup{A, B}$ is a
detached pair (by the definition of a ``detached pair''). This proves
the $\Longleftarrow$ direction of the equivalence
\eqref{pf.lem.dominating.heinrich-lemma2.1}.

Hence, both directions of \eqref{pf.lem.dominating.heinrich-lemma2.1}
are proven.}.

On the other hand, the following equivalence holds:
\begin{equation}
\left( V \setminus \tup{B \cup B'} = \varnothing \right)
\Longleftrightarrow
\left( B \text{ is dominating} \right)
\label{pf.lem.dominating.heinrich-lemma2.2}
\end{equation}
\footnote{\textit{Proof of
\eqref{pf.lem.dominating.heinrich-lemma2.2}.}
% We have the following
% chain of equivalences:
% \begin{align*}
% & \left( V \setminus \tup{B \cup B'} = \varnothing \right) \\
% \Longleftrightarrow
% & \left( \tup{V \setminus B} \setminus B' = \varnothing \right)
% \qquad \left( \text{since } V \setminus \tup{B \cup B'} =
                % \tup{V \setminus B} \setminus B' \right) \\
% & \left( V \setminus B \subseteq B' \right) \\
% & \left( \text{for every vertex } v \in V \setminus B
            % \text{, we have } \underbrace{v \in B'}_{\substack{
                % \Longleftrightarrow
                % \left( v \text{ is a vertex
                % \right) \\
% & \left( \text{for every vertex } v \in V \setminus B
            % \text{, we have } \v \in B' \right) 
% \end{align*}
Again, we are going to
separately prove the $\Longrightarrow$ and $\Longleftarrow$ directions
of the equivalence:

$\Longrightarrow$: Assume that
$V \setminus \tup{B \cup B'} = \varnothing$. We must show that $B$
is dominating.

Let $v \in \verts{G} \setminus B$. Thus, $v \in \verts{G}$ and
$v \notin B$. We have $v \in \verts{G} = V \subseteq B \cup B'$
(since $V \setminus \tup{B \cup B'} = \varnothing$). Combined with
$v \notin B$, we obtain $v \in \tup{B \cup B'} \setminus B \subseteq
B'$. According to the definition of $B'$, this means that $v$ is a
vertex in $V \setminus B$ such that at least one neighbor of
$v$ belongs to $B$. In particular, we thus have shown that at least
one neighbor of $v$ belongs to $B$.

Now, forget that we fixed $v$. We thus have proven that for
every vertex $v \in \verts{G} \setminus B$, at least one neighbor of
$v$ belongs to $B$. In other words, the set $B$ is dominating
(because this is how we defined ``dominating sets''). This proves
the $\Longrightarrow$ direction of the equivalence
\eqref{pf.lem.dominating.heinrich-lemma2.2}.

$\Longleftarrow$: Assume that $B$ is dominating. We must prove that
$V \setminus \tup{B \cup B'} = \varnothing$.

We know that $B$ is dominating. In other words, for every vertex
$v \in \verts{G} \setminus B$, at least one neighbor of $v$ belongs to
$B$ (because this is what ``dominating'' means). In other words, for
every $v \in V \setminus B$, at least one neighbor of $v$ belongs to
$B$ (since $\verts{G} = V$).

Now, let $v \in V \setminus B$ be arbitrary. As we have just seen,
we then know that at least one neighbor of $v$ belongs to $B$.

Let $v \in V \setminus B$. Thus, $v$ is a
vertex in $V \setminus B$ such that at least one neighbor of
$v$ belongs to $B$. This means that $v \in B'$ (by the definition of
$B'$).

Now, forget that we fixed $v$. We thus have shown that $v \in B'$ for
each $v \in V \setminus B$. In other words,
$V \setminus B \subseteq B'$. But now,
$ V \setminus \tup{B \cup B'} = \tup{V \setminus B} \setminus B'
= \varnothing$
(since $V \setminus B \subseteq B'$).
Thus, the $\Longleftarrow$ direction of the equivalence
\eqref{pf.lem.dominating.heinrich-lemma2.2} is proven.}

Now, we can use the equivalence
\eqref{pf.lem.dominating.heinrich-lemma2.1} to rewrite the summation
sign $\sum_{\substack{A \subseteq V; \\ \tup{A, B}
\text{ is a detached pair}}}$ as
$\sum_{\substack{A \subseteq V; \\
A \subseteq V \setminus \tup{B \cup B'}}}$. This latter summation sign
can be further simplified to
$\sum_{A \subseteq V \setminus \tup{B \cup B'}}$ (since every subset
$A$ of $V \setminus \tup{B \cup B'}$ is clearly a subset of $V$ as
well). Hence, we can replace the summation sign
$\sum_{\substack{A \subseteq V; \\ \tup{A, B}
\text{ is a detached pair}}}$ by
$\sum_{A \subseteq V \setminus \tup{B \cup B'}}$. In particular,
\begin{align*}
& \sum_{\substack{A \subseteq V; \\ \tup{A, B}
\text{ is a detached pair}}} \tup{-1}^{\abs{A}} \\
&= \sum_{A \subseteq V \setminus \tup{B \cup B'}} \tup{-1}^{\abs{A}}
\\
&= \ive{ V \setminus \tup{B \cup B'} = \varnothing }
\qquad
\left( \text{by Lemma~\ref{lem.dominating.heinrich-lemma1}, applied
         to } P = V \setminus \tup{B \cup B'} \right) \\
&= \ive{B \text{ is dominating}}
\end{align*}
(by the equivalence \eqref{pf.lem.dominating.heinrich-lemma2.2}).
This proves Lemma~\ref{lem.dominating.heinrich-lemma2}.
\end{proof}

Lemma~\ref{lem.dominating.heinrich-lemma2} has the following
consequence:

\begin{corollary} \label{cor.dominating.heinrich-lemma2c}
Let $B$ be a subset of $V$. Then,
\[
\sum_{\substack{A \subseteq V; \\ A \neq \varnothing; \\
\tup{A, B} \text{ is a detached pair}}}
\tup{-1}^{\abs{A}}
= \ive{B \text{ is dominating}} - 1.
\]
\end{corollary}

\begin{proof}[Proof of Corollary~\ref{cor.dominating.heinrich-lemma2c}.]
The pair $\tup{\varnothing, B}$ is a detached
pair\footnote{\textit{Proof.} The sets
$\varnothing$ and $B$ are two disjoint subsets of $V$
(disjoint because $\varnothing \cap B = \varnothing$)
having the property that there exists no edge
$ab \in E$ satisfying $a \in \varnothing$ and $b \in B$
(this is vacuously true, since there exists no
$a \in \varnothing$). In other words, $\tup{\varnothing, B}$ is
a detached pair (by the definition of a ``detached pair'').}.
Hence, the sum \newline
$\sum_{\substack{A \subseteq V; \\ \tup{A, B}
\text{ is a detached pair}}} \tup{-1}^{\abs{A}}$
has an addend for $A = \varnothing$. If we split out this addend
from this sum, we obtain
\[
\sum_{\substack{A \subseteq V; \\ \tup{A, B}
\text{ is a detached pair}}} \tup{-1}^{\abs{A}}
=
\tup{-1}^{\abs{\varnothing}}
+
\sum_{\substack{A \subseteq V; \\ \tup{A, B}
\text{ is a detached pair}; \\ A \neq \varnothing}}
\tup{-1}^{\abs{A}} .
\]
Hence,
\begin{align*}
\sum_{\substack{A \subseteq V; \\ \tup{A, B}
\text{ is a detached pair}; \\ A \neq \varnothing}}
\tup{-1}^{\abs{A}}
&=
\underbrace{\sum_{\substack{A \subseteq V; \\ \tup{A, B}
             \text{ is a detached pair}}} \tup{-1}^{\abs{A}}}
           _{\substack{= \ive{B \text{ is dominating}} \\
             \text{(by Lemma~\ref{lem.dominating.heinrich-lemma2})}}}
- \underbrace{\tup{-1}^{\abs{\varnothing}}}_{
                =\tup{-1}^0=1} \\
&= \ive{B \text{ is dominating}} - 1.
\end{align*}
Thus,
\begin{align*}
\sum_{\substack{A \subseteq V; \\ A \neq \varnothing; \\
\tup{A, B} \text{ is a detached pair}}}
\tup{-1}^{\abs{A}}
= \sum_{\substack{A \subseteq V; \\ \tup{A, B}
\text{ is a detached pair}; \\ A \neq \varnothing}}
\tup{-1}^{\abs{A}}
= \ive{B \text{ is dominating}} - 1.
\end{align*}
This proves Corollary~\ref{cor.dominating.heinrich-lemma2c}.
\end{proof}

\begin{lemma} \label{lem.dominating.heinrich-lemma3}
Let $A$ and $B$ be two subsets of $V$. Then, $\tup{A, B}$ is a
detached pair if and only if $\tup{B, A}$ is a detached pair.
\end{lemma}

\begin{proof}[Proof of Lemma~\ref{lem.dominating.heinrich-lemma3}.]
We have the following chain of logical equivalences:
\begin{align*}
& \left( \tup{B, A} \text{ is a detached pair} \right) \\
\Longleftrightarrow \ 
& \left( B \text{ and } A \text{ are two disjoint subsets of } V
     \text{ having the property that } \right. \\
& \qquad \qquad \left. \text{ there exists no edge } ab \in E
     \text{ satisfying } a \in B \text{ and } b \in A \right) \\
& \qquad \left( \text{by the definition of ``detached pair''} \right)
\\
\Longleftrightarrow \ 
& \left( B \text{ and } A \text{ are two disjoint subsets of } V
     \text{ having the property that } \right. \\
& \qquad \qquad \left. \text{ there exists no edge } ba \in E
     \text{ satisfying } b \in B \text{ and } a \in A \right) \\
& \qquad \left( \text{here, we renamed } a \text{ and } b \text{ as }
            b \text{ and } a\text{, respectively} \right)
\\
\Longleftrightarrow \ 
& \left( A \text{ and } B \text{ are two disjoint subsets of } V
     \text{ having the property that } \right. \\
& \qquad \qquad \left. \text{ there exists no edge } ba \in E
     \text{ satisfying } a \in A \text{ and } b \in B \right) \\
& \qquad \left( \text{since } B \text{ and } A
            \text{ are disjoint if and only if }
            A \text{ and } B \text{ are disjoint} \right)
\\
\Longleftrightarrow \ 
& \left( A \text{ and } B \text{ are two disjoint subsets of } V
     \text{ having the property that } \right. \\
& \qquad \qquad \left. \text{ there exists no edge } ab \in E
     \text{ satisfying } a \in A \text{ and } b \in B \right) \\
& \qquad \left( \text{since } ba = ab \right)
\\
\Longleftrightarrow \ 
& \left( \tup{A, B} \text{ is a detached pair} \right) \\
& \qquad \left( \text{by the definition of ``detached pair''} \right)
.
\end{align*}
This proves Lemma~\ref{lem.dominating.heinrich-lemma3}.
\end{proof}

Next comes another general lemma about cardinalities of sets:

\begin{lemma} \label{lem.dominating.heinrich-lemma-inv}
Let $S$ be a finite set. Let $\sigma : S \to S$ be a map such that
$\sigma \circ \sigma = \id_S$. (Such a map is called an
\textit{involution} on $S$.)

Let $F = \set{ i \in S \mid \sigma\tup{i} = i }$. (The elements of $F$
are known as the \textit{fixed points} of $\sigma$.) Then,
$\abs{F} \equiv \abs{S} \mod 2$.
\end{lemma}

\begin{proof}[Proof of Lemma~\ref{lem.dominating.heinrich-lemma-inv}.]
In a nutshell, the proof of
Lemma~\ref{lem.dominating.heinrich-lemma-inv} is very simple: We have
$\sigma \circ \sigma = \id_S$; in other words, we have
$\sigma\tup{\sigma\tup{i}} = i$ for each $i \in S$. Each
$i \in S \setminus F$ satisfies $\sigma\tup{i} \neq i$. Thus, we can
assign to each element $i \in S \setminus F$ the two-element subset
$\set{i, \sigma\tup{i}} \subseteq S$. This assignment has the property
that the two-element subset assigned to $\sigma\tup{i}$ is the same
as the one assigned to $i$ (since
$\set{\sigma\tup{i}, \underbrace{\sigma\tup{\sigma\tup{i}}}_{= i}}
= \set{\sigma\tup{i}, i} = \set{i, \sigma\tup{i}}$).
Thus, each two-element subset
that gets assigned at all is assigned to exactly two elements of
$S \setminus F$ (namely, the subset assigned to $i$ is assigned to
$i$ and $\sigma\tup{i}$, and to no other elements of
$S \setminus F$).
As a consequence, the elements of $S \setminus F$ are paired up
(each pair consisting of two elements to which the same subset is
assigned). Correspondingly, $\abs{S \setminus F}$ is even. Thus,
$\abs{S} - \abs{F} = \abs{S \setminus F}$ is even, so that
$\abs{F} \equiv \abs{S} \mod 2$.

If you found this insufficiently rigorous (or unclear), here is a
\textit{rigorous version of this proof:}
For each $i \in S \setminus F$, we have
$\set{i, \sigma\tup{i}} \in \powset[2]{S}$
\ \ \ \ \footnote{\textit{Proof.} Let $i \in S \setminus F$. Thus,
$i \in S$ and $i \notin F$. If we had $\sigma\tup{i} = i$, then we
would have $i \in F$ (by the definition of $F$), which would
contradict $i \notin F$. Hence, we do not have
$\sigma\tup{i} = i$. Thus, we have $\sigma\tup{i} \neq i$. Hence,
$\set{i, \sigma\tup{i}}$ is a $2$-element set. Since
$\set{i, \sigma\tup{i}} \subseteq S$ (because both $i$ and
$\sigma\tup{i}$ are elements of $S$), this shows that
$\set{i, \sigma\tup{i}}$ is a $2$-element subset of $S$. In other
words, $\set{i, \sigma\tup{i}} \in \powset[2]{S}$, qed.}. In other
words, for each $i \in S \setminus F$, there exists a unique
$K \in \powset[2]{S}$ satisfying $\set{i, \sigma\tup{i}} = K$
(namely, $K = \set{i, \sigma\tup{i}}$).

But
\begin{align}
\abs{S \setminus F}
&= \tup{ \text{the number of } i \in S \setminus F } \nonumber \\
&= \sum_{K \in \powset[2]{S}}
      \tup{ \text{the number of } i \in S \setminus F
              \text{ satisfying } \set{i, \sigma\tup{i}} = K }
\label{pf.lem.dominating.heinrich-lemma-inv.1}
\end{align}
(because for each $i \in S \setminus F$, there exists a unique
$K \in \powset[2]{S}$ satisfying $\set{i, \sigma\tup{i}} = K$).

On the other hand, for each $K \in \powset[2]{S}$, we have
\begin{equation}
\tup{ \text{the number of } i \in S \setminus F
      \text{ satisfying } \set{i, \sigma\tup{i}} = K }
\equiv 0 \mod 2
\label{pf.lem.dominating.heinrich-lemma-inv.2}
\end{equation}
\footnote{\textit{Proof of
\eqref{pf.lem.dominating.heinrich-lemma-inv.2}.}
Let $K \in \powset[2]{S}$. If there exists no $i \in S \setminus F$
satisfying $\set{i, \sigma\tup{i}} = K$, then
\eqref{pf.lem.dominating.heinrich-lemma-inv.2} is true (because
in this case, we have
$\tup{ \text{the number of } i \in S \setminus F
       \text{ satisfying } \set{i, \sigma\tup{i}} = K }
= 0 \equiv 0 \mod 2$). Hence, we WLOG assume that there does exist
some $i \in S \setminus F$ satisfying $\set{i, \sigma\tup{i}} = K$.
Fix such an $i$, and denote it by $j$. Thus, $j$ is an element of
$S \setminus F$ and satisfies $\set{j, \sigma\tup{j}} = K$.

Set $k = \sigma\tup{j}$. Thus,
$\sigma\tup{k} = \sigma\tup{\sigma\tup{j}}
= \underbrace{\tup{\sigma \circ \sigma}}_{= \id_S} \tup{j}
= \id_S \tup{j} = j$
and therefore
$\set{\underbrace{k}_{=\sigma\tup{j}},
     \underbrace{\sigma\tup{k}}_{=j}}
= \set{\sigma\tup{j}, j} = \set{j, \sigma\tup{j}} = K$.
Furthermore, from $j \in S \setminus F$, we obtain $j \in S$ and
$j \notin F$. If we had $\sigma\tup{j} = j$, then we
would have $j \in F$ (by the definition of $F$), which would
contradict $j \notin F$. Hence, we do not have
$\sigma\tup{j} = j$. Thus, we have $\sigma\tup{j} \neq j$. Hence,
$j \neq \sigma\tup{j} = k$. In other words, $j$ and $k$ are distinct.

If $i \in S \setminus F$ satisfies $\set{i, \sigma\tup{i}} = K$, then
$i$ must be either $j$ or $k$ (because
$i \in \set{i, \sigma\tup{i}} = K
= \set{j, \underbrace{\sigma\tup{j}}_{=k}} = \set{j, k}$).

If we had $k \in F$, then we would have $\sigma\tup{k} = k$ (by the
definition of $F$), which would contradict $\sigma\tup{k} = j \neq k$.
Hence, we do not have $k \in F$. Thus, $k \notin F$. Combining this
with $k \in S$, we find $k \in S \setminus F$. Hence, $j$ and $k$ are
two elements of $S \setminus F$. Furthermore, $j$ and $k$ are two
$i \in S \setminus F$ satisfying $\set{i, \sigma\tup{i}} = K$ (since
$j$ is an element of $S \setminus F$ satisfying
$\set{j, \sigma\tup{j}} = K$, and since $k$ is an element of
$S \setminus F$ satisfying $\set{k, \sigma\tup{k}} = K$). Furthermore,
$j$ and $k$ are the \textbf{only} such $i$ (because we have shown that
if $i \in S \setminus F$ satisfies $\set{i, \sigma\tup{i}} = K$, then
$i$ must be either $j$ or $k$). Therefore, we conclude that there are
exactly two $i \in S \setminus F$ that satisfy
$\set{i, \sigma\tup{i}} = K$ (namely, $j$ and $k$) (because $j$ and
$k$ are distinct). Hence,
$\tup{ \text{the number of } i \in S \setminus F
        \text{ satisfying } \set{i, \sigma\tup{i}} = K }
= 2 \equiv 0 \mod 2$. This completes the proof of
\eqref{pf.lem.dominating.heinrich-lemma-inv.2}.}.

Now, \eqref{pf.lem.dominating.heinrich-lemma-inv.1} becomes
\begin{align*}
\abs{S \setminus F}
&= \sum_{K \in \powset[2]{S}}
      \underbrace{\tup{ \text{the number of } i \in S \setminus F
                   \text{ satisfying } \set{i, \sigma\tup{i}} = K }}
                 _{\substack{\equiv 0 \mod 2 \\
                              \text{(by \eqref{pf.lem.dominating.heinrich-lemma-inv.2})}}}
\\
& \equiv \sum_{K \in \powset[2]{S}} 0 = 0 \mod 2 .
\end{align*}
Since $\abs{S \setminus F} = \abs{S} - \abs{F}$ (because
$F \subseteq S$), this rewrites as
$\abs{S} - \abs{F} \equiv 0 \mod 2$. In other words,
$\abs{S} \equiv \abs{F} \mod 2$. This proves
Lemma~\ref{lem.dominating.heinrich-lemma-inv}.
\end{proof}

Here is a particular case of
Lemma~\ref{lem.dominating.heinrich-lemma-inv}:

\begin{corollary} \label{cor.dominating.heinrich-lemma-inv0}
Let $S$ be a finite set. Let $\sigma : S \to S$ be a map such that
$\sigma \circ \sigma = \id_S$. Assume that each $i \in S$
satisfies $\sigma\tup{i} \neq i$. Then,
$\abs{S}$ is even.
\end{corollary}

\begin{proof}[Proof of Corollary~\ref{cor.dominating.heinrich-lemma-inv0}.]
Let $F = \set{ i \in S \mid \sigma\tup{i} = i }$.
Lemma~\ref{lem.dominating.heinrich-lemma-inv} yields
$\abs{F} \equiv \abs{S} \mod 2$.

But each $i \in S$ satisfies $\sigma\tup{i} \neq i$. In other
words, no $i \in S$ satisfies $\sigma\tup{i} = i$. In other
words, $\set{ i \in S \mid \sigma\tup{i} = i } = \varnothing$.
Thus, $F = \set{ i \in S \mid \sigma\tup{i} = i }
= \varnothing$. Hence,
$\abs{F} = \abs{\varnothing} = 0$. But from
$\abs{F} \equiv \abs{S} \mod 2$, we obtain
$\abs{S} \equiv \abs{F} = 0 \mod 2$. In other words, $\abs{S}$
is even. Hence, Corollary~\ref{cor.dominating.heinrich-lemma-inv0}
is proven.
\end{proof}

Now, it is time to get rid of part \textbf{(a)} of
Theorem~\ref{thm.dominating.heinrich}:

\begin{lemma} \label{lem.dominating.heinrich-lemma.a}
Let $\alpha$ and $\beta$ be defined as in
Theorem~\ref{thm.dominating.heinrich}. Then, $\alpha$ and $\beta$ are
even.
\end{lemma}

\begin{proof}[Proof of Lemma~\ref{lem.dominating.heinrich-lemma.a}.]
Let us first show that $\beta$ is even.

Indeed, let $S$ be the set of all detached pairs $\tup{A, B}$ for
which both numbers $\abs{A}$ and $\abs{B}$ are odd.
Then, the definition of $\beta$ can be rewritten as
$\beta = \abs{S}$. Clearly, $S$ is a finite set.

For every $\tup{A, B} \in S$, we have
$\tup{B, A} \in S$\ \ \ \ \footnote{\textit{Proof.} Let
$\tup{A, B} \in S$. Thus, $\tup{A, B}$ is a detached pair for which
both numbers $\abs{A}$ and $\abs{B}$ are odd (by the definition of
$S$). Hence, $A$ and $B$ are two subsets of $V$ (since $\tup{A, B}$ is
a detached pair).
But Lemma~\ref{lem.dominating.heinrich-lemma3} shows that $\tup{A, B}$
is a detached pair if and only if $\tup{B, A}$ is a detached pair.
Hence, $\tup{B, A}$ is a detached pair (since $\tup{A, B}$ is a
detached pair). Furthermore, both numbers $\abs{B}$ and $\abs{A}$ are
odd. Thus, $\tup{B, A}$ is a detached pair for which both numbers
$\abs{B}$ and $\abs{A}$ are odd. In other words, $\tup{B, A} \in S$
(by the definition of $S$), qed.}. Hence, we can define a map
$\sigma : S \to S$ by setting
\[
\sigma\tup{A, B} = \tup{B, A}
\qquad \text{for all } \tup{A, B} \in S .
\]
Consider this $\sigma$. It is easy to see that
$\sigma \circ \sigma = \id_S$\ \ \ \ \footnote{\textit{Proof.}
For every $\tup{A, B} \in S$, we have
\begin{align*}
\tup{\sigma \circ \sigma} \tup{A, B}
&= \sigma \tup{\underbrace{\sigma \tup{A, B}}_{= \tup{B, A}}}
= \sigma \tup{B, A} = \tup{A, B}
\end{align*}
(by the definition of $\sigma$). In other words, the map
$\sigma \circ \sigma$ sends each $\tup{A, B} \in S$ to itself. Hence,
$\sigma \circ \sigma = \id_S$.}.

Each $i \in S$ satisfies
$\sigma\tup{i} \neq i$\ \ \ \ \footnote{\text{Proof.}
Let $i \in S$. We must prove that $\sigma\tup{i} \neq i$.

We have $i \in S$.
In other words, $i$ is a detached pair $\tup{A, B}$ for
which both numbers $\abs{A}$ and $\abs{B}$ are odd (by the
definition of $S$). Consider
this $\tup{A, B}$. The number $\abs{B}$ is odd and is a nonnegative
integer. Hence, this number $\abs{B}$ is positive (because any odd
nonnegative integer is positive). Thus, the set
$B$ is nonempty. In other words, $B \neq \varnothing$.

But $\tup{A, B}$ is a detached pair. In other
words, $A$ and $B$ are two disjoint subsets of $V$ having the
property that there exists no edge $ab \in E$ satisfying $a \in A$
and $b \in B$ (by the definition of a ``detached pair'').
Now, $A \cap B = \varnothing$ (since $A$ and $B$ are disjoint).
If we had $A = B$, then we would have
$\underbrace{A}_{=B} \cap B = B \cap B = B \neq \varnothing$,
which would contradict $A \cap B = \varnothing$. Hence, we cannot
have $A = B$. Thus, $A \neq B$.

If we had $\tup{A, B} = \tup{B, A}$, then we would have
$A = B$, which would contradict $A \neq B$. Hence, we cannot have
$\tup{A, B} = \tup{B, A}$. Therefore, $\tup{A, B} \neq
\tup{B, A}$.

But $i = \tup{A, B}$, and thus $\sigma\tup{i} = \sigma\tup{A, B}
= \tup{B, A}$ (by the definition of $\sigma$).
Now, $i = \tup{A, B} \neq \tup{B, A} = \sigma\tup{i}$. In other
words, $\sigma\tup{i} \neq i$, qed.}.
Hence, Corollary~\ref{cor.dominating.heinrich-lemma-inv0}
shows that $\abs{S}$ is even. In other words, $\beta$ is even
(since $\beta = \abs{S}$).

The same argument (with the obvious changes\footnote{For example,
``odd'' has to be replaced by ``even and positive''.}) shows that
$\alpha$ is even. This completes the proof of
Lemma~\ref{lem.dominating.heinrich-lemma.a}.
\end{proof}

We will spend the rest of this section manipulating sums in various
artful ways. These manipulations will culminate in a proof
of Theorem~\ref{thm.dominating.heinrich} \textbf{(b)}; they also
provide examples of techniques that are useful throughout
mathematics.

\begin{lemma} \label{lem.dominating.heinrich-lemma-2sum}
We have
\[
\sum_{\substack{\tup{A, B} \text{ is a detached pair}; \\
                A \neq \varnothing; \  B \neq \varnothing}}
  \tup{\tup{-1}^{\abs{A}} + \tup{-1}^{\abs{B}}}
= 2
\sum_{\substack{\tup{A, B} \text{ is a detached pair}; \\
                A \neq \varnothing; \  B \neq \varnothing}}
  \tup{-1}^{\abs{A}} .
\]
\end{lemma}

\begin{proof}[Proof of Lemma~\ref{lem.dominating.heinrich-lemma-2sum}.]
First of all,
\begin{align}
& \sum_{\substack{\tup{A, B} \text{ is a detached pair}; \\
                A \neq \varnothing; \  B \neq \varnothing}}
  \tup{\tup{-1}^{\abs{A}} + \tup{-1}^{\abs{B}}}
\nonumber \\
& =
\sum_{\substack{\tup{A, B} \text{ is a detached pair}; \\
                A \neq \varnothing; \  B \neq \varnothing}}
  \tup{-1}^{\abs{A}}
+
\sum_{\substack{\tup{A, B} \text{ is a detached pair}; \\
                A \neq \varnothing; \  B \neq \varnothing}}
  \tup{-1}^{\abs{B}}
.
\label{pf.lem.dominating.heinrich-lemma-2sum.1}
\end{align}
Let us now rewrite the second sum on the right hand side of
this equation:
\begin{align*}
& \sum_{\substack{\tup{A, B} \text{ is a detached pair}; \\
                A \neq \varnothing; \  B \neq \varnothing}}
  \tup{-1}^{\abs{B}} \\
& =
\sum_{\substack{\tup{B, A} \text{ is a detached pair}; \\
                B \neq \varnothing; \  A \neq \varnothing}}
  \tup{-1}^{\abs{A}} \\
& \qquad
\left( \text{here, we have renamed the summation index }
         \tup{A, B} \text{ as } \tup{B, A} \right) \\
& =
\sum_{\substack{\tup{B, A} \text{ is a detached pair}; \\
                A \neq \varnothing; \  B \neq \varnothing}}
  \tup{-1}^{\abs{A}}
=
\sum_{\substack{\tup{A, B} \text{ is a detached pair}; \\
                A \neq \varnothing; \  B \neq \varnothing}}
  \tup{-1}^{\abs{A}}
\end{align*}
(because the condition
``$\tup{B, A}$ is a detached pair'' under the summation
sign is equivalent to the condition ``$\tup{A, B}$ is a
detached pair'' (by Lemma~\ref{lem.dominating.heinrich-lemma3})).
Hence, \eqref{pf.lem.dominating.heinrich-lemma-2sum.1} becomes
\begin{align*}
&
\sum_{\substack{\tup{A, B} \text{ is a detached pair}; \\
                A \neq \varnothing; \  B \neq \varnothing}}
  \tup{\tup{-1}^{\abs{A}} + \tup{-1}^{\abs{B}}} \\
& =
\sum_{\substack{\tup{A, B} \text{ is a detached pair}; \\
                A \neq \varnothing; \  B \neq \varnothing}}
  \tup{-1}^{\abs{A}}
+
\underbrace{
\sum_{\substack{\tup{A, B} \text{ is a detached pair}; \\
                A \neq \varnothing; \  B \neq \varnothing}}
  \tup{-1}^{\abs{B}}
}_{
=
\sum_{\substack{\tup{A, B} \text{ is a detached pair}; \\
                A \neq \varnothing; \  B \neq \varnothing}}
  \tup{-1}^{\abs{A}}
} \\
& =
\sum_{\substack{\tup{A, B} \text{ is a detached pair}; \\
                A \neq \varnothing; \  B \neq \varnothing}}
  \tup{-1}^{\abs{A}}
+
\sum_{\substack{\tup{A, B} \text{ is a detached pair}; \\
                A \neq \varnothing; \  B \neq \varnothing}}
  \tup{-1}^{\abs{A}} \\
& = 2
\sum_{\substack{\tup{A, B} \text{ is a detached pair}; \\
                A \neq \varnothing; \  B \neq \varnothing}}
  \tup{-1}^{\abs{A}}
.
\end{align*}
This proves
Lemma~\ref{lem.dominating.heinrich-lemma-2sum}.
\end{proof}

\begin{lemma} \label{lem.dominating.heinrich-lemma-2sum2}
Let $n = \abs{V}$. Assume that $n > 0$. Let $\delta$
be the number of dominating sets of $G$. Then,
\[
\sum_{\substack{\tup{A, B} \text{ is a detached pair}; \\
                A \neq \varnothing; \  B \neq \varnothing}}
  \tup{-1}^{\abs{A}}
= \delta - \tup{2^n - 1}.
\]
\end{lemma}

\begin{proof}[Proof of
Lemma~\ref{lem.dominating.heinrich-lemma-2sum2}.]
We first observe that the set $V$ has $2^{\abs{V}}$ subsets. In other
words, the set $V$ has $2^n$ subsets (since $\abs{V} = n$). Thus, the
sum $\sum_{B \subseteq V} 1$ has $2^n$ addends. Since each of these
addends equals $1$, we conclude that this sum equals $2^n \cdot 1
= 2^n$. In other words,
\begin{equation}
\sum_{B \subseteq V} 1 = 2^n .
\label{pf.lem.dominating.heinrich-lemma-2sum2.1}
\end{equation}

On the other hand, each subset $B$ of $V$ is either dominating or not.
Hence,
\begin{align*}
& \sum_{B \subseteq V} \ive{B \text{ is dominating}}  \\
&=
\sum_{\substack{B \subseteq V; \\ B \text{ is dominating}}}
    \underbrace{\ive{B \text{ is dominating}}}_{\substack{
                 = 1 \\ \text{(since } B \text{ is dominating)}}}
+ \sum_{\substack{B \subseteq V; \\ B \text{ is not dominating}}}
    \underbrace{\ive{B \text{ is dominating}}}_{\substack{
                 = 0 \\ \text{(since } B \text{ is not dominating)}}}
 \\
&=
\sum_{\substack{B \subseteq V; \\ B \text{ is dominating}}}
    1
+ \underbrace{
  \sum_{\substack{B \subseteq V; \\ B \text{ is not dominating}}}
    0}_{=0}
=
\sum_{\substack{B \subseteq V; \\ B \text{ is dominating}}} 1 .
\end{align*}

But the number of dominating sets of $G$ is $\delta$. In other words,
the number of dominating subsets $B \subseteq V$ is $\delta$. Hence,
the sum
$\sum_{\substack{B \subseteq V; \\ B \text{ is dominating}}} 1$
has $\delta$ addends. Since each of these addends equals $1$, we thus
see that this sum equals $\delta \cdot 1 = \delta$. In other words,
\[
\sum_{\substack{B \subseteq V; \\ B \text{ is dominating}}} 1
= \delta .
\]
Hence,
\begin{equation}
\sum_{B \subseteq V} \ive{B \text{ is dominating}}
= \sum_{\substack{B \subseteq V; \\ B \text{ is dominating}}} 1
= \delta .
\label{pf.lem.dominating.heinrich-lemma-2sum2.2}
\end{equation}

The subset $\varnothing$ of $V$ is not
dominating\footnote{\textit{Proof.} Assume the contrary. Thus, the
subset $\varnothing$ of $V$ is dominating.

The set $V$ is nonempty (since $\abs{V} = n > 0$). Thus, there exists
some $q \in V$. Consider this $q$. But the set $\varnothing$ is
dominating. In other words, for every vertex
$v \in \verts{G} \setminus \varnothing$, at least one neighbor of
$v$ belongs to $\varnothing$. Applying this to $v = q$, we conclude
that at least one neighbor of $q$ belongs to $\varnothing$ (since
$q \in V = \verts{G} = \verts{G} \setminus \varnothing$). Hence, at
least some object belongs to $\varnothing$. In other words,
$\varnothing$ is nonempty. This is absurd. This contradiction shows
that our assumption is false, qed.}. Hence,
$\ive{\varnothing \text{ is dominating}} = 0$.

Each detached pair $\tup{A, B}$ consists of two subsets $A$ and $B$
of $V$ (by the definition of a ``detached pair''). Hence,
\begin{align}
& \sum_{\substack{\tup{A, B} \text{ is a detached pair}; \\
                A \neq \varnothing; \  B \neq \varnothing}}
  \tup{-1}^{\abs{A}} \nonumber \\
&=
\sum_{\substack{B \subseteq V; \\ B \neq \varnothing}}
\underbrace{
\sum_{\substack{A \subseteq V; \\ A \neq \varnothing; \\
                \tup{A, B} \text{ is a detached pair}}}
\tup{-1}^{\abs{A}}
}_{\substack{= \ive{B \text{ is dominating}} - 1 \\
        \text{(by Corollary~\ref{cor.dominating.heinrich-lemma2c})}}}
 \nonumber \\
&= \sum_{\substack{B \subseteq V; \\ B \neq \varnothing}}
\tup{\ive{B \text{ is dominating}} - 1} .
\label{pf.lem.dominating.heinrich-lemma-2sum2.3}
\end{align}

But the sum
$\sum_{B \subseteq V} \tup{\ive{B \text{ is dominating}} - 1}$ has an
addend for $B = \varnothing$. If we split off this addend, we find
\begin{align*}
& \sum_{B \subseteq V} \tup{\ive{B \text{ is dominating}} - 1} \\
& = \tup{\underbrace{\ive{\varnothing \text{ is dominating}}}_{=0} - 1}
+ \sum_{\substack{B \subseteq V; \\ B \neq \varnothing}}
\tup{\ive{B \text{ is dominating}} - 1} \\
&= -1
+ \sum_{\substack{B \subseteq V; \\ B \neq \varnothing}}
\tup{\ive{B \text{ is dominating}} - 1} .
\end{align*}
Thus,
\begin{align*}
\sum_{\substack{B \subseteq V; \\ B \neq \varnothing}}
\tup{\ive{B \text{ is dominating}} - 1}
&= \underbrace{
\sum_{B \subseteq V} \tup{\ive{B \text{ is dominating}} - 1}
}_{= \sum_{B \subseteq V} \ive{B \text{ is dominating}}
- \sum_{B \subseteq V} 1}
+ 1 \\
&= \underbrace{\sum_{B \subseteq V} \ive{B \text{ is dominating}}}_{
        \substack{= \delta \\
       \text{(by \eqref{pf.lem.dominating.heinrich-lemma-2sum2.2})}}}
- \underbrace{\sum_{B \subseteq V} 1}_{\substack{= 2^n \\
       \text{(by \eqref{pf.lem.dominating.heinrich-lemma-2sum2.1})}}}
+ 1
= \delta - 2^n + 1.
\end{align*}
Thus, \eqref{pf.lem.dominating.heinrich-lemma-2sum2.3} becomes
\begin{align*}
& \sum_{\substack{\tup{A, B} \text{ is a detached pair}; \\
                A \neq \varnothing; \  B \neq \varnothing}}
  \tup{-1}^{\abs{A}} \\
&= \sum_{\substack{B \subseteq V; \\ B \neq \varnothing}}
\tup{\ive{B \text{ is dominating}} - 1} = \delta - 2^n + 1
= \delta - \tup{2^n - 1} .
\end{align*}
This proves Lemma~\ref{lem.dominating.heinrich-lemma-2sum2}.
\end{proof}

Let us now finally step to the proof of
Theorem~\ref{thm.dominating.heinrich}:

\begin{proof}[Proof of Theorem~\ref{thm.dominating.heinrich}.]
\textbf{(a)} Theorem~\ref{thm.dominating.heinrich} \textbf{(a)} is
precisely Lemma~\ref{lem.dominating.heinrich-lemma.a}, which has
already been proven.

\textbf{(b)} Let $\delta$ be the number of dominating sets of $G$.

We shall compute the sum
$\sum_{\substack{\tup{A, B} \text{ is a detached pair}; \\
                A \neq \varnothing; \  B \neq \varnothing}}
  \tup{\tup{-1}^{\abs{A}} + \tup{-1}^{\abs{B}}}$
in two different ways, and then compare the results.

The first way relies on
Lemma~\ref{lem.dominating.heinrich-lemma-2sum} and on
Lemma~\ref{lem.dominating.heinrich-lemma-2sum2}:
From Lemma~\ref{lem.dominating.heinrich-lemma-2sum}, we obtain
\begin{align}
\sum_{\substack{\tup{A, B} \text{ is a detached pair}; \\
                A \neq \varnothing; \  B \neq \varnothing}}
  \tup{\tup{-1}^{\abs{A}} + \tup{-1}^{\abs{B}}}
&= 2
\underbrace{
\sum_{\substack{\tup{A, B} \text{ is a detached pair}; \\
                A \neq \varnothing; \  B \neq \varnothing}}
  \tup{-1}^{\abs{A}}
}_{\substack{= \delta - \tup{2^n - 1} \\
        \text{(by Lemma~\ref{lem.dominating.heinrich-lemma-2sum2})}}}
\nonumber \\
&= 2 \tup{\delta - \tup{2^n - 1}} .
\label{pf.thm.dominating.heinrich.1}
\end{align}

Now, let us try another way. We observe that each detached pair
$\tup{A, B}$ satisfies one and only one of the following four
conditions:
\begin{enumerate}
\item The number $\abs{A}$ is even, and the number $\abs{B}$ is even.
\item The number $\abs{A}$ is even, and the number $\abs{B}$ is odd.
\item The number $\abs{A}$ is odd, and the number $\abs{B}$ is even.
\item The number $\abs{A}$ is odd, and the number $\abs{B}$ is odd.
\end{enumerate}
Accordingly, the sum on the left hand side of
\eqref{pf.thm.dominating.heinrich.1} can be split into four smaller
sums:
\begin{align*}
& \sum_{\substack{\tup{A, B} \text{ is a detached pair}; \\
                A \neq \varnothing; \  B \neq \varnothing}}
  \tup{\tup{-1}^{\abs{A}} + \tup{-1}^{\abs{B}}}
  \\
&= \sum_{\substack{\tup{A, B} \text{ is a detached pair}; \\
                A \neq \varnothing; \  B \neq \varnothing ; \\
                \abs{A} \text{ is even};
                \ \abs{B} \text{ is even}}}
  \tup{\underbrace{\tup{-1}^{\abs{A}}}_{\substack{= 1 \\
               \text{(since } \abs{A} \text{ is even)}}}
       + \underbrace{\tup{-1}^{\abs{B}}}_{\substack{= 1 \\
               \text{(since } \abs{B} \text{ is even)}}}} \\
& \qquad +  \sum_{\substack{\tup{A, B} \text{ is a detached pair}; \\
                A \neq \varnothing; \  B \neq \varnothing ; \\
                \abs{A} \text{ is even};
                \ \abs{B} \text{ is odd}}}
  \tup{\underbrace{\tup{-1}^{\abs{A}}}_{\substack{= 1 \\
               \text{(since } \abs{A} \text{ is even)}}}
       + \underbrace{\tup{-1}^{\abs{B}}}_{\substack{= -1 \\
               \text{(since } \abs{B} \text{ is odd)}}}} \\
& \qquad +  \sum_{\substack{\tup{A, B} \text{ is a detached pair}; \\
                A \neq \varnothing; \  B \neq \varnothing ; \\
                \abs{A} \text{ is odd};
                \ \abs{B} \text{ is even}}}
  \tup{\underbrace{\tup{-1}^{\abs{A}}}_{\substack{= -1 \\
               \text{(since } \abs{A} \text{ is odd)}}}
       + \underbrace{\tup{-1}^{\abs{B}}}_{\substack{= 1 \\
               \text{(since } \abs{B} \text{ is even)}}}} \\
& \qquad +  \sum_{\substack{\tup{A, B} \text{ is a detached pair}; \\
                A \neq \varnothing; \  B \neq \varnothing ; \\
                \abs{A} \text{ is odd};
                \ \abs{B} \text{ is odd}}}
  \tup{\underbrace{\tup{-1}^{\abs{A}}}_{\substack{= -1 \\
               \text{(since } \abs{A} \text{ is odd)}}}
       + \underbrace{\tup{-1}^{\abs{B}}}_{\substack{= -1 \\
               \text{(since } \abs{B} \text{ is odd)}}}} \\
&= \sum_{\substack{\tup{A, B} \text{ is a detached pair}; \\
                A \neq \varnothing; \  B \neq \varnothing ; \\
                \abs{A} \text{ is even};
                \ \abs{B} \text{ is even}}}
   \underbrace{\tup{1 + 1}}_{= 2}
+ \sum_{\substack{\tup{A, B} \text{ is a detached pair}; \\
                A \neq \varnothing; \  B \neq \varnothing ; \\
                \abs{A} \text{ is even};
                \ \abs{B} \text{ is odd}}}
   \underbrace{\tup{1 + \tup{-1}}}_{= 0} \\
&\qquad + \sum_{\substack{\tup{A, B} \text{ is a detached pair}; \\
                A \neq \varnothing; \  B \neq \varnothing ; \\
                \abs{A} \text{ is odd};
                \ \abs{B} \text{ is even}}}
   \underbrace{\tup{\tup{-1} + 1}}_{= 0}
+ \sum_{\substack{\tup{A, B} \text{ is a detached pair}; \\
                A \neq \varnothing; \  B \neq \varnothing ; \\
                \abs{A} \text{ is odd};
                \ \abs{B} \text{ is odd}}}
   \underbrace{\tup{\tup{-1} + \tup{-1}}}_{= -2} \\
&= \sum_{\substack{\tup{A, B} \text{ is a detached pair}; \\
                A \neq \varnothing; \  B \neq \varnothing ; \\
                \abs{A} \text{ is even};
                \ \abs{B} \text{ is even}}}
   2
+ \underbrace{\sum_{\substack{\tup{A, B} \text{ is a detached pair}; \\
                A \neq \varnothing; \  B \neq \varnothing ; \\
                \abs{A} \text{ is even};
                \ \abs{B} \text{ is odd}}}
   0}_{= 0} \\
&\qquad + \underbrace{
   \sum_{\substack{\tup{A, B} \text{ is a detached pair}; \\
                A \neq \varnothing; \  B \neq \varnothing ; \\
                \abs{A} \text{ is odd};
                \ \abs{B} \text{ is even}}}
   0}_{= 0}
+ \sum_{\substack{\tup{A, B} \text{ is a detached pair}; \\
                A \neq \varnothing; \  B \neq \varnothing ; \\
                \abs{A} \text{ is odd};
                \ \abs{B} \text{ is odd}}}
   \tup{-2}
\end{align*}
\begin{align}
&= \sum_{\substack{\tup{A, B} \text{ is a detached pair}; \\
                A \neq \varnothing; \  B \neq \varnothing ; \\
                \abs{A} \text{ is even};
                \ \abs{B} \text{ is even}}}
   2
+ \sum_{\substack{\tup{A, B} \text{ is a detached pair}; \\
                A \neq \varnothing; \  B \neq \varnothing ; \\
                \abs{A} \text{ is odd};
                \ \abs{B} \text{ is odd}}}
   \tup{-2} .
\label{pf.thm.dominating.heinrich.3}
\end{align}

We shall now analyze the two sums on the right hand side of
\eqref{pf.thm.dominating.heinrich.3} more closely.
Let us begin with the first sum. For any subset $A$ of $V$, the
condition ``$A \neq \varnothing$'' is equivalent to the condition
``$\abs{A}$ is positive'' (because clearly, the finite set $A$ is
distinct from $\varnothing$ if and only if its size $\abs{A}$ is
positive). Likewise, for any subset $B$ of $V$, the condition
``$B \neq \varnothing$'' is equivalent to the condition
``$\abs{B}$ is positive''. Combining the preceding two sentences, we
conclude that the conditions ``$A \neq \varnothing$'' and
``$B \neq \varnothing$'' under the summation sign
$\sum_{\substack{\tup{A, B} \text{ is a detached pair}; \\
                A \neq \varnothing; \  B \neq \varnothing ; \\
                \abs{A} \text{ is even};
                \ \abs{B} \text{ is even}}}$
are equivalent to the conditions ``$\abs{A}$ is positive'' and
``$\abs{B}$ is positive'', respectively. Hence, we can rewrite this
summation sign as
$\sum_{\substack{\tup{A, B} \text{ is a detached pair}; \\
                \abs{A} \text{ is positive} ; 
                \ \abs{B} \text{ is positive} ; \\
                \abs{A} \text{ is even};
                \ \abs{B} \text{ is even}}}$.
Thus,
\begin{align}
\sum_{\substack{\tup{A, B} \text{ is a detached pair}; \\
                A \neq \varnothing; \  B \neq \varnothing ; \\
                \abs{A} \text{ is even};
                \ \abs{B} \text{ is even}}}
   2
&= \sum_{\substack{\tup{A, B} \text{ is a detached pair}; \\
                \abs{A} \text{ is positive} ; 
                \ \abs{B} \text{ is positive} ; \\
                \abs{A} \text{ is even};
                \ \abs{B} \text{ is even}}} 2
= \sum_{\substack{\tup{A, B} \text{ is a detached pair}; \\
                \abs{A} \text{ is even and positive} ; \\ 
                \abs{B} \text{ is even and positive}}} 2
\nonumber \\
&= \sum_{\substack{\tup{A, B} \text{ is a detached pair}; \\
                \text{both numbers } \abs{A} \text{ and } \abs{B}
                \text{ are even and positive}}} 2 .
\label{pf.thm.dominating.heinrich.4a1}
\end{align}
But the sum $\sum_{\substack{\tup{A, B} \text{ is a detached pair}; \\
                \text{both numbers } \abs{A} \text{ and } \abs{B}
                \text{ are even and positive}}} 2$
has exactly $\alpha$ terms (since the number of all detached pairs
$\tup{A, B}$ for which both numbers $\abs{A}$ and $\abs{B}$ are even
and positive is $\alpha$), and thus equals $\alpha \cdot 2$ (since
each of its $\alpha$ terms equals $2$). Thus,
$\sum_{\substack{\tup{A, B} \text{ is a detached pair}; \\
                \text{both numbers } \abs{A} \text{ and } \abs{B}
                \text{ are even and positive}}} 2
= \alpha \cdot 2 = 2 \alpha$. Hence,
\eqref{pf.thm.dominating.heinrich.4a1} becomes
\begin{align}
\sum_{\substack{\tup{A, B} \text{ is a detached pair}; \\
                A \neq \varnothing; \  B \neq \varnothing ; \\
                \abs{A} \text{ is even};
                \ \abs{B} \text{ is even}}}
   2
&= \sum_{\substack{\tup{A, B} \text{ is a detached pair}; \\
                \text{both numbers } \abs{A} \text{ and } \abs{B}
                \text{ are even and positive}}} 2
= 2 \alpha .
\label{pf.thm.dominating.heinrich.4a12}
\end{align}

Let us now attack the second sum. We have
\begin{align}
\sum_{\substack{\tup{A, B} \text{ is a detached pair}; \\
                A \neq \varnothing; \  B \neq \varnothing ; \\
                \abs{A} \text{ is odd};
                \ \abs{B} \text{ is odd}}}
   \tup{-2}
&= \sum_{\substack{\tup{A, B} \text{ is a detached pair}; \\
                \text{both numbers } \abs{A} \text{ and } \abs{B}
                \text{ are odd and positive}}} \tup{-2} .
\label{pf.thm.dominating.heinrich.4b1}
\end{align}
(Indeed, this can be proven in the same way as we proved
\eqref{pf.thm.dominating.heinrich.4a1}, except that we must replace
every word ``even'' by ``odd'', and every appearance of $2$ by $-2$.)

But if $\tup{A, B}$ is a detached pair, then the condition
``both numbers $\abs{A}$ and $\abs{B}$ are odd and positive'' is
equivalent to the simpler condition
``both numbers $\abs{A}$ and $\abs{B}$ are
odd''\footnote{\textit{Proof.} Let $\tup{A, B}$ be a detached pair.
Thus, $A$ and $B$ are two subsets of $V$ (by the definition of
``detached pair''), and thus are two finite sets. Hence, $\abs{A}$ and
$\abs{B}$ are nonnegative integers. Therefore, if $\abs{A}$ and
$\abs{B}$ are odd, then $\abs{A}$ and $\abs{B}$ are automatically
positive (because if a nonnegative integer is odd, then it is
automatically positive). Therefore, the condition
``both numbers $\abs{A}$ and $\abs{B}$ are odd and positive'' is
equivalent to the simpler condition
``both numbers $\abs{A}$ and $\abs{B}$ are odd''. Qed.}. Hence, the
summation sign
$\sum_{\substack{\tup{A, B} \text{ is a detached pair}; \\
                \text{both numbers } \abs{A} \text{ and } \abs{B}
                \text{ are odd and positive}}} \tup{-2}$
can be rewritten as
$\sum_{\substack{\tup{A, B} \text{ is a detached pair}; \\
                \text{both numbers } \abs{A} \text{ and } \abs{B}
                \text{ are odd}}} \tup{-2}$.
Thus,
\[
\sum_{\substack{\tup{A, B} \text{ is a detached pair}; \\
                \text{both numbers } \abs{A} \text{ and } \abs{B}
                \text{ are odd and positive}}} \tup{-2}
= \sum_{\substack{\tup{A, B} \text{ is a detached pair}; \\
                \text{both numbers } \abs{A} \text{ and } \abs{B}
                \text{ are odd}}} \tup{-2} .
\]
But the sum $\sum_{\substack{\tup{A, B} \text{ is a detached pair}; \\
                \text{both numbers } \abs{A} \text{ and } \abs{B}
                \text{ are odd}}} \tup{-2}$
has exactly $\beta$ terms (since the number of all detached pairs
$\tup{A, B}$ for which both numbers $\abs{A}$ and $\abs{B}$ are odd
is $\beta$), and thus equals $\beta \cdot \tup{-2}$ (since
each of its $\beta$ terms equals $-2$). Thus, \newline
$\sum_{\substack{\tup{A, B} \text{ is a detached pair}; \\
                \text{both numbers } \abs{A} \text{ and } \abs{B}
                \text{ are odd}}} \tup{-2}
= \beta \cdot \tup{-2} = -2 \beta$. Hence,
\eqref{pf.thm.dominating.heinrich.4b1} becomes
\begin{align}
\sum_{\substack{\tup{A, B} \text{ is a detached pair}; \\
                A \neq \varnothing; \  B \neq \varnothing ; \\
                \abs{A} \text{ is odd};
                \ \abs{B} \text{ is odd}}}
   \tup{-2}
&= \sum_{\substack{\tup{A, B} \text{ is a detached pair}; \\
                \text{both numbers } \abs{A} \text{ and } \abs{B}
                \text{ are odd and positive}}} \tup{-2} \nonumber \\
&= \sum_{\substack{\tup{A, B} \text{ is a detached pair}; \\
                \text{both numbers } \abs{A} \text{ and } \abs{B}
                \text{ are odd}}} \tup{-2}
= -2 \beta .
\label{pf.thm.dominating.heinrich.4b12}
\end{align}

Now, \eqref{pf.thm.dominating.heinrich.3} becomes

\begin{align*}
& \sum_{\substack{\tup{A, B} \text{ is a detached pair}; \\
                A \neq \varnothing; \  B \neq \varnothing}}
  \tup{\tup{-1}^{\abs{A}} + \tup{-1}^{\abs{B}}}
  \\
&= \underbrace{\sum_{\substack{\tup{A, B} \text{ is a detached pair}; \\
                A \neq \varnothing; \  B \neq \varnothing ; \\
                \abs{A} \text{ is even};
                \ \abs{B} \text{ is even}}}
   2}_{\substack{ = 2 \alpha \\
        \text{(by \eqref{pf.thm.dominating.heinrich.4a12})}}}
+ \underbrace{\sum_{\substack{\tup{A, B} \text{ is a detached pair}; \\
                A \neq \varnothing; \  B \neq \varnothing ; \\
                \abs{A} \text{ is odd};
                \ \abs{B} \text{ is odd}}}
   \tup{-2}}_{\substack{ = - 2 \beta \\
        \text{(by \eqref{pf.thm.dominating.heinrich.4b12})}}} \\
&= 2 \alpha + \tup{-2 \beta} = 2 \tup{\alpha - \beta} .
\end{align*}

Comparing this with \eqref{pf.thm.dominating.heinrich.1}, we find
\[
2 \tup{\delta - \tup{2^n - 1}} = 2 \tup{\alpha - \beta} .
\]
Cancelling $2$ from this equality, we obtain
$\delta - \tup{2^n - 1} = \alpha - \beta$. Solving this equation for
$\delta$, we find $\delta = 2^n - 1 + \alpha - \beta$.

Now, recall that $\delta$ is the number of dominating sets of $G$.
Hence, the number of dominating sets of $G$ is
$\delta = 2^n - 1 + \alpha - \beta$. This proves
Theorem~\ref{thm.dominating.heinrich} \textbf{(b)}.
\end{proof}

\section{\label{sect.hamilton}Hamiltonian paths}

%XXX

[...]

[to be continued]

\begin{thebibliography}{9999999999}                                                                                       %

\bibitem[AigZie]{AigZie}Martin Aigner, G\"{u}nter M. Ziegler,
\textit{Proofs from the Book}, 4th edition, Springer 2010.

\bibitem[AoPS-ISL]{AoPS-ISL}Art of Problem Solving (forum),
\textit{IMO Shortlist} (collection of threads),
\newline
\url{http://www.artofproblemsolving.com/community/c3223_imo_shortlist}

% \bibitem[Artin10]{Artin10}Michael Artin, \textit{Algebra}, 2nd edition,
% Pearson 2010.

\bibitem[Bahran15]{Bahran15}Cihan Bahran,
\textit{Solutions to Math 5707 Spring 2015 homework}.
\newline \url{http://www-users.math.umn.edu/~bahra004/5707.html}

% \bibitem[BarSch73]{BarSch73}Hans Schneider, George Phillip Barker,
% \textit{Matrices and Linear Algebra}, 2nd edition, Dover 1973.

\bibitem[BehCha71]{BehCha71}Mehdi Behzad, Gary Chartrand,
\textit{Introduction to the Theory of Graphs},
Allyn \& Bacon, 1971.

\bibitem[BenWil12]{BenWil12}Edward A. Bender and S. Gill Williamson,
\textit{Foundations of Combinatorics with Applications}.
\newline \url{http://cseweb.ucsd.edu/~gill/FoundCombSite/}

\bibitem[BeChZh15]{BeChZh15}Arthur Benjamin, Gary Chartrand,
Ping Zhang,
\textit{The fascinating world of graph theory},
Princeton University Press 2015.

\bibitem[Berge91]{Berge91}Claude Berge,
\textit{Graphs}, 3rd edition, North-Holland 1991.

\bibitem[Bogomoln]{cut-the-knot}Alexander Bogomolny,
\textit{Cut the Knot} (website devoted to educational applets on
various mathematical subjects),
\url{http://www.cut-the-knot.org/Curriculum/index.shtml#combinatorics} .

\bibitem[Bollob79]{Bollob79}B\'ela Bollob\'as,
\textit{Graph Theory: An Introductory Course},
Graduate Texts in Mathematics \#63, 1st edition, Springer 1971.

\bibitem[Bollob98]{Bollob98}B\'ela Bollob\'as,
\textit{Modern Graph Theory},
Graduate Texts in Mathematics \#184, 1st edition, Springer 1998.

\bibitem[Bona11]{Bona11}
Mikl\'os B\'ona,
\textit{A Walk Through Combinatorics:
An Introduction to Enumeration and Graph Theory},
3rd edition, World Scientific 2011.

\bibitem[BonMur76]{BonMur76}
J. A. Bondy and U. S. R. Murty, \textit{Graph theory with Applications},
North-Holland 1976.
\newline \url{https://www.iro.umontreal.ca/~hahn/IFT3545/GTWA.pdf}.

\bibitem[BonMur08]{BonMur08}
J. A. Bondy and U. S. R. Murty, \textit{Graph theory}, Graduate Texts
in Mathematics \#244, Springer 2008.
\newline \url{https://www.classes.cs.uchicago.edu/archive/2016/spring/27500-1/hw3.pdf}

\bibitem[BonTho77]{BonTho77}
J. A. Bondy, C. Thomassen,
\textit{A short proof of Meyniel's theorem},
Discrete Mathematics 19 (1977), pp. 195--197.
\newline \url{http://www.sciencedirect.com/science/article/pii/0012365X77900346}

\bibitem[Brouwe09]{Brouwe09}
Andries E. Brouwer,
\textit{The number of dominating sets of a finite graph is odd},
\url{http://www.win.tue.nl/~aeb/preprints/domin2.pdf} .

% \bibitem[Camero08]{Camero08}Peter J. Cameron, \textit{Notes on Linear
% Algebra}, version 5 Sep 2008.\newline\url{http://www.maths.qmul.ac.uk/~pjc/notes/linalg.pdf}

\bibitem[ChaLes15]{ChaLes15}
Gary Chartrand, Linda Lesniak, Ping Zhang,
\textit{Graphs \& Digraphs}, 6th edition, CRC Press 2016.

\bibitem[Choo16]{Choo16}David Choo,
\textit{4 proofs to Mantel's Theorem},
\url{http://davinchoo.com/2016/06/13/4-proofs-to-mantels-theorem/} .

\bibitem[Conrad]{Conrad-sign}Keith Conrad, \textit{Sign of permutations},
\newline\url{http://www.math.uconn.edu/~kconrad/blurbs/grouptheory/sign.pdf} .

\bibitem[Day16]{Day-proofs}Martin V. Day,
\textit{An Introduction to Proofs and the Mathematical Vernacular},
\newline\url{https://www.math.vt.edu/people/day/ProofsBook/IPaMV.pdf} .

% \bibitem[deBoor]{deBoor}Carl de Boor, \textit{An empty exercise}.
% \url{ftp://ftp.cs.wisc.edu/Approx/empty.pdf} .

\bibitem[Dieste16]{Dieste16}Reinhard Diestel, \textit{Graph Theory},
Graduate Texts in Mathematics \#173, 5th edition, Springer 2016.
\newline \url{http://diestel-graph-theory.com/basic.html} .

\bibitem[Gessel79]{Gessel-Vand}Ira Gessel, \textit{Tournaments and
Vandermonde's Determinant}, Journal of Graph Theory, Vol. 3 (1979), pp. 305--307.

\bibitem[GrRoSp90]{GrRoSp90}Ronald L. Graham, Bruce L. Rothschild,
Joel H. Spencer, \textit{Ramsey Theory}, 2nd edition,
Wiley 1990.

\bibitem[Griffi15]{Griffi15}Christopher Griffin,
\textit{Graph Theory: Penn State Math 485 Lecture Notes},
version 1.4.2.1 (18 Oct 2015),
\newline\url{https://sites.google.com/site/cgriffin229/} .

\bibitem[Grinbe16]{detnotes}Darij Grinberg, \textit{Notes on the combinatorial
fundamentals of algebra}, 29 December 2016.
\newline\url{http://www.cip.ifi.lmu.de/~grinberg/primes2015/sols.pdf}

\bibitem[HarPal73]{HarPal73}
Frank Harary, Edgar M. Palmer,
\textit{Graphical enumeration},
Academic Press 1973.

\bibitem[HaHiMo08]{HaHiMo08}
John M. Harris, Jeffry L. Hirst, Michael J. Mossinghoff,
\textit{Combinatorics and Graph Theory}, Undergraduate Texts in
Mathematics, Springer 2008.

% \bibitem[Heffer16]{Heffer16}Jim Hefferon, \textit{Linear Algebra},
% 2016.\newline\url{http://joshua.smcvt.edu/linearalgebra/}

\bibitem[HeiTit17]{HeiTit17}
Irene Heinrich, Peter Tittmann,
\textit{Counting Dominating Sets of Graphs},
\arxiv{1701.03453v1}.

\bibitem[Jukna11]{Jukna11}Stasys Jukna,
\textit{Extremal Combinatorics}, 2nd edition, Springer 2011.
An early draft is available at
\url{http://www.thi.informatik.uni-frankfurt.de/~jukna/EC_Book_2nd/} .

% \bibitem[OlvSha06]{OlvSha06}Peter J. Olver, Chehrzad Shakiban, \textit{Applied
% Linear Algebra}, Prentice Hall, 2006.\newline See also
% \url{http://www.math.umn.edu/~olver/ala.html} for corrections.

\bibitem[KelTro15]{KelTro15}Mitchel T. Keller, William T. Trotter,
\textit{Applied Combinatorics},
version 26 May 2015.
\newline \url{https://people.math.gatech.edu/~trotter/book.pdf}

\bibitem[Klarre17]{Klarre17}Erica Klarreich,
\textit{Complexity Theory Problem Strikes Back},
Quanta Magazine, 5 January 2017.
\newline \url{https://www.quantamagazine.org/20170105-graph-isomorphism-retraction/}

\bibitem[Knuth95]{Knuth95} Donald E. Knuth,
\textit{Overlapping Pfaffians},
Electron. J. Combin. 3 (1996), no. 2, \#R5.
Also available as arXiv preprint \arxiv{math/9503234v1}.

% \bibitem[Kowals16]{Kowals16}Emmanuel Kowalski, \textit{Linear Algebra},
% version 15 Sep 2016.\newline\url{https://people.math.ethz.ch/~kowalski/script-la.pdf}

\bibitem[LaNaSc16]{LaNaSc16}Isaiah Lankham, Bruno Nachtergaele, Anne
Schilling, \textit{Linear Algebra As an Introduction to Abstract Mathematics},
2016.\newline\url{https://www.math.ucdavis.edu/~anne/linear_algebra/mat67_course_notes.pdf}

\bibitem[LeLeMe16]{LeLeMe16}Eric Lehman, F. Thomson Leighton, Albert R. Meyer,
\textit{Mathematics for Computer Science}, revised Thursday 28th September,
2016, \newline\url{https://courses.csail.mit.edu/6.042/spring16/mcs.pdf} .

\bibitem[LoPeVe03]{LoPeVe03}L\'aszl\'o Lov\'asz, J\'ozsef Pelik\'an,
Katalin Vesztergombi,
\textit{Discrete Mathematics: Elementary and Beyound},
Springer 2003.

% \bibitem[m.se709196]{m.se709196}Daniela Diaz and others, \textit{Definition of
% General Associativity for binary operations}, math.stackexchange question
% \#709196 .\newline\url{http://math.stackexchange.com/q/709196}

\bibitem[Ore62]{Ore62}
Oystein Ore, \textit{Theory of graphs},
AMS Colloquium Publications \#XXXVIII,
4th printing, AMS 1962.

\bibitem[Ore90]{Ore90}
Oystein Ore, \textit{Graphs and their uses}, revised and updated by
Robin J. Wilson,
Anneli Lax New Mathematical Library \#34, MAA 1990.

\bibitem[Overbe74]{Overbe74}Maria Overbeck-Larisch,
\textit{Hamiltonian paths in oriented graphs},
Journal of Combinatorial Theory, Series B,
Volume 21, Issue 1, August 1976, pp. 76--80.
\newline \url{http://www.sciencedirect.com/science/article/pii/0095895676900307}

\bibitem[Petrov15]{Petrov15}Fedor Petrov,
\textit{mathoverflow post \#198679 (Flooding a cycle digraph via
chip-firing: $n^{k-1} + n^{k-2} + \cdots + 1$ bound (a Norway 1998-99
problem generalized))},
MathOverflow,
\newline \url{http://mathoverflow.net/q/198679}

\bibitem[PoTaWo83]{PoTaWo83}George P\'olya, Robert E. Tarjan,
Donald R. Woods,
\textit{Notes on Introductory Combinatorics},
Birkh\"auser 1983.
\newline See
\url{http://i.stanford.edu/pub/cstr/reports/cs/tr/79/732/CS-TR-79-732.pdf}
for a preliminary version.

\bibitem[Pretzel]{Pretzel}Oliver Pretzel,
\textit{On reorienting graphs by pushing down maximal vertices},
Order, 1986, Volume 3, Issue 2, pp. 135--153.

\bibitem[RaWiRa]{RaWi-Ramsey}RationalWiki,
\textit{Ramsey theory},
\url{http://rationalwiki.org/wiki/Ramsey_theory} .

\bibitem[RotSot92]{RotSot92}Alvin E. Roth, Marilda A. Oliveira
Sotomayor, \textit{Two-sided matching: A study in game-theoretic
modeling and analysis}, Cambridge University Press 1992.

\bibitem[Ruohon13]{Ruohon13}Keijo Ruohonen, \textit{Graph theory},
2013.\newline \url{http://math.tut.fi/~ruohonen/GT_English.pdf} .

\bibitem[Stanle12]{Stanley-EC1}Richard P. Stanley, \textit{Enumerative
Combinatorics, Volume 1}, 2nd edition, CUP 2012.\newline See
\url{http://math.mit.edu/~rstan/ec/ec1/} for a preliminary version.

\bibitem[West01]{West01}Douglas B. West,
\textit{Introduction to Graph Theory}, 2nd edition, Pearson 2001.
\newline See \url{http://www.math.illinois.edu/~dwest/igt/}
for updates and corrections.

\bibitem[Wilson96]{Wilson96}Robin J. Wilson,
\textit{Introduction to Graph Theory}, 4th edition,
Addison Wesley 1996.

\end{thebibliography}


\end{document}