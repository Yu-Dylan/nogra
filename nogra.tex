\documentclass[numbers=enddot,12pt,final,onecolumn,notitlepage]{scrartcl}%
\usepackage[headsepline,footsepline,manualmark]{scrlayer-scrpage}
\usepackage[all,cmtip]{xy}
\usepackage{amssymb}
\usepackage{amsmath}
\usepackage{amsthm}
\usepackage{framed}
\usepackage{comment}
\usepackage{color}
\usepackage{hyperref}
\usepackage{datetime}
\usepackage[sc]{mathpazo}
\usepackage[T1]{fontenc}
\usepackage{needspace}
\usepackage{tabls}
%TCIDATA{OutputFilter=latex2.dll}
%TCIDATA{Version=5.50.0.2960}
%TCIDATA{LastRevised=Tuesday, December 13, 2016 21:43:59}
%TCIDATA{SuppressPackageManagement}
%TCIDATA{<META NAME="GraphicsSave" CONTENT="32">}
%TCIDATA{<META NAME="SaveForMode" CONTENT="1">}
%TCIDATA{BibliographyScheme=Manual}
%TCIDATA{Language=American English}
%BeginMSIPreambleData
\providecommand{\U}[1]{\protect\rule{.1in}{.1in}}
%EndMSIPreambleData
\theoremstyle{definition}
\newtheorem{theo}{Theorem}[subsection]
\newenvironment{theorem}[1][]
{\begin{theo}[#1]\begin{leftbar}}
{\end{leftbar}\end{theo}}
\providecommand*\theoremautorefname{Theorem}
\newtheorem{lem}[theo]{Lemma}
\newenvironment{lemma}[1][]
{\begin{lem}[#1]\begin{leftbar}}
{\end{leftbar}\end{lem}}
\providecommand*\lemmaautorefname{Lemma}
\newtheorem{prop}[theo]{Proposition}
\newenvironment{proposition}[1][]
{\begin{prop}[#1]\begin{leftbar}}
{\end{leftbar}\end{prop}}
\providecommand*\propositionautorefname{Proposition}
\newtheorem{defi}[theo]{Definition}
\newenvironment{definition}[1][]
{\begin{defi}[#1]\begin{leftbar}}
{\end{leftbar}\end{defi}}
\providecommand*\definitionautorefname{Definition}
\newtheorem{remk}[theo]{Remark}
\newenvironment{remark}[1][]
{\begin{remk}[#1]\begin{leftbar}}
{\end{leftbar}\end{remk}}
\providecommand*\remarkautorefname{Remark}
\newtheorem{coro}[theo]{Corollary}
\newenvironment{corollary}[1][]
{\begin{coro}[#1]\begin{leftbar}}
{\end{leftbar}\end{coro}}
\providecommand*\corollaryautorefname{Corollary}
\newtheorem{conv}[theo]{Convention}
\newenvironment{condition}[1][]
{\begin{conv}[#1]\begin{leftbar}}
{\end{leftbar}\end{conv}}
\providecommand*\conditionautorefname{Convention}
\newtheorem{warn}[theo]{Warning}
\newenvironment{conclusion}[1][]
{\begin{warn}[#1]\begin{leftbar}}
{\end{leftbar}\end{warn}}
\providecommand*\conclusionautorefname{Warning}
\newtheorem{conj}[theo]{Conjecture}
\newenvironment{conjecture}[1][]
{\begin{conj}[#1]\begin{leftbar}}
{\end{leftbar}\end{conj}}
\providecommand*\conjectureautorefname{Conjecture}
\newtheorem{exam}[theo]{Example}
\newenvironment{example}[1][]
{\begin{exam}[#1]\begin{leftbar}}
{\end{leftbar}\end{exam}}
\providecommand*\exampleautorefname{Example}
\newtheorem{exmp}[theo]{Exercise}
\newenvironment{exercise}[1][]
{\begin{exmp}[#1]\begin{leftbar}}
{\end{leftbar}\end{exmp}}
\providecommand*\exerciseautorefname{Exercise}
\newenvironment{statement}{\begin{quote}}{\end{quote}}
\newtheorem{quest}[theo]{TODO}
\newenvironment{todo}[1][]
{\begin{quest}[#1]\begin{leftbar}}
{\end{leftbar}\end{quest}}
\iffalse
\newenvironment{proof}[1][Proof]{\noindent\textbf{#1.} }{\ \rule{0.5em}{0.5em}}
\fi
\let\sumnonlimits\sum
\let\prodnonlimits\prod
\let\cupnonlimits\bigcup
\let\capnonlimits\bigcap
\renewcommand{\sum}{\sumnonlimits\limits}
\renewcommand{\prod}{\prodnonlimits\limits}
\renewcommand{\bigcup}{\cupnonlimits\limits}
\renewcommand{\bigcap}{\capnonlimits\limits}
\newcommand\arxiv[1]{\href{http://www.arxiv.org/abs/#1}{\texttt{arXiv:#1}}}
% A command for citing arXiv preprints.
% Example syntax: \arxiv{1009.4134v2} and \arxiv{math/0602634v4}.

\setlength\tablinesep{3pt}
\setlength\arraylinesep{3pt}
\setlength\extrarulesep{3pt}
\voffset=0cm
\hoffset=-0.7cm
\setlength\textheight{22.5cm}
\setlength\textwidth{15.5cm}
\newenvironment{verlong}{}{}
\newenvironment{vershort}{}{}
\newenvironment{noncompile}{}{}
\excludecomment{verlong}
\includecomment{vershort}
\excludecomment{noncompile}
\newcommand{\id}{\operatorname{id}}
\newcommand{\NN}{\mathbb{N}}
\newcommand{\ZZ}{\mathbb{Z}}
\newcommand{\QQ}{\mathbb{Q}}
\newcommand{\RR}{\mathbb{R}}
\newcommand{\powset}[2][]{\ifthenelse{\equal{#2}{}}{\mathcal{P}\left(#1\right)}{\mathcal{P}_{#1}\left(#2\right)}}
% $\powset[k]{S}$ stands for the set of all $k$-element subsets of
% $S$. The argument $k$ is optional, and if not provided, the result
% is the whole powerset of $S$.
\newcommand{\set}[1]{\left\{ #1 \right\}}
% $\set{...}$ yields $\left\{ ... \right\}$.
\newcommand{\abs}[1]{\left| #1 \right|}
% $\abs{...}$ yields $\left| ... \right|$.
\newcommand{\tup}[1]{\left( #1 \right)}
% $\tup{...}$ yields $\left( ... \right)$.
\newcommand{\ive}[1]{\left[ #1 \right]}
% $\ive{...}$ yields $\left[ ... \right]$.
\newcommand{\verts}[1]{\operatorname{V}\left( #1 \right)}
% $\verts{...}$ yields $\operatorname{V}\left( ... \right)$.
\newcommand{\edges}[1]{\operatorname{E}\left( #1 \right)}
% $\edges{...}$ yields $\operatorname{E}\left( ... \right)$.
\newcommand{\arcs}[1]{\operatorname{A}\left( #1 \right)}
% $\arcs{...}$ yields $\operatorname{A}\left( ... \right)$.
\newcommand{\E}{\operatorname{E}}
\newcommand{\A}{\operatorname{A}}
\ihead{Notes on graph theory (\today, \currenttime)}
\ohead{page \thepage}
\cfoot{}
\begin{document}

\title{Notes on graph theory}
\author{Darij Grinberg}
\date{
%TCIMACRO{\TeXButton{today}{\today} }%
%BeginExpansion
\today\ 
%EndExpansion
at \currenttime\ 
(unfinished draft!)}
\maketitle
\tableofcontents

\section{Preface}

These are lecture notes on graph theory -- the part of mathematics
involved with graphs. They are currently work in
progress; once finished, they should contain a semester's worth of
material. I have tried to keep the presentation as self-contained and
elementary as possible; the reader is nevertheless assumed to possess
some mathematical maturity (in particular, to know how to read
combinatorial proofs, filling in simple details)\footnote{I believe
that the MIT lecture notes \cite{LeLeMe16} are a good source for
achieving this mathematical maturity. (Actually, there is some
intersection between \cite[Chapter 12]{LeLeMe16} and our notes, but
\cite{LeLeMe16} mostly keeps to the basics of graph theory.)
Another resource to familiarize oneself with proofs is
\cite{Day-proofs}. Generally, most good books about ``reading and
writing mathematicals'' or ``introductions to abstract mathematics''
should convey these skills, although the extent to which they actually
do so may differ.}
and know how to work
with modular arithmetic (i.e., congruences between integers modulo a
positive integer $n$) and summation signs (such as $\sum_{i=1}^n$ and
$\sum_{a \in A}$, where $n$ is an integer and $A$ is a finite
set)\footnote{See \cite[\S 1.3]{detnotes} for a list of important
properties of summation signs.}.
In some chapters, familiarity with matrices (and their products),
permutations (and their signs)\footnote{A summary of the most
fundamental results about signs of permutations can be found in
\cite[\S 8.1]{LaNaSc16}. These results appear with proofs in
\cite[Chapter 6.B]{Day-proofs}. For an even more detailed treatment
(also including proofs),
see \cite[\S 4.1--4.3]{detnotes}. Another treatment can be found in
\cite{Conrad-sign}, but this requires some familiarity with group
theory.}
and polynomials will be required.
I hope that the proofs are readable; feel free to contact me (at
\texttt{dgrinber@umn.edu}) for any clarifications.

The choice of material surveyed in these notes is idiosyncratic
(sometimes even purposefully trying to wander some seldom trodden
paths). Some standard material (Eulerian walks, Hamiltonian paths,
trees, bipartite matching theory, network flows) is present (at least
in the eventual final form of these notes), whereas
other popular topics (planar graphs, random graphs, adjacency matrices
and spectral graph theory) are missing. Some of these omissions have
specific reasons (e.g., many of the omitted topics would make it much
harder to keep the notes self-contained), whereas others are merely
due to my tastes and lack of time. I am trying to give an elementary
introduction into the (rather new) theory of sandpiles (also known as
chip-firing) as well as two applications of combinatorics to linear
algebra (viz., Gessel's proof of the Vandermonde determinant
\cite{Gessel-Vand}, and some of the applications of matchings to
Pfaffians); I also intend to include some properties of Hamiltonian
paths not commonly exposed in textbooks.

These notes are accompanying
\href{http://www.math.umn.edu/~dgrinber/5707s17/}{a
class on graph theory (Math 5707) I am giving at the University of
Minneapolis in Spring 2017}. They contain both the
material of the class (although with no promise of timeliness!) and the
homework exercises (and possibly some additional exercises).
Sections marked with an asterisk (*) are not part of the Math 5707
course.

Various other texts on graph theory are \cite{Bollob79},
\cite{BonMur76}, \cite{Ore62}, \cite{BehCha71}, \cite{BeChZh15},
\cite{BonMur08}, \cite{Ruohon13}, \cite{Dieste16}, \cite{Ore90},
\cite{HaHiMo08}, \cite{Berge91}, \cite{ChaLes15}, \cite{Griffi15}.
(This is a haphazard list; I have barely touched most of these texts.)
Also, texts on combinatorics and on discrete mathematics (such as
\cite{BenWil12}, \cite{KelTro15}, \cite{PoTaWo83}) often contain
sections on graph theory, since graph theory is considered to be part
of both.
Material on graph theory can also be found in large quantities on
mathematical contests for students (such as the International
Mathematical Olympiad) and, consequently, in collections of problems
from these contests, such as the AoPS collection of IMO Shortlist
problems \cite{AoPS-ISL}.
Finally, some elementary results in graph theory double as puzzles
(or are related to puzzles), which often has the consequence that they
appear on puzzle websites such as Cut-the-Knot \cite{cut-the-knot}.

The notes you are reading are under construction, and will remain so for at
least the whole Spring of 2017. Please let me know of any errors and
unclarities you encounter (my email address is \texttt{dgrinber@umn.edu}%
)\footnote{The sourcecode of the notes is also publicly available at
\url{https://github.com/darijgr/nogra} .}. Thank you!

\subsection{Acknowledgments}

Thanks to Victor Reiner for helpful conversations.

[Your name could be in here!]

\section{\label{chp.intro}Introduction}

In this chapter, we shall define a first notion of graphs (``first''
because there are several others to follow) and various basic notions
related to it, and prove some elementary properties thereof. This
chapter is meant to give a taste of the whole theory, although (not
unexpectedly) it is not a representative sample.

\subsection{\label{sect.intro.notations}Notations and conventions}

Before we get to anything interesting, let me get some technicalities
out of the way. Namely, I shall be using the following notations:

\begin{itemize}
\item In the following, we use the symbol $\NN$ to denote the set
$\left\{ 0, 1, 2, \ldots \right\}$. (Be warned that some other authors
use this symbol for $\left\{ 1, 2, 3, \ldots \right\}$ instead.)

\item We let $\QQ$ denote the set of all rational numbers; we let
$\RR$ be the set of all real numbers.

\item If $X$ and $Y$ are two sets, then we shall use the notation
``$X\rightarrow Y,\ x\mapsto E$'' (where $x$
is some symbol which has no specific meaning in the current context,
and where $E$ is some expression which usually involves $x$) for ``the
map from $X$ to $Y$ that sends every $x\in X$ to $E$''. For
example, ``$\NN\rightarrow\NN,\ x\mapsto x^{2}+x+6$'' means the map
from $\NN$ to $\NN$ that sends every $x\in\NN$ to $x^{2}+x+6$. For
another example, ``$\NN\rightarrow\QQ,\ x\mapsto\dfrac{x}{1+x}$''
denotes the map from $\NN$ to $\QQ$ that
sends every $x\in\NN$ to $\dfrac{x}{1+x}$.\ \ \ \ \footnote{A word of
warning: Of course, the notation ``$X\rightarrow Y,\ x\mapsto E$''
does not always make sense; indeed, the map that it
stands for might sometimes not exist. For instance, the notation
``$\NN\rightarrow\QQ,\ x\mapsto\dfrac{x}{1-x}$'' does not actually
define a map, because the map that it
is supposed to define (i.e., the map from $\NN$ to $\QQ$ that
sends every $x\in\NN$ to $\dfrac{x}{1-x}$) does not exist (since
$\dfrac{x}{1-x}$ is not defined for $x=1$). For another example, the
notation ``$\NN\rightarrow\ZZ,\ x\mapsto\dfrac{x}{1+x}$'' does not
define a map, because the map that it is
supposed to define (i.e., the map from $\NN$ to $\ZZ$ that
sends every $x\in\NN$ to $\dfrac{x}{1+x}$) does not exist (for $x=2$,
we have $\dfrac{x}{1+x}=\dfrac{2}{1+2}\notin\ZZ$, which shows that a
map from $\NN$ to $\ZZ$ cannot send this $x$ to this $\dfrac
{x}{1+x}$). Thus, when defining a map from $X$ to $Y$ (using whatever
notation), do not forget to check that it is well-defined (i.e., that
your definition specifies precisely one image for each $x\in X$, and
that these images all lie in $Y$). In many cases, this is obvious or
very easy to check (I will usually not even mention this check), but
in some cases, this is a difficult task.}

\item If $S$ is a set, then the \textit{powerset} of $S$ means the set
of all subsets of $S$. This powerset will be denoted by
$\powset{S}$. % This expands to $\mathcal{P} \left( S \right)$.
For example, the powerset of $\left\{  1,2\right\}  $ is
$\powset{ \left\{  1,2\right\} }
=\left\{  \varnothing,\left\{
1\right\}  ,\left\{  2\right\}  ,\left\{  1,2\right\}  \right\}  $.

Furthermore, if $S$ is a set and $k$ is an integer, then
$\powset[k]{S}$ % This expands to $\mathcal{P}_k\left(S\right)$.
shall mean the set of all $k$-element sets of $S$. (This is empty if
$k < 0$.)
\end{itemize}

\subsection{\label{sect.intro.simple}Simple graphs}

As already hinted above, there is not one single concept of a
``graph''. Instead, there are several mutually related (but not
equivalent) concepts of ``graph'', which are often kept apart by
adorning them with adjectives (e.g., ``simple graph'', ``directed
graph'', ``loopless graph'', ``loopless weighted undirected
graph'', ``infinite graph'', etc.) or prefixes (``digraph'',
``multigraph'', etc.). Let me first define the simplest one of these:

\begin{definition} \label{def.intro.simple.sg}
A \textit{simple graph} is a pair $\tup{V, E}$, where $V$ is a
finite set, and where $E$ is a subset of $\powset[2]{V}$.
\end{definition}

Let us unpack this definition first. The word ``simple'' in
``simple graph'' (roughly speaking) has the meaning of ``with no
bells and whistles''; i.e., it says that the notion of
``simple graph'' is one of the crudest, most primitive notions of
a graph known.
% i.e., no additional structure.
It does not mean
that everything that can be said about simple graphs is simple (this
is far from the case, as we will see below). The condition
``$E$ is a subset of $\powset[2]{V}$'' in
Definition~\ref{def.intro.simple.sg} can be rewritten as ``$E$ is a
set of $2$-element subsets of $V$'' (since $\powset[2]{V}$ is the set
of all $2$-element subsets of $V$). Thus, a simple graph is a pair
consisting of a finite set $V$, and a set of $2$-element subsets of
$V$.
For example,
$\left(\set{1,2,3}, \set{\set{1,3}, \set{3,2}} \right)$ and
$\left(\set{2,5}, \set{\set{2,5}}\right)$ and
$\left(\varnothing, \varnothing\right)$ are three simple graphs.

\begin{conclusion}
\textbf{(a)}
Our Definition~\ref{def.intro.simple.sg} differs from the definition
of a ``simple graph'' in many sources, in that we are requiring $V$ to
be finite.

\textbf{(b)}
Simple graphs are often just called ``graphs''. But then again, some
other concepts of graphs (such as multigraphs, which we will
encounter further below) are also often just called ``graphs''. Thus,
the precise meaning of the word ``graph'' depends on the context in
which it appears. For example, Bollob\'as (in \cite{Bollob79}) uses
the word ``graph'' for ``simple graph'', whereas Bondy and Murty
(in \cite{BonMur08}) use it for ``multidigraph'' (a concept we will
define further below). When reading literature, always check the
definitions (and, if these are missing, try to take an educated guess,
ruling out options that make some of the claims false).
\end{conclusion}

So far, we have not explained how we should intuitively think of
simple graphs, and why they are interesting. We will spend a
significant part of these notes answering the latter question; but let
us first comment on the former.

Simple graphs can be used to model symmetric relations between
different objects. For example, if you have $n$ integers
(for some $n \in \NN$), then you can define a graph
$\tup{V, E}$ for which $V$ is the set of these $n$ integers,
and $E$ is the set of all $2$-element subsets
$\set{u, v}$ of $V$ for which $\abs{u-v} \leq 3$. (Notice that
$\set{u, u}$ does not count as a $2$-element subset.)
For a non-mathematical example, consider an (idealized) group $P$ of
(finitely many) people, each two of which are either mutual friends or
not\footnote{We assume that if a person $u$ is a friend of a person
$v$, then $v$ is a friend of $u$. We also assume that no person $u$ is
a friend of $u$ itself (or, at least, we don't count this as
friendship).}. Then, you can define a graph $\tup{P, E}$, where $E$
is the set of all $2$-element subsets $\set{u, v}$ of $P$ for which
$u$ and $v$ are mutual friends. This graph then models the friendships
between the people in the group $P$; in a sense, it is a social
network (similar to the ones widespread on the Internet, but much more
rudimentary, since it only knows who is a friend of
whom).\footnote{This kind of ``social graphs'' has been used for many
years as a language for stating theorems about graphs without saying
the word ``graph'' (and without using mathematical notation): Just
speak of people and their mutual friendships. This language was in use
long before the Internet and actual social networks came about.}

The following notations provide a quick way to reference the elements
of $V$ and $E$ when given a graph $\tup{V, E}$:

\begin{definition} \label{def.intro.simple.VE}
Let $G = \tup{V, E}$ be a simple graph.

\textbf{(a)} The set $V$ is called the \textit{vertex set} of $G$;
it is denoted by $\verts{G}$. (Notice that the letter
``$\operatorname{V}$'' in ``$\verts{G}$'' is upright, as opposed to
the letter ``$V$'' in ``$\tup{V, E}$'', which is italic.
These are two different symbols, and have different meanings: The
letter $V$ stands for the specific set $V$ which is the first
component of the pair $G$, whereas the letter
$\operatorname{V}$ is part of the notation $\verts{G}$ for the
vertex set of any graph. Thus, if $H = \left(W, F\right)$ is another
graph, then $\verts{H}$ is $W$, not $V$.)

The elements of $V$ are called the \textit{vertices} (or the
\textit{nodes}) of $G$.

\textbf{(b)} The set $E$ is called the \textit{edge set} of $G$; it
is denoted by $\edges{G}$. (Again, the letter ``$\operatorname{E}$''
in ``$\edges{G}$'' is upright, and stands for a different thing than
the ``$E$''.)

The elements of $E$ are called the \textit{edges} of $G$. When $u$ and
$v$ are two elements of $V$, we shall often use the notation $uv$ for
$\set{u, v}$; thus, each edge of $G$ has the form $uv$ for two
distinct elements $u$ and $v$ of $V$. Of course, we always have
$uv = vu$.

\textbf{(c)} Two vertices $u$ and $v$ of $G$ are said to be
\textit{adjacent} (or \textit{connected by an edge}) if $uv \in E$
(that is, if $uv$ is an edge of $G$).

Two vertices $u$ and $v$ of $G$
are said to be \textit{non-adjacent} if they are not adjacent (i.e.,
if $uv \notin E$).

We say that a vertex $u$ of $G$ is \textit{adjacent to} a vertex $v$
of $G$ if the vertices $u$ and $v$ are adjacent (i.e., if $uv \in E$).
Similarly, we say that a vertex $u$ of $G$ is \textit{non-adjacent to}
a vertex $v$ of $G$ if the vertices $u$ and $v$ are non-adjacent
(i.e., if $uv \notin E$).

\textbf{(d)} Let $v$ be a vertex of $G$ (that is, $v \in V$). Then,
the \textit{neighbors} of $v$ are the vertices $u$ of $G$ that
satisfy $vu \in G$. (Of course, these neighbors depend on both $v$ and
$G$. When $G$ is not clear from the context, we shall call them the
``neighbors of $v$ in $G$'' instead of just ``neighbors of $v$''.)
\end{definition}

Of course, the relation of adjacency is symmetric\footnote{This means
the following: Given two vertices $u$ and $v$ of a simple graph $G$,
the vertex $u$ is adjacent to $v$ if and only if $v$ is adjacent to
$u$.}. Same holds for the relation of non-adjacency.

\begin{example} \label{exa.intro.simple.VE}
Let $U$ be the $5$-element set $\set{1,2,3,4,5}$. Let $E$ be the
subset $\set{\set{u,v} \in \powset[2]{U} \ \mid \ u + v \geq 5 }$
of $\powset[2]{U}$. This set $E$ is well-defined, because the sum
$u + v$ of two integers $u$ and $v$ depends only on the set
$\set{u,v}$ and not on how this set is written (since
$u + v = v + u$). (This is important, because if we had used the
condition $u - v \geq 3$ instead of $u + v \geq 5$, then the set $E$
would not be well-defined, because it would not be clear whether
$\set{1, 5}$ should be inside it or not -- indeed, if we write
$\set{1, 5}$ as $\set{u, v}$ with $u = 5$ and $v = 1$, then
$u - v \geq 3$ is satisfied, but if we write $\set{1, 5}$ as
$\set{u, v}$ with $u = 1$ and $v = 5$, then $u - v \geq 3$ is not
satisfied.)

Let $G$ be the graph $\tup{U, E}$. Then, $\verts{G} = U
= \set{1,2,3,4,5}$ and
\begin{align}
\edges{G} &= E
= \set{\set{u,v} \in \powset[2]{U} \ \mid \ u + v \geq 5 }
\nonumber \\
&= \set{\set{1,4}, \set{1,5},
        \set{2,3}, \set{2,4}, \set{2,5},
        \set{3,4}, \set{3,5},
        \set{4,5}} .
\label{eq.exa.intro.simple.VE.edgesG=}
\end{align}
Thus, $G$ has $\abs{\verts{G}} = \abs{U} = 5$ vertices and
$\abs{\edges{G}} = \abs{E} = 8$ edges.

Using the shorthand notation
$uv$ for $\set{u, v}$ (introduced in
Definition~\ref{def.intro.simple.VE} \textbf{(b)}), the equality
\eqref{eq.exa.intro.simple.VE.edgesG=} rewrites as
\[
\edges{G}
= \set{14, 15, 23, 24, 25, 34, 35, 45} .
\]

The vertices $2$ and $4$ of $G$ are adjacent (since $24 \in E$).
In other words, $4$ is a neighbor of $2$. Equivalently, $2$ is a
neighbor of $4$. On the other hand, the vertices $1$ and $3$ of $G$
are not adjacent (since $13 \notin E$); thus, $1$ is not a neighbor
of $3$. The neighbors of $1$ are $4$ and $5$.
\end{example}

\subsection{\label{sect.intro.draw}Drawing graphs}

There is a common method to represent graphs visually: Namely, a graph
can be drawn as a set of points in the plane and a set of curves
connecting some of these points with each other.
\begin{todo}
Write this section.
\end{todo}

\subsection{\label{sect.intro.R33}A first fact: $R\tup{3,3} = 6$}

After these definitions, it might be time for a first result. The
following classical fact (which is actually the beginning of a deep
theory -- the so-called \textit{Ramsey theory}) should neatly
illustrate the concepts introduced above:

\begin{proposition} \label{prop.simple.R33}
Let $G$ be a simple graph with $\abs{\verts{G}} \geq 6$ (that is,
$G$ has at least $6$ vertices). Then, at least one of the following
two statements holds:

\begin{itemize}
\item \textit{Statement 1:} There exist three distinct vertices $a$,
$b$ and $c$ of $G$ such that $ab$, $bc$ and $ca$ are edges of $G$.

\item \textit{Statement 2:} There exist three distinct vertices $a$,
$b$ and $c$ of $G$ such that none of $ab$, $bc$ and $ca$ is an edge of
$G$.
\end{itemize}
\end{proposition}

In other words, Proposition~\ref{prop.simple.R33} says that if a graph
$G$ has at least $6$ vertices, then we can either find three distinct
vertices that are mutually adjacent\footnote{by which we mean (of
course) that any two \textbf{distinct} ones among these three vertices
are adjacent} or find three distinct vertices that are mutually
non-adjacent (i.e., no two of them are adjacent), or both.

\begin{todo}
Rewrite in terms of a party.
\end{todo}

\begin{todo}
Examples: Statement 1 holds; Statement 2 holds; both hold; neither
holds (pentagon, with 5 vertices).
\end{todo}

\begin{proof}[Proof of Proposition~\ref{prop.simple.R33}.]We need to
prove that either Statement 1 holds or Statement 2 holds (or both).

Choose any vertex $u \in \verts{G}$. (This is clearly possible,
since $\abs{\verts{G}} \geq 6 \geq 1$.) Then,
$\abs{\verts{G} \setminus \set{u}} = \abs{\verts{G}} - 1 \geq 5$
(since $\abs{\verts{G}} \geq 6$). We are in one of the following two
cases:

\textit{Case 1:} At least $3$ vertices in
$\verts{G} \setminus \set{u}$ are adjacent to $u$.

\textit{Case 2:} At most $2$ vertices in
$\verts{G} \setminus \set{u}$ are adjacent to $u$.

Let us consider Case 1 first. In this case, at least $3$ vertices in
$\verts{G} \setminus \set{u}$ are adjacent to $u$. Hence, we can find
three distinct vertices $p$, $q$ and $r$ in
$\verts{G} \setminus \set{u}$ that are adjacent to $u$. Consider
these $p$, $q$ and $r$. If none of $pq$, $qr$ and $rp$ is an edge of
$G$, then Statement 2 holds (in fact, we can just take $a = p$,
$b = q$ and $c = r$). Thus, if none of $pq$, $qr$ and $rp$ is an edge
of $G$, then our proof is complete\footnote{because our goal in this
proof is to show that either Statement 1 holds or Statement 2 holds
(or both)}. Thus, we WLOG\footnote{The word ``WLOG'' means
``without loss of generality''. See, e.g., the corresponding Wikipedia
page \url{https://en.wikipedia.org/wiki/Without_loss_of_generality}
for its meaning.} assume that at least one $pq$, $qr$ and $rp$ is an
edge of $G$. In other words, we can pick two distinct elements $g$
and $h$ of $\set{p, q, r}$ such that $gh$ is an edge of $G$. Consider
these $g$ and $h$.

The vertex $g$ is one of $p$, $q$ and $r$
(since $g \in \set{p, q, r}$).
The vertices $p$, $q$ and $r$ are adjacent to $u$. Hence, the vertex
$g$ is adjacent to $u$ (since the vertex $g$ is one of $p$, $q$ and
$r$). In other words, $ug$ is an edge of $G$. Similarly, $uh$ is an
edge of $G$. In other words, $hu$ is an edge of $G$ (since $hu = uh$).

We have $g \in \set{p, q, r} \subseteq \verts{G} \setminus \set{u}$
(since $p$, $q$ and $r$ belong to $\verts{G} \setminus \set{u}$).
Hence, $g \neq u$. In other words, $u \neq g$.
% Clearly, $ug$ is a $2$-element set (since $ug$ is an edge of $G$, but
% each edge of $G$ is a $2$-element set). Thus, $u \neq g$.
Similarly,
$u \neq h$. Hence, $h \neq u$. Finally, $g \neq h$ (since $g$ and $h$
are distinct). Now, we know that the three vertices $u$, $g$ and $h$
are distinct (since $u \neq g$, $g \neq h$ and $h \neq u$), and have
the property that $ug$, $gh$ and $hu$ are edges of $G$. Therefore,
Statement 1 holds (in fact, we can just take $a = u$, $b = g$ and
$c = h$). Hence, the proof is complete in Case 1.

Let us now consider Case 2. In this case, at most $2$ vertices in
$\verts{G} \setminus \set{u}$ are adjacent to $u$. Thus, at least $3$
vertices in $\verts{G} \setminus \set{u}$ are non-adjacent to $u$
\ \ \ \ \footnote{\textit{Proof.} Let $k$ be the number of vertices
in $\verts{G} \setminus \set{u}$ that are adjacent to $u$. Let $\ell$
be the number of vertices in $\verts{G} \setminus \set{u}$ that are
non-adjacent to $u$. Then,
$k + \ell = \abs{\verts{G} \setminus \set{u}}$ (since each vertex
in $\verts{G} \setminus \set{u}$ is either adjacent to $u$ or
non-adjacent to $u$, but not both at the same time). But $k \leq 2$
(since at most $2$ vertices in $\verts{G} \setminus \set{u}$ are
adjacent to $u$). Hence, $k + \ell \leq 2 + \ell = \ell + 2$, so that
$\ell + 2 \geq k + \ell = \abs{\verts{G} \setminus \set{u}} \geq 5$
and thus $\ell \geq 3$. In other words, at least $3$
vertices in $\verts{G} \setminus \set{u}$ are non-adjacent to $u$.
Qed.}. Hence, we can find
three distinct vertices $p$, $q$ and $r$ in
$\verts{G} \setminus \set{u}$ that are non-adjacent to $u$. Consider
these $p$, $q$ and $r$. If all of $pq$, $qr$ and $rp$ are edges of
$G$, then Statement 1 holds (in fact, we can just take $a = p$,
$b = q$ and $c = r$). Thus, if all of $pq$, $qr$ and $rp$ are edges
of $G$, then our proof is complete. Thus, we WLOG
assume that not all of $pq$, $qr$ and $rp$ are edges of $G$. In other
words, at least one of $pq$, $qr$ and $rp$ is \textbf{not} an edge of
$G$. In other words, we can pick two distinct elements $g$ and $h$ of
$\set{p, q, r}$ such that $gh$ is not an edge of $G$. Consider
these $g$ and $h$.

The vertex $g$ is one of $p$, $q$ and $r$
(since $g \in \set{p, q, r}$).
The vertices $p$, $q$ and $r$ are non-adjacent to $u$. Hence, the
vertex $g$ is non-adjacent to $u$ (since the vertex $g$ is one of $p$,
$q$ and $r$). In other words, $ug$ is not an edge of $G$. Similarly,
$uh$ is not an edge of $G$. In other words, $hu$ is not an edge of $G$
(since $hu = uh$).

We have $g \in \set{p, q, r} \subseteq \verts{G} \setminus \set{u}$
(since $p$, $q$ and $r$ belong to $\verts{G} \setminus \set{u}$).
Thus, $g \neq u$. In other words, $u \neq g$. Similarly, $u \neq h$.
Finally, $g \neq h$ (since $g$ and $h$
are distinct). Now, we know that the three vertices $u$, $g$ and $h$
are distinct (since $u \neq g$, $g \neq h$ and $h \neq u$), and have
the property that none of $ug$, $gh$ and $hu$ is an edge of $G$ (since
$ug$ is not an edge of $G$, since $gh$ is not an edge of $G$, and
since $hu$ is not an edge of $G$). Therefore,
Statement 2 holds (in fact, we can just take $a = u$, $b = g$ and
$c = h$). Hence, the proof is complete in Case 2.

We have now proven Proposition~\ref{prop.simple.R33} in each of the
two Cases 1 and 2. Thus, the proof of
Proposition~\ref{prop.simple.R33} is complete.
\end{proof}

\begin{remark}
I have written the above proof in much detail, since it is the first
proof in these notes. I could have easily made it much shorter if I
had relied on the reader to fill in some details (and in fact, I
\textbf{will} rely on the reader in similar situations further below).
The proof can further be shortened by noticing that part of the
argument for Case 2 was a ``mirror version'' of the argument for
Case 1, with the only difference that ``adjacent'' is replaced by
``non-adjacent'' (and vice versa), and ``is an edge'' is replaced by
``is not an edge'' (and vice versa).
\end{remark}

\begin{remark}
Let me observe that Proposition~\ref{prop.simple.R33} could be proven
by brute force as well (using a computer). Indeed, here is how such a
proof would proceed: Let $x_1, x_2, x_3, x_4, x_5, x_6$ be six
distinct vertices of $G$. (Such six vertices exist, since
$\abs{\verts{G}} \geq 6$.) Let
$X = \set{x_1, x_2, x_3, x_4, x_5, x_6}$ be the set of these six
vertics. Notice that the set $X$ has $6$ elements, and thus the set
$\powset[2]{X}$ has $\dbinom{6}{2} = 15$ elements.
Let $F$ be the set of all edges $uv$ of $G$ for which both
$u$ and $v$ belong to $X$. (In other words, $F$ is the set of all
edges of $G$ having the form $x_i x_j$.)
Clearly, it suffices to prove Proposition~\ref{prop.simple.R33} for
the graph $\tup{X, F}$ instead of $G$ (because if we have found, for
example, three distinct vertices $a$, $b$ and $c$ of $\tup{X, F}$ such
that $ab$, $bc$ and $ca$ are edges of $\tup{X, F}$, then these $a$,
$b$ and $c$ are obviously also three vertices of $G$ such that
$ab$, $bc$ and $ca$ are edges of $G$). However,
$F$ is a subset of $\powset[2]{X}$. Since there
are only finitely many subsets of $\powset[2]{X}$ (in fact, there are
$2^{15}$ such subsets, since $\powset[2]{X}$ has $15$ elements), we
thus see that there are only finitely many choices for $F$ (when $X$
is being regarded as fixed). We can
check, for each of these choices, whether the graph $\tup{X, F}$
satisfies Proposition~\ref{prop.simple.R33}. (Just try each
possible choice of three distinct vertices $a$, $b$ and $c$ of this
graph $\tup{X, F}$, and check that at least one of these choices
satisfies either Statement 1 or Statement 2.) After a huge but
finite amount of checking (which you can automate), you will see that
Proposition~\ref{prop.simple.R33} holds for $\tup{X, F}$. Thus, as we
have already mentioned, Proposition~\ref{prop.simple.R33} also holds
for the original graph $G$.
\end{remark}

\begin{todo}
Ramsey theory in general: binomial coeff bound with ref.
Cite Wikipedia page and at least \cite{GrRoSp90}.
Also mention \cite{RaWi-Ramsey} for intuition (Ramsey theory in wider
meaning of the word: results that say that each pattern of some given
type can be found in a sufficiently large structure).

Mention also \url{http://www.cut-the-knot.org/Curriculum/Combinatorics/ThreeOrThree.shtml}
in \cite{cut-the-knot}.
\end{todo}

\subsection{\label{sect.intro.deg}Degrees}

Next, we introduce the notion of the \textit{degree} of a vertex of a
graph. This is simply the number of neighbors of the vertex:

\begin{definition} \label{def.intro.deg}
Let $G = \tup{V, E}$ be a simple graph. Let $v \in V$ be a vertex of
$G$. Then, the \textit{degree} of $v$ (with respect to $G$) is defined
as the number of all neighbors of $v$ in $G$. This degree is a
nonnegative integer, and is denoted by $\deg v$ (or by
$\deg_G v$, when the graph $G$ is not clear from the context).
Thus,
\begin{align*}
\deg v &= \deg_G v =
\left(\text{the number of all neighbors of } v\right) \\
&= \abs{\set{u \in V \ \mid \ u \text{ is a neighbor of } v }} \\
&= \abs{\set{u \in V \ \mid \ u \text{ is adjacent to } v }} \\
&= \abs{\set{u \in V \ \mid \ uv \text{ is an edge of } G }} \\
&= \abs{\set{u \in V \ \mid \ uv \in E }} .
\end{align*}
\end{definition}

\begin{todo}
Examples of degrees.
\end{todo}

\begin{remark}
Different sources use different notations for the degree of a
vertex $v$ of a simple graph $G$. We call it $\deg v$ (and so do
Ore's \cite{Ore90} and the introductory notes
\cite{LeLeMe16}); Ore's \cite{Ore62} calls it $\rho\left(v\right)$;
Bollob\'as's \cite{Bollob79} and Bondy's and Murty's
\cite{BonMur08} and \cite{BonMur76} call it $d\left(v\right)$.
% Not sure if anyone uses $\delta\left(v\right)$.
\end{remark}

At this point, we can state a few simple facts about degrees:

\begin{proposition} \label{prop.intro.deg.in-set}
Let $G$ be a simple graph. Let $n = \abs{\verts{G}}$. Let $v$ be a
vertex of $G$. Then, $\deg v \in \set{0, 1, \ldots, n-1}$.
\end{proposition}

\begin{todo}
Proof.
\end{todo}

\begin{proposition} \label{prop.intro.2n}
Let $G$ be a simple graph.
The sum of the degrees of all vertices of $G$ equals
twice the number of edges of $G$. In other words,
$\sum_{v \in \verts{G}} \deg v = 2 \abs{\edges{G}}$.
\end{proposition}

\begin{todo}
Proof.
\end{todo}

\begin{proposition} \label{prop.intro.even-odd}
Let $G$ be a simple graph.
Then, the number of vertices $v$ of $G$ whose degree $\deg v$ is odd
is even.
\end{proposition}

\begin{todo}
Proof.
\end{todo}

\begin{todo}
``odd number of friends''.
\end{todo}

\begin{proposition} \label{prop.intro.pigeonhole}
Let $G$ be a simple graph with at least two vertices.
Then, there exist two distinct vertices $v$ and $w$ of $G$ having the
same degree (that is, having $\deg v = \deg w$).
\end{proposition}

\begin{todo}
Proof.
\end{todo}

Degrees of vertices can be used to prove various facts about graphs.
For an example, let us show \textit{Mantel's theorem}:

\begin{theorem} \label{thm.intro.mantel}
Let $G$ be a simple graph. Let $n = \abs{\verts{G}}$ be the number of
vertices of $G$. Assume that $\abs{\edges{G}} > n^2 / 4$. (In other
words, assume that $G$ has more than $n^2 / 4$ edges.) Then, there
exist three distinct vertices $a$,
$b$ and $c$ of $G$ such that $ab$, $bc$ and $ca$ are edges of $G$.
\end{theorem}

\begin{todo}
Example.
\end{todo}

\begin{todo}
Proof.
\end{todo}

\begin{exercise} \label{exe.intro.mantel-co}
Let $G$ be a simple graph. Let $n = \abs{\verts{G}}$ be the number of
vertices of $G$. Assume that $\abs{\edges{G}} < n\tup{n-2} / 4$. (In
other words, assume that $G$ has less than $n\tup{n-2} / 4$ edges.)
Then, there exist three distinct vertices $a$, $b$ and $c$ of $G$ such
that none of $ab$, $bc$ and $ca$ are edges of $G$.
\end{exercise}

\subsection{\label{sect.intro.paths}Examples of graphs}

\begin{todo}
Define complete graphs; empty graphs; path graphs; cycle graphs; a
bunch of basic graphs; mod-3 congruence graph (just as an example);
2-element subsets with common elements.
\end{todo}

\subsection{\label{sect.intro.walks}Walks and paths}

\begin{definition} \label{def.intro.walks}
Let $G$ be a simple graph.

\textbf{(a)} A \textit{walk} (in $G$) means a sequence
$\tup{v_0, v_1, \ldots, v_k}$ of vertices of $G$ (with $k \geq 0$)
such that all of
$v_0 v_1, v_1 v_2, \ldots, v_{k-1} v_k$ are edges of $G$. (We allow
$k$ to be $0$, in which case the condition that
``$v_0 v_1, v_1 v_2, \ldots, v_{k-1} v_k$ are edges of $G$'' is
vacuously true.)

\textbf{(b)} If $\mathbf{w} = \tup{v_0, v_1, \ldots, v_k}$ is a walk
in $G$, then:

\begin{itemize}
\item The \textit{vertices of $\mathbf{w}$} are defined to be
the vertices $v_0, v_1, \ldots, v_k$.
\item The
\textit{edges of $\mathbf{w}$} are defined to be the edges
$v_0 v_1, v_1 v_2, \ldots, v_{k-1} v_k$ of $G$;
\item The nonnegative integer $k$ is called the
\textit{length} of $\mathbf{w}$. (This integer is the number of all
edges of $G$, counted with multiplicity. It is $1$ smaller than the
number of all vertices of $G$, counted with multiplicity.)
\item The vertex $v_0$ is said to be the \textit{starting point} of
$\mathbf{w}$.
\item The vertex $v_k$ is said to be the \textit{ending point} of
$\mathbf{w}$.
\end{itemize}

\textbf{(c)} A \textit{path} (in $G$) means a walk (in $G$) whose
vertices are distinct. (In other words, a \textit{path} in $G$ means
a walk $\tup{v_0, v_1, \ldots, v_k}$ in $G$ such that
$v_0, v_1, \ldots, v_k$ are distinct.)

\textbf{(e)} Let $p$ and $q$ be two vertices of $G$. A
\textit{walk from $p$ to $q$} (in $G$) means a walk (in $G$) whose
starting point is $p$ and whose ending point is $q$. A
\textit{path from $p$ to $q$} (in $G$) means a path (in $G$) whose
starting point is $p$ and whose ending point is $q$.
\end{definition}

\begin{todo}
Examples!
\end{todo}

\begin{exercise} \label{exe.intro.path.edges-dist}
Let $G$ be a simple graph. Let $\mathbf{w}$ be a path in $G$.
Prove that the edges of $\mathbf{w}$ are distinct. (This may look
obvious when you can point to a picture; but we ask you to give a
rigorous proof!)
\end{exercise}

\begin{proposition} \label{prop.intro.paths-and-walks}
Let $G$ be a simple graph. Let $p$ and $q$ be two vertices of $G$.
The following six statements are equivalent:

\begin{itemize}
\item \textit{Statement 1:} There exists a walk from $u$ to $v$.

\item \textit{Statement 2:} There exists a walk from $v$ to $u$.

\item \textit{Statement 3:} There exists a walk from one of the two
vertices $u$ and $v$ to the other.

\item \textit{Statement 4:} There exists a path from $u$ to $v$.

\item \textit{Statement 5:} There exists a path from $v$ to $u$.

\item \textit{Statement 6:} There exists a path from one of the two
vertices $u$ and $v$ to the other.
\end{itemize}
\end{proposition}

\begin{todo}
Proof.
\end{todo}

\begin{todo}
Dijkstra's algorithm?
\end{todo}

\begin{todo}
Cycles and circuits.
\end{todo}

\subsection{\label{sect.intro.teasers}Questions to ask about graphs}

\begin{todo}
Connectedness.
\end{todo}

\begin{todo}
Hamiltonian paths and cycles:
Do they exist? How to find them? (We will have partial results, mainly sufficient conditions.)
\end{todo}

\begin{todo}
Eulerian walks and circuits.
(Theory is really nice here, with easy necessary-and-sufficient conditions, but we'll wait for a better notion of graph.)
\end{todo}

\begin{todo}
Matchings:
how large, how many, structure? (Nice and large theory, still under research; we will see a lot of it but still barely scratch the surface.)
\end{todo}

\subsection{\label{sect.intro.dominating}Dominating sets}

I will next digress to discuss the notion of \textit{dominating sets}.
The reason why I am doing this at this point is not that dominating
sets are of any particular fundamental importance (they are not; we
probably will not use them anywhere after this section), but rather
that they neatly illustrate the concepts we have seen so far and
provide some experience with graph-theoretical proofs, all that while
not requiring any complex theory or advanced techniques.

\begin{definition} \label{def.intro.dominating}
Let $G$ be a simple graph. A subset $U$ of $\verts{G}$ is said to be
\textit{dominating} (for $G$) if it has the following property: For
every vertex $v \in \verts{G} \setminus U$, at least one neighbor of
$v$ belongs to $U$.

A \textit{dominating set} of $G$ means a dominating subset of
$\verts{G}$.
\end{definition}

\begin{todo}
Example and non-example (e.g., on pentagon).
\end{todo}

Clearly, if $G$ is a simple graph, then its vertex set $\verts{G}$ is
a dominating set.
One natural question to ask is how small a dominating set of a graph
can be. When the graph $G$ is empty, only the vertex set $\verts{G}$
itself is dominating. On the other hand, when $G$ is a complete graph
on $n \geq 1$ vertices, every nonempty subset of $\verts{G}$ is
dominating. Clearly, the more edges a simple graph has, the more
dominating sets it has (in the sense that if we add a new edge, then
all sets that are dominating remain dominating, and possibly new
dominating sets appear). It is furthermore clear that if a vertex of
a simple graph $G$ has no neighbors, then it must belong to each
dominating set of $G$ (because otherwise, at least one neighbor of
this vertex would need to lie in the dominating set; but this is
impossible, since it has no neighbors). Such vertices are said to be
\textit{isolated}.

\begin{definition} \label{def.intro.isolated}
Let $G$ be a simple graph. A vertex $v$ of $G$ is said to be
\textit{isolated} if it has no neighbors. (In other words, a vertex
$v$ of $G$ is said to be \textit{isolated} if its degree $\deg v$
equals $0$.)
\end{definition}

\begin{proposition} \label{prop.intro.dominating.|V|/2}
Let $G = \tup{V, E}$ be a simple graph that has no isolated vertices.

\textbf{(a)} There exist two disjoint dominating subsets $A$ and $B$
of $V$ such that $A \cup B = V$.

\textbf{(b)} There exists a dominating subset of $V$ having size
$\geq \abs{V}/2$.
\end{proposition}

We will see a proof of this proposition later, when we have defined
the notion of the \textit{distance} between two vertices in a graph.

\begin{todo}
Bound in Proposition~\ref{prop.intro.dominating.|V|/2} \textbf{(b)}
is unsharp: example.
\end{todo}

Next, we state a surprising recent result by Brouwer (\cite{Brouwe09},
from 2009) about the number of dominating sets of a graph:

\begin{theorem} \label{thm.intro.dominating.brouwer}
Let $G$ be a simple graph. Then, the number of dominating sets of $G$
is odd.
\end{theorem}

Brouwer (in \cite{Brouwe09}) gives three proofs of this theorem. We
are going to give another. Better yet, we shall prove a more
precise result which is even more recent (a preprint \cite{HeiTit17}
from 2017), due to Heinrich and Tittmann:

\begin{theorem} \label{thm.intro.dominating.heinrich}
Let $G = \tup{V, E}$ be a simple graph. Let $n = \abs{V}$. Assume that
$n \geq 1$.

A \textit{detached pair} shall mean a pair $\tup{A, B}$ of two
disjoint subsets $A$ and $B$ of $V$ having the property that
there exists no edge $ab \in E$ satisfying
$a \in A$ and $b \in B$.

Let $\alpha$ be the number of all detached pairs $\tup{A, B}$ for
which both numbers $\abs{A}$ and $\abs{B}$ are even and positive.

Let $\beta$ be the number of all detached pairs $\tup{A, B}$ for
which both numbers $\abs{A}$ and $\abs{B}$ are odd.

Then:

\textbf{(a)} The numbers $\alpha$ and $\beta$ are even.

\textbf{(b)} The number of dominating sets of $G$ is
$2^n - 1 + \alpha - \beta$.
\end{theorem}

Theorem~\ref{thm.intro.dominating.heinrich} is a restatement of
\cite[Theorem 8]{HeiTit17}. The proof we shall give below is much
shorter than the proof in \cite{HeiTit17}, but does not lead us
through as many interesting intermediate results.

\begin{todo}
Derive Theorem~\ref{thm.intro.dominating.brouwer} from
Theorem~\ref{thm.intro.dominating.heinrich}.
\end{todo}

Our proof of Theorem~\ref{thm.intro.dominating.heinrich} will rely on
a few lemmas. But first, we shall introduce a notation:

\begin{definition} \label{def.intro.iverson}
Let $\mathcal{A}$ be a logical statement. Then, an element
$\ive{\mathcal{A}} \in \set{0, 1}$ is defined as follows:
We set $\ive{\mathcal{A}} =
\begin{cases}
1, & \text{if }\mathcal{A}\text{ is true};\\
0, & \text{if }\mathcal{A}\text{ is false}
\end{cases}$.
This element $\ive{\mathcal{A}}$ is called the \textit{truth value}
of $\mathcal{A}$.
(For example, $\ive{1+1 = 2} = 1$ and $\ive{1+1 = 3} = 0$. For
another example,
$\ive{\text{Proposition~\ref{prop.intro.dominating.|V|/2} holds}}
= 1$, although we have not proven this yet.)
The notation $\ive{\mathcal{A}}$ for the truth value of $\mathcal{A}$
is known as the \textit{Iverson bracket notation}.
\end{definition}

\begin{todo}
Simple rules for Iverson brackets. Most importantly, equivalence!
Exercise: prove them. Also, $\abs{\edges{G}}$ and $\deg v$ using
sums of Iverson brackets.
\end{todo}

The following lemma is fundamental to much of combinatorics (if not
to say much of mathematics):

\begin{lemma} \label{lem.intro.dominating.heinrich-lemma1}
Let $P$ be a finite set. Then,
\[
\sum_{\substack{A \subseteq P}} \left(-1\right)^{\abs{A}}
= \ive{P = \varnothing} .
\]
\end{lemma}

\begin{todo}
Proof.
\end{todo}

\begin{lemma} \label{lem.intro.dominating.heinrich-lemma2}
Let $G = \tup{V, E}$ be a simple graph. Let $B$ be a subset of $V$.
Then,
\[
\sum_{\substack{A \subseteq V; \\ \tup{A, B}
\text{ is a detached pair}}} \left(-1\right)^{\abs{A}}
= \ive{B \text{ is dominating}}.
\]
\end{lemma}

\begin{todo}
Proof.
\end{todo}

\begin{todo}
Proof of the whole thing.
\end{todo}

%XXX

[...]

[to be continued]

\begin{thebibliography}{9999999999}                                                                                       %

\bibitem[AigZie]{AigZie}Martin Aigner, G\"{u}nter M. Ziegler,
\textit{Proofs from the Book}, 4th edition, Springer 2010.

\bibitem[AoPS-ISL]{AoPS-ISL}Art of Problem Solving (forum),
\textit{IMO Shortlist} (collection of threads),
\newline
\url{http://www.artofproblemsolving.com/community/c3223_imo_shortlist}

% \bibitem[Artin10]{Artin10}Michael Artin, \textit{Algebra}, 2nd edition,
% Pearson 2010.

\bibitem[Bahran15]{Bahran15}Cihan Bahran,
\textit{Solutions to Math 5707 Spring 2015 homework}.
\newline \url{http://www-users.math.umn.edu/~bahra004/5707.html}

% \bibitem[BarSch73]{BarSch73}Hans Schneider, George Phillip Barker,
% \textit{Matrices and Linear Algebra}, 2nd edition, Dover 1973.

\bibitem[BehCha71]{BehCha71}Mehdi Behzad, Gary Chartrand,
\textit{Introduction to the Theory of Graphs},
Allyn \& Bacon, 1971.

\bibitem[BenWil12]{BenWil12}Edward A. Bender and S. Gill Williamson,
\textit{Foundations of Combinatorics with Applications}.
\newline \url{http://cseweb.ucsd.edu/~gill/FoundCombSite/}

\bibitem[BeChZh15]{BeChZh15}Arthur Benjamin, Gary Chartrand,
Ping Zhang,
\textit{The fascinating world of graph theory},
Princeton University Press 2015.

\bibitem[Berge91]{Berge91}Claude Berge,
\textit{Graphs}, 3rd edition, North-Holland 1991.

\bibitem[Bogomoln]{cut-the-knot}Alexander Bogomolny,
\textit{Cut the Knot} (website devoted to educational applets on
various mathematical subjects),
\url{http://www.cut-the-knot.org/Curriculum/index.shtml#combinatorics} .

\bibitem[Bollob79]{Bollob79}B\'ela Bollob\'as,
\textit{Graph Theory: An Introductory Course},
Graduate Texts in Mathematics \#63, 1st edition, Springer 1971.

\bibitem[BonMur76]{BonMur76}
J. A. Bondy and U. S. R. Murty, \textit{Graph theory with Applications},
North-Holland 1976.
\newline \url{https://www.iro.umontreal.ca/~hahn/IFT3545/GTWA.pdf}.

\bibitem[BonMur08]{BonMur08}
J. A. Bondy and U. S. R. Murty, \textit{Graph theory}, Graduate Texts
in Mathematics \#244, Springer 2008.
\newline \url{https://www.classes.cs.uchicago.edu/archive/2016/spring/27500-1/hw3.pdf}

\bibitem[BonTho77]{BonTho77}
J. A. Bondy, C. Thomassen,
\textit{A short proof of Meyniel's theorem},
Discrete Mathematics 19 (1977), pp. 195--197.
\newline \url{http://www.sciencedirect.com/science/article/pii/0012365X77900346}

\bibitem[Brouwe09]{Brouwe09}
Andries E. Brouwer,
\textit{The number of dominating sets of a finite graph is odd},
\url{http://www.win.tue.nl/~aeb/preprints/domin2.pdf} .

% \bibitem[Camero08]{Camero08}Peter J. Cameron, \textit{Notes on Linear
% Algebra}, version 5 Sep 2008.\newline\url{http://www.maths.qmul.ac.uk/~pjc/notes/linalg.pdf}

\bibitem[ChaLes15]{ChaLes15}
Gary Chartrand, Linda Lesniak, Ping Zhang,
\textit{Graphs \& Digraphs}, 6th edition, CRC Press 2016.

\bibitem[Conrad]{Conrad-sign}Keith Conrad, \textit{Sign of permutations},
\newline\url{http://www.math.uconn.edu/~kconrad/blurbs/grouptheory/sign.pdf} .

\bibitem[Day16]{Day-proofs}Martin V. Day,
\textit{An Introduction to Proofs and the Mathematical Vernacular},
\newline\url{https://www.math.vt.edu/people/day/ProofsBook/IPaMV.pdf} .

% \bibitem[deBoor]{deBoor}Carl de Boor, \textit{An empty exercise}.
% \url{ftp://ftp.cs.wisc.edu/Approx/empty.pdf} .

\bibitem[Dieste16]{Dieste16}Reinhard Diestel, \textit{Graph Theory},
Graduate Texts in Mathematics \#173, 5th edition, Springer 2016.
\newline \url{http://diestel-graph-theory.com/basic.html} .

\bibitem[Gessel79]{Gessel-Vand}Ira Gessel, \textit{Tournaments and
Vandermonde's Determinant}, Journal of Graph Theory, Vol. 3 (1979), pp. 305--307.

\bibitem[GrRoSp90]{GrRoSp90}Ronald L. Graham, Bruce L. Rothschild,
Joel H. Spencer, \textit{Ramsey Theory}, 2nd edition,
Wiley 1990.

\bibitem[Griffi15]{Griffi15}Christopher Griffin,
\textit{Graph Theory: Penn State Math 485 Lecture Notes},
version 1.4.2.1 (18 Oct 2015),
\newline\url{https://sites.google.com/site/cgriffin229/} .

\bibitem[Grinbe16]{detnotes}Darij Grinberg, \textit{Notes on the combinatorial
fundamentals of algebra}, 29 December 2016.
\newline\url{http://www.cip.ifi.lmu.de/~grinberg/primes2015/sols.pdf}

\bibitem[HaHiMo08]{HaHiMo08}
John M. Harris, Jeffry L. Hirst, Michael J. Mossinghoff,
\textit{Combinatorics and Graph Theory}, Undergraduate Texts in
Mathematics, Springer 2008.

% \bibitem[Heffer16]{Heffer16}Jim Hefferon, \textit{Linear Algebra},
% 2016.\newline\url{http://joshua.smcvt.edu/linearalgebra/}

\bibitem[HeiTit17]{HeiTit17}
Irene Heinrich, Peter Tittmann,
\textit{Counting Dominating Sets of Graphs},
\arxiv{1701.03453v1}.

% \bibitem[OlvSha06]{OlvSha06}Peter J. Olver, Chehrzad Shakiban, \textit{Applied
% Linear Algebra}, Prentice Hall, 2006.\newline See also
% \url{http://www.math.umn.edu/~olver/ala.html} for corrections.

\bibitem[KelTro15]{KelTro15}Mitchel T. Keller, William T. Trotter,
\textit{Applied Combinatorics},
version 26 May 2015.
\newline \url{https://people.math.gatech.edu/~trotter/book.pdf}

\bibitem[Knuth95]{Knuth95} Donald E. Knuth,
\textit{Overlapping Pfaffians},
Electron. J. Combin. 3 (1996), no. 2, \#R5.
Also available as arXiv preprint \arxiv{math/9503234v1}.

% \bibitem[Kowals16]{Kowals16}Emmanuel Kowalski, \textit{Linear Algebra},
% version 15 Sep 2016.\newline\url{https://people.math.ethz.ch/~kowalski/script-la.pdf}

\bibitem[LaNaSc16]{LaNaSc16}Isaiah Lankham, Bruno Nachtergaele, Anne
Schilling, \textit{Linear Algebra As an Introduction to Abstract Mathematics},
2016.\newline\url{https://www.math.ucdavis.edu/~anne/linear_algebra/mat67_course_notes.pdf}

\bibitem[LeLeMe16]{LeLeMe16}Eric Lehman, F. Thomson Leighton, Albert R. Meyer,
\textit{Mathematics for Computer Science}, revised Thursday 28th September,
2016, \newline\url{https://courses.csail.mit.edu/6.042/spring16/mcs.pdf} .

% \bibitem[m.se709196]{m.se709196}Daniela Diaz and others, \textit{Definition of
% General Associativity for binary operations}, math.stackexchange question
% \#709196 .\newline\url{http://math.stackexchange.com/q/709196}

\bibitem[Ore62]{Ore62}
Oystein Ore, \textit{Theory of graphs},
AMS Colloquium Publications \#XXXVIII,
4th printing, AMS 1962.

\bibitem[Ore90]{Ore90}
Oystein Ore, \textit{Graphs and their uses}, revised and updated by
Robin J. Wilson,
Anneli Lax New Mathematical Library \#34, MAA 1990.

\bibitem[Overbe74]{Overbe74}Maria Overbeck-Larisch,
\textit{Hamiltonian paths in oriented graphs},
Journal of Combinatorial Theory, Series B,
Volume 21, Issue 1, August 1976, pp. 76--80.
\newline \url{http://www.sciencedirect.com/science/article/pii/0095895676900307}

\bibitem[Petrov15]{Petrov15}Fedor Petrov,
\textit{mathoverflow post \#198679 (Flooding a cycle digraph via
chip-firing: $n^{k-1} + n^{k-2} + \cdots + 1$ bound (a Norway 1998-99
problem generalized))},
MathOverflow,
\newline \url{http://mathoverflow.net/q/198679}

\bibitem[PoTaWo83]{PoTaWo83}George P\'olya, Robert E. Tarjan,
Donald R. Woods,
\textit{Notes on Introductory Combinatorics},
Birkh\"auser 1983.
\newline See
\url{http://i.stanford.edu/pub/cstr/reports/cs/tr/79/732/CS-TR-79-732.pdf}
for a preliminary version.

\bibitem[Pretzel]{Pretzel}Oliver Pretzel, \textit{On reorienting graphs by
pushing down maximal vertices, }Order, 1986, Volume 3, Issue 2, pp. 135--153.

\bibitem[RaWiRa]{RaWi-Ramsey}RationalWiki,
\textit{Ramsey theory},
\url{http://rationalwiki.org/wiki/Ramsey_theory} .

\bibitem[RotSot92]{RotSot92}Alvin E. Roth, Marilda A. Oliveira
Sotomayor, \textit{Two-sided matching: A study in game-theoretic
modeling and analysis}, Cambridge University Press 1992.

\bibitem[Ruohon13]{Ruohon13}Keijo Ruohonen, \textit{Graph theory},
2013.\newline \url{http://math.tut.fi/~ruohonen/GT_English.pdf} .

\bibitem[Stanle12]{Stanley-EC1}Richard P. Stanley, \textit{Enumerative
Combinatorics, Volume 1}, 2nd edition, CUP 2012.\newline See
\url{http://math.mit.edu/~rstan/ec/ec1/} for a preliminary version.

\end{thebibliography}


\end{document}