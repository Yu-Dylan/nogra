\documentclass[numbers=enddot,12pt,final,onecolumn,notitlepage]{scrartcl}%
\usepackage[headsepline,footsepline,manualmark]{scrlayer-scrpage}
\usepackage[all,cmtip]{xy}
\usepackage{amssymb}
\usepackage{amsmath}
\usepackage{amsthm}
\usepackage{framed}
\usepackage{comment}
\usepackage{color}
\usepackage{hyperref}
\usepackage{ifthen}
\usepackage[sc]{mathpazo}
\usepackage[T1]{fontenc}
\usepackage{needspace}
\usepackage{tabls}
%TCIDATA{OutputFilter=latex2.dll}
%TCIDATA{Version=5.50.0.2960}
%TCIDATA{LastRevised=Friday, September 16, 2016 20:39:00}
%TCIDATA{SuppressPackageManagement}
%TCIDATA{<META NAME="GraphicsSave" CONTENT="32">}
%TCIDATA{<META NAME="SaveForMode" CONTENT="1">}
%TCIDATA{BibliographyScheme=Manual}
%TCIDATA{Language=American English}
%BeginMSIPreambleData
\providecommand{\U}[1]{\protect\rule{.1in}{.1in}}
%EndMSIPreambleData
\newcounter{exer}
\theoremstyle{definition}
\newtheorem{theo}{Theorem}[section]
\newenvironment{theorem}[1][]
{\begin{theo}[#1]\begin{leftbar}}
{\end{leftbar}\end{theo}}
\newtheorem{lem}[theo]{Lemma}
\newenvironment{lemma}[1][]
{\begin{lem}[#1]\begin{leftbar}}
{\end{leftbar}\end{lem}}
\newtheorem{prop}[theo]{Proposition}
\newenvironment{proposition}[1][]
{\begin{prop}[#1]\begin{leftbar}}
{\end{leftbar}\end{prop}}
\newtheorem{defi}[theo]{Definition}
\newenvironment{definition}[1][]
{\begin{defi}[#1]\begin{leftbar}}
{\end{leftbar}\end{defi}}
\newtheorem{remk}[theo]{Remark}
\newenvironment{remark}[1][]
{\begin{remk}[#1]\begin{leftbar}}
{\end{leftbar}\end{remk}}
\newtheorem{coro}[theo]{Corollary}
\newenvironment{corollary}[1][]
{\begin{coro}[#1]\begin{leftbar}}
{\end{leftbar}\end{coro}}
\newtheorem{conv}[theo]{Convention}
\newenvironment{condition}[1][]
{\begin{conv}[#1]\begin{leftbar}}
{\end{leftbar}\end{conv}}
\newtheorem{quest}[theo]{Question}
\newenvironment{algorithm}[1][]
{\begin{quest}[#1]\begin{leftbar}}
{\end{leftbar}\end{quest}}
\newtheorem{warn}[theo]{Warning}
\newenvironment{conclusion}[1][]
{\begin{warn}[#1]\begin{leftbar}}
{\end{leftbar}\end{warn}}
\newtheorem{conj}[theo]{Conjecture}
\newenvironment{conjecture}[1][]
{\begin{conj}[#1]\begin{leftbar}}
{\end{leftbar}\end{conj}}
\newtheorem{exam}[theo]{Example}
\newenvironment{example}[1][]
{\begin{exam}[#1]\begin{leftbar}}
{\end{leftbar}\end{exam}}
\newtheorem{exmp}[exer]{Exercise}
\newenvironment{exercise}[1][]
{\begin{exmp}[#1]\begin{leftbar}}
{\end{leftbar}\end{exmp}}
\newenvironment{statement}{\begin{quote}}{\end{quote}}
\iffalse
\newenvironment{proof}[1][Proof]{\noindent\textbf{#1.} }{\ \rule{0.5em}{0.5em}}
\fi
\let\sumnonlimits\sum
\let\prodnonlimits\prod
\let\cupnonlimits\bigcup
\let\capnonlimits\bigcap
\renewcommand{\sum}{\sumnonlimits\limits}
\renewcommand{\prod}{\prodnonlimits\limits}
\renewcommand{\bigcup}{\cupnonlimits\limits}
\renewcommand{\bigcap}{\capnonlimits\limits}
\setlength\tablinesep{3pt}
\setlength\arraylinesep{3pt}
\setlength\extrarulesep{3pt}
\voffset=0cm
\hoffset=-0.7cm
\setlength\textheight{22.5cm}
\setlength\textwidth{15.5cm}
\newenvironment{verlong}{}{}
\newenvironment{vershort}{}{}
\newenvironment{noncompile}{}{}
\excludecomment{verlong}
\includecomment{vershort}
\excludecomment{noncompile}
\newcommand{\id}{\operatorname{id}}
\newcommand{\rev}{\operatorname{rev}}
\newcommand{\conncomp}{\operatorname{conncomp}}
\newcommand{\NN}{\mathbb{N}}
\newcommand{\ZZ}{\mathbb{Z}}
\newcommand{\QQ}{\mathbb{Q}}
\newcommand{\RR}{\mathbb{R}}
\newcommand{\powset}[2][]{\ifthenelse{\equal{#2}{}}{\mathcal{P}\left(#1\right)}{\mathcal{P}_{#1}\left(#2\right)}}
% $\powset[k]{S}$ stands for the set of all $k$-element subsets of
% $S$. The argument $k$ is optional, and if not provided, the result
% is the whole powerset of $S$.
\newcommand{\set}[1]{\left\{ #1 \right\}}
% $\set{...}$ yields $\left\{ ... \right\}$.
\newcommand{\abs}[1]{\left| #1 \right|}
% $\abs{...}$ yields $\left| ... \right|$.
\newcommand{\tup}[1]{\left( #1 \right)}
% $\tup{...}$ yields $\left( ... \right)$.
\newcommand{\ive}[1]{\left[ #1 \right]}
% $\ive{...}$ yields $\left[ ... \right]$.
\newcommand{\verts}[1]{\operatorname{V}\left( #1 \right)}
% $\verts{...}$ yields $\operatorname{V}\left( ... \right)$.
\newcommand{\edges}[1]{\operatorname{E}\left( #1 \right)}
% $\edges{...}$ yields $\operatorname{E}\left( ... \right)$.
\newcommand{\arcs}[1]{\operatorname{A}\left( #1 \right)}
% $\arcs{...}$ yields $\operatorname{A}\left( ... \right)$.
\newcommand{\underbrack}[2]{\underbrace{#1}_{\substack{#2}}}
% $\underbrack{...1}{...2}$ yields
% $\underbrace{...1}_{\substack{...2}}$. This is useful for doing
% local rewriting transformations on mathematical expressions with
% justifications.
\ihead{Math 5707 Spring 2017 (Darij Grinberg): homework set 3}
\ohead{page \thepage}
\cfoot{}
\begin{document}

\begin{center}
\textbf{Math 5707 Spring 2017 (Darij Grinberg): homework set 3}

\textbf{due: Wed, 8 Mar 2017, in class} or by email
(\texttt{dgrinber@umn.edu}) before class

\textbf{Please hand in solutions to FIVE of the 8 problems.}
{\color{red}
(If you hand in more, the grader will choose five to grade.)}
\end{center}

See the
\href{http://www-users.math.umn.edu/~dgrinber/5707s17/nogra.pdf}{lecture notes}
and also the
\href{http://www-users.math.umn.edu/~dgrinber/5707s17/}{handwritten notes}
for relevant material.
Also, some definitions can be found in
\href{http://www-users.math.umn.edu/~dgrinber/5707s17/hw2s.pdf}{the solutions to hw2}.
If you reference results from the lecture notes, please \textbf{mention the date and time} of the version of the notes you are using (as the numbering changes during updates).

If $v$ is a vertex of a simple graph $G = \tup{V, E}$, then the
\textit{eccentricity} of $v$ is defined to be
$\max \set{ d\tup{v, u} \mid u \in V }$ (where $d\tup{v, u}$ is the
distance between $v$ and $u$, as usual). A \textit{center} of a simple
graph $G$ means a vertex whose eccentricity is minimum (among the
eccentricities of all vertices).

\begin{exercise} \label{exe.hw3.centerlp}
Let $T$ be a tree. Let $\tup{v_0, v_1, \ldots, v_k}$ be a longest path
of $T$. Prove that each center of $T$ belongs to this path (i.e., is
one of the vertices $v_0, v_1, \ldots, v_k$).
\end{exercise}

\begin{exercise} \label{exe.hw3.countST}
\textbf{(a)} Consider the cycle graph $C_n$ for some $n \geq 2$. Its
vertices are $1, 2, \ldots, n$, and its edges are $12, 23, \ldots,
\tup{n-1}n, n1$. (Here is how it looks for $n = 5$:
\[
\xymatrix{
& & 1 \ar@{-}[rrd] \\
5 \ar@{-}[rru] \ar@{-}[rd] & & & & 2 \ar@{-}[ld] \\
& 4 \ar@{-}[rr] & & 3
}
\]
)
Find the number of spanning trees of $C_n$.

\textbf{(b)} Consider the directed cycle graph $\overrightarrow{C}_n$
for some $n \geq 2$. It is a digraph; its vertices are
$1, 2, \ldots, n$, and its arcs are $12, 23, \ldots, \tup{n-1}n, n1$.
(Here is how it looks for $n = 5$:
\[
\xymatrix{
& & 1 \ar[rrd] \\
5 \ar[rru] & & & & 2 \ar[ld] \\
& 4 \ar[lu] & & 3 \ar[ll]
}
\]
)
Find the number of oriented spanning trees of $\overrightarrow{C}_n$
with root $1$.

\textbf{(c)} Fix $m \geq 1$. Let $G$ be the simple graph with $3m+2$
vertices
\[
a, b, x_1, x_2, \ldots, x_m, y_1, y_2, \ldots, y_m, z_1, z_2, \ldots,
z_m
\]
and the following $3m+3$ edges:
\begin{align*}
& ax_1, ay_1, az_1, \\
& x_i x_{i+1}, y_i y_{i+1}, z_i z_{i+1} \qquad \text{ for all }
                i \in \set{1, 2, \ldots, m-1}, \\
& x_m b, y_m b, z_m b .
\end{align*}
(Thus, the graph consists of two vertices $a$ and $b$ connected by
three paths, each of length $m+1$, with no overlaps between the paths
except for their starting and ending points. Here is a picture for
$m = 3$:
\[
\xymatrix{
& x_1 \ar@{-}[r] & x_2 \ar@{-}[r] & x_3 \ar@{-}[dr] \\
a \ar@{-}[r] \ar@{-}[ru] \ar@{-}[rd] & y_1 \ar@{-}[r] & y_2 \ar@{-}[r] & y_3 \ar@{-}[r] & b \\
& z_1 \ar@{-}[r] & z_2 \ar@{-}[r] & z_3 \ar@{-}[ru]
}
\]
)
Compute the number of spanning trees of $G$.

[To argue why your number is correct, a sketch of the argument in 1-2
sentences should be enough; a fully rigorous proof is not required.]
\end{exercise}

\begin{exercise} \label{exe.hw3.conn}
If $G$ is a multigraph, then $\operatorname{conn} G$ shall denote the
number of connected components of $G$. (Note that this is $0$ when $G$
has no vertices, and $1$ if $G$ is connected.)

Let $\tup{V, H, \phi}$ be a multigraph. Let $E$ and $F$ be two
subsets of $H$.

\textbf{(a)} Prove that
\begin{align}
& \operatorname{conn} \tup{V, E, \phi\mid_E}
+ \operatorname{conn} \tup{V, F, \phi\mid_F} \nonumber \\
& \leq
\operatorname{conn} \tup{V, E \cup F, \phi\mid_{E \cup F}}
+ \operatorname{conn} \tup{V, E \cap F, \phi\mid_{E \cap F}} .
\label{eq.exe.hw3.conn.ineq}
\end{align}

[Feel free to restrict yourself to the case of a simple graph; in this
case, $E$ and $F$ are two subsets of $\powset[2]{V}$, and you have to
show that
\[
\operatorname{conn} \tup{V, E} + \operatorname{conn} \tup{V, F}
\leq \operatorname{conn} \tup{V, E \cup F}
+ \operatorname{conn} \tup{V, E \cap F} .
\]
This isn't any easier than the general case, but saves you the hassle
of carrying the map $\phi$ around.

Also, feel free to take inspiration from the
\href{http://math.stackexchange.com/questions/500511/dimension-of-the-sum-of-two-vector-subspaces}{proof
of the classical fact that
$\dim X + \dim Y = \dim \tup{X + Y} + \dim \tup{X \cap Y}$ when $X$
and $Y$ are two subspaces of a finite-dimensional vector space $U$}.
That proof relies on choosing a basis of $X \cap Y$ and extending it
to bases of $X$ and $Y$, then merging the extended bases to a basis of
$X + Y$. A ``basis'' of a multigraph $G$ is a spanning forest: a
spanning subgraph that is a forest and has the same number of
connected components as $G$. More precisely, it is the set of the
edges of a spanning forest.]

\textbf{(b)} Give an example where the inequality
\eqref{eq.exe.hw3.conn.ineq} does \textbf{not} become an equality.
\end{exercise}

\begin{exercise} \label{exe.hw3.whamilton}
Let $T$ be a tree having more than $1$ vertex.
Let $L$ be the set of leaves of $T$. Prove that it
is possible to add $\abs{L}-1$ new edges to $T$ in such a way that
the resulting multigraph has a Hamiltonian cycle.
\end{exercise}

If $u$ and $v$ are two vertices of a digraph $G$, then
$d \tup{u, v}$ denotes the \textit{distance} from $u$ to $v$. This
is defined to be the length of the shortest path from $u$ to $v$ if
such a path exists; otherwise it is defined to be the symbol $\infty$.
Notice that $d \tup{u, v}$ is not usually the same as $d \tup{v, u}$
(unlike for simple graphs).

\begin{exercise} \label{exe.hw3.d+d+d.directed}
Let $a$, $b$ and $c$ be three vertices of a strongly connected
digraph $G = \tup{V, A}$ such that $\abs{V} > 4$.

\textbf{(a)} Prove that
$d \tup{b, c} + d \tup{c, a} + d \tup{a, b} \leq 3 \abs{V} - 4$.

\textbf{(b)} For each $n \geq 5$, construct an example in which
$\abs{V} = n$ and
$d \tup{b, c} + d \tup{c, a} + d \tup{a, b} = 3 \abs{V} - 4$.
(No proof required for the example.)
\end{exercise}

Recall that a \textit{$k$-coloring} of a simple graph $G = \tup{V, E}$
means a map $f : V \to \set{1, 2, \ldots, k}$. Such a $k$-coloring $f$
is said to be \textit{proper} if no two adjacent vertices $u$ and $v$
have the same color (i.e., satisfy $f \tup{u} = f \tup{v}$).

\begin{exercise} \label{exe.hw3.not-tripar}
We have learned that a simple graph $G$ (or multigraph $G$)
has a proper $2$-coloring if
and only if all cycles of $G$ have even length.

\textbf{(a)} Is it true that if all cycles of a simple graph $G$ (or
multigraph $G$) have length divisible by $3$, then $G$ has a proper
$3$-coloring?

\textbf{(b)} Is it true that if a simple graph $G$ has a proper
$3$-coloring, then all cycles of $G$ have length divisible by $3$ ?
\end{exercise}

\begin{exercise} \label{exe.hw3.turan}
In class, we have proven the following fact: If $G = \tup{V, E}$ is a
simple graph, then $G$ has an independent set of size
$\geq \dfrac{n}{1+d}$, where $n = \abs{V}$ and
$d = \dfrac{1}{n} \sum_{v \in V} \deg v$. (Notice that $d$ is simply
the average degree of a vertex of $G$.)

Use this to prove Tur\'an's theorem (Theorem 2.5.15 in the
\href{http://www-users.math.umn.edu/~dgrinber/5707s17/nogra.pdf}{lecture notes}).
\end{exercise}

\begin{exercise}
\textbf{Extra credit:}

In this exercise, ``number'' means (e.g.) a real number. (Feel
free to restrict yourself to positive integers  if it helps you.
The most general interpretation would be ``element of a commutative
ring'', but you don't need to work in this generality.)

For any $n$ numbers $x_1, x_2, \ldots, x_n$, we define
$v\left(x_1, x_2, \ldots, x_n\right)$ to be the number
\[
 \prod_{1\leq i<j\leq n} \left( x_j - x_i \right)
 =
 \det \tup{ \tup{ x_j^{i-1} }_{1\leq i\leq n, \  1\leq j\leq n } } .
\]

Let $x_1, x_2, \ldots, x_n$ be $n$ numbers. Let $t$
be a further number.
Prove
at least one of the following facts combinatorially
(i.e., without using any properties of the determinant
other than its definition as a sum over permutations):

\textbf{(a)} We have
\[
\sum_{k=1}^{n} v\left(  x_{1},x_{2},\ldots,x_{k-1},x_{k}+t,x_{k+1}%
,x_{k+2},\ldots,x_{n}\right)
= n v\left(  x_{1},x_{2},\ldots,x_{n}\right)  .
\]

\textbf{(b)} For each $m\in\left\{ 0,1,\ldots,n-1\right\}  $, we have
\[
\sum_{k=1}^{n}x_{k}^{m}v\left(  x_{1},x_{2},\ldots,x_{k-1},t,x_{k+1}%
,x_{k+2},\ldots,x_{n}\right)
= t^{m}v\left(  x_{1},x_{2},\ldots,x_{n}\right)
.
\]

\textbf{(c)} We have
\begin{align*}
&  \sum_{k=1}^{n}x_{k}v\left(  x_{1},x_{2},\ldots,x_{k-1},x_{k}+t,x_{k+1}%
,x_{k+2},\ldots,x_{n}\right) \\
&  =\left(  \dbinom{n}{2}t+\sum_{k=1}^{n}x_{k}\right)  v\left(  x_{1}%
,x_{2},\ldots,x_{n}\right)  .
\end{align*}

\end{exercise}

\end{document}