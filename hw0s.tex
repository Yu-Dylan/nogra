% LaTeX solution template for Math 5707 (Owen Levin, Darij Grinberg)
% ------------------------------------------------------------------

% Like most advanced LaTeX files, this one begins with a lot of
% boilerplate. You don't need to understand (or even read) most of it.
% All you need to do is fill in your name, UMN ID, email address,
% and the number of the pset. (Search for "METADATA" to find the place
% for this.) Then, you can go straight to the "EXERCISE 1"
% section and start writing your solutions.
% The "VARIOUS USEFUL COMMANDS" section is probably worth taking a
% look at at some point.

%----------------------------------------------------------------------------------------
%	PACKAGES AND OTHER DOCUMENT CONFIGURATIONS
%----------------------------------------------------------------------------------------
\documentclass[paper=a4, fontsize=12pt]{scrartcl} % A4 paper and 12pt font size
\usepackage[T1]{fontenc} % Use 8-bit encoding that has 256 glyphs
\usepackage[english]{babel} % English language/hyphenation
\usepackage{amsmath,amsfonts,amsthm,amssymb} % Math packages
\usepackage{mathrsfs}    % More math packages
\usepackage{sectsty}  % Allows customizing section commands
\allsectionsfont{\centering \normalfont\scshape} % Make all section titles centered, the default font and small caps %remove this to left align section tites
\usepackage{hyperref} % Turns cross-references into hyperlinks,
                      % and defines \url and \href commands.
\usepackage{graphicx} % For embedding graphics files.
\usepackage{framed}   % For the "leftbar" environment used below.
\usepackage{ifthen}   % Used for the \powset command below.
\usepackage{lastpage} % for counting the number of pages
\usepackage[headsepline,footsepline,manualmark]{scrlayer-scrpage}
\usepackage[height=10in,a4paper,hmargin={1in,0.8in}]{geometry}
\usepackage[usenames,dvipsnames]{xcolor}
\usepackage{tikz}     % This is a powerful tool to draw vector
                      % graphics inside LaTeX. In particular, you can
                      % use it to draw graphs.
\usepackage{verbatim} % For the "verbatim" environment, in which
                      % special symbols can be used freely without
                      % confusing the compiler. (And it's typeset in
                      % a constant-width font.)
                      % Useful, e.g., for quoting code (or ASCII art).

%\numberwithin{table}{section} % Number tables within sections (i.e. 1.1, 1.2, 2.1, 2.2 instead of 1, 2, 3, 4)

\setlength\parindent{20pt} % Makes indentation for paragraphs longer.
                           % This makes paragraphs stand out more.

%----------------------------------------------------------------------------------------
%	VARIOUS USEFUL COMMANDS
%----------------------------------------------------------------------------------------
% The commands below might be convenient. For example, you probably
% prefer to write $\powset[2]{V}$ for the set of $2$-element subsets
% of $V$, rather than writing $\mathcal{P}_2(V)$.
% Notice that you can easily define your own commands like this.
% Caveat: Some of these commands need to be properly "guarded" when
% they occur in subscripts or superscripts. So you should not write
% $K_\CC$, but rather $K_{\CC}$.
\newcommand{\CC}{\mathbb{C}} % complex numbers
\newcommand{\RR}{\mathbb{R}} % real numbers
\newcommand{\QQ}{\mathbb{Q}} % rational numbers
\newcommand{\NN}{\mathbb{N}} % nonnegative integers
\newcommand{\Z}[1]{\mathbb{Z}/#1\mathbb{Z}} % integers modulo k
                                            % (syntax: "\Z{k}")
\newcommand{\ZZ}{\mathbb{Z}} % integers
\newcommand{\id}{\operatorname{id}} % identity map
\newcommand{\lcm}{\operatorname{lcm}}
% Lowest common multiple. For historical reasons, LaTeX has a \gcd
% command built in, but not an \lcm command. The preceding line
% rectifies that.
\newcommand{\rev}{\operatorname{rev}} % reversal of a walk
\newcommand{\powset}[2][]{\ifthenelse{\equal{#2}{}}{\mathcal{P}\left(#1\right)}{\mathcal{P}_{#1}\left(#2\right)}}
% $\powset[k]{S}$ stands for the set of all $k$-element subsets of
% $S$. The argument $k$ is optional, and if not provided, the result
% is the whole powerset of $S$.
\newcommand{\set}[1]{\left\{ #1 \right\}}
% $\set{...}$ compiles to {...} (set-brackets).
\newcommand{\abs}[1]{\left| #1 \right|}
% $\abs{...}$ compiles to |...| (absolute value, or size of a set).
\newcommand{\tup}[1]{\left( #1 \right)}
% $\tup{...}$ compiles to (...) (parentheses, or tuple-brackets).
\newcommand{\ive}[1]{\left[ #1 \right]}
% $\ive{...}$ compiles to [...] (Iverson bracket, aka truth value).
\newcommand{\verts}[1]{\operatorname{V}\left( #1 \right)}
% $\verts{...}$ compiles to V(...) (vertex set of a graph/digraph).
\newcommand{\edges}[1]{\operatorname{E}\left( #1 \right)}
% $\edges{...}$ compiles to E(...) (edge set of a graph).
\newcommand{\arcs}[1]{\operatorname{A}\left( #1 \right)}
% $\arcs{...}$ compiles to A(...) (arc set of a digraph).
\newcommand{\underbrack}[2]{\underbrace{#1}_{\substack{#2}}}
% $\underbrack{...1}{...2}$ yields
% $\underbrace{...1}_{\substack{...2}}$. This is useful for doing
% local rewriting transformations on mathematical expressions with
% justifications. For example, try this out:
% $ \underbrack{(a+b)^2}{= a^2 + 2ab + b^2 \\ \text{(by the binomial formula)}} $
\newcommand{\horrule}[1]{\rule{\linewidth}{#1}} % Create horizontal rule command with 1 argument of height

%----------------------------------------------------------------------------------------
%	MAKING SUMMATION SIGNS ALWAYS PUT THEIR BOUNDS ABOVE AND BELOW
%	THE SIGN
%----------------------------------------------------------------------------------------
% The following are hacks to ensure that sums (such as
% $\sum_{k=1}^n k$) always put their bounds (i.e., the $k=1$ and the
% $n$) underneath and above the sign, as opposed to on its right.
% Same for products (\prod), set unions (\bigcup) and set
% intersections (\bigcap). Remove the 8 lines below if you do not want
% this behavior.
\let\sumnonlimits\sum
\let\prodnonlimits\prod
\let\cupnonlimits\bigcup
\let\capnonlimits\bigcap
\renewcommand{\sum}{\sumnonlimits\limits}
\renewcommand{\prod}{\prodnonlimits\limits}
\renewcommand{\bigcup}{\cupnonlimits\limits}
\renewcommand{\bigcap}{\capnonlimits\limits}

%----------------------------------------------------------------------------------------
%	ENVIRONMENTS
%----------------------------------------------------------------------------------------
% The incantations below define how theorem environments
% (\begin{theorem} ... \end{theorem}) and their likes will look like.
\newtheoremstyle{plainsl}% <name>
  {8pt plus 2pt minus 4pt}% <Space above>
  {8pt plus 2pt minus 4pt}% <Space below>
  {\slshape}% <Body font>
  {0pt}% <Indent amount>
  {\bfseries}% <Theorem head font>
  {.}% <Punctuation after theorem head>
  {5pt plus 1pt minus 1pt}% <Space after theorem headi>
  {}% <Theorem head spec (can be left empty, meaning `normal')>

% Environments which make the text inside them slanted:
\theoremstyle{plainsl}
  \newtheorem{theorem}{Theorem}[section]
  \newtheorem{proposition}[theorem]{Proposition}
  \newtheorem{lemma}[theorem]{Lemma}
  \newtheorem{corollary}[theorem]{Corollary}
  \newtheorem{conjecture}[theorem]{Conjecture}
% Environments that don't:
\theoremstyle{definition}
  \newtheorem{definition}[theorem]{Definition}
  \newtheorem{example}[theorem]{Example}
  \newtheorem{exercise}[theorem]{Exercise}
  \newtheorem{examples}[theorem]{Examples}
  \newtheorem{algorithm}[theorem]{Algorithm}
  \newtheorem{question}[theorem]{Question}
 \theoremstyle{remark}
  \newtheorem{remark}[theorem]{Remark}
\newenvironment{statement}{\begin{quote}}{\end{quote}}

%----------------------------------------------------------------------------------------
%	METADATA
%----------------------------------------------------------------------------------------
\newcommand{\myname}{Darij Grinberg} % ENTER YOUR NAME HERE
\newcommand{\myid}{00000000} % ENTER YOUR UMN ID HERE
\newcommand{\mymail}{dgrinber@umn.edu} % ENTER YOUR EMAIL HERE
\newcommand{\psetnumber}{0} % ENTER THE NUMBER OF THIS PSET HERE

%----------------------------------------------------------------------------------------
%	HEADER AND FOOTER
%----------------------------------------------------------------------------------------
\ihead{Solutions to homework set \#\psetnumber} % Page header left
\ohead{page \thepage\ of \pageref{LastPage}} % Page header right
\ifoot{\myname, \myid} % left footer
\ofoot{\mymail} % right footer

%----------------------------------------------------------------------------------------
%	TITLE SECTION
%----------------------------------------------------------------------------------------
\title{	
\normalfont \normalsize 
\textsc{University of Minnesota, School of Mathematics} \\ [25pt] % Your university, school and/or department name(s)
\horrule{0.5pt} \\[0.4cm] % Thin top horizontal rule
\huge Math 5707: Graph Theory, Spring 2017\\
Homework \psetnumber\\% The assignment title
\horrule{2pt} \\[0.5cm] % Thick bottom horizontal rule
}
\author{\myname}

\begin{document}

\maketitle % Print the title

%----------------------------------------------------------------------------------------
%	EXERCISE 1
%----------------------------------------------------------------------------------------
\horrule{0.3pt} \\[0.4cm]

\section{Exercise 1}

\subsection{Problem}

Let $G = \tup{V, E}$ be a simple graph.
% The \tup{...} command puts the ... in parentheses.
If $k \in \NN$,
% \NN is an abbreviation for \mathbb{N}, and stands for the
% nonnegative integers.
then a \textit{$k$-coloring} of $G$ means a map
$f : V \to \set{1, 2, \ldots, k}$.
% \ldots is the neatest way to get a "..." in LaTeX. But no one will
% mind if you just write "...".
If $f$ is a $k$-coloring of $G$ for some $k \in \NN$, then the value
$f \tup{v}$ of $f$ at a vertex $v \in V$ is called the
\textit{color} of $v$ in the $k$-coloring $f$.
(It is customary to visualize a $k$-coloring by pretending that the
numbers $1, 2, \ldots, k$ are colors, and so a $k$-coloring assigns to
each vertex a color; i.e., it ``colors'' the vertices.)

Prove that there exists a $2$-coloring of $G$ having the following
property: For each vertex $v \in V$, at most $\dfrac{1}{2} \deg v$
among the neighbors of $v$ have the same color as $v$.

\begin{remark} \label{rmk.exe1.r}
This problem is often restated as follows: You are given a (finite)
set of politicians; some politicians are mutual enemies. (No one is
their own enemy. If $u$ is an enemy of $v$, then $v$ is an enemy of
$u$. An enemy of an enemy is not necessarily a friend. So this is just
a simple graph.) Prove that it is possible to subdivide this set into
two (disjoint) parties such that no politician has more than half of
his enemies in his own party.
\end{remark}

\subsection{Solution}
% The solution below is written in more detail than necessary, to
% illustrate both the use of LaTeX and the concept of an algorithmic
% proof.
The idea of the solution is fairly simple, particularly using the
convenient language of Remark \ref{rmk.exe1.r}:
% The \ref command provides a cross-reference to a theorem,
% remark, definition, section or whatever. Of course, you need to
% specify which theorem etc. you want to reference; this is achieved
% by defining a label (in our case, "rmk.exe1.r"), which is done using
% the \label command.
% Note that you need to compile a LaTeX file *twice* in order for the
% cross-references to appear! During the first compilation, LaTeX
% collects all the labels and stores them in an .aux file; during the
% second compilation, it then actually writes the cross-references.
% Hence, you also need to recompile a LaTeX file twice if you have
% added new theorems/definitions/whatever (or removed some), because
% during the first compilation the .aux file is still unaware of the
% new theorems and so the numbers will be out-of-sync. 
We subdivide our set of
politicians into two parties in some arbitrary way (e.g., by throwing
them all into one party and keeping the other party empty), and then
we improve the situation step by step by picking a politician who has
more than half his enemies in his own party, and moving him to the
opposite party. However, will this procedure necessarily terminate, or
will it result in an eternal process of politicians being kicked
around back and forth between the two parties? Fortunately, it will
terminate, but this needs to be proven.

Let us first make this procedure rigorous:

\begin{algorithm} \label{alg.exe1.1}
\begin{leftbar}
% The "leftbar" environment places a vertical bar on the left of
% whatever is inside it. I find it quite convenient here to clarify
% where this algorithm ends. But handle with care: For example,
% footnotes don't work if you are in a leftbar environment...

\textbf{Input:} a simple graph $G = \tup{V, E}$.

\textbf{Output:} a $2$-coloring of $G$ having the following
property: For each vertex $v \in V$, at most $\dfrac{1}{2} \deg v$
among the neighbors of $v$ have the same color as $v$.

\begin{enumerate}
% The "enumerate" environment creates a numbered list. If you want
% custom numbers (as opposed to just 1, 2, 3, ...), you can use
% \item[X] instead of plain \item.
\item Define a $2$-coloring $f : V \to \set{1, 2}$ arbitrarily (for
      example, by setting $f \tup{v} = 1$ for all $v \in V$).
\item \textbf{While} there exists some $v \in V$ such that
      more than $\dfrac{1}{2} \deg v$ among the neighbors of $v$ have
      the same color as $v$, \textbf{do} the following:
      \begin{itemize}
      % The "itemize" environment creates a bullet-point (unnumbered)
      % list. You can nest such environments inside them.
      \item Flip the color $f \tup{v}$ of this $v$ (that is, change
            this color to $2$ if it is $1$, and change it to $1$ if it
            is $2$).
      \end{itemize}
\item Output the $2$-coloring $f$.
\end{enumerate}
\end{leftbar}
\end{algorithm}

It is clear that if step 3 of this algorithm is ever reached, then the
$2$-coloring $f$ outputted by the algorithm
does have the property that for each vertex
$v \in V$, at most $\dfrac{1}{2} \deg v$ among the neighbors of $v$
have the same color as $v$. (Indeed, this property is saying
precisely that we have fallen out of the while-loop in step 2.) But
we need to show that step 3 will eventually be reached. This is not
obvious, since step 2 contains a while-loop, and a while-loop may go
on indefinitely. We need to prove that this while-loop must eventually
come to an end.

The most common way to prove such a claim is by exhibiting a
\textit{loop monovariant}. In our situation, this means a nonnegative
integer defined for each $2$-coloring $f$ and which has the property
that in every iteration of the while-loop, this integer decreases
(strictly). If we can define such an integer, then we can immediately
conclude that the while-loop must eventually come to an end (because
a nonnegative integer cannot keep decreasing indefinitely while
remaining a nonnegative integer).

The role of this nonnegative integer (the loop monovariant) will be
played by what I call the ``enmity'' of a $2$-coloring $f$:

If $f$ is a $2$-coloring of $G$, then I define the \textit{enmity} of
$f$ to be the number of $f$-monochromatic edges of $G$. Here, an edge
$e$ of $G$ is said to be \textit{$f$-monochromatic} if the two
endpoints of $e$ have the same color in the $2$-coloring $f$.

For each $2$-coloring $f$ of $G$, the enmity of $f$ is clearly a
nonnegative integer. Now, I claim the following:
\begin{statement}
\textit{Claim 1:} Let $f$ be a $2$-coloring of $G$. Let $v \in V$ be
a vertex such that more than $\dfrac{1}{2} \deg v$ among the neighbors
of $v$ have the same color as $v$. If we flip the color $f \tup{v}$ of
$v$, then the enmity of $f$ \textbf{decreases}.
\end{statement}

\begin{proof}[Proof of Claim 1.]
% The "proof" environment automatically puts a square at the end of
% the proof.
Let $a$ be the number of neighbors of $v$ that have the same color as
$v$. Then, $a > \dfrac{1}{2} \deg v$ (since more than
$\dfrac{1}{2} \deg v$ among the neighbors of $v$ have the same color
as $v$). Hence, $2a > \deg v$, so that $a > \deg v - a$.

Clearly, an edge $e$ containing $v$ is monochromatic\footnote{We
abbreviate the word ``$f$-monochromatic'' as ``monochromatic''.}
if and only if the neighbor of $v$ contained in $e$ has the same color
as $v$.
Thus, the monochromatic edges containing $v$ are in one-to-one
correspondence with the neighbors of $v$ that have the same color as
$v$. Since the number of the latter neighbors is $a$, we thus conclude
that the number of the former edges is also $a$. In other words, among
the $\deg v$ edges containing $v$, exactly $a$ are monochromatic.
If we flip the color $f \tup{v}$ of $v$, then these $a$ monochromatic
edges become no longer monochromatic (because they now connect
vertices of different color), whereas the remaining $\deg v - a$ edges
containing $v$ become monochromatic. As for the edges that do not
contain $v$, their status does not change (i.e., if they are
monochromatic before the flipping, then they remain so, and
if they are not monochromatic before the flipping, then they do not
become monochromatic), because none of their vertices changes its
color.
Hence, by flipping the color $f \tup{v}$ of $v$, we lose $a$
monochromatic edges but, at the same time, gain $\deg v - a$ new
monochromatic edges. Since $a > \deg v - a$, we thus lose
more monochromatic edges than we gain. Therefore, the number of
monochromatic edges of $G$ decreases. In other words, the enmity
of $f$ decreases (since the enmity of $f$ is defined as the number
of monochromatic edges of $G$). This proves Claim 1.
\end{proof}

Claim 1 shows that the enmity of $f$ decreases in each iteration of
the while-loop in Algorithm \ref{alg.exe1.1}. Hence, this while-loop
cannot go on indefinitely (since the enmity of $f$ is a nonnegative
integer, and thus cannot keep decreasing forever). Thus,
Algorithm \ref{alg.exe1.1} must eventually terminate (since the
while-loop in step 2 is the only part in which the algorithm might
get stuck). As we already
know, the $2$-coloring $f$ outputted by the algorithm
does have the property that for each vertex
$v \in V$, at most $\dfrac{1}{2} \deg v$ among the neighbors of $v$
have the same color as $v$. As a consequence, such a $2$-coloring
exists. This solves the exercise.

\subsection{Postscriptum}

\begin{remark}
In
\href{http://www.cip.ifi.lmu.de/~grinberg/t/17s/nogra.pdf}{the lecture notes},
% This is an example of a hyperlink created by the \href command.
% The general syntax is \href{URL}{text}. If you want just the URL
% itself to appear, you can also use \url{URL}.
I have stated the fact that
if $G = \tup{V, E}$ is a simple graph that has no isolated vertices,
then there exist two disjoint dominating subsets $A$ and $B$
of $V$ such that $A \cup B = V$. Do you see why this follows from
the above exercise?
\end{remark}

%Copy or imitate the above for each exercise you solve.

\end{document}