\documentclass[numbers=enddot,12pt,final,onecolumn,notitlepage]{scrartcl}%
\usepackage[headsepline,footsepline,manualmark]{scrlayer-scrpage}
\usepackage[all,cmtip]{xy}
\usepackage{amssymb}
\usepackage{amsmath}
\usepackage{amsthm}
\usepackage{framed}
\usepackage{comment}
\usepackage{color}
\usepackage{hyperref}
\usepackage{ifthen}
\usepackage[sc]{mathpazo}
\usepackage[T1]{fontenc}
\usepackage{needspace}
\usepackage{tabls}
%TCIDATA{OutputFilter=latex2.dll}
%TCIDATA{Version=5.50.0.2960}
%TCIDATA{LastRevised=Friday, September 16, 2016 20:39:00}
%TCIDATA{SuppressPackageManagement}
%TCIDATA{<META NAME="GraphicsSave" CONTENT="32">}
%TCIDATA{<META NAME="SaveForMode" CONTENT="1">}
%TCIDATA{BibliographyScheme=Manual}
%TCIDATA{Language=American English}
%BeginMSIPreambleData
\providecommand{\U}[1]{\protect\rule{.1in}{.1in}}
%EndMSIPreambleData
\newcounter{exer}
\theoremstyle{definition}
\newtheorem{theo}{Theorem}[section]
\newenvironment{theorem}[1][]
{\begin{theo}[#1]\begin{leftbar}}
{\end{leftbar}\end{theo}}
\newtheorem{lem}[theo]{Lemma}
\newenvironment{lemma}[1][]
{\begin{lem}[#1]\begin{leftbar}}
{\end{leftbar}\end{lem}}
\newtheorem{prop}[theo]{Proposition}
\newenvironment{proposition}[1][]
{\begin{prop}[#1]\begin{leftbar}}
{\end{leftbar}\end{prop}}
\newtheorem{defi}[theo]{Definition}
\newenvironment{definition}[1][]
{\begin{defi}[#1]\begin{leftbar}}
{\end{leftbar}\end{defi}}
\newtheorem{remk}[theo]{Remark}
\newenvironment{remark}[1][]
{\begin{remk}[#1]\begin{leftbar}}
{\end{leftbar}\end{remk}}
\newtheorem{coro}[theo]{Corollary}
\newenvironment{corollary}[1][]
{\begin{coro}[#1]\begin{leftbar}}
{\end{leftbar}\end{coro}}
\newtheorem{conv}[theo]{Convention}
\newenvironment{condition}[1][]
{\begin{conv}[#1]\begin{leftbar}}
{\end{leftbar}\end{conv}}
\newtheorem{quest}[theo]{Question}
\newenvironment{algorithm}[1][]
{\begin{quest}[#1]\begin{leftbar}}
{\end{leftbar}\end{quest}}
\newtheorem{warn}[theo]{Warning}
\newenvironment{conclusion}[1][]
{\begin{warn}[#1]\begin{leftbar}}
{\end{leftbar}\end{warn}}
\newtheorem{conj}[theo]{Conjecture}
\newenvironment{conjecture}[1][]
{\begin{conj}[#1]\begin{leftbar}}
{\end{leftbar}\end{conj}}
\newtheorem{exam}[theo]{Example}
\newenvironment{example}[1][]
{\begin{exam}[#1]\begin{leftbar}}
{\end{leftbar}\end{exam}}
\newtheorem{exmp}[exer]{Exercise}
\newenvironment{exercise}[1][]
{\begin{exmp}[#1]\begin{leftbar}}
{\end{leftbar}\end{exmp}}
\newenvironment{statement}{\begin{quote}}{\end{quote}}
\iffalse
\newenvironment{proof}[1][Proof]{\noindent\textbf{#1.} }{\ \rule{0.5em}{0.5em}}
\fi
\let\sumnonlimits\sum
\let\prodnonlimits\prod
\let\cupnonlimits\bigcup
\let\capnonlimits\bigcap
\renewcommand{\sum}{\sumnonlimits\limits}
\renewcommand{\prod}{\prodnonlimits\limits}
\renewcommand{\bigcup}{\cupnonlimits\limits}
\renewcommand{\bigcap}{\capnonlimits\limits}
\setlength\tablinesep{3pt}
\setlength\arraylinesep{3pt}
\setlength\extrarulesep{3pt}
\voffset=0cm
\hoffset=-0.7cm
\setlength\textheight{22.5cm}
\setlength\textwidth{15.5cm}
\newenvironment{verlong}{}{}
\newenvironment{vershort}{}{}
\newenvironment{noncompile}{}{}
\excludecomment{verlong}
\includecomment{vershort}
\excludecomment{noncompile}
\newcommand{\id}{\operatorname{id}}
\newcommand{\rev}{\operatorname{rev}}
\newcommand{\conncomp}{\operatorname{conncomp}}
\newcommand{\NN}{\mathbb{N}}
\newcommand{\ZZ}{\mathbb{Z}}
\newcommand{\QQ}{\mathbb{Q}}
\newcommand{\RR}{\mathbb{R}}
\newcommand{\powset}[2][]{\ifthenelse{\equal{#2}{}}{\mathcal{P}\left(#1\right)}{\mathcal{P}_{#1}\left(#2\right)}}
% $\powset[k]{S}$ stands for the set of all $k$-element subsets of
% $S$. The argument $k$ is optional, and if not provided, the result
% is the whole powerset of $S$.
\newcommand{\set}[1]{\left\{ #1 \right\}}
% $\set{...}$ yields $\left\{ ... \right\}$.
\newcommand{\abs}[1]{\left| #1 \right|}
% $\abs{...}$ yields $\left| ... \right|$.
\newcommand{\tup}[1]{\left( #1 \right)}
% $\tup{...}$ yields $\left( ... \right)$.
\newcommand{\ive}[1]{\left[ #1 \right]}
% $\ive{...}$ yields $\left[ ... \right]$.
\newcommand{\floor}[1]{\left\lfloor #1 \right\rfloor}
% $\floor{...}$ yields $\left\lfloor ... \right\rfloor$.
\newcommand{\verts}[1]{\operatorname{V}\left( #1 \right)}
% $\verts{...}$ yields $\operatorname{V}\left( ... \right)$.
\newcommand{\edges}[1]{\operatorname{E}\left( #1 \right)}
% $\edges{...}$ yields $\operatorname{E}\left( ... \right)$.
\newcommand{\arcs}[1]{\operatorname{A}\left( #1 \right)}
% $\arcs{...}$ yields $\operatorname{A}\left( ... \right)$.
\newcommand{\underbrack}[2]{\underbrace{#1}_{\substack{#2}}}
% $\underbrack{...1}{...2}$ yields
% $\underbrace{...1}_{\substack{...2}}$. This is useful for doing
% local rewriting transformations on mathematical expressions with
% justifications.
\ihead{Math 5707 Spring 2017 (Darij Grinberg): midterm 1}
\ohead{page \thepage}
\cfoot{}
\begin{document}

\begin{center}
\textbf{Math 5707 Spring 2017 (Darij Grinberg): midterm 1}

\textbf{Solution sketches (DRAFT).}
\end{center}

\tableofcontents

\subsection{Reminders}

See the
\href{http://www-users.math.umn.edu/~dgrinber/5707s17/nogra.pdf}{lecture notes}
and also the
\href{http://www-users.math.umn.edu/~dgrinber/5707s17/}{handwritten notes}
for relevant material.
See also
\href{http://www-users.math.umn.edu/~dgrinber/5707s17/hw2s.pdf}{the solutions to homework set 2}
for various conventions and notations that are in use here.

\subsection{Exercise \ref{exe.mt1.from-dom}: a from-dominating set
from a Hamiltonian path}

\begin{exercise} \label{exe.mt1.from-dom}
Let $D = \tup{V, A}$ be a digraph. A \textit{from-dominating set} of
$D$ shall mean a subset $S$ of $V$ such that for each vertex $v \in
V \setminus S$, there exists at least one arc $uv \in A$ with
$u \in S$.

Assume that $D$ has a Hamiltonian path. Prove that $D$ has a
from-dominating set of size $\leq \dfrac{\abs{V}+1}{2}$.
\end{exercise}

\begin{proof}[Solution sketch to Exercise~\ref{exe.mt1.from-dom}.]
We have assumed that $D$ has a Hamiltonian path. Fix such a path, and
denote it by $\tup{v_1, v_2, \ldots, v_n}$.

Each vertex of $D$ appears exactly once in the path
$\tup{v_1, v_2, \ldots, v_n}$ (since $\tup{v_1, v_2, \ldots, v_n}$ is
a Hamiltonian path). In other words, each element of $V$ appears
exactly once in the path $\tup{v_1, v_2, \ldots, v_n}$. Hence,
$\abs{V} = n$ and $V = \set{v_1, v_2, \ldots, v_n}$.

Define a subset $S$ of $V$ by
\[
S = \set{v_i \ \mid \  i \in \set{1,2,\ldots,n} \text{ is odd}}
  = \set{v_1, v_3, v_5, \ldots}.
\]
Then,\footnote{Here, we are using the following notation: If $x$ is a
real number, then $\floor{x}$ denotes the largest integer that is
smaller or equal to $x$. For example, $\floor{2.8} = 2$ and
$\floor{3} = 3$ and $\floor{-1.6} = -2$.}
$\abs{S} = \floor{\dfrac{n+1}{2}} \leq \dfrac{n+1}{2}
= \dfrac{\abs{V}+1}{2}$ (since $n = \abs{V}$). Hence, the set $S$ has
size $\leq \dfrac{\abs{V}+1}{2}$. It thus merely remains to prove that
$S$ is from-dominating.

According to the definition of ``from-dominating'', this means proving
that for each vertex $v \in V \setminus S$, there exists at least one
arc $uv \in A$ with $u \in S$. So let us prove this now.

Let $v \in V \setminus S$. Thus,
\begin{align*}
v &\in V \setminus S
= \set{v_1, v_2, \ldots, v_n} \setminus
     \set{v_1, v_3, v_5, \ldots} \\
&\qquad
\left( \text{since } V = \set{v_1, v_2, \ldots, v_n} \text{ and }
        S = \set{v_1, v_3, v_5, \ldots} \right) \\
&= \set{v_2, v_4, v_6, \ldots}
= \set{v_i \ \mid \  i \in \set{1,2,\ldots,n} \text{ is even}} .
\end{align*}
In other words, $v = v_i$ for some even $i \in \set{1,2,\ldots,n}$.
Consider this $i$. The integer $i$ is even and positive; hence, the
integer $i-1$ is odd and positive. Hence,
$i-1 \in \set{1,2,\ldots,n-1}$ (since $i-1 < i \leq n$), so that
$i-1 \in \set{1, 2, \ldots, n}$.
Thus, $v_{i-1} \in \set{v_1, v_3, v_5, \ldots}$ (since $i-1$ is odd).
Hence, $v_{i-1} \in \set{v_1, v_3, v_5, \ldots} = S$.

Furthermore, recall that $\tup{v_1, v_2, \ldots, v_n}$ is a path in
$D$. Thus, $v_j v_{j+1}$ is an arc of $D$ for each
$j \in \set{1,2,\ldots,n-1}$. Applying this to $j=i-1$, we conclude
that $v_{i-1} v_i$ is an arc of $D$. In other words,
$v_{i-1} v_i \in A$. Since $v_i = v$, this rewrites as
$v_{i-1} v \in A$. Hence, there exists at least one arc $uv \in A$
with $u \in S$ (namely, the arc $v_{i-1} v$ with $u = v_{i-1}$).
This is exactly what we wanted to prove. Hence, we have shown that $S$
is from-dominating.
\end{proof}

\subsection{Exercise \ref{exe.mt1.L-hamil}: Hamiltonian paths of a
line graph}

\begin{exercise} \label{exe.mt1.L-hamil}
Let $G = \tup{V, E}$ be a simple graph. The \textit{line graph}
$L \tup{G}$ is defined as the simple graph $\tup{E, F}$, where
\[
F = \set{ \set{e_1, e_2} \in \powset[2]{E}
            \ \mid \ e_1 \cap e_2 \neq \varnothing } .
\]
(In other words, $L \tup{G}$ is the graph whose \textbf{vertices}
are the \textbf{edges} of $G$, and in which two vertices $e_1$ and
$e_2$ are adjacent if and only if the edges $e_1$ and $e_2$ of $G$
share a common vertex.)

Assume that $\abs{V} > 1$.

\textbf{(a)} If $G$ has a Hamiltonian path, then prove that
$L \tup{G}$ has a Hamiltonian path.

\textbf{(b)} If $G$ has a Eulerian walk, then prove that $L \tup{G}$
has a Hamiltonian path.
\end{exercise}

\begin{proof}[Hints to Exercise~\ref{exe.mt1.L-hamil}.]
\textbf{(a)} Let $\tup{v_0, v_1, \ldots, v_n}$ be a Hamiltonian path
in $G$.
Let $e_1, e_2, \ldots, e_n$ be the edges of this path (so that
$e_i = v_{i-1} v_i$ for each $i$). Then,
$\mathbf{p} = \tup{e_1, e_2, \ldots, e_n}$ is a path in $L \tup{G}$.
This path is
not necessarily a Hamiltonian path; but we can turn it into a
Hamiltonian path by the following procedure:
\begin{itemize}
\item Insert all edges of $G$ that contain $v_0$ and are not already
in the path $\mathbf{p}$ into $\mathbf{p}$, placing them
at the beginning of $\mathbf{p}$ (just before $e_1$).
\item Insert all edges of $G$ that contain $v_1$ and are not already
in the path $\mathbf{p}$ into $\mathbf{p}$, placing them
between $e_1$ and $e_2$.
\item Insert all edges of $G$ that contain $v_2$ and are not already
in the path $\mathbf{p}$ into $\mathbf{p}$, placing them
between $e_2$ and $e_3$.
\item And so on, until all edges of $G$ have been inserted.
\end{itemize}
It is easy to check that the result is a Hamiltonian path in
$L \tup{G}$.

\textbf{(b)} Let $e_1, e_2, \ldots, e_m$ be the edges of an Eulerian
walk in $G$. Then, $\tup{e_1, e_2, \ldots, e_m}$ is a Hamiltonian
path in $L \tup{G}$.

[\textit{Remark:} It is also true that if $G$ has an Eulerian circuit,
then $L \tup{G}$ has an Eulerian circuit. To prove this, show that
$L \tup{G}$ is connected and that each vertex of $L \tup{G}$ has even
degree.]
\end{proof}

\subsection{Exercise \ref{exe.mt1.deg-cycle-dir}: a condition for a
digraph to have a cycle}

\begin{exercise} \label{exe.mt1.deg-cycle-dir}
Let $D = \tup{V, A}$ be a digraph with $\abs{V} > 0$. Assume
that each vertex $v \in V$ satisfies $\deg^- v > 0$. Prove that
$D$ has at least one cycle.

(Keep in mind that a length-1 circuit $\tup{v, v}$ counts as a cycle
when $A$ contains the loop $\tup{v, v}$.)
\end{exercise}

\begin{proof}[Hints to Exercise~\ref{exe.mt1.deg-cycle-dir}.]
Fix a longest path $\tup{v_0, v_1, \ldots, v_k}$ in $D$.
There is at least one arc with target
$v_0$ (since $\deg^- \tup{v_0} > 0$). Let $v_{-1}$ be the source of
this arc. Then, $\tup{v_{-1}, v_0, v_1, \ldots, v_k}$ is a walk, but
not a path (since $\tup{v_0, v_1, \ldots, v_k}$ is a longest path).
Hence, $v_{-1} = v_i$ for some $i \in \set{0, 1, \ldots, k}$.
Fix the \textbf{smallest} such $i$.
Then, $\tup{v_{-1}, v_0, \ldots, v_i}$ is a cycle.

[\textit{Remark:} This is very similar to Lemma 0.2 in
\href{http://www-users.math.umn.edu/~dgrinber/5707s17/hw2s.pdf}{the solutions to homework set 2}.]
\end{proof}

\subsection{Exercise \ref{exe.mt1.deg-cycle-2}: a condition for a
multigraph to have a cycle}

Recall that the degree $\deg v$ of a vertex $v$ of a multigraph $G$
is defined as the number of all edges of $G$ containing $v$.

\begin{exercise} \label{exe.mt1.deg-cycle-2}
Let $G$ be a multigraph with at least one edge.
Assume that each vertex of $G$ has even degree.
Prove that $G$ has a cycle.
\end{exercise}

\begin{proof}[Hints to Exercise~\ref{exe.mt1.deg-cycle-2}.]
Fix a longest path
$\tup{v_0, e_1, v_1, e_2, v_2, \ldots, e_k, v_k}$ in $G$.
There is at least one edge containing $v_0$ (namely, $e_1$).
Thus, the number $\deg \tup{v_0}$ is positive.
This number $\deg \tup{v_0}$ is furthermore even (since each
vertex of $G$ has even degree).
Hence, this number $\deg \tup{v_0}$ is $\geq 2$
(because it is even and positive).
In other words, there are at least two edges containing $v_0$.
Thus, there is at least one edge containing $v_0$ that is distinct
from $e_1$.
Denote this edge by $e_0$, and let $v_{-1}$ be its other endpoint.
Then, $\tup{v_{-1}, e_0, v_0, e_1, v_1, e_2, v_2, \ldots, e_k, v_k}$
is a walk, but not a path (since
$\tup{v_0, e_1, v_1, e_2, v_2, \ldots, e_k, v_k}$ is a longest path).
Hence, $v_{-1} = v_i$ for some $i \in \set{0, 1, \ldots, k}$.
Fix the \textbf{smallest} such $i$.
Then,
$\tup{v_{-1}, e_0, v_0, e_1, v_1, \ldots, e_i, v_i}$ is a cycle.

[\textit{Remark:} This is (so to speak) the undirected version of
Lemma 0.2 in
\href{http://www-users.math.umn.edu/~dgrinber/5707s17/hw2s.pdf}{the solutions to homework set 2}.]
\end{proof}

\subsection{Exercise \ref{exe.mt1.enmity}: a coloring where neighbors
shun equal colors}

\begin{exercise} \label{exe.mt1.enmity}
Let $k \in \NN$. Let $p_1, p_2, \ldots, p_k$ be $k$ nonnegative
real numbers such that $p_1 + p_2 + \cdots + p_k \geq 1$.

Let $G = \tup{V, E}$ be a simple graph. A \textit{$k$-coloring} of $G$
shall mean a map $f : V \to \set{1, 2, \ldots, k}$.

Prove that there exists a
$k$-coloring $f$ of $G$ with the following property: For each vertex
$v \in V$, at most $p_{f\tup{v}} \deg v$ neighbors of $v$ have the
same color
as $v$. Here, the \textit{color} of a vertex $w \in V$ (under the
$k$-coloring $f$) means the value $f\tup{w}$.
\end{exercise}

\begin{proof}[Hints to Exercise~\ref{exe.mt1.enmity}.]
This is a generalization of Exercise 1 on
\href{http://www-users.math.umn.edu/~dgrinber/5707s17/hw0s.pdf}{homework set 0}.
Indeed, you obtain the latter exercise if you set $k = 2$, $p_1 = 1/2$
and $p_2 = 1/2$.

To solve Exercise~\ref{exe.mt1.enmity}, we can generalize the solution
of Exercise 1 on
\href{http://www-users.math.umn.edu/~dgrinber/5707s17/hw0s.pdf}{homework set 0}.
Four changes need to be made:

\begin{enumerate}
\item We need to deal with the cases $k \leq 1$ separately.
\item In the algorithm, we need to explain how precisely to ``flip''
      the color $f\tup{v}$ of the vertex $v$. (Indeed, for $k > 2$,
      there is more than one possibility.)
\item We need to change the definition of ``enmity''.
\item We need to prove that the enmity cannot keep decreasing forever.
\end{enumerate}

I leave change 1 to the reader (the cases $k \leq 1$ are essentially
trivial).

As for change 2: We have some $v \in V$ such that more than
$p_{f\tup{v}} \deg v$ among the neighbors of $v$ have the same color
of $v$. Then, there exists \textbf{some} $i \in \set{1, 2, \ldots, k}$
such that \textbf{at most} $p_i \deg v$ among the neighbors of $v$
have the color $i$ (because otherwise, by summing over all $i$, we
conclude that $v$ has more than
$\sum_{i=1}^k p_i \deg v
= \underbrace{\tup{p_1+p_2+\ldots+p_k}}_{\geq 1} \deg v
\geq \deg v$ neighbors in total; but this is absurd)\footnote{You can
even find an $i$ such that \textbf{fewer than} $p_i \deg v$ among the
neighbors of $v$ have the color $i$. But that's not necessary for
us.}. Fix such an $i$,
and replace the color $f \tup{v}$ of $v$ by this $i$.

What about change 3? We formerly defined the enmity of a coloring $f$
to be the number of $f$-monochromatic edges. This definition no longer
works. However, what works is the following subtler definition: For
each $i \in \set{1, 2, \ldots, k}$, an edge $e$ of $G$ is said to be
\textit{$\tup{f, i}$-chromatic} if the two endpoints of $e$ both have
color $i$ in the $k$-coloring $f$. The \textit{enmity} of a
$k$-coloring $f$ is now defined as the sum
\[
\sum_{i=1}^k \dfrac{1}{p_i} \tup{ \text{the number of all }
                                  \tup{f, i}\text{-chromatic edges} }.
\]
This definition requires a minor technical fix: It only works if all
$p_i$ are positive (i.e., nonzero). Fortunately, we can WLOG assume
that all $p_i$ are positive (indeed, if some $p_i$ is zero, then
we can discard this $p_i$, and correspondingly agree to never use the
color $i$ in our coloring). Proving that the enmity decreases
throughout the algorithm is rather easy (a straightforward
modification of the original argument).

Change 4 is simple but subtle; it is really easy to miss its
importance. In the solution of Exercise 1 on
\href{http://www-users.math.umn.edu/~dgrinber/5707s17/hw0s.pdf}{homework set 0},
we just argued that the enmity of a $2$-coloring cannot keep
decreasing forever because it is a nonnegative integer. However, in
our more general setting, the enmity no longer is a nonnegative
integer, and thus one could imagine it decreasing indefinitely
(e.g., from $1$ to
$1/2$, then on to $1/3$, then on to $1/4$, etc.). So we need a new
argument. Fortunately, there is an easy one: There are only finitely
many $k$-colorings of $G$ (namely, $k^{\abs{V}}$ many). Hence, there
are only finitely many values that the enmity of a $k$-coloring of $G$
can take. Hence, the enmity cannot keep decreasing forever (because
that would force it to take infinitely many different values along
the way). So we are done.
\end{proof}

\begin{noncompile}
A \textit{forest} means a simple graph that has no cycles.

\begin{exercise} \label{exe.mt1.forests-matroid}
Let $G = \tup{V, E}$ be a forest. Let $H = \tup{V, F}$ be a forest
with the same vertex set as $G$ but satisfying $\abs{F} > \abs{E}$.
Prove that there exists some $f \in F \setminus E$ such that
$\tup{V, E \cup \set{f}}$ is still a forest. (In other words, prove
that we can add some edge from $H$ to the graph $G$ such that the
resulting graph is still a forest, but this edge was not contained in
$G$ to begin with.)
\end{exercise}
\end{noncompile}

\subsection{Exercise \ref{exe.mt1.euler-add}: adding edges to get an
Eulerian circuit}

\begin{exercise} \label{exe.mt1.euler-add}
Let $G$ be a connected multigraph. Let $m$ be the number of vertices
of $G$ that have odd degree. Prove that we can add $m/2$ new edges to
$G$ in such a way that the resulting multigraph will have an Eulerian
circuit. (It is allowed to add an edge even if there is already an
edge between the same two vertices.)
\end{exercise}

\begin{proof}[Hints to Exercise~\ref{exe.mt1.euler-add}.]
Proposition 2.5.8 in the
\href{http://www-users.math.umn.edu/~dgrinber/5707s17/nogra.pdf}{lecture notes}
says that the number of vertices of $G$ having odd degree is
even\footnote{To be fully precise, Proposition 2.5.8 in the
\href{http://www-users.math.umn.edu/~dgrinber/5707s17/nogra.pdf}{lecture notes}
only states this fact for simple graphs, not for multigraphs. But the
proof for multigraphs is almost the same. (The only difference is that
``$v \in e$'' must be replaced by ``$v \in \phi\tup{e}$''.)}.
In other words, $m$ is even. Let $v_1, v_2, \ldots, v_m$ be the $m$
vertices of $G$ that have odd degree.

Now, let us add $m/2$ new edges
$e_1, e_2, \ldots, e_{m/2}$ to $G$, where each $e_i$ has
$\phi\tup{e_i} = \set{v_{2i-1}, v_{2i}}$.
(This is well-defined, since $m$ is even.) The resulting multigraph is
clearly connected (since $G$ was connected), and has the property that
each of its vertices has even degree\footnote{\textit{Proof.}
Let us see how the degrees of the vertices of our multigraph have been
affected by adding the $m/2$ new edges $e_1, e_2, \ldots, e_{m/2}$:
\begin{itemize}
\item The degrees of the vertices $v_1, v_2, \ldots, v_m$ have been
      incremented by $1$ when we added our $m/2$ new edges. This
      caused these degrees to become even (because back in the
      original multigraph $G$, they were odd).
\item The degrees of all other vertices have not changed when we added
      our $m/2$ new edges (because none of these new edges contains
      any of the other vertices). Hence, these degrees have remained
      even (because they were even in the original multigraph $G$).
\end{itemize}
Thus, in the resulting multigraph, the degrees of \textbf{all}
vertices have become even.}.
Hence, by the Euler-Hierholzer theorem, this new graph has an Eulerian
circuit.
\end{proof}

\subsection{Exercise \ref{exe.mt1.d+d+d}: how large can the perimeter
of a triangle on a graph be?}

\subsubsection{Distances in a graph}

If $u$ and $v$ are two vertices of a simple graph $G$, then
$d \tup{u, v}$ denotes the \textit{distance} between $u$ and $v$. This
is defined to be the minimum length of a path from $u$ to $v$ if
such a path exists; otherwise it is defined to be the symbol $\infty$.

We observe the following simple facts:

\begin{lemma} \label{lem.mt1.d-leq-V}
Let $u$ and $v$ be two vertices of a connected simple graph
$G = \tup{V, E}$.
Then, $d \tup{u, v} \leq \abs{V} - 1$.
\end{lemma}

\begin{proof}[Proof of Lemma~\ref{lem.mt1.d-leq-V}.]
The simple graph $G$ is connected. Hence, there exists a walk from $u$
to $v$ in $G$. Let $k$ be the length of this walk. Therefore, there
exists a walk from $u$ to $v$ in $G$ having length $\leq k$ (namely,
the walk we have just constructed).
Hence, Corollary 2.8.9 in the
\href{http://www-users.math.umn.edu/~dgrinber/5707s17/nogra.pdf}{lecture notes}
shows that there exists a path from $u$ to $v$ having length
$\leq k$. Let $\tup{u_0, u_1, \ldots, u_g}$ be this path. Then, the
vertices $u_0, u_1, \ldots, u_g$ are pairwise distinct (since
$\tup{u_0, u_1, \ldots, u_g}$ is a path). Hence,
$\abs{\set{u_0, u_1, \ldots, u_g}} = g+1$. But from
$\set{u_0, u_1, \ldots, u_g} \subseteq V$, we obtain
$\abs{\set{u_0, u_1, \ldots, u_g}} \leq \abs{V}$. Thus,
$g + 1 = \abs{\set{u_0, u_1, \ldots, u_g}} \leq \abs{V}$.
Hence, $g \leq \abs{V} - 1$.

Now, there exists a path from $u$ to $v$ having length $g$ (namely,
the path $\tup{u_0, u_1, \ldots, u_g}$). Hence, the minimum length of
a path from $u$ to $v$ is $\leq g$. But this minimum length is
$d \tup{u, v}$ (by the definition of $d \tup{u, v}$). Hence, we obtain
$d \tup{u, v} \leq g \leq \abs{V} - 1$. This proves
Lemma~\ref{lem.mt1.d-leq-V}.
\end{proof}

Lemma~\ref{lem.mt1.d-leq-V} shows that if $u$ and $v$ are two
vertices of a connected simple graph $G$, then $d \tup{u, v}$ is an
actual integer (as opposed to $\infty$).

\begin{lemma} \label{lem.mt1.walk-to-distance}
Let $u$ and $v$ be two vertices of a simple graph $G$. Let
$k \in \NN$. If there exists a walk from $u$ to $v$ in $G$ having
length $k$, then $d \tup{u, v} \leq k$.
\end{lemma}

\begin{proof}[Proof of Lemma~\ref{lem.mt1.walk-to-distance}.]
We assumed that there exists a walk from $u$ to $v$ in $G$ having
length $k$. Therefore, Corollary 2.8.9 in the
\href{http://www-users.math.umn.edu/~dgrinber/5707s17/nogra.pdf}{lecture notes}
shows that there exists a path from $u$ to $v$ having length
$\leq k$. Therefore, the minimum length of a path from $u$ to $v$
is $\leq k$. But this minimum length is $d \tup{u, v}$ (by the
definition of $d \tup{u, v}$). Hence, we obtain
$d \tup{u, v} \leq k$. This proves
Lemma~\ref{lem.mt1.walk-to-distance}.
\end{proof}

\begin{lemma} \label{lem.mt1.distances-metric}
Let $G = \tup{V, E}$ be a simple graph.

\textbf{(a)} Each $u \in V$ satisfies $d \tup{u, u} = 0$.

\textbf{(b)} Each $u \in V$ and $v \in V$ satisfy
$d \tup{u, v} = d \tup{v, u}$.

\textbf{(c)} Each $u \in V$, $v \in V$ and $w \in V$ satisfy
$d \tup{u, v} + d \tup{v, w} \geq d \tup{u, w}$.
(This inequality has to be interpreted appropriately when one of the
distances is infinite: For example, we understand $\infty$ to be
greater than any integer, and we understand $\infty + m$ to be
$\infty$ whenever $m \in \ZZ$.)

\textbf{(d)} If $u \in V$ and $v \in V$ satisfy $d \tup{u, v} = 0$,
then $u = v$.
\end{lemma}

\begin{proof}[Proof of Lemma~\ref{lem.mt1.distances-metric}
(sketched).]
Parts \textbf{(a)}, \textbf{(b)} and \textbf{(d)} of
Lemma~\ref{lem.mt1.distances-metric} are easy to check, and thus left
to the reader.

\textbf{(c)} We WLOG assume that none of the two distances
$d \tup{u, v}$ and $d \tup{v, w}$ is $\infty$ (since otherwise,
Lemma~\ref{lem.mt1.distances-metric} \textbf{(c)} holds for obvious
reasons).

If there was no path from $u$ to $v$, then $d \tup{u, v}$ would be
$\infty$ (by the definition of $d \tup{u, v}$), which would contradict
the fact that none of the two distances
$d \tup{u, v}$ and $d \tup{v, w}$ is $\infty$. Hence, there must exist
at least one path from $u$ to $v$. Thus,
$d \tup{u, v}$ is the minimum length of a path from $u$ to $v$ (by the
definition of $d \tup{u, v}$). Hence, there exists a path from $u$ to
$v$ having length $d \tup{u, v}$. Fix such a path, and denote it by
$\mathbf{p} = \tup{p_0, p_1, \ldots, p_g}$. Hence,
$\tup{\text{the length of the path } \mathbf{p}} = g$. Therefore,
$g = \tup{\text{the length of the path } \mathbf{p}} = d \tup{u, v}$
(since $\mathbf{p}$ has length $d \tup{u, v}$).

We have shown that there exists a path from $u$ to
$v$ having length $d \tup{u, v}$. Similarly, we can show that there
exists a path from $v$ to $w$ having length $d \tup{v, w}$. Fix such
a path, and denote it by
$\mathbf{q} = \tup{q_0, q_1, \ldots, q_h}$. Hence,
$\tup{\text{the length of the path } \mathbf{q}} = h$. Therefore,
$h = \tup{\text{the length of the path } \mathbf{q}} = d \tup{v, w}$
(since $\mathbf{h}$ has length $d \tup{v, w}$).

Now, $\tup{p_0, p_1, \ldots, p_g}$ is a path from $u$ to $v$. Hence,
$p_0 = u$ and $p_g = v$.
Also, $\tup{q_0, q_1, \ldots, q_h}$ is a path from $v$ to $w$. Hence,
$q_0 = v$ and $q_h = w$.

The lists $\tup{p_0, p_1, \ldots, p_g}$ and
$\tup{q_0, q_1, \ldots, q_h}$ are paths, and therefore walks. The
ending point $p_g$ of the first of these two walks is the starting
point $q_0$ of the second (because $p_g = v = q_0$).
Hence, we can combine these two walks to a walk
$\tup{p_0, p_1, \ldots, p_g, q_1, q_2, \ldots, q_h}
= \tup{p_0, p_1, \ldots, p_{g-1}, q_0, q_1, \ldots, q_h}$. This latter
walk has length $g + h$, and is a walk from $u$ to $w$ (since
$p_0 = u$ and $q_h = w$). Thus, there exists a walk from $u$ to $w$
having length $g + h$ (namely, the walk that we have just
constructed). Hence,
Lemma~\ref{lem.mt1.walk-to-distance} (applied to $w$ and $g + h$
instead of $v$ and $k$) shows that $d \tup{u, w} \leq g + h
= d \tup{u, v} + d \tup{v, w}$ (since $g = d \tup{u, v}$ and
$h = d \tup{v, w}$). In other words,
$d \tup{u, v} + d \tup{v, w} \geq d \tup{u, w}$.
This proves Lemma~\ref{lem.mt1.distances-metric} \textbf{(c)}.
\end{proof}

Lemma~\ref{lem.mt1.distances-metric} \textbf{(c)} is known as the
\textit{triangle inequality} for distances on a graph.
Of course, this is due to its similarity to the well-known triangle
inequality in Euclidean geometry. In fact, this similarity goes
deeper:
Lemma~\ref{lem.mt1.d-leq-V} shows that if $G = \tup{V, E}$ is a
connected simple graph, then the definition of the distance
$d \tup{u, v}$ for each pair $\tup{u, v} \in V \times V$ gives rise
to a well-defined map $d : V \times V \to \NN$.
Lemma~\ref{lem.mt1.distances-metric} shows that this map $d$ is a
\href{https://en.wikipedia.org/wiki/Metric_(mathematics)}{metric}.
We shall not use this in the following, but it is a useful fact to
keep in one's mind.

\subsubsection{Statement of the exercise}

\begin{exercise} \label{exe.mt1.d+d+d}
Let $a$, $b$ and $c$ be three vertices of a connected simple graph
$G = \tup{V, E}$.
Prove that
$d \tup{b, c} + d \tup{c, a} + d \tup{a, b} \leq 2 \abs{V} - 2$.
\end{exercise}

\subsubsection{First solution}

\begin{proof}[Solution sketch to Exercise~\ref{exe.mt1.d+d+d}.]
The following solution was found by Jiali Huang and
Nicholas Rancourt.

Fix some path $\mathbf{z} = \tup{z_0, z_1, \ldots, z_g}$ from
$a$ to $b$ having minimum length. Then, the length of $\mathbf{z}$ is
$d \tup{a, b}$ (since $d \tup{a, b}$ is defined as the minimum
length of a path from $a$ to $b$). Hence,
\begin{equation}
d \tup{a, b}
= \tup{\text{the length of the path } \mathbf{z}}
= g
\label{pf.exe.mt1.d+d+d.walk0}
\end{equation}
(since $\mathbf{z} = \tup{z_0, z_1, \ldots, z_g}$).

Since $\tup{z_0, z_1, \ldots, z_g}$ is a path from $a$ to $b$,
we have $z_0 = a$ and $z_g = b$.

Since $\tup{z_0, z_1, \ldots, z_g}$ is a path, the $g+1$
vertices $z_0, z_1, \ldots, z_g$ are distinct. Hence,
$\abs{\set{z_0, z_1, \ldots, z_g}} = g+1$.

Now, pick an element $i \in \set{0, 1, \ldots, g}$ for which the
number $d \tup{c, z_i}$ is minimum. Hence,
\begin{equation}
d \tup{c, z_i} \leq d \tup{c, z_j}
\qquad \text{for each } j \in \set{0, 1, \ldots, g} .
\label{sol.mt1.d+d+d.zi-minimum}
\end{equation}

Fix some path $\mathbf{t} = \tup{t_0, t_1, \ldots, t_h}$ from
$c$ to $z_i$ having minimum length. Then, the length of $\mathbf{t}$
is $d \tup{c, z_i}$ (since $d \tup{c, z_i}$ is defined as the
minimum length of a path from $c$ to $z_i$). Hence,
\[
d \tup{c, z_i}
= \tup{\text{the length of the path } \mathbf{t}}
= h
\]
(since $\mathbf{t} = \tup{t_0, t_1, \ldots, t_h}$).

Since $\tup{t_0, t_1, \ldots, t_h}$ is a path from $c$ to $z_i$,
we have $t_0 = c$ and $t_h = z_i$.

Since $\tup{t_0, t_1, \ldots, t_h}$ is a path, the $h+1$
vertices $t_0, t_1, \ldots, t_h$ are distinct. In particular,
the $h$ vertices $t_0, t_1, \ldots, t_{h-1}$ are distinct. Hence,
$\abs{\set{t_0, t_1, \ldots, t_{h-1}}} = h$.

Now, there is a walk from $c$ to $a$ in $G$ having length
$h + i$\ \ \ \ \footnote{\textit{Proof.} We know that
$\tup{z_0, z_1, \ldots, z_g}$ is a path, thus a walk. Hence,
$\tup{z_0, z_1, \ldots, z_i}$ is a walk as well. Therefore,
$\tup{z_i, z_{i-1}, \ldots, z_0}$ is a walk (being the reversal
of the walk $\tup{z_0, z_1, \ldots, z_i}$). On the other hand,
$\tup{t_0, t_1, \ldots, t_h}$ is a walk.
Since the ending point of the walk $\tup{t_0, t_1, \ldots, t_h}$
is the starting point of the walk
$\tup{z_i, z_{i-1}, \ldots, z_0}$ (because $t_h = z_i$), we can
combine these two walks, obtaining a new walk
$\tup{t_0, t_1, \ldots, t_{h-1}, z_i, z_{i-1}, \ldots, z_0}$.
This new walk is a walk from $c$ to $a$ (since $t_0 = c$ and
$z_0 = a$) and has length $h + i$. Hence, there is a
walk from $c$ to $a$ in $G$ having length $h + i$ (namely, the
walk that we have just constructed).}. Hence,
Lemma~\ref{lem.mt1.walk-to-distance} (applied to $u = c$, $v = a$
and $k = h + i$) shows that
\begin{equation}
d \tup{c, a} \leq h + i .
\label{pf.exe.mt1.d+d+d.walk1}
\end{equation}

On the other hand, there is a walk from $b$ to $c$ in $G$ having
length $\tup{g-i} + h$\ \ \ \ \footnote{\textit{Proof.} We know that
$\tup{z_0, z_1, \ldots, z_g}$ is a path, thus a walk. Hence,
$\tup{z_i, z_{i+1}, \ldots, z_g}$ is a walk as well. Therefore,
$\tup{z_g, z_{g-1}, \ldots, z_i}$ is a walk (being the reversal
of the walk $\tup{z_i, z_{i+1}, \ldots, z_g}$). On the other hand,
$\tup{t_0, t_1, \ldots, t_h}$ is a walk. Hence,
$\tup{t_h, t_{h-1}, \ldots, t_0}$ is a walk as well
(being the reversal of the walk $\tup{t_0, t_1, \ldots, t_h}$).
Since the ending point of the walk $\tup{z_g, z_{g-1}, \ldots, z_i}$
is the starting point of the walk
$\tup{t_h, t_{h-1}, \ldots, t_0}$ (because $z_i = t_h$), we can
combine these two walks, obtaining a new walk
$\tup{z_g, z_{g-1}, \ldots, z_i, t_{h-1}, t_{h-2}, \ldots, t_0}$.
This new walk is a walk from $b$ to $c$ (since $z_g = b$ and
$t_0 = c$) and has length $\tup{g-i} + h$. Hence, there is a
walk from $b$ to $c$ in $G$ having length $\tup{g-i} + h$ (namely,
the walk that we have just constructed).}. Hence,
Lemma~\ref{lem.mt1.walk-to-distance} (applied to $u = b$, $v = c$
and $k = \tup{g-i} + h$) shows that
\begin{equation}
d \tup{b, c} \leq \tup{g-i} + h .
\label{pf.exe.mt1.d+d+d.walk2}
\end{equation}

It is easy to see that the sets
$\set{t_0, t_1, \ldots, t_{h-1}}$ and
$\set{z_0, z_1, \ldots, z_g}$ are disjoint\footnote{\textit{Proof.}
Let $v \in
\set{t_0, t_1, \ldots, t_{h-1}} \cap \set{z_0, z_1, \ldots, z_g}$.
We shall derive a contradiction.

We have
$v \in
\set{t_0, t_1, \ldots, t_{h-1}} \cap \set{z_0, z_1, \ldots, z_g}
\subseteq \set{t_0, t_1, \ldots, t_{h-1}}$. Hence,
$v = t_p$ for some $p \in \set{0, 1, \ldots, h-1}$. Consider this
$p$.

We have $v \in
\set{t_0, t_1, \ldots, t_{h-1}} \cap \set{z_0, z_1, \ldots, z_g}
\subseteq \set{z_0, z_1, \ldots, z_g}$. Hence,
$v = z_j$ for some $j \in \set{0, 1, \ldots, g}$. Consider this
$j$. 

Recall that $\tup{t_0, t_1, \ldots, t_h}$ is a walk. Hence,
$\tup{t_0, t_1, \ldots, t_p}$ is a walk as well. This walk
$\tup{t_0, t_1, \ldots, t_p}$ is a walk from $c$ to $v$ (since
$t_0 = c$ and $t_p = v$) and has length $p$.
Hence, there is a walk from $c$ to $v$ in $G$ having length
$p$ (namely, the walk $\tup{t_0, t_1, \ldots, t_p}$).
Consequently, Lemma~\ref{lem.mt1.walk-to-distance}
(applied to $u = c$ and $k = p$) shows that
$d \tup{c, v} \leq p \leq h-1$
(since $p \in \set{0, 1, \ldots, h-1}$).

But \eqref{sol.mt1.d+d+d.zi-minimum} yields
$d \tup{c, z_i} \leq d \tup{c, z_j} = d \tup{c, v}$
(since $z_j = v$). Thus, $d \tup{c, v} \geq d \tup{c, z_i}
= h > h-1$. This contradicts $d \tup{c, v} \leq h-1$.

Now, forget that we fixed $v$. Thus, we have obtained a
contradiction for each
$v \in
\set{t_0, t_1, \ldots, t_{h-1}} \cap \set{z_0, z_1, \ldots, z_g}$.
Hence, there exists no
$v \in
\set{t_0, t_1, \ldots, t_{h-1}} \cap \set{z_0, z_1, \ldots, z_g}$.
Thus,
$\set{t_0, t_1, \ldots, t_{h-1}} \cap \set{z_0, z_1, \ldots, z_g}
= \varnothing$. In other words, the sets
$\set{t_0, t_1, \ldots, t_{h-1}}$ and
$\set{z_0, z_1, \ldots, z_g}$ are disjoint.}. Hence, the sum of
the sizes of these sets equals the size of their union. In other
words,
\begin{align*}
\abs{\set{t_0, t_1, \ldots, t_{h-1}}}
+ \abs{\set{z_0, z_1, \ldots, z_g}}
&=
\abs{\underbrace{\set{t_0, t_1, \ldots, t_{h-1}} \cup
                   \set{z_0, z_1, \ldots, z_g}}_{\subseteq V}}
\leq \abs{V} .
\end{align*}
Since $\abs{\set{z_0, z_1, \ldots, z_g}} = g+1$
and $\abs{\set{t_0, t_1, \ldots, t_{h-1}}} = h$, this rewrites as
$h + \tup{g+1} \leq \abs{V}$. Hence,
$\tup{g+h} + 1 = h + \tup{g+1} \leq \abs{V}$, so that
$g+h \leq \abs{V} - 1$.

Adding together the two inequalities \eqref{pf.exe.mt1.d+d+d.walk2}
and \eqref{pf.exe.mt1.d+d+d.walk1} and the equation
\eqref{pf.exe.mt1.d+d+d.walk0}, we obtain
\begin{align*}
d \tup{b, c} + d \tup{c, a} + d \tup{a, b}
&\leq \tup{\tup{g-i} + h} + \tup{h + i} + g
= 2 \tup{\underbrace{g+h}_{\leq \abs{V} - 1}} \\
&\leq 2 \tup{\abs{V} - 1} = 2 \abs{V} - 2 .
\end{align*}
This solves the exercise.
\end{proof}

\subsubsection{Second solution}

\begin{proof}[Hints to a second solution of Exercise~\ref{exe.mt1.d+d+d}.]
The following solution is how I originally solved the exercise.

Let $\ell\tup{\mathbf{w}}$ denote the length of a walk $\mathbf{w}$.

Fix
\begin{itemize}
\item a shortest path $\mathbf{x}$ from $b$ to $c$;
\item a shortest path $\mathbf{y}$ from $c$ to $a$.
\item a shortest path $\mathbf{z}$ from $a$ to $b$.
\end{itemize}
Let $X$ be the set of all vertices of $\mathbf{x}$. Similarly define
$Y$ and $Z$. Thus,
\begin{align}
d \tup{b, c} &= \ell\tup{\mathbf{x}} = \abs{X} - 1 \qquad \text{and}
\label{pf.exe.mt1.d+d+d.hint.d1} \\
d \tup{c, a} &= \ell\tup{\mathbf{y}} = \abs{Y} - 1 \qquad \text{and}
\label{pf.exe.mt1.d+d+d.hint.d2} \\
d \tup{a, b} &= \ell\tup{\mathbf{z}} = \abs{Z} - 1.
\label{pf.exe.mt1.d+d+d.hint.d3}
\end{align}

Now, we claim that $\abs{X \cap Y \cap Z} \leq 1$. Indeed, assume the
contrary. Then, $\abs{X \cap Y \cap Z} \geq 2$. Hence, there exist two
distinct vertices $p$ and $q$ in $X \cap Y \cap Z$. Consider these $p$
and $q$.

Both vertices $p$ and $q$ belong to $X \cap Y \cap Z \subseteq X$,
thus appear on the path $\mathbf{x}$.
We WLOG assume that $p$ appears before $q$ on this path (i.e., the
path $\mathbf{x}$ has the form
$\tup{\ldots, p, \ldots, q, \ldots}$, where some of the $\ldots$ may
be empty). (This WLOG assumption is legitimate, since we can switch
$p$ with $q$.)

But the vertices $p$ and $q$ also appear on the path $\mathbf{y}$
(since they belong to $X \cap Y \cap Z \subseteq Y$). The vertex $p$
must appear after $q$ on this path\footnote{\textit{Proof.} Assume
the contrary. Thus, $p$ appears before $q$ on the path $\mathbf{y}$.

Now, let us split the path $\mathbf{x}$ into three parts: Namely,
\begin{itemize}
\item let $\mathbf{x}_1$ be the part between $b$ and $p$;
\item let $\mathbf{x}_2$ be the part between $p$ and $q$;
\item let $\mathbf{x}_3$ be the part between $q$ and $c$.
\end{itemize}
(This is possible because $p$ appears before $q$ in $\mathbf{x}$.)
Then, $\ell\tup{\mathbf{x}} = \ell\tup{\mathbf{x}_1}
+ \ell\tup{\mathbf{x}_2} + \ell\tup{\mathbf{x}_3}$.

Next, let us split the path $\mathbf{y}$ into three parts: Namely,
\begin{itemize}
\item let $\mathbf{y}_1$ be the part between $c$ and $p$;
\item let $\mathbf{y}_2$ be the part between $p$ and $q$;
\item let $\mathbf{y}_3$ be the part between $q$ and $a$.
\end{itemize}
(This is possible because $p$ appears before $q$ in $\mathbf{y}$.)
Then, $\ell\tup{\mathbf{y}} = \ell\tup{\mathbf{y}_1}
+ \ell\tup{\mathbf{y}_2} + \ell\tup{\mathbf{y}_3}$.

The path $\mathbf{x}_2$ connects $p$ and $q$, and thus has length $>0$
(since $p$ and $q$ are distinct). Hence, $\ell\tup{\mathbf{x}_2} > 0$.
Thus,
\[
\ell\tup{\mathbf{x}}
= \ell\tup{\mathbf{x}_1}
   + \underbrace{\ell\tup{\mathbf{x}_2}}_{> 0}
   + \ell\tup{\mathbf{x}_3}
> \ell\tup{\mathbf{x}_1} + \ell\tup{\mathbf{x}_3} ,
\]
so that
\begin{align}
\ell\tup{\mathbf{x}_1} + \ell\tup{\mathbf{x}_3}
< \ell\tup{\mathbf{x}} = d \tup{b, c} .
\label{pf.exe.mt1.d+d+d.hint.fn1.more1}
\end{align}
Similarly,
\begin{align}
\ell\tup{\mathbf{y}_1} + \ell\tup{\mathbf{y}_3}
< d \tup{c, a} .
\label{pf.exe.mt1.d+d+d.hint.fn1.more2}
\end{align}

Combining the reversal of the walk $\mathbf{x}_3$ with the walk
$\mathbf{y}_3$, we obtain a walk from $c$ to $a$ (via $q$). This walk
has length $\ell\tup{\mathbf{x}_3} + \ell\tup{\mathbf{y}_3}$. Hence,
there exists a walk from $c$ to $a$ having length
$\ell\tup{\mathbf{x}_3} + \ell\tup{\mathbf{y}_3}$. Consequently,
Lemma~\ref{lem.mt1.walk-to-distance} (applied to $u = c$, $v = a$
and $k = \ell\tup{\mathbf{x}_3} + \ell\tup{\mathbf{y}_3}$) shows that
\begin{equation}
d \tup{c, a} \leq \ell\tup{\mathbf{x}_3} + \ell\tup{\mathbf{y}_3} .
\label{pf.exe.mt1.d+d+d.hint.fn1.1}
\end{equation}

Combining the walk $\mathbf{x}_1$ with the reversal of the walk
$\mathbf{y}_1$, we obtain a walk from $b$ to $c$ (via $p$). This walk
has length $\ell\tup{\mathbf{x}_1} + \ell\tup{\mathbf{y}_1}$. Hence,
there exists a walk from $b$ to $c$ having length
$\ell\tup{\mathbf{x}_1} + \ell\tup{\mathbf{y}_1}$. Consequently,
Lemma~\ref{lem.mt1.walk-to-distance} (applied to $u = b$, $v = c$
and $k = \ell\tup{\mathbf{x}_1} + \ell\tup{\mathbf{y}_1}$) shows that
\begin{equation}
d \tup{b, c} \leq \ell\tup{\mathbf{x}_1} + \ell\tup{\mathbf{y}_1} .
\label{pf.exe.mt1.d+d+d.hint.fn1.2}
\end{equation}

Adding together the inequalities \eqref{pf.exe.mt1.d+d+d.hint.fn1.1}
and \eqref{pf.exe.mt1.d+d+d.hint.fn1.2}, we obtain
\begin{align*}
d \tup{c, a} + d \tup{b, c}
&\leq \tup{\ell\tup{\mathbf{x}_3} + \ell\tup{\mathbf{y}_3}}
      + \tup{\ell\tup{\mathbf{x}_1} + \ell\tup{\mathbf{y}_1}} \\
&=
  \underbrace{ \ell\tup{\mathbf{y}_1}
               + \ell\tup{\mathbf{y}_3} }
             _{\substack{< d\tup{c, a} \\
                 \text{(by \eqref{pf.exe.mt1.d+d+d.hint.fn1.more2})}}}
+ \underbrace{ \ell\tup{\mathbf{x}_1}
               + \ell\tup{\mathbf{x}_3} }
             _{\substack{< d\tup{b, c} \\
                 \text{(by \eqref{pf.exe.mt1.d+d+d.hint.fn1.more1})}}}
< d \tup{c, a} + d \tup{b, c}.
\end{align*}
This is absurd. Hence, we have found a contradiction, qed.}.
In other words, the vertex $q$ appears before $p$ on the path
$\mathbf{y}$.

The same reasoning (but applied to $b$, $c$, $a$, $\mathbf{y}$,
$\mathbf{z}$, $\mathbf{x}$, $q$ and $p$ instead of $a$, $b$, $c$,
$\mathbf{x}$, $\mathbf{y}$, $\mathbf{z}$, $p$ and $q$) now shows that
the vertex $p$ appears before $q$ on the path $\mathbf{z}$ (because
the vertex $q$ appears before $p$ on the path $\mathbf{y}$). The same
reasoning (but applied to $b$, $c$, $a$, $\mathbf{y}$,
$\mathbf{z}$ and $\mathbf{x}$ instead of $a$, $b$, $c$,
$\mathbf{x}$, $\mathbf{y}$ and $\mathbf{z}$) therefore shows
that the vertex $p$ appears before $q$ on the path $\mathbf{y}$. But
this contradicts the fact that the vertex $q$ appears before $p$ on
the path $\mathbf{y}$.

This contradiction shows that our assumption was wrong. Hence,
$\abs{X \cap Y \cap Z} \leq 1$ is proven. From this, we can easily
obtain $\abs{X} + \abs{Y} + \abs{Z} \leq 2 \abs{X \cup Y \cup Z} + 1$
\ \ \ \ \footnote{\textit{Proof.} The sum
$\abs{X} + \abs{Y} + \abs{Z}$ counts the elements of
$X \cup Y \cup Z$, but it counts some of them twice and some thrice:
Namely, an element is counted thrice if it belongs to
$X \cap Y \cap Z$; otherwise, it is counted at most twice. Since
$\abs{X \cap Y \cap Z} \leq 1$, we know that at most one element is
counted thrice. All other elements are counted at most twice. Hence,
the total sum $\abs{X} + \abs{Y} + \abs{Z}$ is at most
$3 + 2 \tup{\abs{X \cup Y \cup Z} - 1} = 2 \abs{X \cup Y \cup Z} + 1$.
Qed.}. Now, adding the equalities
\eqref{pf.exe.mt1.d+d+d.hint.d1}, \eqref{pf.exe.mt1.d+d+d.hint.d2}
and \eqref{pf.exe.mt1.d+d+d.hint.d3} together, we obtain
\begin{align*}
d \tup{b, c} + d \tup{c, a} + d \tup{a, b}
&= \tup{\abs{X} - 1} + \tup{\abs{Y} - 1} + \tup{\abs{Z} - 1}
= \underbrace{\abs{X} + \abs{Y} + \abs{Z}}
             _{\leq 2 \abs{X \cup Y \cup Z} + 1}
  - 3 \\
&\leq 2 \abs{\underbrace{X \cup Y \cup Z}_{\subseteq V}} + 1 - 3
\leq 2 \abs{V} + 1 - 3 = 2 \abs{V} - 2.
\end{align*}
\end{proof}

\subsubsection{Third solution}

Let me finally sketch a third solution of the exercise. The idea of
the below solution is due to Sasha Pevzner, although I am restating it
in slightly different terms. First, let me generalize the problem as
follows:

\Needspace{4cm}
\begin{theorem} \label{thm.mt1.d+d+d.gene}
Let $k \in \NN$ be odd. Let $a_1, a_2, \ldots, a_k$ be $k$ vertices of
a connected simple graph $G$. Let us set $a_{k+1} = a_1$.
Then,
\[
\sum_{i=1}^k d \tup{a_i, a_{i+1}}
\leq \tup{k-1} \tup{\abs{\verts{G}}-1} .
\]
\end{theorem}

Before we start proving this, let us notice that
Theorem~\ref{thm.mt1.d+d+d.gene} does not hold for $k$ even, and the
whole question of maximizing $\sum_{i=1}^k d \tup{a_i, a_{i+1}}$ for
$k$ even is rather trivial\footnote{To wit, it is obvious that
  $\sum_{i=1}^k d \tup{a_i, a_{i+1}} \leq k \tup{\abs{\verts{G}}-1}$
  (since Lemma~\ref{lem.mt1.d-leq-V} shows that each
  of the $k$ numbers $d \tup{a_i, a_{i+1}}$ is
  $\leq \abs{\verts{G}}-1$).
  When $k$ is even, this bound cannot be improved,
  because it is attained whenever $G$ is the path graph $P_n$, the
  vertices $a_1, a_3, a_5, \ldots$ all equal to one endpoint of this
  path, and the vertices $a_2, a_4, a_6, \ldots$ all equal to the
  other endpoint.}.
This makes Theorem~\ref{thm.mt1.d+d+d.gene} (which answers the
question of maximizing $\sum_{i=1}^k d \tup{a_i, a_{i+1}}$ for
$k$ odd\footnote{Indeed, it is not hard to see that equality
  can be obtained in the inequality in
  Theorem~\ref{thm.mt1.d+d+d.gene}; hence, it really maximizes
  $\sum_{i=1}^k d \tup{a_i, a_{i+1}}$.})
all the more interesting.

Exercise~\ref{exe.mt1.d+d+d} is the particular case of
Theorem~\ref{thm.mt1.d+d+d.gene} for $k = 3$.

For the proof of Theorem~\ref{thm.mt1.d+d+d.gene}, we shall need a
slightly more advanced notation for the distance between two vertices
in a graph: Namely, if $u$ and $v$ are two vertices of a simple graph
$G$, then the distance $d \tup{u, v}$ will often be denoted by
$d_G \tup{u, v}$ (in order to stress the dependence on $G$). This
allows us to unambiguously speak of distances between $u$ and $v$ even
when there are several different graphs containing $u$ and $v$
as vertices.

Sasha's argument begins by reducing the problem to the situation in
which the graph is a tree. Thus, we will need to prove the following
lemma, which of course is a particular case of
Theorem~\ref{thm.mt1.d+d+d.gene}:

\begin{lemma} \label{lem.mt1.d+d+d.tree}
Let $k \in \NN$ be odd. Let $a_1, a_2, \ldots, a_k$ be $k$ vertices of
a tree $G$. Let us set $a_{k+1} = a_1$. Then,
\[
\sum_{i=1}^k d \tup{a_i, a_{i+1}}
\leq \tup{k-1} \tup{\abs{\verts{G}}-1} .
\]
(Here, we regard a tree as a simple graph.)
\end{lemma}

Let us first see why proving this lemma is sufficient:

\begin{proof}[Proof of Theorem~\ref{thm.mt1.d+d+d.gene} using
Lemma~\ref{lem.mt1.d+d+d.tree} (sketched).]
Assume that Lemma~\ref{lem.mt1.d+d+d.tree} is already proven.

The graph $G$ is connected, and thus has a spanning tree $T$. Fix such
a $T$. Clearly, $\verts{T} = \verts{G}$.
Being a tree, $T$ is of course still connected.

But $T$ is a tree. Therefore, Lemma~\ref{lem.mt1.d+d+d.tree} (applied
to $T$ instead of $G$) shows that
$\sum_{i=1}^k d_T \tup{a_i, a_{i+1}} \leq \tup{k-1}
\tup{\abs{\verts{T}}-1} = \tup{k-1} \tup{\abs{\verts{G}} - 1}$
(since $\verts{T} = \verts{G}$).

Since $T$ is a subgraph of $G$, each walk in $T$ is a walk in
$G$. Hence, $d_G \tup{u, v} \leq d_T \tup{u, v}$ for any two vertices
$u$ and $v$ of $G$. Thus,
\[
\sum_{i=1}^k \underbrack{d_G \tup{a_i, a_{i+1}}}
                        {\leq d_T \tup{a_i, a_{i+1}}}
\leq
\sum_{i=1}^k d_T \tup{a_i, a_{i+1}} \leq \tup{k-1}
\tup{\abs{\verts{G}}-1} .
\]
Hence, Theorem~\ref{thm.mt1.d+d+d.gene} is proven.
\end{proof}

It thus remains to prove Lemma~\ref{lem.mt1.d+d+d.tree}.

\begin{proof}[Proof of Lemma~\ref{lem.mt1.d+d+d.tree} (sketched).]
We proceed by induction on $\abs{\verts{G}}$.

The \textit{induction base} is the case when $\abs{\verts{G}} = 1$.
This case is easy (indeed, $G$ has only one vertex in this case, so
that all $k$ vertices $a_1, a_2, \ldots, a_k$ must be identical,
and therefore their distances $d \tup{a_i, a_{i+1}}$ are all $0$).

Now, to the \textit{induction step}: Fix an integer $N > 1$.
Assume (as the induction hypothesis)
that Lemma~\ref{lem.mt1.d+d+d.tree} is proven in the case when
$\abs{\verts{G}} = N-1$.
We must now prove Lemma~\ref{lem.mt1.d+d+d.tree} in the case when
$\abs{\verts{G}} = N$.

Thus, let us consider the situation of
Lemma~\ref{lem.mt1.d+d+d.tree} under the assumption that
$\abs{\verts{G}} = N$.
We need to prove that
\begin{equation}
\sum_{i=1}^k d_G \tup{a_i, a_{i+1}}
\leq \tup{k-1} \tup{\abs{\verts{G}}-1} .
\label{pf.lem.mt1.d+d+d.tree.goal}
\end{equation}

The graph $G$ is a tree with more than $1$ vertex (since
$\abs{\verts{G}} = N > 1$).
Hence, $G$ has at least one leaf.
Pick such a leaf, and denote it by $v$.
Let $v'$ be the only neighbor of $v$ in $G$.

Let $G'$ be the subgraph of $G$ obtained from $G$ by removing the
vertex $v$ and the unique edge with endpoint $v$ (that is, the edge
$vv'$).
Thus, $G'$ is a tree with $\abs{\verts{G} \setminus \set{v}} = N - 1$
vertices. Hence, $\abs{\verts{G'}} = N - 1$.

For each vertex $u$ of $G$, we define a vertex $\overline{u}$ of $G'$
as follows:
\[
\overline{u}
= \begin{cases}
    u   , & \text{ if } u \neq v; \\
    v'  , & \text{ if } u = v
  \end{cases}
.
\]
Then, all of $\overline{a_1}, \overline{a_2}, \ldots, \overline{a_k}$
are vertices of $G'$. Hence, (by the induction hypothesis) we can
apply Lemma~\ref{lem.mt1.d+d+d.tree} to $G'$ and $\overline{a_i}$
instead of $G$ and $a_i$. We thus obtain
\begin{equation}
\sum_{i=1}^k d_{G'} \tup{\overline{a_i}, \overline{a_{i+1}}}
\leq \tup{k-1} \tup{\abs{\verts{G'}}-1} .
\label{pf.lem.mt1.d+d+d.tree.4}
\end{equation}

Recall that $v$ is a leaf of the tree $G$. Hence, any path in $G$ that
neither starts nor ends at $v$ must be a path in $G'$ as well (because
otherwise, it would have to traverse $v$, but this would entail that
$v$ is contained in at least two edges, which would contradict the
fact that $v$ is a leaf). Thus,
\begin{equation}
d_G \tup{x, y}
= d_{G'} \tup{x, y}
\label{pf.lem.mt1.d+d+d.tree.5}
\end{equation}
for any two vertices $x$ and $y$ of $G'$.

Let us, however, notice that
\begin{equation}
d_G \tup{x, y}
= d_{G'} \tup{\overline{x}, \overline{y}}
  + \ive{\text{exactly one of } x \text{ and } y
         \text{ equals } v }
\label{pf.lem.mt1.d+d+d.tree.6}
\end{equation}
for any two vertices $x$ and $y$ of $G$, where we are using the
Iverson bracket notation\footnote{\textit{Proof
of \eqref{pf.lem.mt1.d+d+d.tree.6}.} Let $x$ and $y$ be two vertices
of $G$.
We must prove \eqref{pf.lem.mt1.d+d+d.tree.6}.
If none of $x$ and $y$ equals $v$, then
\eqref{pf.lem.mt1.d+d+d.tree.6} follows from
\eqref{pf.lem.mt1.d+d+d.tree.5} (because in this case, we have
$\overline{x} = x$ and $\overline{y} = y$).
If both $x$ and $y$ equal $v$, then \eqref{pf.lem.mt1.d+d+d.tree.6}
holds as well (since in this case, we have $x = y$).
Hence, it remains to prove \eqref{pf.lem.mt1.d+d+d.tree.6} in the case
when exactly one of $x$ and $y$ equals $v$.
Thus, consider this case.
WLOG assume that $x = v$ and $y \neq v$ (since otherwise, we can
switch $x$ with $y$). Thus,
$\ive{\text{exactly one of } x \text{ and } y
         \text{ equals } v } = 1$.
From $x = v$, we obtain $\overline{x} = v'$
(by the definition of $\overline{x}$).
From $y \neq v$, we obtain $\overline{y} = y$
(by the definition of $\overline{y}$).
Recall that $v'$ is the only neighbor of $v$.
Hence, $vv'$ is the only edge that contains $v$.
In other words, $vv'$ is the only edge that contains $x$ (since
$x = v$).
Recall that $G$ is a tree. Hence, there is only one path from $x$ to
$y$. The length of this path must therefore be $d_G \tup{x, y}$. This
path has length $> 0$ (since $y \neq v = x$) and therefore begins with
an edge that contains $x$. Hence, it begins with the edge $vv'$
(because the only edge that contains $x$ is $vv'$). After traversing
this edge, the path must proceed from $v'$ to $y$, which requires
$d_G \tup{v', y}$ edges (again because $G$ is a tree, and thus there
is exactly one path from $v'$ to $y$). Hence, the length of this path
is $1 + d_G \tup{v', y}$. Since we already know that the length of
this path is $d_G \tup{x, y}$, we thus obtain
\begin{align*}
d_G \tup{x, y}
&= 1 + \underbrack{d_G \tup{v', y}}
                  {= d_{G'} \tup{v', y} \\
                   \text{(by \eqref{pf.lem.mt1.d+d+d.tree.5}, applied }
                   \text{to } v' \text{ instead of } x
                   \text{ (since } v' \text{ and } y \text{ are }
                   \text{vertices of } G' \text{))}}
 = 1 + d_{G'} \tup{v', y} \\
&= d_{G'} \tup{\underbrack{v'}{= \overline{x}},
               \underbrack{y}{= \overline{y}}} + 1
 = d_{G'} \tup{\overline{x}, \overline{y}} + 1
 = d_{G'} \tup{\overline{x}, \overline{y}}
  + \ive{\text{exactly one of } x \text{ and } y
         \text{ equals } v }
\end{align*}
(since $1 = \ive{\text{exactly one of } x \text{ and } y
         \text{ equals } v }$).
This proves \eqref{pf.lem.mt1.d+d+d.tree.6}.}.

Hence,
\begin{align}
& \sum_{i=1}^k
  \underbrack{d_G \tup{a_i, a_{i+1}}}
             {= d_{G'} \tup{\overline{a_i}, \overline{a_{i+1}}}
              + \ive{\text{exactly one of } a_i \text{ and } a_{i+1}
                     \text{ equals } v } \\
              \text{(by \eqref{pf.lem.mt1.d+d+d.tree.6}, applied to }
              x = a_i \text{ and } y = a_{i+1} \text{)}}
\nonumber \\
&= \sum_{i=1}^k
   \tup{d_{G'} \tup{\overline{a_i}, \overline{a_{i+1}}}
          + \ive{\text{exactly one of } a_i \text{ and } a_{i+1}
              \text{ equals } v }}
\nonumber \\
&= \underbrack{\sum_{i=1}^k
                d_{G'} \tup{\overline{a_i}, \overline{a_{i+1}}}}
              {\leq \tup{k-1} \tup{\abs{\verts{G'}}-1} \\
                \text{(by \eqref{pf.lem.mt1.d+d+d.tree.4})}}
 + \sum_{i=1}^k
      \underbrack{\ive{\text{exactly one of } a_i \text{ and } a_{i+1}
                    \text{ equals } v }}
                 {\leq 1}
\nonumber \\
&\leq \tup{k-1} \tup{\abs{\verts{G'}}-1}
    + \underbrack{\sum_{i=1}^k 1}{= k}
= \tup{k-1} \tup{\underbrack{\abs{\verts{G'}}}{= N-1}
                 -1} + k
\nonumber \\
&= \tup{k-1} \tup{\tup{N-1} - 1} + k
= \tup{k-1} \tup{N-1} + 1 .
\label{pf.lem.mt1.d+d+d.tree.antigoal}
\end{align}

But recall that our goal is to prove
\eqref{pf.lem.mt1.d+d+d.tree.goal}. We assume the contrary (for the
sake of contradiction). Hence, we have
\[
\sum_{i=1}^k d_G \tup{a_i, a_{i+1}}
> \tup{k-1} \tup{\underbrack{\abs{\verts{G}}}{= N} -1}
= \tup{k-1} \tup{N-1} .
\]
Since both sides of this
inequality are integers, we thus obtain
\[
\sum_{i=1}^k d_G \tup{a_i, a_{i+1}}
\geq \tup{k-1} \tup{N-1} + 1.
\]
Combining this with \eqref{pf.lem.mt1.d+d+d.tree.antigoal}, we see
that the inequality \eqref{pf.lem.mt1.d+d+d.tree.antigoal} must be an
equality. Hence, each inequality that was used in the derivation of
\eqref{pf.lem.mt1.d+d+d.tree.antigoal} must also be an
equality\footnote{This is not to be taken fully literally. For
example, we can derive the inequality $2\cdot 0 \leq 0$ by multiplying
the two inequalities $2 \leq 0$ and $0 \leq 0$, and it is not true
that each of the latter two inequalities must be an equality, even
though the former inequality is an equality.
However, the derivation of the inequality
\eqref{pf.lem.mt1.d+d+d.tree.antigoal} involved only
\textbf{addition} (not multiplication) of other inequalities; and
therefore, equality in \eqref{pf.lem.mt1.d+d+d.tree.antigoal} forces
equality in each of the inequalities that were added.}.
In particular, the inequality
\[
\ive{\text{exactly one of } a_i \text{ and } a_{i+1}
                     \text{ equals } v } \leq 1
\]
must be an equality for each $i \in \set{1, 2, \ldots, k}$ (because
all of these inequalities were used in the derivation of
\eqref{pf.lem.mt1.d+d+d.tree.antigoal}). In other words, the following
claim holds:
\begin{statement}
\textit{Claim 1:} For each $i \in \set{1, 2, \ldots, k}$, exactly one
of $a_i$ and $a_{i+1}$ equals $v$.
\end{statement}

Now, let us assume that $a_1 = v$. Then, Claim 1 (applied to $i=1$)
shows that exactly one of $a_1$ and $a_2$ equals $v$. Hence, $a_2 \neq
v$ (since $a_1 = v$). But Claim 2 (applied to $i=2$) shows that
exactly one of $a_2$ and $a_3$ equals $v$. Hence, $a_3 = v$ (since
$a_2 \neq v$). We can continue this line of reasoning, thus showing
that $a_i = v$ for each odd $i \in \set{1, 2, \ldots, k+1}$ and that
$a_i \neq v$ for each even $i \in \set{1, 2, \ldots, k+1}$.
In particular, this shows that $a_{k+1} \neq v$ (since $k+1$ is even
(since $k$ is odd)).
But this contradicts $a_{k+1} = a_1 = v$.

Thus, we have obtained a contradiction under the assumption that
$a_1 = v$. But we could similarly have obtained a contradiction under
the assumption that $a_1 \neq v$ (indeed, the very same argument would
have worked, except that the roles of ``being equal to $v$'' and
``being not equal to $v$'' would be interchanged). Thus, we always
have a contradiction. This completes the proof of
Lemma~\ref{lem.mt1.d+d+d.tree}.
\end{proof}

\subsubsection{The generalization derived from the problem}

As already mentioned, Exercise~\ref{exe.mt1.d+d+d} can be obtained
as a particular case of Theorem~\ref{thm.mt1.d+d+d.gene} for $k = 3$.
Conversely, we can also prove Theorem~\ref{thm.mt1.d+d+d.gene} using
a fairly simple induction argument that relies on
Exercise~\ref{exe.mt1.d+d+d} and Lemma~\ref{lem.mt1.distances-metric}.
Unlike Sasha Pevzner's proof above, this does not give a new solution
to Exercise~\ref{exe.mt1.d+d+d}, but might still be of interest.

\begin{proof}[Proof of Theorem~\ref{thm.mt1.d+d+d.gene} using
Exercise~\ref{exe.mt1.d+d+d} and
Lemma~\ref{lem.mt1.distances-metric}.]
Write the graph $G$ in the form $G = \tup{V, E}$.
Thus, $\verts{G} = V$ and $\edges{G} = E$.
Set $n = \abs{V}$.
Also, set $n' = n-1$.

We shall first show that for each $j \in \NN$ satisfying $2j \leq k$,
we have
\begin{equation}
\sum_{i=1}^{2j} d \tup{a_i, a_{i+1}}
\leq 2j n' - d \tup{a_{2j+1}, a_1} .
\label{pf.thm.mt1.d+d+d.gene.indproof.1}
\end{equation}

\begin{proof}[Proof of \eqref{pf.thm.mt1.d+d+d.gene.indproof.1}:]
We shall prove \eqref{pf.thm.mt1.d+d+d.gene.indproof.1} by induction
over $j$.

\textit{Induction base:} The inequality
\eqref{pf.thm.mt1.d+d+d.gene.indproof.1} holds for
$j = 0$\ \ \ \ \footnote{\textit{Proof.} We have
$d \tup{\underbrace{a_{2\cdot 0+1}}_{= a_1}, a_1}
= d \tup{a_1, a_1} = 0$
(by Lemma~\ref{lem.mt1.distances-metric} \textbf{(a)}), so that
$\underbrace{2 \cdot 0 n'}_{= 0}
- \underbrace{d \tup{a_1, a_{2\cdot 0+1}}}_{= 0}
= 0 - 0 = 0$. Now,
$\sum_{i=1}^{2\cdot 0} d \tup{a_i, a_{i+1}}
= \tup{\text{empty sum}} = 0
\leq 0 = 2 \cdot 0 n' - d \tup{a_{2\cdot 0+1}, a_1}$.
In other words, the inequality
\eqref{pf.thm.mt1.d+d+d.gene.indproof.1} holds for $j = 0$.}.
Hence, the induction base is complete.

\textit{Induction step:} Fix a positive $J \in \NN$ satisfying
$2J \leq k$.
Assume that \eqref{pf.thm.mt1.d+d+d.gene.indproof.1} holds for
$j = J-1$.
We must then show that \eqref{pf.thm.mt1.d+d+d.gene.indproof.1} holds
for $j = J$.

Since $J$ is a positive integer, we have $J-1 \in \NN$. Moreover,
$2 \tup{J-1} \leq 2J \leq k$. Hence,
\eqref{pf.thm.mt1.d+d+d.gene.indproof.1} is applicable to $j = J-1$.
Since we have assumed that \eqref{pf.thm.mt1.d+d+d.gene.indproof.1}
holds for $j = J-1$, we thus obtain
\begin{align}
\sum_{i=1}^{2 \tup{J-1}} d \tup{a_i, a_{i+1}}
&\leq 2\tup{J-1} n' - d \tup{a_{2\tup{J-1}+1}, a_1} \nonumber \\
&= 2\tup{J-1} n' - d \tup{a_{2J-1}, a_1}
\label{pf.thm.mt1.d+d+d.gene.indproof.1.pf.1}
\end{align}
(since $2\tup{J-1}+1 = 2J-1$).

Lemma~\ref{lem.mt1.distances-metric} \textbf{(c)} (applied to
$u = a_{2J-1}$, $v = a_1$ and $w = a_{2J}$) yields
$d \tup{a_{2J-1}, a_1} + d \tup{a_1, a_{2J}}
\geq d \tup{a_{2J-1}, a_{2J}}$.
Hence,
\[
d \tup{a_{2J-1}, a_{2J}}
\leq d \tup{a_{2J-1}, a_1} + d \tup{a_1, a_{2J}} .
\]

But Exercise~\ref{exe.mt1.d+d+d} (applied to $a = a_{2J}$,
$b = a_{2J+1}$ and $c = a_1$) yields
\[
d \tup{a_{2J+1}, a_1} + d \tup{a_1, a_{2J}} + d \tup{a_{2J}, a_{2J+1}}
\leq 2 \underbrace{\abs{V}}_{= n} - 2 = 2 n - 2
= 2 \underbrace{\tup{n-1}}_{= n'} = 2 n' .
\]
Subtracting $d \tup{a_{2J+1}, a_1}$ from both sides of this
inequality, we obtain
\[
d \tup{a_1, a_{2J}} + d \tup{a_{2J}, a_{2J+1}}
\leq 2 n' - d \tup{a_{2J+1}, a_1} .
\]

Now,
\begin{align*}
\sum_{i=1}^{2J} d \tup{a_i, a_{i+1}}
&= \sum_{i=1}^{2J-2} d \tup{a_i, a_{i+1}}
        + d \tup{a_{2J-1}, a_{2J}} + d \tup{a_{2J}, a_{2J+1}} \\
&= \underbrack{\sum_{i=1}^{2 \tup{J-1}} d \tup{a_i, a_{i+1}}}
              {\leq 2\tup{J-1} n' - d \tup{a_{2J-1}, a_1} \\
               \text{(by \eqref{pf.thm.mt1.d+d+d.gene.indproof.1.pf.1})}}
        + \underbrack{d \tup{a_{2J-1}, a_{2J}}}
                     {\leq d \tup{a_{2J-1}, a_1} + d \tup{a_1, a_{2J}}}
        + d \tup{a_{2J}, a_{2J+1}} \\
&\qquad \left(\text{since } 2J-2 = 2 \tup{J-1} \right) \\
&\leq 2\tup{J-1} n' - d \tup{a_1, a_{2J-1}}
        + d \tup{a_1, a_{2J-1}} + d \tup{a_1, a_{2J}}
        + d \tup{a_{2J}, a_{2J+1}} \\
&= 2\tup{J-1} n'
        + \underbrack{d \tup{a_1, a_{2J}} + d \tup{a_{2J}, a_{2J+1}}}
                     {\leq 2 n' - d \tup{a_{2J+1}, a_1} } \\
&\leq \underbrace{2\tup{J-1} n' + 2n'}_{= 2J n'}
      - d \tup{a_{2J+1}, a_1}
 = 2J n' - d \tup{a_{2J+1}, a_1} .
\end{align*}
In other words, \eqref{pf.thm.mt1.d+d+d.gene.indproof.1} holds
for $j = J$.
This completes the induction step.
Thus, \eqref{pf.thm.mt1.d+d+d.gene.indproof.1} is proven by induction.
\end{proof}

We know that $k$ is an odd nonnegative integer.
Hence, there exists some $j \in \NN$ satisfying $k = 2j+1$.
Consider this $j$.
From $2j+1 = k$, we obtain $a_{\tup{2j+1}+1} = a_{k+1} = a_1$.
Also, $2j = k-1$ (since $k = 2j+1$) and
$n' = \underbrace{n}_{= \abs{V}} - 1
= \abs{\underbrace{V}_{= \verts{G}}} - 1
= \abs{\verts{G}} - 1$.

From $k = 2j+1$, we obtain
\begin{align*}
\sum_{i=1}^k d \tup{a_i, a_{i+1}}
&= \sum_{i=1}^{2j+1} d \tup{a_i, a_{i+1}}
 = \underbrack{\sum_{i=1}^{2j} d \tup{a_i, a_{i+1}}}
              {\leq 2j n' - d \tup{a_{2j+1}, a_1} \\
               \text{(by \eqref{pf.thm.mt1.d+d+d.gene.indproof.1})}}
         + d \tup{a_{2j+1}, \underbrace{a_{\tup{2j+1}+1}}_{= a_1}} \\
&\leq 2j n' - d \tup{a_{2j+1}, a_1} + d \tup{a_{2j+1}, a_1}
 = \underbrace{2j}_{= k-1} \underbrace{n'}_{= \abs{\verts{G}} - 1} \\
&= \tup{k-1} \tup{\abs{\verts{G}} - 1} .
\end{align*}
This proves Theorem~\ref{thm.mt1.d+d+d.gene}.
\end{proof}

\subsection{Exercise \ref{exe.mt1.cyctree}: two steps away from a
forest}

We prepare for Exercise~\ref{exe.mt1.cyctree} by showing several
lemmas.

\begin{lemma} \label{lem.mt1.cyctree.H-has-cycle}
Let $H$ be a multigraph such that
$\abs{\edges{H}} \geq \abs{\verts{H}}$.
Assume that $H$ has at least one vertex.
Then, $H$ contains a cycle.
\end{lemma}

\begin{proof}[Proof of Lemma~\ref{lem.mt1.cyctree.H-has-cycle}
(sketched).]
Assume the contrary. Hence, $H$
contains no cycle, i.e., is a forest. Therefore, Corollary 20 from
\href{http://www-users.math.umn.edu/~dgrinber/5707s17/5707lec9.pdf}{lecture 9}
shows that\footnote{Recall that $b_0 \tup{H}$ denotes the number of
connected components of $H$. Since $H$ has at least one connected
component (because $H$ has at least one vertex), we have
$b_0 \tup{H} > 0$.}
$\abs{\edges{H}} = \abs{\verts{H}} - b_0 \tup{H} < \abs{\verts{H}}$
(since $b_0 \tup{H} > 0$), which contradicts
$\abs{\edges{H}} \geq \abs{\verts{H}}$.
Hence, our assumption was false, qed.
\end{proof}

In the following, we shall use the Iverson bracket notation.
In other words, for any logical statement $\mathcal{A}$, we set
$\ive{\mathcal{A}} =
\begin{cases}
1, & \text{if }\mathcal{A}\text{ is true};\\
0, & \text{if }\mathcal{A}\text{ is false}
\end{cases}$.

\begin{verlong}
\begin{lemma} \label{lem.mt1.cyctree.iverson}
Let $G = \tup{V, E, \phi}$ be a multigraph. Then,
$\deg v = \sum_{e \in E} \ive{v \in \phi\tup{e}}$.
\end{lemma}

\begin{proof}[Proof of Lemma~\ref{lem.mt1.cyctree.iverson}.]
Recall that $\deg v$ is the number of all edges of $G$ containing $v$.
In other words,
$\deg v = \tup{\text{the number of all } e \in E \text{ such that }
                v \in \phi\tup{e}}$.
But
\begin{align*}
\sum_{e \in E} \ive{v \in \phi\tup{e}}
&= \sum_{\substack{e \in E; \\ v \in \phi\tup{e}}}
            \underbrack{\ive{v \in \phi\tup{e}}}
                       {= 1 \\ \text{(since } v \in \phi\tup{e}
                            \text{)}}
 + \sum_{\substack{e \in E; \\ \text{not } v \in \phi\tup{e}}}
            \underbrack{\ive{v \in \phi\tup{e}}}
                       {= 0 \\ \text{(since we do not have }
                            v \in \phi\tup{e} \text{)}} \\
&\qquad \left(\text{since each } e \in E \text{ either satisfies }
                v \in \phi\tup{e} \text{ or not } \right) \\
&= \sum_{\substack{e \in E; \\ v \in \phi\tup{e}}} 1
 + \underbrack{\sum_{\substack{e \in E; \\ \text{not }
                            v \in \phi\tup{e}}} 0}
              {= 0}
=  \sum_{\substack{e \in E; \\ v \in \phi\tup{e}}} 1 \\
&= \tup{\text{the number of all } e \in E \text{ such that }
                v \in \phi\tup{e}} \cdot 1 \\
&= \tup{\text{the number of all } e \in E \text{ such that }
                v \in \phi\tup{e}}
= \deg v.
\end{align*}
This proves Lemma~\ref{lem.mt1.cyctree.iverson}.
\end{proof}
\end{verlong}

Let us introduce one more notation:

\begin{itemize}
\item If $A$ and $B$ are two sets, then $A \bigtriangleup B$ shall
denote the
\href{https://en.wikipedia.org/wiki/Symmetric_difference}{\textit{symmetric difference}}
of $A$ and $B$; this is defined as the
set $\tup{A \cup B} \setminus \tup{A \cap B}
= \tup{A \setminus B} \cup \tup{B \setminus A}$. In other words,
$A \bigtriangleup B$ is the set of all elements that belong to
\textbf{exactly one} of $A$ and $B$. (This is the operation on sets
that corresponds to
\href{https://en.wikipedia.org/wiki/Exclusive_or}{the logical operation XOR}.)
\end{itemize}

\begin{lemma} \label{lem.mt1.cyctree.AsymB}
Let $G = \tup{V, E, \phi}$ be a multigraph. Let $A$ and $B$ be two
subsets of $E$. Let $v \in V$. Then,
\[
\deg_{\tup{V, A \bigtriangleup B, \phi\mid_{A \bigtriangleup B}}} v
= \deg_{\tup{V, A, \phi\mid_A}} v
  + \deg_{\tup{V, B, \phi\mid_B}} v
  - 2 \deg_{\tup{V, A \cap B, \phi\mid_{A \cap B}}} v.
\]
\end{lemma}

\begin{proof}[Proof of Lemma~\ref{lem.mt1.cyctree.AsymB}.]
Each edge $e \in E$ satisfies
\begin{equation}
\ive{e \in A \bigtriangleup B}
= \ive{e \in A} + \ive{e \in B} - 2 \ive{e \in A \cap B}
\label{pf.lem.mt1.cyctree.AsymB.1}
\end{equation}
\footnote{\textit{Proof of \eqref{pf.lem.mt1.cyctree.AsymB.1}.}
Let $e \in E$. We must prove the equality
\eqref{pf.lem.mt1.cyctree.AsymB.1}.
We are in one of the following four cases:
\begin{itemize}
\item \textit{Case 1:} We have $e \in A$ and $e \in B$.
\item \textit{Case 2:} We have $e \in A$ but not $e \in B$.
\item \textit{Case 3:} We have $e \in B$ but not $e \in A$.
\item \textit{Case 4:} We have neither $e \in A$ nor $e \in B$.
\end{itemize}
Let us prove \eqref{pf.lem.mt1.cyctree.AsymB.1} in Case 1. In this
case, we have $e \in A$ and $e \in B$. Thus, $e \in A \cap B$, so
that $e \notin \tup{A \cup B} \setminus \tup{A \cap B}
= A \bigtriangleup B$. Thus, $\ive{e \in A \bigtriangleup B} = 0$.
Also, $\ive{e \in A \cap B} = 1$ (since $e \in A \cap B$) and
$\ive{e \in A} = 1$ (since $e \in A$) and $\ive{e \in B} = 1$ (since
$e \in B$). Thus,
\[
\underbrace{\ive{e \in A}}_{= 1}
 + \underbrace{\ive{e \in B}}_{= 1}
 - 2 \underbrace{\ive{e \in A \cap B}}_{= 1}
= 1 + 1 - 2 \cdot 1 = 0 = \ive{e \in A \bigtriangleup B} .
\]
Hence, \eqref{pf.lem.mt1.cyctree.AsymB.1} is proven in Case 1.
Similarly, \eqref{pf.lem.mt1.cyctree.AsymB.1} can be proven in Case 2,
in Case 3, and in Case 4. Therefore,
\eqref{pf.lem.mt1.cyctree.AsymB.1} always holds. This completes the
proof of \eqref{pf.lem.mt1.cyctree.AsymB.1}.}.

But for each subset $F$ of $E$, we have
\begin{equation}
\deg_{\tup{V, F, \phi\mid_F}} v
= \sum_{\substack{e \in E;\ v \in \phi\tup{e}}} \ive{e \in F}
\label{pf.lem.mt1.cyctree.AsymB.2}
\end{equation}
\footnote{\textit{Proof of \eqref{pf.lem.mt1.cyctree.AsymB.2}.}
Let $F$ be a subset of $E$.
The definition of $\deg_{\tup{V, F, \phi\mid_F}} v$ shows that
$\deg_{\tup{V, F, \phi\mid_F}} v$ is the number of edges of the
multigraph $\tup{V, F, \phi\mid_F}$ that contain $v$. Since the edges
of the multigraph $\tup{V, F, \phi\mid_F}$ are the elements of $F$,
this rewrites as follows:
$\deg_{\tup{V, F, \phi\mid_F}} v$ is the number of elements of $F$
that contain $v$. In other words,
$\deg_{\tup{V, F, \phi\mid_F}} v$ is the number of all $e \in F$ that
satisfy $v \in \tup{\phi\mid_F}\tup{e}$.
In other words,
$\deg_{\tup{V, F, \phi\mid_F}} v$ is the number of all $e \in F$ that
satisfy $v \in \phi\tup{e}$ (because obviously, we have
$\tup{\phi\mid_F}\tup{e} = \phi\tup{e}$ for each $e \in F$).
In other words,
\begin{equation}
\deg_{\tup{V, F, \phi\mid_F}} v
= \tup{\text{the number of all } e \in F \text{ that satisfy }
          v \in \phi\tup{e}} .
\label{pf.lem.mt1.cyctree.AsymB.2.pf.1}
\end{equation}
But
\begin{align*}
\sum_{\substack{e \in E;\ v \in \phi\tup{e}}} \ive{e \in F}
&= \sum_{\substack{e \in E;\ v \in \phi\tup{e}; \  e \in F}}
            \underbrack{\ive{e \in F}}
                       {= 1 \\ \text{(since } e \in F \text{)}}
 + \sum_{\substack{e \in E;\ v \in \phi\tup{e};
                \ \text{not } e \in F}}
            \underbrack{\ive{e \in F}}
                       {= 0 \\ \text{(since we do not have }
                                e \in F \text{)}} \\\
&\qquad \left(\text{since each } e \in E \text{ either satisfies }
                    e \in F \text{ or does not}\right) \\
&= \sum_{\substack{e \in E;\ v \in \phi\tup{e}; \  e \in F}} 1
 + \underbrack{\sum_{\substack{e \in E;\ v \in \phi\tup{e};
                        \ \text{not } e \in F}} 0}
              {= 0}
\\
&= \sum_{\substack{e \in E;\ v \in \phi\tup{e}; \  e \in F}} 1 \\
&= \tup{\text{the number of all } e \in E \text{ that satisfy }
          v \in \phi\tup{e} \text{ and } e \in F} \cdot 1 \\
&= \tup{\text{the number of all } e \in E \text{ that satisfy }
          v \in \phi\tup{e} \text{ and } e \in F} \\
&= \tup{\text{the number of all } e \in F \text{ that satisfy }
          v \in \phi\tup{e}}
\end{align*}
(because the $e \in E$ that satisfy $e \in F$ are precisely the
$e \in F$). Comparing this with
\eqref{pf.lem.mt1.cyctree.AsymB.2.pf.1}, we obtain
\[
\deg_{\tup{V, F, \phi\mid_F}} v
= \sum_{\substack{e \in E;\ v \in \phi\tup{e}}} \ive{e \in F} .
\]
This proves \eqref{pf.lem.mt1.cyctree.AsymB.2}.}.
Applying this to $F = A \bigtriangleup B$, we obtain
\begin{align*}
\deg_{\tup{V, A \bigtriangleup B, \phi\mid_{A \bigtriangleup B}}} v
&= \sum_{\substack{e \in E;\ v \in \phi\tup{e}}}
         \underbrack{\ive{e \in A \bigtriangleup B}}
                    {= \ive{e \in A} + \ive{e \in B}
                         - 2 \ive{e \in A \cap B} \\
                     \text{(by \eqref{pf.lem.mt1.cyctree.AsymB.1})}}
         \\
&= \sum_{\substack{e \in E;\ v \in \phi\tup{e}}}
         \tup{\ive{e \in A} + \ive{e \in B} - 2 \ive{e \in A \cap B}}
         \\
&= \sum_{\substack{e \in E;\ v \in \phi\tup{e}}} \ive{e \in A}
   + \sum_{\substack{e \in E;\ v \in \phi\tup{e}}} \ive{e \in B}
   - 2 \sum_{\substack{e \in E;\ v \in \phi\tup{e}}}
                \ive{e \in A \cap B} .
\end{align*}
Comparing this with
\begin{align*}
&
\underbrack{\deg_{\tup{V, A, \phi\mid_A}} v}
           {= \sum_{\substack{e \in E;\ v \in \phi\tup{e}}}
                    \ive{e \in A} \\
            \text{(by \eqref{pf.lem.mt1.cyctree.AsymB.2}, applied to }
            F = A \text{)}}
+
\underbrack{\deg_{\tup{V, B, \phi\mid_B}} v}
           {= \sum_{\substack{e \in E;\ v \in \phi\tup{e}}}
                    \ive{e \in B} \\
            \text{(by \eqref{pf.lem.mt1.cyctree.AsymB.2}, applied to }
            F = B \text{)}}
-
2 \underbrack{\deg_{\tup{V, A \cap B, \phi\mid_{A \cap B}}} v}
             {= \sum_{\substack{e \in E;\ v \in \phi\tup{e}}}
                      \ive{e \in A \cap B} \\
              \text{(by \eqref{pf.lem.mt1.cyctree.AsymB.2}, applied to }
              F = A \cap B \text{)}} \\
&= \sum_{\substack{e \in E;\ v \in \phi\tup{e}}} \ive{e \in A}
   + \sum_{\substack{e \in E;\ v \in \phi\tup{e}}} \ive{e \in B}
   - 2 \sum_{\substack{e \in E;\ v \in \phi\tup{e}}}
                \ive{e \in A \cap B} ,
\end{align*}
we obtain
\[
\deg_{\tup{V, A \bigtriangleup B, \phi\mid_{A \bigtriangleup B}}} v
= \deg_{\tup{V, A, \phi\mid_A}} v
  + \deg_{\tup{V, B, \phi\mid_B}} v
  - 2 \deg_{\tup{V, A \cap B, \phi\mid_{A \cap B}}} v.
\]
This proves Lemma~\ref{lem.mt1.cyctree.AsymB}.
\end{proof}

\begin{lemma} \label{lem.mt1.cyctree.two-cycles}
Let $G = \tup{V, E, \phi}$ be a multigraph. Let $\mathbf{a}$ and
$\mathbf{b}$ be two cycles of $G$. Let $A$ be the set of the edges
of $\mathbf{a}$. Let $B$ be the set of the edges of $\mathbf{b}$.

\textbf{(a)} For each vertex $v \in V$, the number
$\deg_{\tup{V, A \bigtriangleup B, \phi\mid_{A \bigtriangleup B}}} v$
is even.

\textbf{(b)} If $A \neq B$, then the multigraph
$\tup{V, A \bigtriangleup B, \phi\mid_{A \bigtriangleup B}}$ has a
cycle.

\textbf{(c)} Let $\mathbf{c}$ be a cycle of the multigraph
$\tup{V, A \bigtriangleup B, \phi\mid_{A \bigtriangleup B}}$. Then,
at least one of the cycles $\mathbf{a}$, $\mathbf{b}$ and
$\mathbf{c}$ has length $\leq \dfrac{2 \abs{E}}{3}$.
\end{lemma}

\begin{proof}[Proof of Lemma~\ref{lem.mt1.cyctree.two-cycles}.]
\textbf{(a)} Let $v \in V$. Then,
$\deg_{\tup{V, A, \phi\mid_A}} v \equiv 0 \mod 2$
\ \ \ \ \footnote{\textit{Proof.} We are in one of the following two
cases:

\begin{itemize}
\item \textit{Case 1:} The vertex $v$ does not lie on the cycle
        $\mathbf{a}$.
\item \textit{Case 2:} The vertex $v$ lies on the cycle $\mathbf{a}$.
\end{itemize}

Let us consider Case 1 first. In this case, the vertex $v$ does not
lie on the cycle $\mathbf{a}$. Hence, no edge of $\mathbf{a}$ contains
$v$. In other words, no edge in $A$ contains $v$ (since the edges in
$A$ are precisely the edges of $\mathbf{a}$). In other words,
$\deg_{\tup{V, A, \phi\mid_A}} v = 0$. Hence,
$\deg_{\tup{V, A, \phi\mid_A}} v = 0 \equiv 0 \mod 2$. Thus,
$\deg_{\tup{V, A, \phi\mid_A}} v \equiv 0 \mod 2$ is proven in Case 1.

Let us next consider Case 2. In this case, the vertex $v$ lies on the
cycle $\mathbf{a}$. Hence, exactly two edges of $\mathbf{a}$ contain
$v$ (because a vertex on a cycle is entered exactly once by the cycle,
and exited exactly once by the cycle). These two edges are distinct
(since the edges of a cycle are always distinct). Thus, exactly two
distinct edges of $\mathbf{a}$ contain $v$. In other words, exactly
two edges in $A$ contain $v$ (since the edges in $A$ are precisely
the edges of $\mathbf{a}$). In other words,
$\deg_{\tup{V, A, \phi\mid_A}} v = 2$. Hence,
$\deg_{\tup{V, A, \phi\mid_A}} v = 2 \equiv 0 \mod 2$. Thus,
$\deg_{\tup{V, A, \phi\mid_A}} v \equiv 0 \mod 2$ is proven in Case 2.

We have now proven
$\deg_{\tup{V, A, \phi\mid_A}} v \equiv 0 \mod 2$ in each of the two
Cases 1 and 2. Hence,
$\deg_{\tup{V, A, \phi\mid_A}} v \equiv 0 \mod 2$ always holds.}.
Similarly,
$\deg_{\tup{V, B, \phi\mid_B}} v \equiv 0 \mod 2$.
Now, Lemma~\ref{lem.mt1.cyctree.AsymB} yields
\begin{align*}
\deg_{\tup{V, A \bigtriangleup B, \phi\mid_{A \bigtriangleup B}}} v
&= \underbrace{\deg_{\tup{V, A, \phi\mid_A}} v}_{\equiv 0 \mod 2}
   + \underbrace{\deg_{\tup{V, B, \phi\mid_B}} v}_{\equiv 0 \mod 2}
   - \underbrace{2}_{\equiv 0 \mod 2}
        \deg_{\tup{V, A \cap B, \phi\mid_{A \cap B}}} v \\
&\equiv 0 + 0 - 0 = 0 \mod 2 .
\end{align*}
In other words, the number
$\deg_{\tup{V, A \bigtriangleup B, \phi\mid_{A \bigtriangleup B}}} v$
is even.
This proves Lemma~\ref{lem.mt1.cyctree.two-cycles} \textbf{(a)}.

\textbf{(b)} It is well-known that if $X$ and $Y$ are two sets
satisfying $X \neq Y$, then $X \bigtriangleup Y \neq \varnothing$.
Applying this to $X = A$ and $Y = B$, we obtain
$A \bigtriangleup B \neq \varnothing$. Hence, the multigraph
$\tup{V, A \bigtriangleup B, \phi\mid_{A \bigtriangleup B}}$ has at
least one edge. Furthermore, each vertex of this multigraph has even
degree (because of
Lemma~\ref{lem.mt1.cyctree.two-cycles} \textbf{(a)}). Thus,
Exercise~\ref{exe.mt1.deg-cycle-2} (applied to
$\tup{V, A \bigtriangleup B, \phi\mid_{A \bigtriangleup B}}$ instead
of $G$) shows that this multigraph
$\tup{V, A \bigtriangleup B, \phi\mid_{A \bigtriangleup B}}$ has a
cycle.
This proves Lemma~\ref{lem.mt1.cyctree.two-cycles} \textbf{(b)}.

\textbf{(c)} Assume the contrary.
Thus, none of the cycles $\mathbf{a}$, $\mathbf{b}$ and
$\mathbf{c}$ has length $\leq \dfrac{2 \abs{E}}{3}$. In other words, each
of the cycles $\mathbf{a}$, $\mathbf{b}$ and
$\mathbf{c}$ has length $> \dfrac{2 \abs{E}}{3}$.

Let $C$ be the set of edges of $\mathbf{c}$. Then,
$C \subseteq A \bigtriangleup B$ (since $\mathbf{c}$ is a cycle
of the multigraph
$\tup{V, A \bigtriangleup B, \phi\mid_{A \bigtriangleup B}}$). Hence,
\begin{align*}
\abs{C}
&\leq \abs{\underbrack{A \bigtriangleup B}
                      {= \tup{A \cup B} \setminus \tup{A \cap B}}}
= \abs{\tup{A \cup B} \setminus \tup{A \cap B}} \\
&= \abs{A \cup B} - \underbrack{\abs{A \cap B}}
                              {= \abs{A} + \abs{B} - \abs{A \cup B}}
\qquad
\left(\text{since } A \cap B \subseteq A \subseteq A \cup B \right) \\
&= \abs{A \cup B} - \tup{\abs{A} + \abs{B} - \abs{A \cup B}}
= 2 \abs{\underbrace{A \cup B}_{\subseteq E}} - \abs{A} - \abs{B}
\leq 2 \abs{E} - \abs{A} - \abs{B} .
\end{align*}
Adding $\abs{A} + \abs{B}$ to both sides of this inequality, we obtain
$\abs{A} + \abs{B} + \abs{C} \leq 2 \abs{E}$.

But
\begin{align*}
\abs{A}
&=
\tup{\text{the number of all elements of } A }
=
\tup{\text{the number of all edges of } \mathbf{a}} \\
& \qquad
\left(\text{since the elements of } A \text{ are the edges of }
        \mathbf{a}\text{, and since these edges are all distinct}
\right) \\
&= \tup{\text{the length of } \mathbf{a}}
> \dfrac{2 \abs{E}}{3}
\end{align*}
(since the cycle $\mathbf{a}$ has length $> \dfrac{2 \abs{E}}{3}$).
Similarly, $\abs{B} > \dfrac{2 \abs{E}}{3}$ and
$\abs{C} > \dfrac{2 \abs{E}}{3}$. Adding these three inequalities
together, we obtain
\[
\abs{A} + \abs{B} + \abs{C}
> \dfrac{2 \abs{E}}{3} + \dfrac{2 \abs{E}}{3}
        + \dfrac{2 \abs{E}}{3}
= 2 \abs{E} .
\]
This contradicts $\abs{A} + \abs{B} + \abs{C} \leq 2 \abs{E}$.
This contradiction shows that our assumption was wrong. Hence,
Lemma~\ref{lem.mt1.cyctree.two-cycles} \textbf{(c)} is proven.
\end{proof}

\begin{lemma} \label{lem.mt1.cyctree.for-multigraphs}
Let $G = \tup{V, E, \phi}$ be a multigraph such that
$\abs{E} > \abs{V}$. Let $n = \abs{V}$. Then,
$G$ has a cycle of length $\leq \dfrac{2n+2}{3}$.
\end{lemma}

\begin{proof}[Proof of Lemma~\ref{lem.mt1.cyctree.for-multigraphs}
(sketched).]
Recall that $\abs{E} > \abs{V}$. Thus,
$\abs{E} \geq \abs{V} + 1$. Hence, we can WLOG assume that
$\abs{E} = \abs{V} + 1$ (since otherwise, we can keep deleting edges
from $G$ until $\abs{E} = \abs{V} + 1$ holds; if we can find a cycle
of length $\leq \dfrac{2n+2}{3}$ after that, then we clearly also get
such a cycle in the original graph).

Thus, $\abs{E} = \underbrack{\abs{V}}{= n} + 1 = n+1$.

The multigraph $G$ has at least one edge (since $\abs{E} > \abs{V}
\geq 0$), and therefore has at least one vertex. Also,
$\abs{\edges{G}} = \abs{E} \geq \abs{V} = \abs{\verts{G}}$.
Hence, Lemma~\ref{lem.mt1.cyctree.H-has-cycle}
(applied to $H = G$) yields that $G$ contains a
cycle. Fix such a cycle, and denote it by $\mathbf{a}$.
Fix an edge $a$ of $\mathbf{a}$.

Removing the edge $a$ from $G$ yields a new multigraph $G'$, which
has one fewer edge than $G$. Thus, this new graph $G'$ satisfies
$\abs{\edges{G'}} = \abs{E} - 1 \geq \abs{V}$ (since $\abs{E} >
\abs{V}$). This rewrites as
$\abs{\edges{G'}} \geq \abs{\verts{G'}}$
(since $V = \verts{G'}$).
Consequently, Lemma~\ref{lem.mt1.cyctree.H-has-cycle}
(applied to $H = G'$) yields that $G'$
contains a cycle. Fix such a cycle, and denote it by $\mathbf{b}$.
Of course, $\mathbf{b}$ is a cycle of $G$ as well (since $G'$ is a
subgraph of $G$).

Let $A$ be the set of the edges of $\mathbf{a}$, and let $B$ be the
set of edges of $\mathbf{b}$. Then, $A \neq B$ (since
$a \in A$ but $a \notin B$). Hence,
Lemma~\ref{lem.mt1.cyctree.two-cycles} \textbf{(b)} shows that the
multigraph
$\tup{V, A \bigtriangleup B, \phi\mid_{A \bigtriangleup B}}$
has a cycle. Fix such a
cycle, and denote it by $\mathbf{c}$. Clearly, $\mathbf{c}$ is a cycle
of $G$ (since
$\tup{V, A \bigtriangleup B, \phi\mid_{A \bigtriangleup B}}$ is a
subgraph of $G$).
Lemma~\ref{lem.mt1.cyctree.two-cycles} \textbf{(c)} shows that
at least one of the cycles $\mathbf{a}$, $\mathbf{b}$ and
$\mathbf{c}$ has length $\leq \dfrac{2 \abs{E}}{3}$.
Since $2 \underbrack{\abs{E}}{= n+1} = 2 \tup{n+1} = 2n+2$, this
rewrites as follows: At least
one of the cycles $\mathbf{a}$, $\mathbf{b}$ and
$\mathbf{c}$ has length $\leq \dfrac{2n + 2}{3}$. Hence, $G$ has a
cycle of length $\leq \dfrac{2n + 2}{3}$ (since all of
$\mathbf{a}$, $\mathbf{b}$ and $\mathbf{c}$ are cycles of $G$).
This proves Lemma~\ref{lem.mt1.cyctree.for-multigraphs}.
\end{proof}

\begin{exercise} \label{exe.mt1.cyctree}
Let $G = \tup{V, E}$ be a simple graph such that $\abs{E} > \abs{V}$.
Prove that $G$ has a cycle of length $\leq \dfrac{2n+2}{3}$,
where $n = \abs{V}$.
\end{exercise}

\begin{proof}[Solution sketch to Exercise~\ref{exe.mt1.cyctree}.]
Recall that each simple graph can be viewed as a multigraph in an
obvious way. Thus, we can view $G$ as a multigraph this way.
Hence, Exercise~\ref{exe.mt1.cyctree} follows immediately from
Lemma~\ref{lem.mt1.cyctree.for-multigraphs}.
\end{proof}

\begin{noncompile}
OLD version of the solution:

\begin{proof}[Hints to Exercise~\ref{exe.mt1.cyctree}.]
Recall the following fact:
\begin{statement}
\textit{Fact 1:} If $H$ is a multigraph with
$\abs{\edges{H}} \geq \abs{\verts{H}}$ and having at least one vertex,
then $H$ contains a cycle.
\end{statement}
(One way to prove Fact 1 is to assume the contrary. Hence, $H$
contains no cycle, i.e., is a forest. But then, Corollary 20 from
\href{http://www-users.math.umn.edu/~dgrinber/5707s17/5707lec9.pdf}{lecture 9}
shows that
$\abs{\edges{H}} = \abs{\verts{H}} - b_0 \tup{H} < \abs{\verts{H}}$
(since $b_0 \tup{H} > 0$), which contradicts
$\abs{\edges{H}} \geq \abs{\verts{H}}$.)

Recall that $\abs{E} > \abs{V}$. Thus,
$\abs{E} \geq \abs{V} + 1$. Hence, we can WLOG assume that
$\abs{E} = \abs{V} + 1$ (since otherwise, we can keep deleting edges
from $G$ until $\abs{E} = \abs{V} + 1$ holds; if we can find a cycle
of length $\leq \dfrac{2n+2}{3}$ after that, then we clearly also get
such a cycle in the original graph).

Now, $G$ clearly has at least one vertex (since $\abs{E} > \abs{V}
\geq 0$) and satisfies
$\abs{\edges{G}} = \abs{E} \geq \abs{V} = \abs{\verts{G}}$.
Hence, Fact 1 (applied to $H = G$) yields that $G$ contains a
cycle. Fix such a cycle, and denote it by $\mathbf{a}$.
Fix an edge $a$ of $\mathbf{a}$.

Removing the edge $a$ from $G$ yields a new simple graph $G'$, which
has one fewer edge than $G$. Thus, this new graph $G'$ satisfies
$\abs{\edges{G'}} = \abs{E} - 1 \geq \abs{V}$ (since $\abs{E} >
\abs{V}$). This rewrites as
$\abs{\edges{G'}} \geq \abs{\verts{G'}}$.
Consequently, Fact 1 (applied to $H = G'$) yields that $G'$
contains a cycle. Fix such a cycle, and denote it by $\mathbf{b}$.
Of course, $\mathbf{b}$ is a cycle of $G$ as well (since $G'$ is a
subgraph of $G$).

If $X$ and $Y$ are two sets, then $X \bigtriangleup Y$ denotes the
\href{https://en.wikipedia.org/wiki/Symmetric_difference}{\textit{symmetric difference}}
of $X$ and $Y$; this is defined as the
set $\tup{X \cup Y} \setminus \tup{X \cap Y}
= \tup{X \setminus Y} \cup \tup{Y \setminus X}$. In other words,
$X \bigtriangleup Y$ is the set of all elements that belong to
\textbf{exactly one} of $X$ and $Y$. (This is the operation on sets
that corresponds to
\href{https://en.wikipedia.org/wiki/Exclusive_or}{the logical operation XOR}.)

Let $A$ be the set of the edges of $\mathbf{a}$, and let $B$ be the
set of edges of $\mathbf{b}$. Then, $a \in A \bigtriangleup B$ (since
$a \in A$ but $a \notin B$). Hence, the graph
$\tup{V, A \bigtriangleup B}$ has at least one edge (namely, $a$).

Moreover, each vertex of this graph
$\tup{V, A \bigtriangleup B}$ has even degree. (In fact, the degree
of a vertex $v$ of this graph can be computed as follows: Start with
$0$. Then, add $2$ if $v$ lies on the cycle $\mathbf{a}$. Then, add
$2$ if $v$ lies on the cycle $\mathbf{b}$. Then, subtract $2$ times
the number of edges that contain $v$ that belong to both cycles
$\mathbf{a}$ and $\mathbf{b}$. Clearly, the result is an even
integer.)

Hence, Exercise~\ref{exe.mt1.deg-cycle-2} (applied to
$\tup{V, A \bigtriangleup B}$ instead of $G$) shows that the graph
$\tup{V, A \bigtriangleup B}$ has a cycle. Fix such a cycle, and
denote it by $\mathbf{c}$. Clearly, $\mathbf{c}$ is a cycle of $G$
(since $\tup{V, A \bigtriangleup B}$ is a subgraph of $G$). Let $C$
be the set of edges of $\mathbf{c}$.

We have $C \subseteq A \bigtriangleup B$. Hence, no edge can belong
to all three sets $A$, $B$ and $C$ at the same time (because if it
belonged to $A$ and $B$, then it would not belong to
$A \bigtriangleup B$, and therefore would not belong to $C$ either,
since $C \subseteq A \bigtriangleup B$).

Hence,
$\abs{A} + \abs{B} + \abs{C} \leq 2 \abs{A \cup B \cup C}$
\ \ \ \ \footnote{\textit{Proof.} The sum
$\abs{A} + \abs{B} + \abs{C}$ counts the edges lying in
$A \cup B \cup C$, but it counts some of them twice and some thrice:
Namely, an edge is counted thrice if it belongs to all three sets
$A$, $B$ and $C$ at the same time;
otherwise, it is counted at most twice. Since no edge can belong to
all three sets $A$, $B$ and $C$ at the same time, we realize that no
edge is counted thrice. Hence, each edge is counted at most twice.
Thus, the total sum $\abs{A} + \abs{B} + \abs{C}$ is at most
$2 \abs{A \cup B \cup C}$. Qed.}. Therefore,
\[
\abs{A} + \abs{B} + \abs{C}
\leq 2 \abs{\underbrace{A \cup B \cup C}_{\subseteq E}}
\leq 2 \underbrace{\abs{E}}_{= \abs{V} + 1}
= 2 \tup{\underbrace{\abs{V}}_{= n} + 1}
= 2 \tup{n+1} = 2n+2.
\]

Now, let us assume (for the sake of contradiction) that
$G$ has no cycle of length $\leq \dfrac{2n+2}{3}$. Hence, each cycle
of $G$ has length $> \dfrac{2n+2}{3}$. In particular, $\mathbf{a}$
has length $> \dfrac{2n+2}{3}$.

But $A$ is the set of the edges of the cycle $\mathbf{a}$. Hence, the
length of $\mathbf{a}$ is $\abs{A}$. Thus, $\abs{A} > \dfrac{2n+2}{3}$
(since $\mathbf{a}$ has length $> \dfrac{2n+2}{3}$). Similarly,
$\abs{B} > \dfrac{2n+2}{3}$ and $\abs{C} > \dfrac{2n+2}{3}$. Adding
these three inequalities together, we obtain
\[
\abs{A} + \abs{B} + \abs{C}
> \dfrac{2n+2}{3} + \dfrac{2n+2}{3} + \dfrac{2n+2}{3}
= 2n+2 .
\]
This contradicts $\abs{A} + \abs{B} + \abs{C} \leq 2n+2$.
This contradiction proves that our assumption was false, qed.
\end{noncompile}

[\textit{Remarks:}

\begin{itemize}
\item A statement fairly close to
      Exercise~\ref{exe.mt1.cyctree} appears in
      \cite[Lemma 6]{BoCaDu13}.

\item The bound $\dfrac{2n+2}{3}$ cannot be improved.
      For a graph $G$ which achieves this bound, see
      Exercise 2 \textbf{(c)} on
      \href{http://www-users.math.umn.edu/~dgrinber/5707s17/hw3.pdf}{homework set 3}.
      A slight modification of this construction allows us to
      find a graph $G$ with $n$ vertices achieving the bound
      $\floor{\dfrac{2n+2}{3}}$ for each $n > 3$.

\item We have generalized Exercise~\ref{exe.mt1.cyctree} to
      Lemma~\ref{lem.mt1.cyctree.for-multigraphs} by replacing
      the simple graph $G$ by a multigraph. However, this
      generalization does not add any significant new power to
      the statement, because each multigraph with no two
      parallel edges (= two distinct edges $e_1$ and $e_2$
      satisfying $\phi\tup{e_1} = \phi\tup{e_2}$) can be
      regarded as a simple graph (as long as we are willing to
      forget the names of the edges, which in our case is
      harmless), whereas Lemma~\ref{lem.mt1.cyctree.for-multigraphs}
      holds obviously for a multigraph with two parallel edges
      (in fact, two parallel edges form a cycle of length $2$).
      Nevertheless, I find the generalization worth stating,
      since it appears to me that the setting of multigraphs
      is more natural for this exercise.

\item Given two integers $k \geq 0$ and $n > 1$, we can
      define $\rho\tup{n, k}$ to be the smallest integer such
      that each multigraph $G$ satisfying
      $\abs{\edges{G}} \geq \abs{\verts{G}} + k$ and
      $\abs{\verts{G}} = n$ must have a cycle of length
      $\leq \rho\tup{n, k}$. Can we compute this
      $\rho\tup{n, k}$, or at least find a good upper bound on
      it? From Lemma~\ref{lem.mt1.cyctree.H-has-cycle} (and the
      example of the cycle graph $C_n$), we can easily obtain
      $\rho\tup{n, 0} = n$. From
      Lemma~\ref{lem.mt1.cyctree.for-multigraphs} (and an
      example of a graph $G$ achieving the bound), we obtain
      $\rho\tup{n, 1} = \floor{\dfrac{2n+2}{3}}$ for each
      $n > 3$. What can we say about $\rho\tup{n, 2}$ ?
      The notion of
      \href{http://www.win.tue.nl/~aeb/drg/graphs/cages/cages.html}{cages}
      seems relevant (even though not directly applicable).
\end{itemize}
]

\begin{thebibliography}{999999}
\bibitem[BoCaDu13]{BoCaDu13}
Prosenjit Bose, Paz Carmi, Stephane Durocher,
\textit{Bounding the locality of distributed routing
algorithms},
Distrib. Comput. (2013) 26, pp. 39--58.
\newline A preprint can be found at
\url{https://pdfs.semanticscholar.org/424b/649b86c848cde7072c4a3500ecb40119e442.pdf}
\end{thebibliography}


\end{document}