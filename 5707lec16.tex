\documentclass[numbers=enddot,12pt,final,onecolumn,notitlepage]{scrartcl}%
\usepackage[headsepline,footsepline,manualmark]{scrlayer-scrpage}
\usepackage[all,cmtip]{xy}
\usepackage{amssymb}
\usepackage{amsmath}
\usepackage{amsthm}
\usepackage{framed}
\usepackage{comment}
\usepackage{color}
\usepackage{hyperref}
\usepackage{ifthen}
\usepackage[sc]{mathpazo}
\usepackage[T1]{fontenc}
\usepackage{needspace}
\usepackage{tabls}
\usepackage{amsfonts}
\usepackage{graphicx}%
\setcounter{MaxMatrixCols}{30}
%TCIDATA{OutputFilter=latex2.dll}
%TCIDATA{Version=5.50.0.2960}
%TCIDATA{LastRevised=Friday, April 27, 2018 10:36:50}
%TCIDATA{<META NAME="GraphicsSave" CONTENT="32">}
%TCIDATA{<META NAME="SaveForMode" CONTENT="1">}
%TCIDATA{BibliographyScheme=Manual}
%BeginMSIPreambleData
\providecommand{\U}[1]{\protect\rule{.1in}{.1in}}
%EndMSIPreambleData
\newcounter{exer}
\theoremstyle{definition}
\newtheorem{theo}{Theorem}[section]
\newenvironment{theorem}[1][]
{\begin{theo}[#1]\begin{leftbar}}
{\end{leftbar}\end{theo}}
\newtheorem{lem}[theo]{Lemma}
\newenvironment{lemma}[1][]
{\begin{lem}[#1]\begin{leftbar}}
{\end{leftbar}\end{lem}}
\newtheorem{prop}[theo]{Proposition}
\newenvironment{proposition}[1][]
{\begin{prop}[#1]\begin{leftbar}}
{\end{leftbar}\end{prop}}
\newtheorem{defi}[theo]{Definition}
\newenvironment{definition}[1][]
{\begin{defi}[#1]\begin{leftbar}}
{\end{leftbar}\end{defi}}
\newtheorem{remk}[theo]{Remark}
\newenvironment{remark}[1][]
{\begin{remk}[#1]\begin{leftbar}}
{\end{leftbar}\end{remk}}
\newtheorem{coro}[theo]{Corollary}
\newenvironment{corollary}[1][]
{\begin{coro}[#1]\begin{leftbar}}
{\end{leftbar}\end{coro}}
\newtheorem{conv}[theo]{Convention}
\newenvironment{condition}[1][]
{\begin{conv}[#1]\begin{leftbar}}
{\end{leftbar}\end{conv}}
\newtheorem{quest}[theo]{Question}
\newenvironment{algorithm}[1][]
{\begin{quest}[#1]\begin{leftbar}}
{\end{leftbar}\end{quest}}
\newtheorem{warn}[theo]{Warning}
\newenvironment{conclusion}[1][]
{\begin{warn}[#1]\begin{leftbar}}
{\end{leftbar}\end{warn}}
\newtheorem{conj}[theo]{Conjecture}
\newenvironment{conjecture}[1][]
{\begin{conj}[#1]\begin{leftbar}}
{\end{leftbar}\end{conj}}
\newtheorem{exam}[theo]{Example}
\newenvironment{example}[1][]
{\begin{exam}[#1]\begin{leftbar}}
{\end{leftbar}\end{exam}}
\newtheorem{exmp}[exer]{Exercise}
\newenvironment{exercise}[1][]
{\begin{exmp}[#1]\begin{leftbar}}
{\end{leftbar}\end{exmp}}
\newenvironment{statement}{\begin{quote}}{\end{quote}}
\iffalse
\newenvironment{proof}[1][Proof]{\noindent\textbf{#1.} }{\ \rule{0.5em}{0.5em}}
\fi
\let\sumnonlimits\sum
\let\prodnonlimits\prod
\let\cupnonlimits\bigcup
\let\capnonlimits\bigcap
\renewcommand{\sum}{\sumnonlimits\limits}
\renewcommand{\prod}{\prodnonlimits\limits}
\renewcommand{\bigcup}{\cupnonlimits\limits}
\renewcommand{\bigcap}{\capnonlimits\limits}
\setlength\tablinesep{3pt}
\setlength\arraylinesep{3pt}
\setlength\extrarulesep{3pt}
\voffset=0cm
\hoffset=-0.7cm
\setlength\textheight{22.5cm}
\setlength\textwidth{15.5cm}
\newenvironment{verlong}{}{}
\newenvironment{vershort}{}{}
\newenvironment{noncompile}{}{}
\excludecomment{verlong}
\includecomment{vershort}
\excludecomment{noncompile}
\newcommand{\id}{\operatorname{id}}
\newcommand{\rev}{\operatorname{rev}}
\newcommand{\conncomp}{\operatorname{conncomp}}
\newcommand{\conn}{\operatorname{conn}}
\newcommand{\NN}{\mathbb{N}}
\newcommand{\ZZ}{\mathbb{Z}}
\newcommand{\QQ}{\mathbb{Q}}
\newcommand{\RR}{\mathbb{R}}
\newcommand{\powset}[2][]{\ifthenelse{\equal{#2}{}}{\mathcal{P}\left(#1\right)}{\mathcal{P}_{#1}\left(#2\right)}}
\newcommand{\set}[1]{\left\{ #1 \right\}}
\newcommand{\abs}[1]{\left| #1 \right|}
\newcommand{\tup}[1]{\left( #1 \right)}
\newcommand{\ive}[1]{\left[ #1 \right]}
\newcommand{\verts}[1]{\operatorname{V}\left( #1 \right)}
\newcommand{\edges}[1]{\operatorname{E}\left( #1 \right)}
\newcommand{\arcs}[1]{\operatorname{A}\left( #1 \right)}
\newcommand{\underbrack}[2]{\underbrace{#1}_{\substack{#2}}}
\newcommand{\are}{\ar@{-}}
\newcommand{\arebi}[1][]{\ar@{<-}@/_/[#1] \ar@/^/[#1]}
\newcommand{\of}{\text{ of }}
\ihead{Math 5707 Spring 2017 (Darij Grinberg): Lecture 16}
\ohead{page \thepage}
\cfoot{}
\begin{document}

\title{UMN, Spring 2017, Math 5707: Lecture 16 (flows and cuts in networks)}
\author{Darij Grinberg}
\date{digitized and updated version,
%TCIMACRO{\TeXButton{TeX field}{\today}}%
%BeginExpansion
\today
%EndExpansion
}
\maketitle
\tableofcontents

\section{Network flows}

\subsection{Introduction}

We shall now study the basics of the theory of network flows. We will barely
scratch the surface; for a more comprehensive survey, see Schrijver's
\cite[Chapter 4]{Schrij17}. Also, \cite[\S 8.2]{Martin17} gives a neat
introduction, and \cite{ForFul62} is the classical text on the subject. Also,
\cite{Thalwi08} appears to be a thorough treatment of the subject with lots of
technical details. See also the \textquotedblleft Lecture notes on maximum
flows and minimum cut problems\textquotedblright\ in \cite{Goeman17}.

We shall mostly follow \cite[\S 8.2]{Martin17}. In particular, we will only
use elementary methods. More advanced disciplines (such as linear optimization
and polyhedral geometry) offer alternative points of view on the theory of
network flows; we shall ignore these. We shall also ignore real-life
applications, although there are many (see \cite[Chapter 4]{Schrij17} for a
few, and see \cite[\S 2]{Schrij12} for the Cold War origins of the subject).

We let $\mathbb{N}$ denote the set $\left\{  0,1,2,\ldots\right\}  $ of all
nonnegative integers.

We let $\mathbb{Q}_{+}$ denote the set $\left\{  x\in\mathbb{Q}\mid
x\geq0\right\}  $ of all nonnegative rational numbers.

We let $\mathbb{R}_{+}$ denote the set $\left\{  x\in\mathbb{R}\mid
x\geq0\right\}  $ of all nonnegative real numbers.

\subsection{The concept of a network}

Recall that a \textit{digraph} is defined to be a pair $\left(  V,A\right)  $
consisting of a finite set $V$ and of a subset $A$ of $V\times V$. The
elements of $V$ are called the \textit{vertices} of this digraph; the elements
of $A$ are called its \textit{arcs}. If $a=\left(  u,v\right)  $ is an arc of
a digraph $\left(  V,A\right)  $, then $u$ is called the \textit{source} of
this arc $a$, and $v$ is called the \textit{target} of this arc $a$.

The whole theory of network flows can just as well be built on the concept of
a multidigraph instead of a digraph. But we shall restrict ourselves to
digraphs for the sake of simplicity. (Also, the basic results for
multidigraphs can be easily derived from the corresponding results for
digraphs; so we aren't even sacrificing any real power.)

\begin{definition}
\label{def.network.basics}A \textit{network} consists of:

\begin{itemize}
\item a digraph $\left(  V,A\right)  $;

\item two distinct vertices $s\in V$ and $t\in V$, called the \textit{source}
and the \textit{sink}, respectively (although we do not require $s$ to have
indegree $0$ or $t$ to have outdegree $0$);

\item a function $c:A\rightarrow\mathbb{Q}_{+}$, called the \textit{capacity
function}.
\end{itemize}
\end{definition}

\begin{definition}
\label{def.network.notations}Let $N$ be a network consisting of a digraph
$\left(  V,A\right)  $, a source $s\in V$ and a sink $t\in V$, and a capacity
function $c:A\rightarrow\mathbb{Q}_{+}$. Then, we define the following notations:

\begin{itemize}
\item For any arc $a\in A$, we call the number $c\left(  a\right)
\in\mathbb{Q}_{+}$ the \textit{capacity} of the arc $a$.

\item For any subset $S$ of $V$, we let $\overline{S}$ denote the subset
$V\setminus S$ of $V$.

\item If $P$ and $Q$ are two subsets of $V$, then $\left[  P,Q\right]  $ shall
mean the set of all arcs $a\in A$ whose source belongs to $P$ and whose target
belongs to $Q$. (In other words, $\left[  P,Q\right]  =A\cap\left(  P\times
Q\right)  $.)

\item If $P$ and $Q$ are two subsets of $V$, and if $d:A\rightarrow
\mathbb{Q}_{+}$ is any function, then the number $d\left(  P,Q\right)
\in\mathbb{Q}_{+}$ is defined by
\[
d\left(  P,Q\right)  =\sum_{a\in\left[  P,Q\right]  }d\left(  a\right)  .
\]

\end{itemize}
\end{definition}

We draw a network $N$ (with notations as in Definition
\ref{def.network.notations}) as follows: We first draw the digraph $\left(
V,A\right)  $ (with source $s$ and sink $t$ marked as such), and then we write
the capacity $c\left(  a\right)  $ of each arc $a\in A$ atop the arc. For
example, the picture%
\begin{equation}%
%TCIMACRO{\TeXButton{network}{\xymatrix@C=7pc@R=5pc{
%s = 1 \ar[r]^2 \ar[dr]_1 & 3 \ar[r]^1 \ar[dr]_2 & 2 \ar[r]^2 & 5 \ar
%[dl]_1 \ar[dr]^2 \\
%& 7 \ar[r]_3 & 4 \ar[u]^1 \ar[r]_1 & 6 \ar[r]_1 & 8 = t
%}} }%
%BeginExpansion
\xymatrix@C=7pc@R=5pc{
s = 1 \ar[r]^2 \ar[dr]_1 & 3 \ar[r]^1 \ar[dr]_2 & 2 \ar[r]^2 & 5 \ar
[dl]_1 \ar[dr]^2 \\
& 7 \ar[r]_3 & 4 \ar[u]^1 \ar[r]_1 & 6 \ar[r]_1 & 8 = t
}
%EndExpansion
\label{eq.network.example.1}%
\end{equation}
represents a network $N$ whose underlying digraph $\left(  V,A\right)  $ has
$8$ vertices \newline$1,2,3,4,5,6,7,8$ and $11$ arcs \newline$\left(
1,3\right)  ,\left(  1,7\right)  ,\left(  3,2\right)  ,\left(  3,4\right)
,\left(  7,4\right)  ,\left(  4,2\right)  ,\left(  2,5\right)  ,\left(
4,6\right)  ,\left(  5,4\right)  ,\left(  5,8\right)  ,\left(  6,8\right)  $,
with $1$ chosen as source and $8$ chosen as sink, and with capacities given by
$c\left(  \left(  1,3\right)  \right)  =2$, $c\left(  \left(  1,7\right)
\right)  =1$, $c\left(  \left(  3,2\right)  \right)  =1$ etc.

\subsection{The concept of a flow}

\begin{definition}
\label{def.flow.def}Let $N$, $V$, $A$, $s$, $t$ and $c$ be as in Definition
\ref{def.network.notations}.

A \textit{flow (on the network }$N$\textit{)} means a function $f:A\rightarrow
\mathbb{Q}_{+}$ with the following properties:

\begin{itemize}
\item We have $0\leq f\left(  a\right)  \leq c\left(  a\right)  $ for all
$a\in A$. This is called the \textit{capacity constraints}.

\item For any vertex $v\in V\setminus\left\{  s,t\right\}  $, we have%
\[
f^{-}\left(  v\right)  =f^{+}\left(  v\right)  ,
\]
where the rational numbers $f^{-}\left(  v\right)  $ and $f^{+}\left(
v\right)  $ are defined as follows:%
\[
f^{-}\left(  v\right)  =\sum_{\substack{a\in A\text{ is an arc}\\\text{with
target }v}}f\left(  a\right)  \ \ \ \ \ \ \ \ \ \ \text{and}%
\ \ \ \ \ \ \ \ \ \ f^{+}\left(  v\right)  =\sum_{\substack{a\in A\text{ is an
arc}\\\text{with source }v}}f\left(  a\right)  .
\]
This is called the \textit{conservation constraints}.
\end{itemize}
\end{definition}

We can visualize a network $N$ as a collection of water pipes (the pipes are
the arcs $a\in A$; the capacity $c\left(  a\right)  $ of a pipe $a$ is how
much water it can maximally transport in a second); then, a flow $f$ on $N$
can be visualized as water flowing through the pipes (namely, the amount of
water traveling through a pipe $a$ in a second is $f\left(  a\right)  $). The
capacity constraints say that no pipe is over its capacity or carries a
negative amount of water\footnote{Notice that each pipe has a pre-determined
direction; water can only flow in that direction!}. The conservation
constraints say that at every vertex $v$ other than $s$ and $t$, the amount of
water coming in (that is, $f^{-}\left(  v\right)  $) equals the amount of
water moving out (that is, $f^{+}\left(  v\right)  $); that is, there are no
leaks and no water being injected into the system other than at $s$ and $t$.
This is why $s$ is called the \textquotedblleft source\textquotedblright\ and
$t$ is the \textquotedblleft sink\textquotedblright\footnote{although these
words are not to be taken fully at face value: it is possible that the source
has more water coming in than moving out, and that the sink has more water
moving out than coming in}.

To draw a flow $f$ on a network $N$ (with notations as in Definition
\ref{def.network.notations}), we proceed in the same way as when drawing the
network $N$ itself, but instead of writing the capacity $c\left(  a\right)  $
atop each arc $a\in A$, we write \textquotedblleft$f\left(  a\right)  $ of
$c\left(  a\right)  $\textquotedblright\ atop each arc $a\in A$. For example,
here is a flow $f$ on the network $N$ shown on (\ref{eq.network.example.1}):%
\begin{equation}%
%TCIMACRO{\TeXButton{network with flow}{\xymatrix@C=7pc@R=5pc{
%s = 1 \ar[r]^{1 \of2} \ar[dr]_{1 \of1} & 3 \ar[r]^{1 \of1} \ar[dr]_{0 \of2}
%& 2 \ar[r]^{2 \of2} & 5 \ar[dl]_{1 \of1} \ar[dr]^{1 \of2} \\
%& 7 \ar[r]_{1 \of3} & 4 \ar[u]^{1 \of1} \ar[r]_{1 \of1} & 6 \ar[r]_{1 \of1}
%& 8 = t
%}} }%
%BeginExpansion
\xymatrix@C=7pc@R=5pc{
s = 1 \ar[r]^{1 \of2} \ar[dr]_{1 \of1} & 3 \ar[r]^{1 \of1} \ar[dr]_{0 \of2}
& 2 \ar[r]^{2 \of2} & 5 \ar[dl]_{1 \of1} \ar[dr]^{1 \of2} \\
& 7 \ar[r]_{1 \of3} & 4 \ar[u]^{1 \of1} \ar[r]_{1 \of1} & 6 \ar[r]_{1 \of1}
& 8 = t
}
%EndExpansion
\label{eq.flow.example.1}%
\end{equation}
(so, for example, $f\left(  \left(  1,3\right)  \right)  =1$, $f\left(
\left(  2,5\right)  \right)  =2$ and $f\left(  \left(  3,4\right)  \right)
=0$).

Let us make a definition (which we already made temporarily in Definition
\ref{def.flow.def}):

\begin{definition}
\label{def.f-f+}Let $N$, $V$, $A$, $s$, $t$ and $c$ be as in Definition
\ref{def.network.notations}. Let $f:A\rightarrow\mathbb{Q}_{+}$ be any map.
Let $v\in V$. Then, we define two rational numbers $f^{-}\left(  v\right)
\in\mathbb{Q}_{+}$ and $f^{+}\left(  v\right)  \in\mathbb{Q}_{+}$ as follows:%
\[
f^{-}\left(  v\right)  =\sum_{\substack{a\in A\text{ is an arc}\\\text{with
target }v}}f\left(  a\right)  \ \ \ \ \ \ \ \ \ \ \text{and}%
\ \ \ \ \ \ \ \ \ \ f^{+}\left(  v\right)  =\sum_{\substack{a\in A\text{ is an
arc}\\\text{with source }v}}f\left(  a\right)  .
\]
(We may call $f^{-}\left(  v\right)  $ the \textit{inflow} of $f$ into $v$,
and we may call $f^{+}\left(  v\right)  $ the \textit{outflow} of $f$ from $v$.)
\end{definition}

We can now define the \textit{value} of a flow:

\begin{definition}
\label{def.flow.value}Let $N$, $V$, $A$, $s$, $t$ and $c$ be as in Definition
\ref{def.network.notations}. Let $f:A\rightarrow\mathbb{Q}_{+}$ be a flow on
$N$.

The \textit{value} of the flow $f$ is defined to be the number $f^{+}\left(
s\right)  -f^{-}\left(  s\right)  $. It is denoted by $\left\vert f\right\vert
$.
\end{definition}

For example, the value of the flow $f$ shown in (\ref{eq.flow.example.1}) is
$\left\vert f\right\vert =f^{+}\left(  s\right)  -f^{-}\left(  s\right)
=2-0=2$.

\begin{proposition}
\label{prop.1}Let $N$, $V$, $A$, $s$, $t$ and $c$ be as in Definition
\ref{def.network.notations}. Let $f:A\rightarrow\mathbb{Q}_{+}$ be a flow on
$N$. Then,%
\begin{align}
\left\vert f\right\vert  &  =f^{+}\left(  s\right)  -f^{-}\left(  s\right)
\label{eq.prop.1.1}\\
&  =f^{-}\left(  t\right)  -f^{+}\left(  t\right)  . \label{eq.prop.1.2}%
\end{align}

\end{proposition}

\begin{proof}
[Proof of Proposition \ref{prop.1} (sketched).]The definition of $\left\vert
f\right\vert $ yields $\left\vert f\right\vert =f^{+}\left(  s\right)
-f^{-}\left(  s\right)  $. It thus remains to show that $\left\vert
f\right\vert =f^{-}\left(  t\right)  -f^{+}\left(  t\right)  $.

Each arc $a\in A$ has exactly one source. Thus,%
\begin{equation}
\sum_{a\in A}f\left(  a\right)  =\sum_{v\in V}\underbrace{\sum_{\substack{a\in
A\text{ is an arc}\\\text{with source }v}}f\left(  a\right)  }%
_{\substack{=f^{+}\left(  v\right)  \\\text{(by the definition of }%
f^{+}\left(  v\right)  \text{)}}}=\sum_{v\in V}f^{+}\left(  v\right)  .
\label{pf.prop.1.inflow}%
\end{equation}
Similarly,%
\begin{equation}
\sum_{a\in A}f\left(  a\right)  =\sum_{v\in V}f^{-}\left(  v\right)  .
\label{pf.prop.1.outflow}%
\end{equation}
Comparing this with (\ref{pf.prop.1.inflow}), we obtain%
\[
\sum_{v\in V}f^{+}\left(  v\right)  =\sum_{v\in V}f^{-}\left(  v\right)  .
\]
Hence,%
\begin{equation}
\sum_{v\in V}\left(  f^{+}\left(  v\right)  -f^{-}\left(  v\right)  \right)
=\underbrace{\sum_{v\in V}f^{+}\left(  v\right)  }_{=\sum_{v\in V}f^{-}\left(
v\right)  }-\sum_{v\in V}f^{-}\left(  v\right)  =0. \label{pf.prop.1.2}%
\end{equation}


But the conservation constraints (which hold, since $f$ is a flow) say that
for any vertex $v\in V\setminus\left\{  s,t\right\}  $, we have%
\begin{equation}
f^{-}\left(  v\right)  =f^{+}\left(  v\right)  . \label{pf.prop.1.cons}%
\end{equation}
Now, recall that $s$ and $t$ are distinct elements of $V$. Hence, we can split
off the addends for $v=s$ and for $v=t$ from the sum $\sum_{v\in V}\left(
f^{+}\left(  v\right)  -f^{-}\left(  v\right)  \right)  $. We thus obtain%
\begin{align*}
&  \sum_{v\in V}\left(  f^{+}\left(  v\right)  -f^{-}\left(  v\right)  \right)
\\
&  =\underbrace{\left(  f^{+}\left(  s\right)  -f^{-}\left(  s\right)
\right)  }_{=\left\vert f\right\vert }+\left(  f^{+}\left(  t\right)
-f^{-}\left(  t\right)  \right)  +\sum_{v\in V\setminus\left\{  s,t\right\}
}\underbrace{\left(  f^{+}\left(  v\right)  -f^{-}\left(  v\right)  \right)
}_{\substack{=0\\\text{(by (\ref{pf.prop.1.cons}))}}}\\
&  =\left\vert f\right\vert +\left(  f^{+}\left(  t\right)  -f^{-}\left(
t\right)  \right)  .
\end{align*}
Comparing this with (\ref{pf.prop.1.2}), we find $\left\vert f\right\vert
+\left(  f^{+}\left(  t\right)  -f^{-}\left(  t\right)  \right)  =0$. Thus,%
\[
\left\vert f\right\vert =-\left(  f^{+}\left(  t\right)  -f^{-}\left(
t\right)  \right)  =f^{-}\left(  t\right)  -f^{+}\left(  t\right)  .
\]
This completes the proof of Proposition \ref{prop.1}.
\end{proof}

For the next proposition, let us recall the conventions we made in Definition
\ref{def.network.notations}. In particular, if $S$ is any subset of $V$ (where
notations as in Definition \ref{def.network.notations}), then $\overline{S}$
denotes the complement $V\setminus S$ of $S$. Also, for any two subsets $P$
and $Q$ of $V$ and any map $d:A\rightarrow\mathbb{Q}_{+}$, we have $d\left(
P,Q\right)  =\sum_{a\in\left[  P,Q\right]  }d\left(  a\right)  $.

\begin{proposition}
\label{prop.2}Let $N$, $V$, $A$, $s$, $t$ and $c$ be as in Definition
\ref{def.network.notations}. Let $f:A\rightarrow\mathbb{Q}_{+}$ be a flow on
$N$. Let $S$ be a subset of $V$.

\textbf{(a)} We have%
\[
f\left(  S,\overline{S}\right)  -f\left(  \overline{S},S\right)  =\sum_{v\in
S}\left(  f^{+}\left(  v\right)  -f^{-}\left(  v\right)  \right)  .
\]


\textbf{(b)} If $s\in S$ and $t\notin S$, then%
\[
\left\vert f\right\vert =f\left(  S,\overline{S}\right)  -f\left(
\overline{S},S\right)  .
\]


\textbf{(c)} If $s\in S$ and $t\notin S$, then $\left\vert f\right\vert \leq
c\left(  S,\overline{S}\right)  $.

\textbf{(d)} Assume that $s\in S$ and $t\notin S$. Then, $\left\vert
f\right\vert =c\left(  S,\overline{S}\right)  $ if and only if%
\begin{equation}
\left(  f\left(  a\right)  =0\text{ for all }a\in\left[  \overline
{S},S\right]  \right)  \label{eq.prop.2.d.b.1}%
\end{equation}
and%
\begin{equation}
\left(  f\left(  a\right)  =c\left(  a\right)  \text{ for all }a\in\left[
S,\overline{S}\right]  \right)  . \label{eq.prop.2.d.b.2}%
\end{equation}

\end{proposition}

\begin{proof}
[Proof of Proposition \ref{prop.2} (sketched).]\textbf{(a)} We have%
\begin{align}
&  \sum_{v\in S}\left(  f^{+}\left(  v\right)  -f^{-}\left(  v\right)  \right)
\nonumber\\
&  =\sum_{v\in S}\underbrace{f^{+}\left(  v\right)  }_{\substack{=\sum
_{\substack{a\in A\text{ is an arc}\\\text{with source }v}}f\left(  a\right)
\\\text{(by the definition of }f^{+}\left(  v\right)  \text{)}}}-\sum_{v\in
S}\underbrace{f^{-}\left(  v\right)  }_{\substack{=\sum_{\substack{a\in
A\text{ is an arc}\\\text{with target }v}}f\left(  a\right)  \\\text{(by the
definition of }f^{-}\left(  v\right)  \text{)}}}\nonumber\\
&  =\underbrace{\sum_{v\in S}\sum_{\substack{a\in A\text{ is an arc}%
\\\text{with source }v}}}_{\substack{=\sum_{\substack{a\in A\text{ is an
arc}\\\text{with source in }S}}}}f\left(  a\right)  -\underbrace{\sum_{v\in
S}\sum_{\substack{a\in A\text{ is an arc}\\\text{with target }v}%
}}_{\substack{=\sum_{\substack{a\in A\text{ is an arc}\\\text{with target in
}S}}}}f\left(  a\right) \nonumber\\
&  =\sum_{\substack{a\in A\text{ is an arc}\\\text{with source in }S}}f\left(
a\right)  -\sum_{\substack{a\in A\text{ is an arc}\\\text{with target in }%
S}}f\left(  a\right)  . \label{pf.prop.2.a.1}%
\end{align}


But recall that $\overline{S}$ denotes the complement $V\setminus S$ of $S$.
Thus, any vertex in $V$ must lie either in $S$ or in $\overline{S}$ (but not
in both). Hence, any arc $a\in A$ must either have target in $S$ or target in
$\overline{S}$ (but not both). Hence, the sum $\sum_{\substack{a\in A\text{ is
an arc}\\\text{with source in }S}}f\left(  a\right)  $ can be split as
follows:
\begin{align}
&  \sum_{\substack{a\in A\text{ is an arc}\\\text{with source in }S}}f\left(
a\right) \nonumber\\
&  =\underbrace{\sum_{\substack{a\in A\text{ is an arc}\\\text{with source in
}S\\\text{and target in }S}}}_{\substack{=\sum_{a\in\left[  S,S\right]
}\\\text{(by the definition of }\left[  S,S\right]  \text{)}}}f\left(
a\right)  +\underbrace{\sum_{\substack{a\in A\text{ is an arc}\\\text{with
source in }S\\\text{and target in }\overline{S}}}}_{\substack{=\sum
_{a\in\left[  S,\overline{S}\right]  }\\\text{(by the definition of }\left[
S,\overline{S}\right]  \text{)}}}f\left(  a\right) \nonumber\\
&  =\sum_{a\in\left[  S,S\right]  }f\left(  a\right)  +\sum_{a\in\left[
S,\overline{S}\right]  }f\left(  a\right)  . \label{pf.prop.2.a.2}%
\end{align}


Similarly, we can split $\sum_{\substack{a\in A\text{ is an arc}\\\text{with
target in }S}}f\left(  a\right)  $ according to whether the source (not the
target this time) of an arc lies in $S$ or in $\overline{S}$. We thus obtain%
\begin{equation}
\sum_{\substack{a\in A\text{ is an arc}\\\text{with target in }S}}f\left(
a\right)  =\sum_{a\in\left[  S,S\right]  }f\left(  a\right)  +\sum
_{a\in\left[  \overline{S},S\right]  }f\left(  a\right)  .
\label{pf.prop.2.a.3}%
\end{equation}


Now, (\ref{pf.prop.2.a.1}) becomes%
\begin{align*}
&  \sum_{v\in S}\left(  f^{+}\left(  v\right)  -f^{-}\left(  v\right)  \right)
\\
&  =\underbrace{\sum_{\substack{a\in A\text{ is an arc}\\\text{with source in
}S}}f\left(  a\right)  }_{\substack{=\sum_{a\in\left[  S,S\right]  }f\left(
a\right)  +\sum_{a\in\left[  S,\overline{S}\right]  }f\left(  a\right)
\\\text{(by (\ref{pf.prop.2.a.2}))}}}-\underbrace{\sum_{\substack{a\in A\text{
is an arc}\\\text{with target in }S}}f\left(  a\right)  }_{\substack{=\sum
_{a\in\left[  S,S\right]  }f\left(  a\right)  +\sum_{a\in\left[  \overline
{S},S\right]  }f\left(  a\right)  \\\text{(by (\ref{pf.prop.2.a.3}))}}}\\
&  =\left(  \sum_{a\in\left[  S,S\right]  }f\left(  a\right)  +\sum
_{a\in\left[  S,\overline{S}\right]  }f\left(  a\right)  \right)  -\left(
\sum_{a\in\left[  S,S\right]  }f\left(  a\right)  +\sum_{a\in\left[
\overline{S},S\right]  }f\left(  a\right)  \right) \\
&  =\sum_{a\in\left[  S,\overline{S}\right]  }f\left(  a\right)  -\sum
_{a\in\left[  \overline{S},S\right]  }f\left(  a\right)  .
\end{align*}
Comparing this with%
\[
\underbrace{f\left(  S,\overline{S}\right)  }_{\substack{=\sum_{a\in\left[
S,\overline{S}\right]  }f\left(  a\right)  \\\text{(by the definition of
}f\left(  S,\overline{S}\right)  \text{)}}}-\underbrace{f\left(  \overline
{S},S\right)  }_{\substack{=\sum_{a\in\left[  \overline{S},S\right]  }f\left(
a\right)  \\\text{(by the definition of }f\left(  \overline{S},S\right)
\text{)}}}=\sum_{a\in\left[  S,\overline{S}\right]  }f\left(  a\right)
-\sum_{a\in\left[  \overline{S},S\right]  }f\left(  a\right)  ,
\]
we obtain $f\left(  S,\overline{S}\right)  -f\left(  \overline{S},S\right)
=\sum_{v\in S}\left(  f^{+}\left(  v\right)  -f^{-}\left(  v\right)  \right)
$. This proves Proposition \ref{prop.2} \textbf{(a)}.

\textbf{(b)} Assume that $s\in S$ and $t\notin S$. Thus, $S\setminus\left\{
s,t\right\}  =S\setminus\left\{  s\right\}  $ (since $t\notin S$).

We can split off the addend for $v=s$ from the sum $\sum_{v\in S}\left(
f^{+}\left(  v\right)  -f^{-}\left(  v\right)  \right)  $ (since $s\in S$). We
thus obtain%
\[
\sum_{v\in S}\left(  f^{+}\left(  v\right)  -f^{-}\left(  v\right)  \right)
=\underbrace{\left(  f^{+}\left(  s\right)  -f^{-}\left(  s\right)  \right)
}_{=\left\vert f\right\vert }+\sum_{v\in S\setminus\left\{  s\right\}
}\underbrace{\left(  f^{+}\left(  v\right)  -f^{-}\left(  v\right)  \right)
}_{\substack{=0\\\text{(by (\ref{pf.prop.1.cons})}\\\text{(since }v\in
S\setminus\left\{  s\right\}  =S\setminus\left\{  s,t\right\}  \subseteq
V\setminus\left\{  s,t\right\}  \text{))}}}=\left\vert f\right\vert .
\]
Hence,%
\[
\left\vert f\right\vert =\sum_{v\in S}\left(  f^{+}\left(  v\right)
-f^{-}\left(  v\right)  \right)  =f\left(  S,\overline{S}\right)  -f\left(
\overline{S},S\right)
\]
(by Proposition \ref{prop.2} \textbf{(a)}). This proves Proposition
\ref{prop.2} \textbf{(b)}.

\textbf{(c)} Assume that $s\in S$ and $t\notin S$. The definition of $f\left(
S,\overline{S}\right)  $ yields%
\begin{equation}
f\left(  S,\overline{S}\right)  =\sum_{a\in\left[  S,\overline{S}\right]
}\underbrace{f\left(  a\right)  }_{\substack{\leq c\left(  a\right)
\\\text{(by the capacity}\\\text{constraints)}}}\leq\sum_{a\in\left[
S,\overline{S}\right]  }c\left(  a\right)  =c\left(  S,\overline{S}\right)
\label{pf.prop.2.b.1}%
\end{equation}
(since $c\left(  S,\overline{S}\right)  $ is defined to be $\sum_{a\in\left[
S,\overline{S}\right]  }c\left(  a\right)  $). The definition of $f\left(
\overline{S},S\right)  $ yields%
\begin{equation}
f\left(  \overline{S},S\right)  =\sum_{a\in\left[  \overline{S},S\right]
}\underbrace{f\left(  a\right)  }_{\substack{\geq0\\\text{(by the
capacity}\\\text{constraints)}}}\geq\sum_{a\in\left[  \overline{S},S\right]
}0=0. \label{pf.prop.2.b.2}%
\end{equation}
Proposition \ref{prop.2} \textbf{(b)} yields%
\[
\left\vert f\right\vert =\underbrace{f\left(  S,\overline{S}\right)
}_{\substack{\leq c\left(  S,\overline{S}\right)  \\\text{(by
(\ref{pf.prop.2.b.1}))}}}-\underbrace{f\left(  \overline{S},S\right)
}_{\substack{\geq0\\\text{(by (\ref{pf.prop.2.b.2}))}}}\leq c\left(
S,\overline{S}\right)  -0=c\left(  S,\overline{S}\right)  .
\]
This proves Proposition \ref{prop.2} \textbf{(c)}.

\textbf{(d)} Let us analyze our above proof of Proposition \ref{prop.2}
\textbf{(c)}. We have obtained the inequality (\ref{pf.prop.2.b.1}) by adding
together the inequalities $f\left(  a\right)  \leq c\left(  a\right)  $ for
all $a\in\left[  S,\overline{S}\right]  $. Thus, the inequality
(\ref{pf.prop.2.b.1}) becomes an equality if and only if all of the latter
inequalities $f\left(  a\right)  \leq c\left(  a\right)  $ for all
$a\in\left[  S,\overline{S}\right]  $ become equalities. Hence, we have the
following chain of equivalences:%
\begin{align*}
&  \ \left(  \text{the inequality (\ref{pf.prop.2.b.1}) becomes an
equality}\right) \\
&  \Longleftrightarrow\ \left(  \text{all of the inequalities }f\left(
a\right)  \leq c\left(  a\right)  \text{ for all }a\in\left[  S,\overline
{S}\right]  \text{ become equalities}\right) \\
&  \Longleftrightarrow\ \left(  f\left(  a\right)  =c\left(  a\right)  \text{
for all }a\in\left[  S,\overline{S}\right]  \right)  .
\end{align*}


We have obtained the inequality (\ref{pf.prop.2.b.2}) by adding together the
inequalities $f\left(  a\right)  \geq0$ for all $a\in\left[  \overline
{S},S\right]  $. Thus, the inequality (\ref{pf.prop.2.b.2}) becomes an
equality if and only if all of the latter inequalities $f\left(  a\right)
\geq0$ for all $a\in\left[  \overline{S},S\right]  $ become equalities. Hence,
we have the following chain of equivalences:%
\begin{align*}
&  \ \left(  \text{the inequality (\ref{pf.prop.2.b.2}) becomes an
equality}\right) \\
&  \Longleftrightarrow\ \left(  \text{all of the inequalities }f\left(
a\right)  \geq0\text{ for all }a\in\left[  \overline{S},S\right]  \text{
become equalities}\right) \\
&  \Longleftrightarrow\ \left(  f\left(  a\right)  =0\text{ for all }%
a\in\left[  \overline{S},S\right]  \right)  .
\end{align*}


But we have proven the inequality $\left\vert f\right\vert \leq c\left(
S,\overline{S}\right)  $ by subtracting the inequality (\ref{pf.prop.2.b.2})
from the inequality (\ref{pf.prop.2.b.1}). Thus, the inequality $\left\vert
f\right\vert \leq c\left(  S,\overline{S}\right)  $ becomes an equality if and
only if both inequalities (\ref{pf.prop.2.b.2}) and (\ref{pf.prop.2.b.1})
become equalities. Hence, we have the following chain of equivalences:%
\begin{align*}
&  \ \left(  \text{the inequality }\left\vert f\right\vert \leq c\left(
S,\overline{S}\right)  \text{ becomes an equality}\right) \\
&  \Longleftrightarrow\ \left(  \text{both inequalities (\ref{pf.prop.2.b.2})
and (\ref{pf.prop.2.b.1}) become equalities}\right) \\
&  \Longleftrightarrow\ \underbrace{\left(  \text{the inequality
(\ref{pf.prop.2.b.1}) becomes an equality}\right)  }_{\Longleftrightarrow
\ \left(  f\left(  a\right)  =c\left(  a\right)  \text{ for all }a\in\left[
S,\overline{S}\right]  \right)  }\\
&  \ \ \ \ \ \ \ \ \ \ \wedge\ \underbrace{\left(  \text{the inequality
(\ref{pf.prop.2.b.2}) becomes an equality}\right)  }_{\Longleftrightarrow
\ \left(  f\left(  a\right)  =0\text{ for all }a\in\left[  \overline
{S},S\right]  \right)  }\\
&  \Longleftrightarrow\ \underbrace{\left(  f\left(  a\right)  =c\left(
a\right)  \text{ for all }a\in\left[  S,\overline{S}\right]  \right)
}_{\Longleftrightarrow\ \left(  \text{(\ref{eq.prop.2.d.b.2}) holds}\right)
}\wedge\underbrace{\left(  f\left(  a\right)  =0\text{ for all }a\in\left[
\overline{S},S\right]  \right)  }_{\Longleftrightarrow\ \left(
\text{(\ref{eq.prop.2.d.b.1}) holds}\right)  }\\
&  \Longleftrightarrow\ \left(  \text{(\ref{eq.prop.2.d.b.2}) holds}\right)
\wedge\left(  \text{(\ref{eq.prop.2.d.b.1}) holds}\right) \\
&  \Longleftrightarrow\ \left(  \text{both (\ref{eq.prop.2.d.b.1}) and
(\ref{eq.prop.2.d.b.2}) hold}\right)  .
\end{align*}
In other words, the inequality $\left\vert f\right\vert \leq c\left(
S,\overline{S}\right)  $ becomes an equality if and only if both
(\ref{eq.prop.2.d.b.1}) and (\ref{eq.prop.2.d.b.2}) hold. In other words,
$\left\vert f\right\vert =c\left(  S,\overline{S}\right)  $ if and only if
both (\ref{eq.prop.2.d.b.1}) and (\ref{eq.prop.2.d.b.2}) hold. This proves
Proposition \ref{def.flow.def} \textbf{(d)}.
\end{proof}

\begin{remark}
Let $N$, $V$, $A$, $s$, $t$ and $c$ be as in Definition
\ref{def.network.notations}. Let $f:A\rightarrow\mathbb{Q}_{+}$ be a flow on
$N$. Assume that $\left(  v,v\right)  \notin A$. Then, it is easy to see that
$f^{+}\left(  v\right)  =f\left(  \left\{  v\right\}  ,\overline{\left\{
v\right\}  }\right)  $ and $f^{-}\left(  v\right)  =f\left(  \overline
{\left\{  v\right\}  },\left\{  v\right\}  \right)  $.
\end{remark}

\subsection{Cuts in networks}

\begin{definition}
\label{def.cuts}Let $N$, $V$, $A$, $s$, $t$ and $c$ be as in Definition
\ref{def.network.notations}.

\textbf{(a)} An $s$\textit{-}$t$\textit{-cutting subset of }$V$ shall mean a
subset $S$ of $V$ satisfying $s\in S$ and $t\notin S$.

\textbf{(b)} A \textit{cut of }$N$ shall mean a subset of $A$ having the form
$\left[  S,\overline{S}\right]  $ for some $s$-$t$-cutting subset $S$ of $V$.

\textbf{(c)} The \textit{capacity} of a cut $\left[  S,\overline{S}\right]  $
is defined to be $c\left(  S,\overline{S}\right)  $. (Note that this is indeed
well-defined: In fact, $c\left(  S,\overline{S}\right)  =\sum_{a\in\left[
S,\overline{S}\right]  }c\left(  a\right)  $ clearly depends only on the cut
$\left[  S,\overline{S}\right]  $ rather than on the set $S$.)
\end{definition}

For example, if $N$ is the network shown in (\ref{eq.network.example.1}), then
$\left\{  1,3,4\right\}  $ is an $s$-$t$-cutting subset; the cut $\left[
\left\{  1,3,4\right\}  ,\overline{\left\{  1,3,4\right\}  }\right]  $
corresponding to this subset is $\left\{  \left(  1,7\right)  ,\left(
3,2\right)  ,\left(  4,2\right)  ,\left(  4,6\right)  \right\}  $ and has
capacity%
\begin{align*}
c\left(  \left\{  1,3,4\right\}  ,\overline{\left\{  1,3,4\right\}  }\right)
&  =\sum_{a\in\left\{  \left(  1,7\right)  ,\left(  3,2\right)  ,\left(
4,2\right)  ,\left(  4,6\right)  \right\}  }c\left(  a\right) \\
&  =\underbrace{c\left(  \left(  1,7\right)  \right)  }_{=1}%
+\underbrace{c\left(  \left(  3,2\right)  \right)  }_{=1}+\underbrace{c\left(
\left(  4,2\right)  \right)  }_{=1}+\underbrace{c\left(  \left(  4,6\right)
\right)  }_{=1}=4.
\end{align*}
We can illustrate this cut by drawing all arcs belonging to the cut as double
arrows ($\Longrightarrow$):%

\[%
%TCIMACRO{\TeXButton{network with cut}{\xymatrix@C=7pc@R=5pc{
%\boxed{s = 1} \ar[r]^2 \ar@{=>}[dr]_1 & \boxed{3} \ar@{=>}[r]^1 \ar
%[dr]_2 & 2 \ar[r]^2 & 5 \ar[dl]_1 \ar[dr]^2 \\
%& 7 \ar[r]_3 & \boxed{4} \ar@{=>}[u]^1 \ar@{=>}[r]_1 & 6 \ar[r]_1 & 8 = t
%}}}%
%BeginExpansion
\xymatrix@C=7pc@R=5pc{
\boxed{s = 1} \ar[r]^2 \ar@{=>}[dr]_1 & \boxed{3} \ar@{=>}[r]^1 \ar
[dr]_2 & 2 \ar[r]^2 & 5 \ar[dl]_1 \ar[dr]^2 \\
& 7 \ar[r]_3 & \boxed{4} \ar@{=>}[u]^1 \ar@{=>}[r]_1 & 6 \ar[r]_1 & 8 = t
}%
%EndExpansion
\]
(where the vertices in $\left\{  1,3,4\right\}  $ have been marked by boxes).

Notice that every network $N$ (with notations as in Definition
\ref{def.network.notations}) has two special cuts: the first is the cut
\[
\left[  \left\{  s\right\}  ,\overline{\left\{  s\right\}  }\right]  =\left\{
a\in A\ \mid\ \text{the source of }a\text{ is }s\text{, but the target is
not}\right\}
\]
corresponding to the $s$-$t$-cutting subset $\left\{  s\right\}  $; the second
is the cut%
\[
\left[  \overline{\left\{  t\right\}  },\left\{  t\right\}  \right]  =\left\{
a\in A\ \mid\ \text{the target of }a\text{ is }t\text{, but the source is
not}\right\}
\]
corresponding to the $s$-$t$-cutting subset $\overline{\left\{  t\right\}  }$.

\subsection{The max-flow-min-cut theorems}

Proposition \ref{prop.2} \textbf{(c)} thus says that the value of a flow is
always $\leq$ to the capacity of a cut. But can we achieve equality?

One of the most important results in combinatorics -- the
\textit{max-flow-min-cut theorem} -- says that \textquotedblleft
yes\textquotedblright: In each network, we can find a flow and a cut such that
the value of the flow equals the capacity of the cut. More precisely, there
are three \textquotedblleft max-flow-min-cut theorems\textquotedblright,
corresponding to different kinds of flows. The first one is about the kind of
flows we have defined above:

\begin{theorem}
\label{thm.3a}Let $N$, $V$, $A$, $s$, $t$ and $c$ be as in Definition
\ref{def.network.notations}. Then,%
\begin{equation}
\max\left\{  \left\vert f\right\vert \ \mid\ f\text{ is a flow}\right\}
=\min\left\{  c\left(  S,\overline{S}\right)  \ \mid\ S\subseteq V;\ s\in
S;\ t\notin S\right\}  . \label{eq.thm.3a.1}%
\end{equation}
In particular, the left-hand side of this equation is well-defined (i.e.,
there exists a flow $f$ for which $\left\vert f\right\vert $ is maximum). (Of
course, the right-hand side of (\ref{eq.thm.3a.1}) is well-defined, because
there are only finitely many subsets $S$ of $V$ satisfying $s\in S$ and
$t\notin S$.)
\end{theorem}

The equality (\ref{eq.thm.3a.1}) in Theorem \ref{thm.3a} says that the maximum
value of a flow equals the minimum value of $c\left(  S,\overline{S}\right)  $
where $S$ ranges over the $s$-$t$-cutting subsets of $V$. In other words, the
maximum value of a flow equals the minimum capacity of a cut.

Another variant of the max-flow-min-cut theorem makes the same claim about
\textit{integer flows} -- i.e., flows $f:A\rightarrow\mathbb{Q}_{+}$ such that
every arc $a\in A$ satisfies $f\left(  a\right)  \in\mathbb{N}$. Accordingly,
it requires that the capacities $c\left(  a\right)  $ of arcs also are
integers. Let us state it precisely:

\begin{definition}
Let $N$, $V$, $A$, $s$, $t$ and $c$ be as in Definition
\ref{def.network.notations}. An \textit{integer flow} means a flow
$f:A\rightarrow\mathbb{Q}_{+}$ satisfying $\left(  f\left(  a\right)
\in\mathbb{N}\text{ for each }a\in A\right)  $.
\end{definition}

\begin{theorem}
\label{thm.3b}Let $N$, $V$, $A$, $s$, $t$ and $c$ be as in Definition
\ref{def.network.notations}. Assume that $c\left(  a\right)  \in\mathbb{N}$
for each $a\in A$. Then,%
\begin{align*}
&  \max\left\{  \left\vert f\right\vert \ \mid\ f\text{ is an integer
flow}\right\} \\
&  =\min\left\{  c\left(  S,\overline{S}\right)  \ \mid\ S\subseteq V;\ s\in
S;\ t\notin S\right\}  .
\end{align*}
In particular, the left-hand side of this equation is well-defined (i.e.,
there exists a flow $f$ for which $\left\vert f\right\vert $ is maximum).
\end{theorem}

Finally, a third variant (which we shall not prove) makes the same statement
as Theorem \ref{thm.3a} but with $\mathbb{Q}_{+}$ replaced by $\mathbb{R}_{+}$:

\begin{theorem}
\label{thm.3c}Let $N$, $V$, $A$, $s$, $t$ and $c$ be as in Definition
\ref{def.network.notations}, except that $c$ is now a map $A\rightarrow
\mathbb{R}_{+}$ instead of being a map $A\rightarrow\mathbb{Q}_{+}$. (That is,
the capacities of arcs are now allowed to be irrational.) Also, let us
temporarily modify the definition of a flow in such a way that a flow is a map
$A\rightarrow\mathbb{R}_{+}$ instead of being a map $A\rightarrow
\mathbb{Q}_{+}$. Then,%
\[
\max\left\{  \left\vert f\right\vert \ \mid\ f\text{ is a flow}\right\}
=\min\left\{  c\left(  S,\overline{S}\right)  \ \mid\ S\subseteq V;\ s\in
S;\ t\notin S\right\}  .
\]
In particular, the left-hand side of this equation is well-defined (i.e.,
there exists a flow $f$ for which $\left\vert f\right\vert $ is maximum).
\end{theorem}

There are various proofs of these three theorems. In particular, Theorem
\ref{thm.3a} and Theorem \ref{thm.3c} can be viewed as an application of
\textit{linear programming duality}, a fundamental principle in linear
optimization. We will instead proceed elementarily. Along the way, we will see
how to \textbf{construct} a flow $f$ maximizing $\left\vert f\right\vert $ and
an $s$-$t$-cutting subset $S$ minimizing $c\left(  S,\overline{S}\right)  $
(two problems that occur in real life).

\subsection{The residual digraph}

First, let us define the so-called \textit{residual digraph} of a flow:

\begin{definition}
\label{def.Df}Let $N$, $V$, $A$, $s$, $t$ and $c$ be as in Definition
\ref{def.network.notations}.

\textbf{(a)} If $a$ is a pair $\left(  u,v\right)  \in V\times V$, then we let
$a^{-1}$ denote the pair $\left(  v,u\right)  \in V\times V$. (This is simply
a suggestive notation; it has nothing to do with reciprocals of numbers.) We
call $a^{-1}$ the \textit{reversal} of the arc $a$. Notice that $\left(
a^{-1}\right)  ^{-1}=a$ for each $a\in V\times V$.

\textbf{(b)} Let $f:A\rightarrow\mathbb{Q}_{+}$ be any flow on $N$. We define
the \textit{residual digraph }$D_{f}$ to be the digraph $\left(
V,A_{f}\right)  $, where%
\[
A_{f}=\left\{  a\in A\ \mid\ f\left(  a\right)  <c\left(  a\right)  \right\}
\cup\left\{  a^{-1}\ \mid\ a\in A;\ f\left(  a\right)  >0\right\}  .
\]

\end{definition}

Notice that the residual digraph $D_{f}$ is not a sub-digraph of $\left(
V,A\right)  $; indeed, it can have more arcs than $\left(  V,A\right)  $ has.
(When $a\in A$ is an arc of $\left(  V,A\right)  $, there is no guarantee that
$a^{-1}$ is also an arc of $\left(  V,A\right)  $.)

\begin{example}
Let $f$ be the following flow:%
\[%
%TCIMACRO{\TeXButton{network with flow}{\xymatrix@C=7pc@R=5pc{
%s = 1 \ar[r]^{1 \of2} \ar[dr]_{1 \of3} & 2 \ar[r]^{0 \of1} \ar[dr]^{1 \of1}
%& 4 \ar[dr]^{0 \of1} \\
%& 3 \ar[r]_{1 \of2} & 5 \ar[r]_{2 \of2} & 6 = t
%}}}%
%BeginExpansion
\xymatrix@C=7pc@R=5pc{
s = 1 \ar[r]^{1 \of2} \ar[dr]_{1 \of3} & 2 \ar[r]^{0 \of1} \ar[dr]^{1 \of1}
& 4 \ar[dr]^{0 \of1} \\
& 3 \ar[r]_{1 \of2} & 5 \ar[r]_{2 \of2} & 6 = t
}%
%EndExpansion
\]
(where, as before, we write \textquotedblleft$f\left(  a\right)  $ of
$c\left(  a\right)  $\textquotedblright\ atop each arc $a$). Then, the
residual digraph $D_{f}$ is%
\[%
%TCIMACRO{\TeXButton{residual digraph}{\xymatrix@C=7pc@R=5pc{
%s = 1 \ar@/^1pc/[r] \ar@/_3pc/[dr] & 2 \ar[r] \ar@/^1pc/[l] & 4 \ar[dr] \\
%& 3 \ar@/^1pc/[lu] \ar@/^1pc/[r] & 5 \ar@/^1pc/[l] \ar[lu] & 6 = t \ar[l]
%}}}%
%BeginExpansion
\xymatrix@C=7pc@R=5pc{
s = 1 \ar@/^1pc/[r] \ar@/_3pc/[dr] & 2 \ar[r] \ar@/^1pc/[l] & 4 \ar[dr] \\
& 3 \ar@/^1pc/[lu] \ar@/^1pc/[r] & 5 \ar@/^1pc/[l] \ar[lu] & 6 = t \ar[l]
}%
%EndExpansion
.
\]
Notice that $D_{f}$ has cycles even though $\left(  V,A\right)  $ has none!
\end{example}

The intuition for the digraph $D_{f}$ is that $D_{f}$ shows the
\textquotedblleft wiggle-room\textquotedblright\ for modifying $f$ without
breaking the capacity constraints (ignoring, for the time being, the
conservation constraints). Indeed, the arcs of the digraph $D_{f}$ are the
arcs $a\in A$ satisfying $f\left(  a\right)  <c\left(  a\right)  $ (that is,
the arcs $a$ used under their capacity in $f$) as well as the reverses of the
arcs $a\in A$ satisfying $f\left(  a\right)  >0$ (that is, of the arcs $a$
that have nonzero flow going through them). The former arcs can afford an
increase in the flow going through them (without violating the capacity
constraints), whereas the latter arcs can afford a reduction in the flow
through them. Of course, if we modify the flow on a single arc, then we will
most likely break the conservation constraints. The key to finding a maximum
flow is thus to change a flow in such a way that both capacity and
conservation constraints are preserved; the residual digraph $D_{f}$ is merely
the first step.

\subsection{The augmenting path lemma}

Recall that if $s$ and $t$ are two vertices of a digraph $D$, then a
\textit{walk} from $s$ to $t$ in $D$ means a sequence $\left(  v_{0}%
,v_{1},\ldots,v_{k}\right)  $, where $v_{0},v_{1},\ldots,v_{k}$ are vertices
of $D$ satisfying $v_{0}=s$ and $v_{k}=t$, and where $v_{i}v_{i+1}$ is an arc
of $D$ for each $i\in\left\{  0,1,\ldots,k-1\right\}  $. The \textit{arcs} of
this walk are defined to be the arcs $v_{0}v_{1},v_{1}v_{2},\ldots
,v_{k-1}v_{k}$ of $D$. A walk $\left(  v_{0},v_{1},\ldots,v_{k}\right)  $ from
$s$ to $t$ in a digraph $D$ is said to be a \textit{path} from $s$ to $t$ if
the vertices $v_{0},v_{1},\ldots,v_{k}$ are distinct.

The main workhorse of our proof will be the following lemma:

\begin{lemma}
\label{lem.4}Let $N$, $V$, $A$, $s$, $t$ and $c$ be as in Definition
\ref{def.network.notations}.

Let $f:A\rightarrow\mathbb{Q}_{+}$ be a flow.

\textbf{(a)} If the digraph $D_{f}$ has a path from $s$ to $t$, then there is
a flow $f^{\prime}$ with a larger value than $f$.

\textbf{(b)} If the digraph $D_{f}$ has no path from $s$ to $t$, then there
exists a subset $S$ of $V$ satisfying $s\in V$ and $t\notin V$ and $c\left(
S,\overline{S}\right)  =\left\vert f\right\vert $.
\end{lemma}

Before we prove this, we need to lay some more groundwork.

\begin{definition}
Let $N$, $V$, $A$, $s$, $t$ and $c$ be as in Definition
\ref{def.network.notations}.

\textbf{(a)} We identify the maps $g:A\rightarrow\mathbb{Q}$ satisfying
$\left(  g\left(  a\right)  \geq0\text{ for all }a\in A\right)  $ (that is,
the maps $g:A\rightarrow\mathbb{Q}$ satisfying $g\left(  A\right)
\subseteq\mathbb{Q}_{+}$) with the maps $A\rightarrow\mathbb{Q}_{+}$.

\textbf{(b)} We extend the notations $f^{-}\left(  v\right)  $ and
$f^{+}\left(  v\right)  $ from Definition \ref{def.f-f+} to arbitrary maps
$f:A\rightarrow\mathbb{Q}$ (not just maps $f:A\rightarrow\mathbb{Q}_{+}$). Of
course, $f^{-}\left(  v\right)  $ and $f^{+}\left(  v\right)  $ will then be
elements of $\mathbb{Q}$, not necessarily of $\mathbb{Q}_{+}$.

We also extend the notation $\left\vert f\right\vert $ from Definition
\ref{def.flow.value} to arbitrary maps $f:A\rightarrow\mathbb{Q}$ (not just
flows). Thus, $\left\vert f\right\vert =f^{+}\left(  s\right)  -f^{-}\left(
s\right)  $ for any map $f:A\rightarrow\mathbb{Q}$.

\textbf{(c)} If $f:A\rightarrow\mathbb{Q}$ and $g:A\rightarrow\mathbb{Q}$ are
two maps, then $f+g$ denotes a new map $A\rightarrow\mathbb{Q}$ that is
defined by%
\[
\left(  f+g\right)  \left(  a\right)  =f\left(  a\right)  +g\left(  a\right)
\ \ \ \ \ \ \ \ \ \ \text{for all }a\in A.
\]


\textbf{(d)} We say that a map $f:A\rightarrow\mathbb{Q}$ \textit{satisfies
the conservation constraints} if and only if each $v\in V\setminus\left\{
s,t\right\}  $ satisfies $f^{-}\left(  v\right)  =f^{+}\left(  v\right)  $.
\end{definition}

Notice that any flow $f:A\rightarrow\mathbb{Q}_{+}$ satisfies the conservation constraints.

\begin{lemma}
\label{lem.5}Let $N$, $V$, $A$, $s$, $t$ and $c$ be as in Definition
\ref{def.network.notations}.

\textbf{(a)} Any two maps $f:A\rightarrow\mathbb{Q}$ and $g:A\rightarrow
\mathbb{Q}$ satisfy $\left\vert f+g\right\vert =\left\vert f\right\vert
+\left\vert g\right\vert $.

\textbf{(b)} Let $f:A\rightarrow\mathbb{Q}$ and $g:A\rightarrow\mathbb{Q}$ be
two maps satisfying the conservation constraints. Then, the map
$f+g:A\rightarrow\mathbb{Q}$ also satisfies the conservation constraints.
\end{lemma}

\begin{proof}
[Proof of Lemma \ref{lem.5} (sketched).]\textbf{(b)} This is exactly as
straightforward as you would expect: Fix $v\in V\setminus\left\{  s,t\right\}
$. Then, the definition of $\left(  f+g\right)  ^{-}\left(  v\right)  $ yields%
\begin{align*}
\left(  f+g\right)  ^{-}\left(  v\right)   &  =\sum_{\substack{a\in A\text{ is
an arc}\\\text{with target }v}}\underbrace{\left(  f+g\right)  \left(
a\right)  }_{\substack{=f\left(  a\right)  +g\left(  a\right)  \\\text{(by the
definition of }f+g\text{)}}}=\sum_{\substack{a\in A\text{ is an arc}%
\\\text{with target }v}}\left(  f\left(  a\right)  +g\left(  a\right)  \right)
\\
&  =\underbrace{\sum_{\substack{a\in A\text{ is an arc}\\\text{with target }%
v}}f\left(  a\right)  }_{\substack{=f^{-}\left(  v\right)  \\\text{(by the
definition of }f^{-}\left(  v\right)  \text{)}}}+\underbrace{\sum
_{\substack{a\in A\text{ is an arc}\\\text{with target }v}}g\left(  a\right)
}_{\substack{=g^{-}\left(  v\right)  \\\text{(by the definition of }%
g^{-}\left(  v\right)  \text{)}}}\\
&  =f^{-}\left(  v\right)  +g^{-}\left(  v\right)  .
\end{align*}
Similarly, $\left(  f+g\right)  ^{+}\left(  v\right)  =f^{+}\left(  v\right)
+g^{+}\left(  v\right)  $. But the map $f$ satisfies the conservation
constraints; hence, $f^{-}\left(  v\right)  =f^{+}\left(  v\right)  $.
Similarly, $g^{-}\left(  v\right)  =g^{+}\left(  v\right)  $.

Now, $\left(  f+g\right)  ^{-}\left(  v\right)  =\underbrace{f^{-}\left(
v\right)  }_{=f^{+}\left(  v\right)  }+\underbrace{g^{-}\left(  v\right)
}_{=g^{+}\left(  v\right)  }=f^{+}\left(  v\right)  +g^{+}\left(  v\right)
=\left(  f+g\right)  ^{+}\left(  v\right)  $.

Now forget that we fixed $v$. We thus have proven that each $v\in
V\setminus\left\{  s,t\right\}  $ satisfies $\left(  f+g\right)  ^{-}\left(
v\right)  =\left(  f+g\right)  ^{+}\left(  v\right)  $. In other words, the
map $f+g$ satisfies the conservation constraints. This proves Lemma
\ref{lem.5} \textbf{(b)}.

\textbf{(a)} The proof (which uses the same ideas as the proof of Lemma
\ref{lem.5} \textbf{(b)}) is left to the reader.
\end{proof}

\begin{definition}
\label{def.A-}Let $N$, $V$, $A$, $s$, $t$ and $c$ be as in Definition
\ref{def.network.notations}.

Then, $A^{-1}$ shall denote the set $\left\{  a^{-1}\ \mid\ a\in A\right\}  $.
This is a subset of $V\times V$.
\end{definition}

\begin{lemma}
\label{lem.6}Let $N$, $V$, $A$, $s$, $t$ and $c$ be as in Definition
\ref{def.network.notations}.

Let $\mathbf{p}$ be a path from $s$ to $t$ in the digraph $\left(  V,A\cup
A^{-1}\right)  $.

Let $P$ be the set of arcs of $\mathbf{p}$. Thus, $P\subseteq A\cup A^{-1}$.

Let $\rho\in\mathbb{Q}$.

Assume that each arc in $P$ is colored either red or blue, in such a way that:

\begin{itemize}
\item if an arc $b\in P$ is colored red, then $b\in A$;

\item if an arc $b\in P$ is colored blue, then $b\in A^{-1}$ (that is,
$b^{-1}\in A$).
\end{itemize}

(Such a coloring can always be found: Indeed, every arc $b\in P$ that does not
belong to $A^{-1}$ must be colored blue; every arc $b\in P$ that does not
belong to $A$ must be colored red; every arc $b\in P$ that belongs to both $A$
and $A^{-1}$ can be colored either red or blue.)

Define a map $g:A\rightarrow\mathbb{Q}$ by setting%
\[
g\left(  a\right)  =%
\begin{cases}
\rho, & \text{if }a\in P\text{ and }a\text{ is red};\\
-\rho, & \text{if }a^{-1}\in P\text{ and }a^{-1}\text{ is blue};\\
0, & \text{otherwise}%
\end{cases}
\ \ \ \ \ \ \ \ \ \ \text{for all }a\in A.
\]
(This is well-defined, because the conditions $a\in P$ and $a^{-1}\in P$
cannot hold at the same time (since $\mathbf{p}$ is a path).)

\textbf{(a)} This map $g$ satisfies the conservation constraints (i.e., each
$v\in V\setminus\left\{  s,t\right\}  $ satisfies $g^{-}\left(  v\right)
=g^{+}\left(  v\right)  $).

\textbf{(b)} We have $\left\vert g\right\vert =\rho$.
\end{lemma}

\begin{proof}
[Proof of Lemma \ref{lem.6} (sketched).]\textbf{(a)} Let $v\in V\setminus
\left\{  s,t\right\}  $ be arbitrary. We must prove that $g^{-}\left(
v\right)  =g^{+}\left(  v\right)  $.

If $v$ does not lie on the path $\mathbf{p}$, then this is obvious (since in
this case, each arc $a$ having source $v$ or target $v$ satisfies $g\left(
a\right)  =0$, and thus we have $g^{-}\left(  v\right)  =0$ and $g^{+}\left(
v\right)  =0$).

Thus, WLOG assume that $v$ does lie on the path $\mathbf{p}$. Since $v$ is
neither the starting point nor the ending point of this path $\mathbf{p}$
(because $v\in V\setminus\left\{  s,t\right\}  $, but the starting and ending
points of $\mathbf{p}$ are $s$ and $t$), we thus conclude that there is
exactly one arc $x\in P$ having source $v$, and exactly one arc $y\in P$
having target $v$. Consider these arcs $x$ and $y$. Notice that $x$ and $y$
are arcs of the digraph $\left(  V,A\cup A^{-1}\right)  $, not necessarily
arcs of $\left(  V,A\right)  $.

The arcs $x$ and $y$ are distinct\footnote{\textit{Proof.} Assume the
contrary. Thus, $x=y$. Hence, the arc $x=y$ has source $v$ (since $x$ has
source $v$) and target $v$ (since $y$ has target $v$). In other words, this
arc $x=y$ equals $\left(  v,v\right)  $. Thus, it is a loop. But the arc $x=y$
is an arc of the path $\mathbf{p}$ (since $x\in P$), and thus cannot be a loop
(since a loop can never be an arc of a path). This is a contradiction, qed.}.

We have $x\in P$. Thus, the arc $x$ is colored either red or blue. Hence, we
are in one of the following two cases:

\begin{statement}
\textit{Case 1:} The arc $x$ is colored red.
\end{statement}

\begin{statement}
\textit{Case 2:} The arc $x$ is colored blue.
\end{statement}

Let us consider Case 2. In this case, the arc $x$ is colored blue. Hence,
$x\in A^{-1}$ (by the requirements on the coloring), so that $x^{-1}\in A$.
The definition of $g$ thus shows that $g\left(  x^{-1}\right)  =-\rho$ (since
$\left(  x^{-1}\right)  ^{-1}=x\in P$ and since $\left(  x^{-1}\right)
^{-1}=x$ is blue).

But $y\in P$. Thus, the arc $y$ is colored either red or blue. Hence, we are
in one of the following two subcases:

\begin{statement}
\textit{Subcase 2.1:} The arc $y$ is colored red.
\end{statement}

\begin{statement}
\textit{Subcase 2.2:} The arc $y$ is colored blue.
\end{statement}

Let us consider Subcase 2.2. In this subcase, the arc $y$ is colored blue.
Hence, $y\in A^{-1}$ (by the requirements on the coloring), so that $y^{-1}\in
A$. The definition of $g$ thus shows that $g\left(  y^{-1}\right)  =-\rho$
(since $\left(  y^{-1}\right)  ^{-1}=y\in P$ and since $\left(  y^{-1}\right)
^{-1}=y$ is blue).

Here is how the two arcs $x$ and $y$ look like in the digraph $\left(  V,A\cup
A^{-1}\right)  $:%
\[%
%TCIMACRO{\TeXButton{two arcs}{\xymatrix{
%& \ar[r]^y & v \ar[r]^x &
%}}}%
%BeginExpansion
\xymatrix{
& \ar[r]^y & v \ar[r]^x &
}%
%EndExpansion
.
\]
And here is how the corresponding arcs $x^{-1}$ and $y^{-1}$ look like in the
underlying digraph $\left(  V,A\right)  $ of our network:%
\begin{equation}%
%TCIMACRO{\TeXButton{two arcs}{\xymatrix{
%& & v \ar[l]_{y^{-1}} & \ar[l]_{x^{-1}}
%}}}%
%BeginExpansion
\xymatrix{
& & v \ar[l]_{y^{-1}} & \ar[l]_{x^{-1}}
}%
%EndExpansion
. \label{pf.lem.6.a.sc2.2.3}%
\end{equation}
Both arcs $x^{-1}$ and $y^{-1}$ are sent to $-\rho$ by $g$ (as we have seen).
All other arcs in $A$ having source or target $v$ are sent to $0$ by
$g$\ \ \ \ \footnote{\textit{Proof.} Let $d$ be any arc in $A$ having source
or target $v$. Assume that $d$ is distinct from both $x^{-1}$ and $y^{-1}$. We
must prove that $g\left(  d\right)  =0$.
\par
Assume the contrary. Thus, $g\left(  d\right)  \neq0$. According to the
definition of $g$, we thus conclude that we have either $\left(  d\in P\text{
and }d\text{ is red}\right)  $ or $\left(  d^{-1}\in P\text{ and }d\text{ is
blue}\right)  $.
\par
If $\left(  d\in P\text{ and }d\text{ is red}\right)  $, then $d$ is either
$x$ or $y$ (since $d\in P$, but the only arcs in $P$ having source or target
$v$ are $x$ and $y$), which yields that $d$ must be blue (since both $x$ and
$y$ are blue), contradicting the fact that $d$ is red.
\par
If $\left(  d^{-1}\in P\text{ and }d\text{ is blue}\right)  $, then $d^{-1}$
is either $x$ or $y$ (since $d\in P$, but the only arcs in $P$ having source
or target $v$ are $x$ and $y$), which yields that $d$ is either $x^{-1}$ or
$y^{-1}$, contradicting the fact that $d$ is distinct from both $x^{-1}$ and
$y^{-1}$.
\par
Thus, we have obtained a contradiction from both possible options. This
completes our proof of $g\left(  d\right)  =0$.}. Hence, in particular, all
arcs in $A$ having target $v$ are sent to $0$ by $g$, except for the arc
$x^{-1}$. Thus, $g^{-}\left(  v\right)  =g\left(  x^{-1}\right)  =-\rho$. A
similar argument shows that $g^{+}\left(  v\right)  =-\rho$. Thus,
$g^{-}\left(  v\right)  =-\rho=g^{+}\left(  v\right)  $. Thus, $g^{-}\left(
v\right)  =g^{+}\left(  v\right)  $ is proven in Subcase 2.2.

The proof in Subcase 2.1 is rather similar, with the little difference that
now the picture (\ref{pf.lem.6.a.sc2.2.3}) is replaced by%
\[%
%TCIMACRO{\TeXButton{two arcs}{\xymatrix{
%& \ar[r]^y & v & \ar[l]_{x^{-1}}
%}}}%
%BeginExpansion
\xymatrix{
& \ar[r]^y & v & \ar[l]_{x^{-1}}
}%
%EndExpansion
,
\]
and so we have $g^{-}\left(  v\right)  =\left(  -\rho\right)  +\rho=0$ and
$g^{+}\left(  v\right)  =0$. But again the result is the same.

Thus, Case 2 is settled. The proof in Case 1 is similar (again, we need to
consider two subcases, depending on whether the arc $y$ is colored red or blue).

Altogether, we have now proven $g^{-}\left(  v\right)  =g^{+}\left(  v\right)
$ in each possible case.

Now, forget that we fixed $v$. We thus have shown that each $v\in
V\setminus\left\{  s,t\right\}  $ satisfies $g^{-}\left(  v\right)
=g^{+}\left(  v\right)  $. In other words, $g$ satisfies the conservation
constraints. This proves Lemma \ref{lem.6} \textbf{(a)}.

\textbf{(b)} This is somewhat similar to our above proof of Lemma \ref{lem.6}
\textbf{(a)}, with the vertex $s$ playing the role of $v$.

The vertex $s$ is the starting point of the path $\mathbf{p}$, but not its
ending point (since its ending point is $t\neq s$). Hence, there is exactly
one arc $x\in P$ having source $s$, and there are no arcs $y\in P$ having
target $s$.

We have $x\in P$. Thus, the arc $x$ is colored either red or blue. Hence, we
are in one of the following two cases:

\begin{statement}
\textit{Case 1:} The arc $x$ is colored red.
\end{statement}

\begin{statement}
\textit{Case 2:} The arc $x$ is colored blue.
\end{statement}

Let us consider Case 2. In this case, the arc $x$ is colored blue. Hence,
$x\in A^{-1}$ (by the requirements on the coloring), so that $x^{-1}\in A$.
The definition of $g$ thus shows that $g\left(  x^{-1}\right)  =-\rho$ (since
$\left(  x^{-1}\right)  ^{-1}=x\in P$ and since $\left(  x^{-1}\right)
^{-1}=x$ is blue).

All other arcs in $A$ having source or target $s$ are sent to $0$ by
$g$\ \ \ \ \footnote{This is easy to check -- just as in the above proof of
Lemma \ref{lem.6} \textbf{(a)}.}. Hence, in particular, all arcs in $A$ having
target $s$ are sent to $0$ by $g$, except for the arc $x^{-1}$. Thus,
$g^{-}\left(  s\right)  =g\left(  x^{-1}\right)  =-\rho$. A similar argument
shows that $g^{+}\left(  s\right)  =0$. Now, the definition of $\left\vert
g\right\vert $ yields $\left\vert g\right\vert =\underbrace{g^{+}\left(
s\right)  }_{=0}-\underbrace{g^{-}\left(  s\right)  }_{=-\rho}=0-\left(
-\rho\right)  =\rho$. Thus, Lemma \ref{lem.6} \textbf{(b)} is proven in Case 2.

A similar argument works in Case 1. Thus, Lemma \ref{lem.6} \textbf{(b)} is
always proven.
\end{proof}

We are now ready to prove Lemma \ref{lem.4}:

\begin{proof}
[Proof of Lemma \ref{lem.4} (sketched).]The definition of $D_{f}$ yields
$D_{f}=\left(  V,A_{f}\right)  $, where%
\[
A_{f}=\left\{  a\in A\ \mid\ f\left(  a\right)  <c\left(  a\right)  \right\}
\cup\left\{  a^{-1}\ \mid\ a\in A;\ f\left(  a\right)  >0\right\}  .
\]
Consider this $A_{f}$. Clearly, $A_{f}\subseteq A\cup A^{-1}$, so that $D_{f}$
is a sub-digraph of the digraph $\left(  V,A\cup A^{-1}\right)  $.

\textbf{(a)} Assume that the digraph $D_{f}$ has a path from $s$ to $t$. Fix
such a path, and denote it by $\mathbf{p}$.

Hence, $\mathbf{p}$ is a path in the digraph $\left(  V,A\cup A^{-1}\right)  $
(since $\mathbf{p}$ is a path in $D_{f}$, but $D_{f}$ is a sub-digraph of the
digraph $\left(  V,A\cup A^{-1}\right)  $).

Let $P$ be the set of arcs of $\mathbf{p}$. Thus, $P\subseteq A_{f}$ (since
$\mathbf{p}$ is a path in $D_{f}=\left(  V,A_{f}\right)  $). The path
$\mathbf{p}$ is a path from $s$ to $t$, and thus has at least one arc (since
$s\neq t$); hence, the set $P$ is nonempty.

We color each arc $b\in P$ either red or blue, according to the following rules:

\begin{itemize}
\item If $b\in\left\{  a\in A\ \mid\ f\left(  a\right)  <c\left(  a\right)
\right\}  $, then we color $b$ red.

\item Otherwise, if $b\in\left\{  a^{-1}\ \mid\ a\in A;\ f\left(  a\right)
>0\right\}  $, then we color $b$ blue.
\end{itemize}

This yields a well-defined coloring, because every $b\in P$ must belong to at
least one of the two sets $\left\{  a\in A\ \mid\ f\left(  a\right)  <c\left(
a\right)  \right\}  $ and $\left\{  a^{-1}\ \mid\ a\in A;\ f\left(  a\right)
>0\right\}  $\ \ \ \ \footnote{indeed, $b\in P\subseteq A_{f}=\left\{  a\in
A\ \mid\ f\left(  a\right)  <c\left(  a\right)  \right\}  \cup\left\{
a^{-1}\ \mid\ a\in A;\ f\left(  a\right)  >0\right\}  $}.

Furthermore, our coloring of the arcs in $P$ clearly has the property that:

\begin{itemize}
\item if an arc $b\in P$ is colored red, then $b\in A$;

\item if an arc $b\in P$ is colored blue, then $b\in A^{-1}$ (that is,
$b^{-1}\in A$).
\end{itemize}

For each arc $b\in P$, we define a number $\rho_{b}\in\mathbb{Q}$ by%
\[
\rho_{b}=%
\begin{cases}
c\left(  b\right)  -f\left(  b\right)  , & \text{if }b\text{ is red};\\
f\left(  b^{-1}\right)  , & \text{if }b\text{ is blue}%
\end{cases}
.
\]
This number $\rho_{b}$ is always positive\footnote{\textit{Proof.} Let $b\in
P$ be any arc. We must prove that $\rho_{b}$ is positive.
\par
We have $b\in P\subseteq A_{f}=\left\{  a\in A\ \mid\ f\left(  a\right)
<c\left(  a\right)  \right\}  \cup\left\{  a^{-1}\ \mid\ a\in A;\ f\left(
a\right)  >0\right\}  $. Thus, either $b\in\left\{  a\in A\ \mid\ f\left(
a\right)  <c\left(  a\right)  \right\}  $ or $b\in\left\{  a^{-1}\ \mid\ a\in
A;\ f\left(  a\right)  >0\right\}  $. So we are in one of the following two
cases:
\par
\textit{Case 1:} We have $b\in\left\{  a\in A\ \mid\ f\left(  a\right)
<c\left(  a\right)  \right\}  $.
\par
\textit{Case 2:} We don't have $b\in\left\{  a\in A\ \mid\ f\left(  a\right)
<c\left(  a\right)  \right\}  $, but we have $b\in\left\{  a^{-1}\ \mid\ a\in
A;\ f\left(  a\right)  >0\right\}  $.
\par
Let us first consider Case 1. In this case, we have $b\in\left\{  a\in
A\ \mid\ f\left(  a\right)  <c\left(  a\right)  \right\}  $. Hence, the
definition of the coloring yields that $b$ is colored red. But $b\in\left\{
a\in A\ \mid\ f\left(  a\right)  <c\left(  a\right)  \right\}  $; in other
words, $b\in A$ and $f\left(  b\right)  <c\left(  b\right)  $. Since $b$ is
red, the definition of $\rho_{b}$ yields $\rho_{b}=c\left(  b\right)
-f\left(  b\right)  >0$ (since $f\left(  b\right)  <c\left(  b\right)  $).
Thus, we have shown that $\rho_{b}$ is positive in Case 1.
\par
Let us now consider Case 2. In this case, we don't have $b\in\left\{  a\in
A\ \mid\ f\left(  a\right)  <c\left(  a\right)  \right\}  $, but we have
$b\in\left\{  a^{-1}\ \mid\ a\in A;\ f\left(  a\right)  >0\right\}  $. Hence,
the definition of the coloring yields that $b$ is colored blue. But
$b\in\left\{  a^{-1}\ \mid\ a\in A;\ f\left(  a\right)  >0\right\}  $. In
other words, $b=a^{-1}$ for some $a\in A$ satisfying $f\left(  a\right)  >0$.
In other words, $b^{-1}\in A$ and $f\left(  b^{-1}\right)  >0$. Since $b$ is
blue, the definition of $\rho_{b}$ yields $\rho_{b}=f\left(  b^{-1}\right)
>0$. Thus, we have shown that $\rho_{b}$ is positive in Case 2.
\par
We have now proven that $\rho_{b}$ is positive in both Cases 1 and 2. Hence,
$\rho_{b}$ is always positive, qed.}.

We now define $\rho\in\mathbb{Q}$ by $\rho=\min\left\{  \rho_{b}\mid b\in
P\right\}  $. (This is well-defined, since $P$ is nonempty and finite.) Then,
$\rho$ is positive (since all $\rho_{b}$ are positive), so that $\rho\geq0$.

Now, define a map $g:A\rightarrow\mathbb{Q}$ as in Lemma \ref{lem.6}. Then,
Lemma \ref{lem.6} \textbf{(a)} shows that the map $g$ satisfies the
conservation constraints. Also, the map $f$ satisfies the conservation
constraints (since $f$ is a flow). Hence, Lemma \ref{lem.5} \textbf{(b)} shows
that the map $f+g$ satisfies the conservation constraints. Also, Lemma
\ref{lem.5} \textbf{(a)} shows that $\left\vert f+g\right\vert =\left\vert
f\right\vert +\underbrace{\left\vert g\right\vert }_{\substack{=\rho
\\\text{(by Lemma \ref{lem.6} \textbf{(b)})}}}=\left\vert f\right\vert +\rho$.

Next, we claim that the map $f+g$ also satisfies the capacity constraints --
i.e., that we have%
\begin{equation}
0\leq\left(  f+g\right)  \left(  a\right)  \leq c\left(  a\right)
\ \ \ \ \ \ \ \ \ \ \text{for each }a\in A. \label{pf.lem.4.a.capac1}%
\end{equation}


[\textit{Proof of (\ref{pf.lem.4.a.capac1}):} Fix $a\in A$. Recall that $f$ is
a flow; thus, $f$ satisfies the capacity constraints. Hence, $0\leq f\left(
a\right)  \leq c\left(  a\right)  $.

Now, we are in one of the following three cases:

\begin{statement}
\textit{Case 1:} We have $a\in P$, and the arc $a$ is red.
\end{statement}

\begin{statement}
\textit{Case 2:} We have $a^{-1}\in P$, and the arc $a^{-1}$ is blue.
\end{statement}

\begin{statement}
\textit{Case 3:} Neither of these two.
\end{statement}

Let us first consider Case 3. In this case, the definition of $g$ yields
$g\left(  a\right)  =0$. Thus, $\left(  f+g\right)  \left(  a\right)
=f\left(  a\right)  +\underbrace{g\left(  a\right)  }_{=0}=f\left(  a\right)
$. Hence, $0\leq f\left(  a\right)  \leq c\left(  a\right)  $ rewrites as
$0\leq\left(  f+g\right)  \left(  a\right)  \leq c\left(  a\right)  $. Thus,
(\ref{pf.lem.4.a.capac1}) is proven in Case 3.

Let us now consider Case 2. In this case, we have $a^{-1}\in P$, and the arc
$a^{-1}$ is blue. Hence, the definition of $g$ yields $g\left(  a\right)
=-\rho$. But the definition of $\rho$ yields $\rho=\min\left\{  \rho_{b}\mid
b\in P\right\}  \leq\rho_{a^{-1}}$ (since $a^{-1}\in P$ and thus $\rho
_{a^{-1}}\in\left\{  \rho_{b}\mid b\in P\right\}  $). Since $a^{-1}$ is blue,
the definition of $\rho_{a^{-1}}$ yields $\rho_{a^{-1}}=f\left(
\underbrace{\left(  a^{-1}\right)  ^{-1}}_{=a}\right)  =f\left(  a\right)  $.
Hence, $\rho\leq\rho_{a^{-1}}=f\left(  a\right)  $. Now, $\left(  f+g\right)
\left(  a\right)  =f\left(  a\right)  +\underbrace{g\left(  a\right)
}_{=-\rho}=f\left(  a\right)  -\rho\geq0$ (since $\rho\leq f\left(  a\right)
$). Combining this with $\left(  f+g\right)  \left(  a\right)  =f\left(
a\right)  +\underbrace{g\left(  a\right)  }_{=-\rho}=f\left(  a\right)
-\underbrace{\rho}_{\geq0}\leq f\left(  a\right)  \leq c\left(  a\right)  $,
we obtain $0\leq\left(  f+g\right)  \left(  a\right)  \leq c\left(  a\right)
$. Thus, (\ref{pf.lem.4.a.capac1}) is proven in Case 2.

Let us finally consider Case 1. In this case, we have $a\in P$, and the arc
$a$ is red. Hence, the definition of $g$ yields $g\left(  a\right)  =\rho$.
But the definition of $\rho$ yields $\rho=\min\left\{  \rho_{b}\mid b\in
P\right\}  \leq\rho_{a}$ (since $a\in P$ and thus $\rho_{a}\in\left\{
\rho_{b}\mid b\in P\right\}  $). Since $a$ is red, the definition of $\rho
_{a}$ yields $\rho_{a}=c\left(  a\right)  -f\left(  a\right)  $. Hence,
$\rho\leq\rho_{a}=c\left(  a\right)  -f\left(  a\right)  $. Now, $\left(
f+g\right)  \left(  a\right)  =f\left(  a\right)  +\underbrace{g\left(
a\right)  }_{=\rho}=f\left(  a\right)  +\rho\leq c\left(  a\right)  $ (since
$\rho\leq c\left(  a\right)  -f\left(  a\right)  $). Combining this with
$\left(  f+g\right)  \left(  a\right)  =f\left(  a\right)
+\underbrace{g\left(  a\right)  }_{=\rho}=f\left(  a\right)  +\underbrace{\rho
}_{\geq0}\geq f\left(  a\right)  \geq0$, we obtain $0\leq\left(  f+g\right)
\left(  a\right)  \leq c\left(  a\right)  $. Thus, (\ref{pf.lem.4.a.capac1})
is proven in Case 1.

We have thus proven (\ref{pf.lem.4.a.capac1}) in all three Cases 1, 2 and 3.
Thus, (\ref{pf.lem.4.a.capac1}) is proven.]

From (\ref{pf.lem.4.a.capac1}), we conclude in particular that $\left(
f+g\right)  \left(  a\right)  \geq0$ for each $a\in A$. Thus, $f+g$ is a map
$A\rightarrow\mathbb{Q}_{+}$. We have shown that this map $f+g:A\rightarrow
\mathbb{Q}_{+}$ satisfies both the capacity constraints and the conservation
constraints. In other words, $f+g$ is a flow (by the definition of a flow).
Furthermore, this flow has value $\left\vert f+g\right\vert =\left\vert
f\right\vert +\rho>\left\vert f\right\vert $ (since $\rho$ is positive). In
other words, it has a larger value than $f$. Thus, there is a flow $f^{\prime
}$ with a larger value than $f$ (namely, $f^{\prime}=f+g$). This proves Lemma
\ref{lem.4} \textbf{(a)}.

\textbf{(b)} Assume that the digraph $D_{f}$ has no path from $s$ to $t$. We
must prove that there exists a subset $S$ of $V$ satisfying $s\in V$ and
$t\notin V$ and $c\left(  S,\overline{S}\right)  =\left\vert f\right\vert $.

Indeed, define a subset $S$ of $V$ by%
\[
S=\left\{  v\in V\ \mid\ \text{the digraph }D_{f}\text{ has a path from
}s\text{ to }v\right\}  .
\]
We shall show that $s\in S$ and $t\notin S$ and $c\left(  S,\overline
{S}\right)  =\left\vert f\right\vert $. This will clearly complete the proof
of Lemma \ref{lem.4} \textbf{(b)}.

First of all, the digraph $D_{f}$ clearly has a path from $s$ to $s$ (namely,
the trivial path $\left(  s\right)  $). In other words, $s\in S$ (by the
definition of $S$).

Furthermore, the digraph $D_{f}$ has no path from $s$ to $t$ (by assumption).
In other words, $t\notin S$ (by the definition of $S$).

It thus remains to show that $c\left(  S,\overline{S}\right)  =f$.

Let us first observe that%
\begin{equation}
\text{any }\left(  u,v\right)  \in A_{f}\text{ satisfying }u\in S\text{ must
satisfy }v\in S. \label{pf.lem.4.b.2}%
\end{equation}


[\textit{Proof of (\ref{pf.lem.4.b.2}):} Let $\left(  u,v\right)  \in A_{f}$
be such that $u\in S$. We must prove $v\in S$.

We have $u\in S$. In other words, the digraph $D_{f}$ has a path from $s$ to
$u$ (by the definition of $S$). Since $\left(  u,v\right)  $ is an arc of
$D_{f}$ (because $\left(  u,v\right)  \in A_{f}$), we can extend this path by
the arc $\left(  u,v\right)  $ to obtain a walk from $s$ to $v$. We can then
trim down this walk to a path from $s$ to $v$ (by removing circuits). Thus,
the digraph $D_{f}$ has a path from $s$ to $v$. In other words, $v\in S$ (by
the definition of $S$). This proves (\ref{pf.lem.4.b.2}).]

Next, we observe that%
\begin{equation}
f\left(  a\right)  =0\ \ \ \ \ \ \ \ \ \ \text{for each }a\in\left[
\overline{S},S\right]  . \label{pf.lem.4.b.3}%
\end{equation}


[\textit{Proof of (\ref{pf.lem.4.b.3}):} Let $a\in\left[  \overline
{S},S\right]  $. Thus, $a\in A$ and $a=\left(  v,u\right)  $ for some
$v\in\overline{S}$ and $u\in S$ (by the definition of $\left[  \overline
{S},S\right]  $). Consider these $v$ and $u$.

We must prove that $f\left(  a\right)  =0$. Assume the contrary. Thus,
$f\left(  a\right)  \neq0$, so that $f\left(  a\right)  >0$ (since $f\left(
a\right)  \geq0$). By the definition of $A_{f}$, we thus have $a^{-1}\in
A_{f}$. But $a=\left(  v,u\right)  $, so that $a^{-1}=\left(  u,v\right)  $.
Thus, $\left(  u,v\right)  =a^{-1}\in A_{f}$. Therefore, (\ref{pf.lem.4.b.2})
yields $v\in S$. This contradicts $v\in\overline{S}$. This contradiction shows
that our assumption was false. Hence, $f\left(  a\right)  =0$ holds, and
(\ref{pf.lem.4.b.3}) is proven.]

Similarly, we have%
\begin{equation}
f\left(  a\right)  =c\left(  a\right)  \ \ \ \ \ \ \ \ \ \ \text{for each
}a\in\left[  S,\overline{S}\right]  . \label{pf.lem.4.b.4}%
\end{equation}


[\textit{Proof of (\ref{pf.lem.4.b.4}):} Let $a\in\left[  S,\overline
{S}\right]  $. Thus, $a\in A$ and $a=\left(  u,v\right)  $ for some $u\in S$
and $v\in\overline{S}$ (by the definition of $\left[  S,\overline{S}\right]
$). Consider these $u$ and $v$.

We must prove that $f\left(  a\right)  =c\left(  a\right)  $. Assume the
contrary. Thus, $f\left(  a\right)  \neq c\left(  a\right)  $, so that
$f\left(  a\right)  <c\left(  a\right)  $ (since $f\left(  a\right)  \leq
c\left(  a\right)  $ (by the capacity constraints)). By the definition of
$A_{f}$, we thus have $a\in A_{f}$. Thus, $\left(  u,v\right)  =a\in A_{f}$.
Therefore, (\ref{pf.lem.4.b.2}) yields $v\in S$. This contradicts
$v\in\overline{S}$. This contradiction shows that our assumption was false.
Hence, $f\left(  a\right)  = c \left( a\right) $ holds, and
(\ref{pf.lem.4.b.4}) is proven.]

Now, the definition of $f\left(  \overline{S},S\right)  $ yields $f\left(
\overline{S},S\right)  =\sum_{a\in\left[  \overline{S},S\right]
}\underbrace{f\left(  a\right)  }_{\substack{=0\\\text{(by (\ref{pf.lem.4.b.3}%
))}}}=0$.

Also, the definition of $f\left(  S,\overline{S}\right)  $ yields $f\left(
S,\overline{S}\right)  =\sum_{a\in\left[  S,\overline{S}\right]
}\underbrace{f\left(  a\right)  }_{\substack{=c\left(  a\right)  \\\text{(by
(\ref{pf.lem.4.b.4}))}}}=\sum_{a\in\left[  S,\overline{S}\right]  }c\left(
a\right)  =c\left(  S,\overline{S}\right)  $ (by the definition of $c\left(
S,\overline{S}\right)  $).

Now, Proposition \ref{prop.2} \textbf{(b)} yields%
\[
\left\vert f\right\vert =\underbrace{f\left(  S,\overline{S}\right)
}_{=c\left(  S,\overline{S}\right)  }-\underbrace{f\left(  \overline
{S},S\right)  }_{=0}=c\left(  S,\overline{S}\right)  .
\]
In other words, $c\left(  S,\overline{S}\right)  =\left\vert f\right\vert $.
This completes our proof of Lemma \ref{lem.4} \textbf{(b)}.
\end{proof}

An analogue of Lemma \ref{lem.4} holds for integer flows (assuming integer capacities):

\begin{lemma}
\label{lem.4int}Let $N$, $V$, $A$, $s$, $t$ and $c$ be as in Definition
\ref{def.network.notations}. Assume that $c\left(  a\right)  \in\mathbb{N}$
for each $a\in A$.

Let $f:A\rightarrow\mathbb{Q}_{+}$ be an integer flow. Then,

\textbf{(a)} If the digraph $D_{f}$ has a path from $s$ to $t$, then there is
an integer flow $f^{\prime}$ with a larger value than $f$.

\textbf{(b)} If the digraph $D_{f}$ has no path from $s$ to $t$, then there
exists a subset $S$ of $V$ satisfying $s\in V$ and $t\notin V$ and $c\left(
S,\overline{S}\right)  =\left\vert f\right\vert $.
\end{lemma}

\begin{proof}
[Proof of Lemma \ref{lem.4int} (sketched).]The same argument we gave above for
Lemma \ref{lem.4} works here, but we need to notice that all the $\rho_{b}$
are integers (since the $f\left(  a\right)  $ and the $c\left(  a\right)  $
are integers), and therefore $\rho$ is an integer.
\end{proof}

\subsection{The Ford-Fulkerson algorithm}

We next claim the following:

\begin{lemma}
\label{lem.3bat}Let $N$, $V$, $A$, $s$, $t$ and $c$ be as in Definition
\ref{def.network.notations}. Assume that $c\left(  a\right)  \in\mathbb{N}$
for each $a\in A$. Then, there exist an integer flow $f:A\rightarrow
\mathbb{Q}_{+}$ and a subset $S$ of $V$ satisfying $s\in S$ and $t\notin S$
and $c\left(  S,\overline{S}\right)  =\left\vert f\right\vert $.
\end{lemma}

\begin{proof}
[Proof of Lemma \ref{lem.3bat} (sketched).]Let us first recall that if we are
given a digraph and two of its vertices, we can easily check whether there is
a path from one vertex to the other. (A stupid way to do this would be to try
all possible paths; but there are fast algorithms, such as
\href{https://en.wikipedia.org/wiki/Dijkstra's_algorithm}{Dijkstra's
algorithm}.)

We claim that the following algorithm always terminates and constructs an
integer flow $f:A\rightarrow\mathbb{Q}_{+}$ and a subset $S$ of $V$ satisfying
$s\in S$ and $t\notin S$ and $c\left(  S,\overline{S}\right)  =\left\vert
f\right\vert $:

\begin{statement}
\underline{\textbf{Ford-Fulkerson algorithm:}}

\underline{\textbf{Input:}} $N$, $V$, $A$, $s$, $t$ and $c$ as in Definition
\ref{def.network.notations}, with the property that $c\left(  a\right)
\in\mathbb{N}$ for each $a\in A$.

\underline{\textbf{Output:}} an integer flow $f:A\rightarrow\mathbb{Q}_{+}$
and a subset $S$ of $V$ satisfying $s\in S$ and $t\notin S$ and $c\left(
S,\overline{S}\right)  =\left\vert f\right\vert $.

\underline{\textbf{Algorithm:}}

\begin{itemize}
\item Define an integer flow $f:A\rightarrow\mathbb{Q}_{+}$ by $\left(
f\left(  a\right)  =0\text{ for all }a\in A\right)  $. (This is clearly a
flow; it is called the \textit{zero flow}.)

\item \textbf{While}\footnote{This word \textquotedblleft\textbf{While}%
\textquotedblright\ marks the beginning of a
\href{https://en.wikipedia.org/wiki/While_loop}{while loop}. The end of this
loop is marked by the word \textquotedblleft\textbf{Endwhile}%
\textquotedblright. Roughly speaking, the instructions inside a while loop
need to be performed over and over until the condition is no longer true
(i.e., in our case, until the digraph $D_{f}$ no longer has a path from $s$ to
$t$). So we need to check whether $D_{f}$ has a path from $s$ to $t$, then --
if there is such a path -- perform whatever is inside the while loop, then
again check whether $D_{f}$ has a path from $s$ to $t$, then -- if there is
such a path -- again perform whatever is inside the while loop, and so on,
until we finally fail the check (i.e., until $D_{f}$ no longer has a path from
$s$ to $t$). Only then do we proceed beyond the while loop.} the digraph
$D_{f}$ has a path from $s$ to $t$\ \ \ \ \footnote{This can be checked, e.g.,
using Dijkstra's algorithm, as mentioned above.}, \textbf{do the following:}

\begin{itemize}
\item Apply Lemma \ref{lem.4int} \textbf{(a)} to construct an integer flow
$f^{\prime}$ with a larger value than $f$. (Strictly speaking, what we mean is
to apply the construction given in the proof of Lemma \ref{lem.4int}
\textbf{(a)}.)

\item Replace $f$ by $f^{\prime}$. (Thus, the flow $f$ now has a larger value
than it had before.)
\end{itemize}

\textbf{Endwhile}.

\item Now, the digraph $D_{f}$ has no path from $s$ to $t$. Therefore, Lemma
\ref{lem.4int} \textbf{(b)} shows that there exists a subset $S$ of $V$
satisfying $s\in V$ and $t\notin V$ and $c\left(  S,\overline{S}\right)
=\left\vert f\right\vert $. Consider this $S$. (Again, the actual construction
of $S$ is hidden in the proof of Lemma \ref{lem.4int} \textbf{(b)}.)

\item Output (the current values of) $f$ and $S$.
\end{itemize}
\end{statement}

Why does this algorithm work? In other words, why does it necessarily
terminate (i.e., why don't we get stuck in the while loop forever?), and why
does the output satisfy the requirements (that $f:A\rightarrow\mathbb{Q}_{+}$
is an integer flow and that $S$ is a subset of $V$ satisfying $s\in S$ and
$t\notin S$ and $c\left(  S,\overline{S}\right)  =\left\vert f\right\vert $) ?

The second question has already been answered in the description of the
algorithm; so all we need to do is answer the first question. In other words,
we need to prove the following claim:

\begin{statement}
\textit{Claim 1:} The algorithm cannot get stuck in the while loop (i.e., it
cannot happen that it keeps running this loop forever).
\end{statement}

[\textit{Proof of Claim 1:} The value of any integer flow is an integer. Thus,
$\left\vert f\right\vert $ is an integer throughout the execution of the
algorithm (since $f$ is an integer flow throughout the execution of the algorithm).

On the other hand, Proposition \ref{prop.2} \textbf{(c)} (applied to $\left\{
s\right\}  $ instead of $S$) shows that $\left\vert f\right\vert \leq c\left(
\left\{  s\right\}  ,\overline{\left\{  s\right\}  }\right)  $ throughout the
execution of the algorithm (since $s\in\left\{  s\right\}  $ and
$t\notin\left\{  s\right\}  $).

Each iteration of the while loop increases the value $\left\vert f\right\vert
$ (because $f$ is replaced by a new integer flow $f^{\prime}$ with a larger
value than the previous $f$), and thus increases it by at least $1$ (since
$\left\vert f\right\vert $ is an integer throughout the execution of the
algorithm, but if an integer increases, then it increases by at least $1$).
Hence, if the while loop is traversed $K$ times (for some $K\in\mathbb{N}$),
then the value $\left\vert f\right\vert $ increases by at least $K$, and thus
this value $\left\vert f\right\vert $ must be at least $K$ afterwards (because
the value $\left\vert f\right\vert $ at the beginning of the algorithm is
$0$). Applying this to $K=c\left(  \left\{  s\right\}  ,\overline{\left\{
s\right\}  }\right)  +1$, we conclude that if the while loop is traversed
$c\left(  \left\{  s\right\}  ,\overline{\left\{  s\right\}  }\right)  +1$
times, then the value $\left\vert f\right\vert $ must be at least $c\left(
\left\{  s\right\}  ,\overline{\left\{  s\right\}  }\right)  +1$ afterwards.
Since the value $\left\vert f\right\vert $ can never be at least $c\left(
\left\{  s\right\}  ,\overline{\left\{  s\right\}  }\right)  +1$ (because we
have $\left\vert f\right\vert \leq c\left(  \left\{  s\right\}  ,\overline
{\left\{  s\right\}  }\right)  <c\left(  \left\{  s\right\}  ,\overline
{\left\{  s\right\}  }\right)  +1$ throughout the execution of the algorithm),
we thus conclude that the while loop cannot be traversed $c\left(  \left\{
s\right\}  ,\overline{\left\{  s\right\}  }\right)  +1$ times. Thus, the while
loop can only be traversed at most $c\left(  \left\{  s\right\}
,\overline{\left\{  s\right\}  }\right)  $ times. In particular, the algorithm
must eventually leave the while loop. This proves Claim 1.]

So we have shown that the algorithm works. Clearly, its output gives us
precisely the $f$ and the $S$ whose existence is claimed by Lemma
\ref{lem.3bat}. Thus, Lemma \ref{lem.3bat} is proven.
\end{proof}

Finally, we can prove Theorem \ref{thm.3b}:

\begin{proof}
[Proof of Theorem \ref{thm.3b} (sketched).]Lemma \ref{lem.3bat} shows that
there exist an integer flow $f:A\rightarrow\mathbb{Q}_{+}$ and a subset $S$ of
$V$ satisfying $s\in S$ and $t\notin S$ and $c\left(  S,\overline{S}\right)
=\left\vert f\right\vert $. Denote these $f$ and $S$ by $g$ and $Q$. Thus,
$g:A\rightarrow\mathbb{Q}_{+}$ is an integer flow, and $Q$ is a subset of $V$
satisfying $s\in Q$, $t\notin Q$ and $c\left(  Q,\overline{Q}\right)
=\left\vert g\right\vert $.

The set $\left\{  c\left(  S,\overline{S}\right)  \ \mid\ S\subseteq V;\ s\in
S;\ t\notin S\right\}  $ is nonempty and finite. Hence, its minimum
$\min\left\{  c\left(  S,\overline{S}\right)  \ \mid\ S\subseteq V;\ s\in
S;\ t\notin S\right\}  $ is well-defined.

We have $\left\vert g\right\vert =c\left(  Q,\overline{Q}\right)  \in\left\{
c\left(  S,\overline{S}\right)  \ \mid\ S\subseteq V;\ s\in S;\ t\notin
S\right\}  $ (since $Q\subseteq V$ and $s\in Q$ and $t\notin Q$) and thus%
\begin{equation}
\left\vert g\right\vert \geq\min\left\{  c\left(  S,\overline{S}\right)
\ \mid\ S\subseteq V;\ s\in S;\ t\notin S\right\}  . \label{pf.thm.3b.1}%
\end{equation}


On the other hand, $\left\vert f\right\vert \leq c\left(  Q,\overline
{Q}\right)  $ for every integer flow $f$ (by Proposition \ref{prop.2}
\textbf{(c)}, applied to $S=Q$). In other words, $\left\vert f\right\vert
\leq\left\vert g\right\vert $ for every integer flow $f$ (since $c\left(
Q,\overline{Q}\right)  =\left\vert g\right\vert $). In other words, any
element of the set $\left\{  \left\vert f\right\vert \ \mid\ f\text{ is an
integer flow}\right\}  $ is $\leq\left\vert g\right\vert $. Since we also know
that $\left\vert g\right\vert $ is an element of this set (because $g$ is an
integer flow), we thus conclude that $\left\vert g\right\vert $ is the maximum
of this set. In other words,%
\begin{equation}
\left\vert g\right\vert =\max\left\{  \left\vert f\right\vert \ \mid\ f\text{
is an integer flow}\right\}  \label{pf.thm.3b.2}%
\end{equation}
(and in particular, $\max\left\{  \left\vert f\right\vert \ \mid\ f\text{ is
an integer flow}\right\}  $ exists). Now, (\ref{pf.thm.3b.1}) yields%
\[
\min\left\{  c\left(  S,\overline{S}\right)  \ \mid\ S\subseteq V;\ s\in
S;\ t\notin S\right\}  \leq\left\vert g\right\vert =\max\left\{  \left\vert
f\right\vert \ \mid\ f\text{ is an integer flow}\right\}  .
\]
Combining this with%
\[
\max\left\{  \left\vert f\right\vert \ \mid\ f\text{ is an integer
flow}\right\}  \leq\min\left\{  c\left(  S,\overline{S}\right)  \ \mid
\ S\subseteq V;\ s\in S;\ t\notin S\right\}
\]
(because every subset $S$ of $V$ satisfying $s\in S$ and $t\notin S$ and every
integer flow $f$ satisfy $\left\vert f\right\vert \leq c\left(  S,\overline
{S}\right)  $ (by Proposition \ref{prop.2} \textbf{(c)})), we obtain%
\[
\max\left\{  \left\vert f\right\vert \ \mid\ f\text{ is an integer
flow}\right\}  =\min\left\{  c\left(  S,\overline{S}\right)  \ \mid
\ S\subseteq V;\ s\in S;\ t\notin S\right\}  .
\]
This proves Theorem \ref{thm.3b}.
\end{proof}

\begin{remark}
\label{rmk.3balg}The Ford-Fulkerson algorithm that we used in our proof of
Lemma \ref{lem.3bat} constructs an integer flow $f$ and a subset $S$ of $V$.
Denote these $f$ and $S$ by $g$ and $S$. Then, $g$ is an integer flow of
maximum value (among all integer flows on $N$). This follows immediately from
(\ref{pf.thm.3b.2}). Thus, we have found an algorithm to construct an integer
flow of maximum value.
\end{remark}

The proof of Theorem \ref{thm.3a} is no longer far away. We first need an
analogue of Lemma \ref{lem.3bat}:

\begin{lemma}
\label{lem.3aat}Let $N$, $V$, $A$, $s$, $t$ and $c$ be as in Definition
\ref{def.network.notations}. Then, there exist a flow $f:A\rightarrow
\mathbb{Q}_{+}$ and a subset $S$ of $V$ satisfying $s\in S$ and $t\notin S$
and $c\left(  S,\overline{S}\right)  =\left\vert f\right\vert $.
\end{lemma}

\begin{proof}
[Proof of Lemma \ref{lem.3aat} (sketched).]The numbers $c\left(  a\right)  $
for $a\in A$ are finitely many nonnegative rational numbers (since $c$ is a
map $A\rightarrow\mathbb{Q}_{+}$). Thus, they can be brought to a common
denominator: i.e., there exists a positive integer $p$ such that $p\cdot
c\left(  a\right)  \in\mathbb{N}$ for each $a\in A$. Consider this $p$. Define
the function $c_{p}:A\rightarrow\mathbb{Q}_{+}$ by
\[
c_{p}\left(  a\right)  =p\cdot c\left(  a\right)
\ \ \ \ \ \ \ \ \ \ \text{for each }a\in A.
\]
Thus, $c_{p}\left(  a\right)  =p\cdot c\left(  a\right)  \in\mathbb{N}$ for
each $a\in A$.

Let $N_{p}$ be the network which is the same as $N$, except that the capacity
function $c$ has been replaced by $c_{p}$. Thus, Lemma \ref{lem.3bat} (applied
to $N_{p}$ and $c_{p}$ instead of $N$ and $c$) yields that there exist an
integer flow $f:A\rightarrow\mathbb{Q}_{+}$ on the network $N_{p}$ and a
subset $S$ of $V$ satisfying $s\in S$ and $t\notin S$ and $c_{p}\left(
S,\overline{S}\right)  =\left\vert f\right\vert $. Consider these $f$ and $S$,
and denote them by $f_{p}$ and $S$. Thus, $f_{p}$ is an integer flow on the
network $N_{p}$, and $S$ is a subset of $V$ satisfying $s\in S$ and $t\notin
S$ and $c_{p}\left(  S,\overline{S}\right)  =\left\vert f_{p}\right\vert $.

It is easy to see that $c_{p}\left(  P,Q\right)  =p\cdot c\left(  P,Q\right)
$ for any two subsets $P$ and $Q$ of $V$. Thus, $c_{p}\left(  S,\overline
{S}\right)  =p\cdot c\left(  S,\overline{S}\right)  $, so that $c\left(
S,\overline{S}\right)  =\underbrace{c_{p}\left(  S,\overline{S}\right)
}_{=\left\vert f_{p}\right\vert }/p=\left\vert f_{p}\right\vert /p$.

Now, define a map $f:A\rightarrow\mathbb{Q}_{+}$ (not an integer flow, in
general) by setting%
\[
f\left(  a\right)  =f_{p}\left(  a\right)  /p\ \ \ \ \ \ \ \ \ \ \text{for
each }a\in A.
\]
This map $f$ is obtained by rescaling the flow $f_{p}$ on the network $N_{p}$
by the factor $1/p$; thus, it is easy to see that $f$ is a flow on the network
$N$. (Roughly speaking, the capacity constraints for $f$ are obtained from the
capacity constraints for $f_{p}$ by dividing by $p$; the same holds for the
conservation constraints.)

Also, it is easy to see that $\left\vert f\right\vert =\left\vert
f_{p}\right\vert /p$. Comparing this with $c\left(  S,\overline{S}\right)
=\left\vert f_{p}\right\vert /p$, we obtain $c\left(  S,\overline{S}\right)
=\left\vert f\right\vert $.

We thus have found a flow $f:A\rightarrow\mathbb{Q}_{+}$ on the network $N$
and a subset $S$ of $V$ satisfying $s\in S$ and $t\notin S$ and $c\left(
S,\overline{S}\right)  =\left\vert f\right\vert $. This proves Lemma
\ref{lem.3aat}.
\end{proof}

\begin{proof}
[Proof of Theorem \ref{thm.3a} (sketched).]The proof of Theorem \ref{thm.3a}
is analogous to the proof of Theorem \ref{thm.3b}, but we need to use Lemma
\ref{lem.3aat} instead of Lemma \ref{lem.3bat}.
\end{proof}

As we said, we will not prove Theorem \ref{thm.3c} here; its proof is harder.
An analogue of Lemma \ref{lem.3aat} for $\mathbb{R}_{+}$ instead of
$\mathbb{Q}_{+}$ indeed holds, but the way we proved Lemma \ref{lem.3aat} does
not generalize to $\mathbb{R}_{+}$ (since there are no common denominators for
incommensurable irrational numbers). Notice also that the Ford-Fulkerson
algorithm (which we showed in the proof of Lemma \ref{lem.3bat}) works for
flows $f:A\rightarrow\mathbb{Q}_{+}$, but may fail when $\mathbb{Q}_{+}$ is
replaced by $\mathbb{R}_{+}$; the paper \cite{Zwick95} gives an example where
it gets stuck in the while loop due to an edge with irrational capacity (the
golden ratio). The good news is that Theorem \ref{thm.3c} is almost entirely
useless in combinatorics, to my knowledge at least. If you want to see a proof
of Theorem \ref{thm.3c}, check out texts on linear programming.

\begin{exercise}
Assume that we generalize the concept of a network by allowing the capacity
function $c$ to take the value $\infty$ as well (where $\infty$ is a symbol
that is understood to satisfy $\infty+q=\infty$ and $\infty+\infty=\infty$ and
$\infty>q$ for all $q\in\mathbb{Q}$). In other words, some arcs $a$ may have
\textquotedblleft infinite capacity\textquotedblright, which means that the
capacity constraints for these arcs $a$ simply say that $0\leq f\left(
a\right)  $ (without requiring $f\left(  a\right)  $ to be $\leq$ to anything
specific). Prove the following:

\textbf{(a)} Theorem \ref{thm.3a} and Theorem \ref{thm.3b} still hold in this
generality, if we additionally assume that there exists \textbf{some} subset
$S$ of $V$ satisfying $s\in S$ and $t\notin S$ and $c\left(  S,\overline
{S}\right)  <\infty$.

\textbf{(b)} If every subset $S$ of $V$ satisfying $s\in S$ and $t\notin S$
satisfies $c\left(  S,\overline{S}\right)  =\infty$, then prove that, for
every $n\in\mathbb{N}$, there exists an integer flow $f$ of value $\left\vert
f\right\vert =n$.
\end{exercise}

\begin{exercise}
State and prove analogues of the above results in which the underlying digraph
of the network $N$ is replaced by a multidigraph.

[\textbf{Hint:} There are two ways to prove these analogues. One is to modify
the above proofs to make them support multidigraphs. Another is to reduce the
analogues to the above results for digraphs.]
\end{exercise}

\subsection{\label{sect.hall}Application: Bipartite matching and Hall's
marriage theorem}

As we said, the max-flow-min-cut theorems (particularly, Theorem \ref{thm.3b})
have multiple applications. In many of these applications, the network isn't
visible right away, but instead has to be constructed. Let me present one such
application: the bipartite matching problem.

If $W$ is a set, then $\mathcal{P}_{2}\left(  W\right)  $ shall mean the set
of all $2$-element subsets of $W$. Recall that a \textit{simple graph} is a
pair $\left(  W,E\right)  $ consisting of a finite set $W$ and a subset $E$ of
$\mathcal{P}_{2}\left(  W\right)  $. The elements of $W$ are called the
\textit{vertices} of this simple graph; the elements of $E$ are called its
\textit{edges}. If $e$ is an edge of a graph, then the two vertices contained
in $e$ are called the \textit{endpoints} of $e$.

\begin{definition}
\label{def.bipgraph}A \textit{bipartite graph} means a triple $\left(
G;X,Y\right)  $ (the semicolon means the same thing as a comma), where
$G=\left(  W,E\right)  $ is a simple graph, and where $X$ and $Y$ are two
subsets of $W$ with the following properties:

\begin{itemize}
\item We have $X\cap Y=\varnothing$ and $X\cup Y=W$.

\item Each edge of $G$ contains exactly one vertex in $X$ and exactly one
vertex in $Y$.
\end{itemize}
\end{definition}

For example, if $G$ is the following graph:%
\begin{equation}%
%TCIMACRO{\TeXButton{a graph not yet made bipartite}{\xymatrix{
%1 \are[r] \are[d] & 2 \are[d] & 5 \are[d] \\
%4 \are[r] & 3 & 6
%}}}%
%BeginExpansion
\xymatrix{
1 \are[r] \are[d] & 2 \are[d] & 5 \are[d] \\
4 \are[r] & 3 & 6
}%
%EndExpansion
, \label{eq.bipartite.example1}%
\end{equation}
then $\left(  G;\left\{  1,3,5\right\}  ,\left\{  2,4,6\right\}  \right)  $ is
a bipartite graph, and $\left(  G;\left\{  2,4,6\right\}  ,\left\{
1,3,5\right\}  \right)  $ is another bipartite graph, and $\left(  G;\left\{
1,3,6\right\}  ,\left\{  2,4,5\right\}  \right)  $ is yet another bipartite
graph, but $\left(  G;\left\{  1,2,3\right\}  ,\left\{  4,5,6\right\}
\right)  $ is not a bipartite graph (because the edge $\left\{  1,2\right\}  $
of $G$ contains two vertices in $\left\{  1,2,3\right\}  $, rather than one in
$\left\{  1,2,3\right\}  $ and one in $\left\{  4,5,6\right\}  $).

We often draw bipartite graphs in a rather special way. Namely, in order to
draw a bipartite graph $\left(  G;X,Y\right)  $, we draw the graph $G$, but
making sure that all vertices are aligned in two columns, where the left
column contains all the vertices in $X$ and the right column contains all the
vertices in $Y$. For example, if $G$ is the graph shown in
(\ref{eq.bipartite.example1}), then the bipartite graph $\left(  G;\left\{
1,3,5\right\}  ,\left\{  2,4,6\right\}  \right)  $ is drawn as%
\[%
%TCIMACRO{\TeXButton{a bipartite graph}{\xymatrix{
%1 \are[r] \are[rd] & 2 \\
%3 \are[r] \are[ru] & 4 \\
%5 \are[r] & 6
%}}}%
%BeginExpansion
\xymatrix{
1 \are[r] \are[rd] & 2 \\
3 \are[r] \are[ru] & 4 \\
5 \are[r] & 6
}%
%EndExpansion
,
\]
whereas the bipartite graph $\left(  G;\left\{  2,4,6\right\}  ,\left\{
1,3,5\right\}  \right)  $ is drawn as%
\[%
%TCIMACRO{\TeXButton{a bipartite graph}{\xymatrix{
%2 \are[r] \are[rd] & 1 \\
%4 \are[r] \are[ru] & 3 \\
%6 \are[r] & 5
%}}}%
%BeginExpansion
\xymatrix{
2 \are[r] \are[rd] & 1 \\
4 \are[r] \are[ru] & 3 \\
6 \are[r] & 5
}%
%EndExpansion
.
\]


The concept of a bipartite graph can be extended from simple graphs to
multigraphs, but we will not have much use for this extension here.

\begin{definition}
Let $G=\left(  W,E\right)  $ be a simple graph. A \textit{matching} in $G$
means a subset $M$ of $E$ such that no two distinct edges in $M$ have a vertex
in common.
\end{definition}

For example, the set $\left\{  \left\{  1,2\right\}  ,\left\{  5,6\right\}
\right\}  $ is a matching in the graph $G$ shown in
(\ref{eq.bipartite.example1}); so is the set $\left\{  \left\{  1,2\right\}
,\left\{  3,4\right\}  ,\left\{  5,6\right\}  \right\}  $ (but not the set
$\left\{  \left\{  1,2\right\}  ,\left\{  2,3\right\}  ,\left\{  5,6\right\}
\right\}  $, because its two edges $\left\{  1,2\right\}  $ and $\left\{
2,3\right\}  $ have the vertex $2$ in common). Also, the empty set
$\varnothing$ is a matching in any graph.

In graph theory, we are often interested in matchings that contain as many
edges as possible. There are, of course, simple bounds on how many edges a
matching can contain: For example, a matching in a simple graph $G=\left(
W,E\right)  $ can never have more than $\left\vert W\right\vert /2$ edges
(since each edge \textquotedblleft uses up\textquotedblright\ two vertices).
Also, if $\left(  G;X,Y\right)  $ is a bipartite graph, then a matching in $G$
can never have more than $\left\vert X\right\vert $ edges (since each edge
\textquotedblleft uses up\textquotedblright\ a vertex in $X$). How can we find
a maximum-sized matching in $G$ ? This is known as the \textit{bipartite
matching problem} (when $\left(  G;X,Y\right)  $ is a bipartite graph). It
turns out that the Ford-Fulkerson algorithm (developed in the above proof of
Lemma \ref{lem.3bat}) gives a way to solve this problem in polynomial time.

Let us introduce a couple more concepts:

\begin{definition}
Let $M$ be a matching in a simple graph $G=\left(  W,E\right)  $.

\textbf{(a)} A vertex $v$ of $G$ is said to be \textit{matched in $M$} if
there exists an edge $e\in M$ such that $v$ is an endpoint of $e$. In this
case, this edge is unique (since $M$ is a matching), and the other endpoint of
this edge (i.e., the one distinct from $v$) is called the \textit{$M$-partner}
of $v$.

\textbf{(b)} Let $S$ be a subset of $W$. The matching $M$ is said to be
\textit{$S$-complete} if each vertex $v\in S$ is matched in $M$.
\end{definition}

For example, if $G$ is the graph shown in (\ref{eq.bipartite.example1}), then
the matching $\left\{  \left\{  1,2\right\}  ,\left\{  3,4\right\}  ,\left\{
5,6\right\}  \right\}  $ is $\left\{  1,3,6\right\}  $-complete (since all
three vertices $1,3,6$ are matched in it\footnote{Their $\left\{  \left\{
1,2\right\}  ,\left\{  3,4\right\}  ,\left\{  5,6\right\}  \right\}
$-partners are $2,4,5$, respectively.}, but the matching $\left\{  \left\{
1,2\right\}  ,\left\{  5,6\right\}  \right\}  $ is not (since the vertex $3$
is not matched in it).

\begin{definition}
Let $G=\left(  W,E\right)  $ be a simple graph.

\textbf{(a)} If $v$ is a vertex of $G$, then a \textit{neighbor} of $v$ means
any vertex $w$ of $G$ such that $\left\{  v,w\right\}  \in E$.

\textbf{(b)} Let $U$ be a subset of $W$. Then, $N\left(  U\right)  $ shall
denote the subset%
\[
\left\{  v\in W\ \mid\ v\text{ has a neighbor in }U\right\}
\]
of $W$.
\end{definition}

For example, if $G$ is the graph shown in (\ref{eq.bipartite.example1}), then
the neighbors of the vertex $1$ are $2$ and $4$, and we have $N\left(
\left\{  1,2\right\}  \right)  =\left\{  1,2,3,4\right\}  $ and $N\left(
\left\{  1,3\right\}  \right)  =\left\{  2,4\right\}  $ and $N\left(  \left\{
2,5\right\}  \right)  =\left\{  1,3,6\right\}  $ and $N\left(  \varnothing
\right)  =\varnothing$.

The following is almost trivial:

\begin{proposition}
\label{prop.bipartite.NXY}Let $\left(  G;X,Y\right)  $ be a bipartite graph.
Let $U$ be a subset of $X$. Then, $N\left(  U\right)  \subseteq Y$.
\end{proposition}

\begin{proof}
[Proof of Proposition \ref{prop.bipartite.NXY}.]Write the graph $G$ in the
form $G=\left(  W,E\right)  $. Recall that $\left(  G;X,Y\right)  $ is a
bipartite graph; hence, $X\cap Y=\varnothing$ and $X\cup Y=W$. Thus,
$Y=W\setminus X$ and $X=W\setminus Y$.

Now, let $p\in N\left(  U\right)  $ be arbitrary. Thus, $p\in N\left(
U\right)  =\left\{  v\in W\ \mid\ v\text{ has a neighbor in }U\right\}  $ (by
the definition of $N\left(  U\right)  $). In other words, $p$ is an element of
$W$ that has a neighbor in $U$.

The vertex $p$ has a neighbor in $U$. Fix such a neighbor, and denote it by
$q$. Thus, $q\in U\subseteq X=W\setminus Y$, so that $q\notin Y$. We know that
$q$ is a neighbor of $p$; in other words, $\left\{  p,q\right\}  \in E$. In
other words, $\left\{  p,q\right\}  $ is an edge of $G$.

But $\left(  G;X,Y\right)  $ is a bipartite graph. Hence, each edge of $G$
contains exactly one vertex in $X$ and exactly one vertex in $Y$. Thus, in
particular, each edge of $G$ contains a vertex in $Y$. Applying this to
$\left\{  p,q\right\}  $, we conclude that $\left\{  p,q\right\}  $ contains a
vertex in $Y$. In other words, one of the two vertices $p$ and $q$ lies in
$Y$. This vertex cannot be $q$ (since $q\notin Y$), and thus must be $p$.
Hence, $p$ lies in $Y$. In other words, $p\in Y$.

Now, forget that we fixed $p$. We have thus shown that $p\in Y$ for each $p\in
N\left(  U\right)  $. In other words, $N\left(  U\right)  \subseteq Y$. This
proves Proposition \ref{prop.bipartite.NXY}.
\end{proof}

We shall now study matchings in bipartite graphs. The most important result
about such matchings is the following fact, known as \textit{Hall's marriage
theorem}:

\begin{theorem}
\label{thm.hall}Let $\left(  G;X,Y\right)  $ be a bipartite graph. Then, $G$
has an $X$-complete matching if and only if each subset $U$ of $X$ satisfies
$\left\vert N\left(  U\right)  \right\vert \geq\left\vert U\right\vert $.
\end{theorem}

Theorem \ref{thm.hall} has applications throughout mathematics, and several
equivalent versions; it also has fairly elementary (but tricky) proofs (see,
e.g., \cite[\S 12.5.2]{LeLeMe17}). We shall derive it from Theorem
\ref{thm.3b}.

In order to do so (and also, in order to reduce the bipartite matching problem
to the Ford-Fulkerson algorithm), we need to construct a network from a given
bipartite graph such that the integer flows in the network shall correspond to
the matchings in the graph. Let us do this.

\begin{condition}
For the rest of Section \ref{sect.hall}, we fix a bipartite graph $\left(
G;X,Y\right)  $. Write the simple graph $G$ in the form $G=\left(  W,E\right)
$.
\end{condition}

Thus, each edge of $G$ contains exactly one vertex in $X$ and exactly one
vertex in $Y$. In other words, if $e$ is an edge of $G$, then exactly one
endpoint of $e$ lies in $X$, and exactly one endpoint of $e$ lies in $Y$.
Thus, we can make the following definition:

\begin{definition}
For each edge $e$ of $G$, we let $\overrightarrow{e}$ be the pair $\left(
x,y\right)  \in X\times Y$, where $x$ is the unique endpoint of $e$ lying in
$X$, and where $y$ is the unique endpoint of $e$ lying in $Y$. Thus, if
$\overrightarrow{e}=\left(  x,y\right)  $, then $e=\left\{  x,y\right\}  $.
\end{definition}

Now, we shall construct a network $N$ out of our bipartite graph $\left(
G;X,Y\right)  $. Before we give the rigorous definition, let us show it on an
example:\newpage

\begin{example}
For this example, let $G=\left(  V,E\right)  $ be the simple graph with
vertices $1,2,3,4,5,6,7$ and edges $\left\{  1,4\right\}  ,\left\{
1,5\right\}  ,\left\{  2,4\right\}  ,\left\{  2,7\right\}  ,\left\{
3,5\right\}  ,\left\{  3,6\right\}  $. Set $X=\left\{  1,2,3\right\}  $ and
$Y=\left\{  4,5,6,7\right\}  $; then, $\left(  G;X,Y\right)  $ is a bipartite
graph which can be drawn as follows:%
\[%
%TCIMACRO{\TeXButton{a bipartite graph}{\xymatrix@C=5pc{
%1 \are[r] \are[rd] & 4 \\
%2 \are[ru] \are[rdd] & 5 \\
%3 \are[ru] \are[r] & 6 \\
%& 7
%}}}%
%BeginExpansion
\xymatrix@C=5pc{
1 \are[r] \are[rd] & 4 \\
2 \are[ru] \are[rdd] & 5 \\
3 \are[ru] \are[r] & 6 \\
& 7
}%
%EndExpansion
.
\]


Now, we replace each edge $e$ of $G$ by the corresponding pair
$\overrightarrow{e}$; this transforms our simple graph into the following
digraph:%
\[%
%TCIMACRO{\TeXButton{a bipartite graph with directed arcs}{\xymatrix@C=5pc{
%1 \ar[r] \ar[rd] & 4 \\
%2 \ar[ru] \ar[rdd] & 5 \\
%3 \ar[ru] \ar[r] & 6 \\
%& 7
%}}}%
%BeginExpansion
\xymatrix@C=5pc{
1 \ar[r] \ar[rd] & 4 \\
2 \ar[ru] \ar[rdd] & 5 \\
3 \ar[ru] \ar[r] & 6 \\
& 7
}%
%EndExpansion
.
\]
The arcs of this digraph will be called the $G$\textit{-arcs} (to stress that
they come directly from the edges of $G$, as opposed to the next arcs that we
are going to add).

Next, we add a new vertex, which we call $s$ and which we draw on the very
left. This $s$ will be the source of our network. For each $x\in X$, we add an
arc $\left(  s,x\right)  $ to our digraph; these new arcs (a total of
$\left\vert X\right\vert $ arcs, one for each $x\in X$) will be called the
$s$\textit{-arcs}. Here is how our digraph now looks like:%
\[%
%TCIMACRO{\TeXButton{a bipartite graph with directed arcs and s}{\xymatrix@C
%=5pc{
%& 1 \ar[r] \ar[rd] & 4 \\
%s \ar[ru] \ar[r] \ar[rd] & 2 \ar[ru] \ar[rdd] & 5 \\
%& 3 \ar[ru] \ar[r] & 6 \\
%& & 7
%}}}%
%BeginExpansion
\xymatrix@C=5pc{
& 1 \ar[r] \ar[rd] & 4 \\
s \ar[ru] \ar[r] \ar[rd] & 2 \ar[ru] \ar[rdd] & 5 \\
& 3 \ar[ru] \ar[r] & 6 \\
& & 7
}%
%EndExpansion
.
\]


Next, we add a new vertex, which we call $t$ and which we draw on the very
right. This $t$ will be the sink of our network. For each $y\in Y$, we add an
arc $\left(  y,t\right)  $ to our digraph; these new arcs (a total of
$\left\vert Y\right\vert $ arcs, one for each $y\in Y$) will be called the
$t$\textit{-arcs}. Here is how our digraph now looks like:%
\[%
%TCIMACRO{\TeXButton{a bipartite graph with directed arcs and s and
%t}{\xymatrix@C=5pc{
%& 1 \ar[r] \ar[rd] & 4 \ar[rd] \\
%s \ar[ru] \ar[r] \ar[rd] & 2 \ar[ru] \ar[rdd] & 5 \ar[r] & t \\
%& 3 \ar[ru] \ar[r] & 6 \ar[ru] \\
%& & 7 \ar[ruu]
%}}}%
%BeginExpansion
\xymatrix@C=5pc{
& 1 \ar[r] \ar[rd] & 4 \ar[rd] \\
s \ar[ru] \ar[r] \ar[rd] & 2 \ar[ru] \ar[rdd] & 5 \ar[r] & t \\
& 3 \ar[ru] \ar[r] & 6 \ar[ru] \\
& & 7 \ar[ruu]
}%
%EndExpansion
.
\]
This $9$-vertex digraph will be the underlying digraph of our network $N$. The
source and the target of $N$ shall be $s$ and $t$, respectively. The capacity
function $c$ is determined by setting $c\left(  a\right)  =1$ for each arc of
the network.

We claim that the integer flows on the network $N$ are in bijection with the
matchings in $G$. Again, let us show how this bijection acts on an example.
Consider the following integer flow on $N$:%
\begin{equation}%
%TCIMACRO{\TeXButton{an integer flow on N}{\xymatrix@C=5pc{
%& 1 \ar[r] \ar[rd] & 4 \ar@{=>}[rd] \\
%s \ar[ru] \ar@{=>}[r] \ar@{=>}[rd] & 2 \ar@{=>}[ru] \ar[rdd] & 5 \ar[r] & t \\
%& 3 \ar[ru] \ar@{=>}[r] & 6 \ar@{=>}[ru] \\
%& & 7 \ar[ruu]
%}}}%
%BeginExpansion
\xymatrix@C=5pc{
& 1 \ar[r] \ar[rd] & 4 \ar@{=>}[rd] \\
s \ar[ru] \ar@{=>}[r] \ar@{=>}[rd] & 2 \ar@{=>}[ru] \ar[rdd] & 5 \ar[r] & t \\
& 3 \ar[ru] \ar@{=>}[r] & 6 \ar@{=>}[ru] \\
& & 7 \ar[ruu]
}%
%EndExpansion
, \label{eq.exam.hall.bij1}%
\end{equation}
where a simple arrow ($\longrightarrow$) stands for an arc which the flow
sends to $0$, and where a double arrow ($\Longrightarrow$) stands for an arc
which the flow sends to $1$. (This is just an alternative way of drawing a
flow when each arc has capacity $1$.) Consider the arcs that are sent to $1$
by this flow. Two of these arcs are $G$-arcs, namely $\left(  2,4\right)  $
and $\left(  3,6\right)  $. The corresponding edges $\left\{  2,4\right\}  $
and $\left\{  3,6\right\}  $ of $G$ form a matching: the matching $\left\{
\left\{  2,4\right\}  ,\left\{  3,6\right\}  \right\}  $. This matching has
size $2$, which is exactly the value of our integer flow.
\end{example}

Let us now formalize what we did in this example. First, here is the general
definition of the network $N$:

\begin{definition}
\textbf{(a)} Pick two distinct new objects $s$ and $t$. Let $V$ be the set
$W\cup\left\{  s,t\right\}  $. Define three subsets $A_{G}$, $A_{s}$ and
$A_{t}$ of $V\times V$ as follows:%
\[
A_{G}=\left\{  \overrightarrow{e}\ \mid\ e\in E\right\}
;\ \ \ \ \ \ \ \ \ \ A_{s}=\left\{  \left(  s,x\right)  \ \mid\ x\in
X\right\}  ;\ \ \ \ \ \ \ \ \ \ A_{t}=\left\{  \left(  y,t\right)
\ \mid\ y\in Y\right\}  .
\]
Let $A$ be the union $A_{G}\cup A_{s}\cup A_{t}$; this is also a subset of
$V\times V$. Thus, $\left(  V,A\right)  $ is a digraph. The elements of
$A_{G}$ shall be called the $G$\textit{-arcs}; the elements of $A_{s}$ shall
be called the $s$\textit{-arcs}; the elements of $A_{t}$ shall be called the
$t$\textit{-arcs}. Thus, the arcs of the digraph $\left(  V,A\right)  $ are
the $G$-arcs, the $s$-arcs and the $t$-arcs.

\textbf{(b)} Define the function $c:A\rightarrow\mathbb{Q}_{+}$ by $\left(
c\left(  a\right)  =1\text{ for each }a\in A\right)  $. Thus, $c$ is a
constant function.

\textbf{(c)} Let $N$ be the network consisting of the digraph $\left(
V,A\right)  $, the source $s$, the sink $t$ and the capacity function $c$.
\end{definition}

We notice that the sets $A_{G}$, $A_{s}$ and $A_{t}$ are mutually disjoint;
i.e., there is no overlap between the $G$-arcs, the $s$-arcs and the $t$-arcs.
This is essentially obvious (for example, all $s$-arcs have source $s$,
whereas none of the other types of arcs do; likewise, all $t$-arcs have target
$t$, whereas none of the other types of arcs do).

Our specific choice of capacity function $c$ ensures that integer flows on $N$
have a very simple form: If $f$ is an integer flow on $N$, and if $a\in A$ is
any arc, then $f\left(  a\right)  $ is either $0$ or $1$. (Indeed, $f\left(
a\right)  $ must be an integer since $f$ is an integer flow; but the capacity
constraints enforce $0\leq f\left(  a\right)  \leq1$, so that this integer
$f\left(  a\right)  $ must be either $0$ or $1$.) Thus, an integer flow $f$ on
$N$ is uniquely determined by knowing which of the arcs $a\in A$ it sends to
$1$ (because then, it has to send all the other arcs to $0$).

Another consequence of our definition of capacity function $c$ is the following:

\begin{lemma}
\label{lem.hall.cPQ}Let $P$ and $Q$ be two subsets of $V$. Then, $c\left(
P,Q\right)  =\left\vert \left[  P,Q\right]  \right\vert $.
\end{lemma}

\begin{proof}
[Proof of Lemma \ref{lem.hall.cPQ}.]The definition of $c\left(  P,Q\right)  $
yields%
\[
c\left(  P,Q\right)  =\sum_{a\in\left[  P,Q\right]  }\underbrace{c\left(
a\right)  }_{\substack{=1\\\text{(by the definition of }c\text{)}}}=\sum
_{a\in\left[  P,Q\right]  }1=\left\vert \left[  P,Q\right]  \right\vert
\cdot1=\left\vert \left[  P,Q\right]  \right\vert .
\]
This proves Lemma \ref{lem.hall.cPQ}.
\end{proof}

Now, we want to formulate the bijection between integer flows on $N$ and
matchings in $G$. First, we need a simple property of integer flows on our
specific network $N$:\ \ \ \ \footnote{We are using the \textit{Iverson
bracket notation}: If $\mathcal{A}$ is any logical statement, then $\left[
\mathcal{A}\right]  $ denotes the integer $%
\begin{cases}
1, & \text{if }\mathcal{A}\text{ is true};\\
0, & \text{if }\mathcal{A}\text{ is false}%
\end{cases}
$.}

\begin{proposition}
\label{prop.hall.lem1}Let $f$ be any integer flow on $N$. Let $M$ be the
subset $\left\{  e\in E\ \mid\ f\left(  \overrightarrow{e}\right)  =1\right\}
$ of $E$.

\textbf{(a)} We have $f\left(  \left(  s,x\right)  \right)  =\sum
_{\substack{e\in E;\\x\in e}}f\left(  \overrightarrow{e}\right)  $ for each
$x\in X$.

\textbf{(b)} We have $f\left(  \left(  y,t\right)  \right)  =\sum
_{\substack{e\in E;\\y\in e}}f\left(  \overrightarrow{e}\right)  $ for each
$y\in Y$.

\textbf{(c)} We have $M=\left\{  \left\{  x,y\right\}  \ \mid\ \left(
x,y\right)  \in A_{G};\ f\left(  \left(  x,y\right)  \right)  =1\right\}  $.

\textbf{(d)} The set $M=\left\{  e\in E\ \mid\ f\left(  \overrightarrow{e}%
\right)  =1\right\}  $ is a matching in $G$.

\textbf{(e)} We have $\left\vert M\right\vert =\left\vert f\right\vert $.

\textbf{(f)} For any vertex $x\in X$, we have $\left[  x\text{ is matched in
}M\right]  =f\left(  \left(  s,x\right)  \right)  $.

\textbf{(g)} For any vertex $y\in Y$, we have $\left[  y\text{ is matched in
}M\right]  =f\left(  \left(  y,t\right)  \right)  $.
\end{proposition}

For example, if $f$ is the flow shown in (\ref{eq.exam.hall.bij1}), then the
set $M$ in Proposition \ref{prop.hall.lem1} is the matching $\left\{  \left\{
2,4\right\}  ,\left\{  3,6\right\}  \right\}  $ in $G$.

\begin{proof}
[Proof of Proposition \ref{prop.hall.lem1} (sketched).]We shall only prove the
parts of Proposition \ref{prop.hall.lem1} used in the below proof of Theorem
\ref{thm.hall} (namely, parts \textbf{(a)}, \textbf{(b)}, \textbf{(d)} and
\textbf{(e)}); the rest is left to the reader.

We first make the following observations:

\begin{statement}
\textit{Observation 1:} Let $x\in X$ and $e\in E$. Then, the arc
$\overrightarrow{e}$ has source $x$ if and only if $x\in e$.
\end{statement}

[\textit{Proof of Observation 1:} $\Longrightarrow:$ Assume that the arc
$\overrightarrow{e}$ has source $x$. We must prove that $x\in e$.

Recall that $\overrightarrow{e}$ is defined to be the pair $\left(
x_{1},y_{1}\right)  $, where $x_{1}$ is the unique endpoint of $e$ lying in
$X$, and where $y_{1}$ is the unique endpoint of $e$ lying in $Y$. Consider
these $x_{1}$ and $y_{1}$. The source of the arc $\overrightarrow{e}$ is
$x_{1}$ (since $\overrightarrow{e}=\left(  x_{1},y_{1}\right)  $). But we have
assumed that the arc $\overrightarrow{e}$ has source $x$. Combining the
previous two sentences, we conclude that $x_{1}=x$. But $x_{1}$ is an endpoint
of $e$ (since $x_{1}$ is the unique endpoint of $e$ lying in $X$). In other
words, $x_{1}\in e$. Hence, $x=x_{1}\in e$. This proves the $\Longrightarrow$
direction of Observation 1.

$\Longleftarrow:$ Assume that $x\in e$. We must show that the arc
$\overrightarrow{e}$ has source $x$.

Recall that $\overrightarrow{e}$ is defined to be the pair $\left(
x_{1},y_{1}\right)  $, where $x_{1}$ is the unique endpoint of $e$ lying in
$X$, and where $y_{1}$ is the unique endpoint of $e$ lying in $Y$. Consider
these $x_{1}$ and $y_{1}$. But $x$ is an endpoint of $e$ (since $x\in e$) and
lies in $X$ (since $x\in X$). Thus, the unique endpoint of $e$ lying in $X$
must be $x$. In other words, $x_{1}$ must be $x$ (since $x_{1}$ is the unique
endpoint of $e$ lying in $X$). Now, the arc $\overrightarrow{e}$ has source
$x_{1}$ (since $\overrightarrow{e}=\left(  x_{1},y_{1}\right)  $). In other
words, the arc $\overrightarrow{e}$ has source $x$ (since $x_{1}=x$). Thus,
the $\Longleftarrow$ direction of Observation 1 is proven.]

\begin{statement}
\textit{Observation 2:} Let $y\in Y$ and $e\in E$. Then, the arc
$\overrightarrow{e}$ has target $y$ if and only if $y\in e$.
\end{statement}

[\textit{Proof of Observation 2:} Analogous to Observation 1.]

\begin{statement}
\textit{Observation 3:} The arcs $\overrightarrow{e}$ for $e\in E$ are
pairwise distinct.
\end{statement}

[\textit{Proof of Observation 3:} If $e\in E$ is any edge, then $e$ can be
reconstructed uniquely from the pair $\overrightarrow{e}$ (indeed, if
$\overrightarrow{e}=\left(  x,y\right)  $, then $e=\left\{  x,y\right\}  $, as
we have shown before). Thus, if $e_{1}$ and $e_{2}$ are two edges in $E$ such
that $\overrightarrow{e_{1}}=\overrightarrow{e_{2}}$, then $e_{1}=e_{2}$. In
other words, the arcs $\overrightarrow{e}$ for $e\in E$ are pairwise distinct.
This proves Observation 3.]

Now, we observe that $s\notin W$, so that $s\notin X$ and $s\notin Y$. Also,
$X\cap Y=\varnothing$ (since $\left(  G;X,Y\right)  $ is a bipartite graph),
so that $X\subseteq V\setminus Y$.

\begin{statement}
\textit{Observation 4:} Let $x\in X$. Then, the arcs $a\in A$ having source
$x$ are exactly the arcs $\overrightarrow{e}$ for $e\in E$ satisfying $x\in e$.
\end{statement}

[\textit{Proof of Observation 4:} We have $x\in X$ but $s\notin X$ (since
$s\notin W$). Thus, $x\neq s$. Hence, there are no $s$-arcs having source $x$
(since all $s$-arcs have source $s\neq x$). Also, $x\notin Y$ (since $x\in
X\subseteq V\setminus Y$). Thus, there are no $t$-arcs having source $x$
(since all $s$-arcs have source lying in $Y$).

Recall that there are three types of arcs $a\in A$: the $G$-arcs, the $s$-arcs
and the $t$-arcs. Among these three types, only $G$-arcs can have source $x$
(because we have seen that there are no $s$-arcs having source $x$, and that
there are no $t$-arcs having source $x$). Hence,%
\begin{align*}
&  \left\{  \text{the arcs }a\in A\text{ having source }x\right\} \\
&  =\left\{  \text{the }G\text{-arcs having source }x\right\} \\
&  =\left\{  \text{the arcs }\overrightarrow{e}\text{ (with }e\in E\text{)
having source }x\right\} \\
&  \ \ \ \ \ \ \ \ \ \ \left(  \text{since the }G\text{-arcs are precisely the
arcs }\overrightarrow{e}\text{ (with }e\in E\text{)}\right) \\
&  =\left\{  \overrightarrow{e}\ \mid\ e\in E;\ \underbrace{\text{the arc
}\overrightarrow{e}\text{ has source }x}_{\substack{\Longleftrightarrow
\ \left(  x\in e\right)  \\\text{(by Observation 1)}}}\right\} \\
&  =\left\{  \overrightarrow{e}\ \mid\ e\in E;\ x\in e\right\}  =\left\{
\text{the arcs }\overrightarrow{e}\text{ for }e\in E\text{ satisfying }x\in
e\right\}  .
\end{align*}
In other words, the arcs $a\in A$ having source $x$ are exactly the arcs
$\overrightarrow{e}$ for $e\in E$ satisfying $x\in e$. This proves Observation 4.]

\begin{statement}
\textit{Observation 5:} Let $y\in Y$. Then, the arcs $a\in A$ having target
$y$ are exactly the arcs $\overrightarrow{e}$ for $e\in E$ satisfying $y\in e$.
\end{statement}

[\textit{Proof of Observation 5:} Analogous to Observation 4.]

\begin{statement}
\textit{Observation 6:} Let $x\in X$. Then, there is a unique arc $a\in A$
having target $x$, namely the $s$-arc $\left(  s,x\right)  $.
\end{statement}

[\textit{Proof of Observation 6:} We have $x\in X\subseteq W$ but $t\notin W$.
Thus, $x\neq t$. Hence, there are no $t$-arcs having target $x$ (since all
$t$-arcs have target $t\neq x$). Also, $x\notin Y$ (since $x\in X\subseteq
V\setminus Y$). Thus, there are no $G$-arcs having target $x$ (since all
$G$-arcs have target lying in $Y$).

Recall that there are three types of arcs $a\in A$: the $G$-arcs, the $s$-arcs
and the $t$-arcs. Among these three types, only $s$-arcs can have target $x$
(because we have seen that there are no $t$-arcs having target $x$, and that
there are no $G$-arcs having target $x$). Hence,%
\[
\left\{  \text{the arcs }a\in A\text{ having target }x\right\}  =\left\{
\text{the }s\text{-arcs having target }x\right\}  .
\]


But the $s$-arcs are simply the pairs of the form $\left(  s,x^{\prime
}\right)  $ for all $x^{\prime}\in X$ (by the definition of the $s$-arcs).
Thus, there is a unique $s$-arc having target $x$, namely $\left(  s,x\right)
$ (since $x\in X$). In other words, $\left\{  \text{the }s\text{-arcs having
target }x\right\}  =\left\{  \left(  s,x\right)  \right\}  $. Hence,%
\[
\left\{  \text{the arcs }a\in A\text{ having target }x\right\}  =\left\{
\text{the }s\text{-arcs having target }x\right\}  =\left\{  \left(
s,x\right)  \right\}  .
\]
In other words, there is a unique arc $a\in A$ having target $x$, namely the
$s$-arc $\left(  s,x\right)  $. This proves Observation 6.]

\begin{statement}
\textit{Observation 7:} Let $y\in Y$. Then, there is a unique arc $a\in A$
having source $y$, namely the $t$-arc $\left(  y,t\right)  $.
\end{statement}

[\textit{Proof of Observation 7:} Analogous to Observation 6.]

\begin{statement}
\textit{Observation 8:} The arcs $a\in A$ having source $s$ are exactly the
arcs $\left(  s,x\right)  $ for $x\in X$, and these arcs are pairwise distinct.
\end{statement}

[\textit{Proof of Observation 8:} We have $s\notin W$ and thus $s\notin X$
(since $X\subseteq W$). Hence, there are no $G$-arcs having source $s$ (since
all $G$-arcs have source lying in $X$). Also, $s\notin W$ and thus $s\notin Y$
(since $Y\subseteq W$). Thus, there are no $t$-arcs having source $s$ (since
all $t$-arcs have source lying in $Y$).

Recall that there are three types of arcs $a\in A$: the $G$-arcs, the $s$-arcs
and the $t$-arcs. Among these three types, only $s$-arcs can have source $s$
(because we have seen that there are no $G$-arcs having source $s$, and that
there are no $t$-arcs having source $s$). Hence,%
\begin{align*}
&  \left\{  \text{the arcs }a\in A\text{ having source }s\right\} \\
&  =\left\{  \text{the }s\text{-arcs having source }s\right\} \\
&  =\left\{  \text{the }s\text{-arcs}\right\}  \ \ \ \ \ \ \ \ \ \ \left(
\text{since \textbf{all} }s\text{-arcs have source }s\right) \\
&  =A_{s}=\left\{  \left(  s,x\right)  \ \mid\ x\in X\right\}  .
\end{align*}
In other words, the arcs $a\in A$ having source $s$ are exactly the arcs
$\left(  s,x\right)  $ for $x\in X$. Furthermore, these arcs are pairwise
distinct (since the targets $x$ of these arcs are pairwise distinct). This
proves Observation 8.]

\begin{statement}
\textit{Observation 9:} There are no arcs $a\in A$ having target $s$.
\end{statement}

[\textit{Proof of Observation 9:} We have $s\notin W$ and thus $s\notin X$
(since $X\subseteq W$). Hence, there are no $s$-arcs having target $s$ (since
all $s$-arcs have target lying in $X$). Also, $s\notin W$ and thus $s\notin Y$
(since $Y\subseteq W$). Hence, there are no $G$-arcs having target $s$ (since
all $G$-arcs have target lying in $Y$). Finally, $s\neq t$. Thus, there are no
$t$-arcs having target $s$ (since all $t$-arcs have target $t\neq s$).

Recall that there are three types of arcs $a\in A$: the $G$-arcs, the $s$-arcs
and the $t$-arcs. Among these three types, none can have target $s$ (since we
have seen that there are no $G$-arcs having target $s$, no $s$-arcs having
target $s$, and no $t$-arcs having target $s$). Thus, there are no arcs $a\in
A$ having target $s$. This proves Observation 9.]

Recall that $f$ is a flow; thus, $f$ satisfies the capacity constraints and
the conservation constraints.

\textbf{(a)} Let $x\in X$. Then, $x\in X\subseteq W=V\setminus\left\{
s,t\right\}  $. Hence, $f^{-}\left(  x\right)  =f^{+}\left(  x\right)  $
(since $f$ satisfies the conservation constraints).

Observation 6 shows that there is a unique arc $a\in A$ having target $x$,
namely the $s$-arc $\left(  s,x\right)  $. Thus, $\sum_{\substack{a\in A\text{
is an arc}\\\text{with target }x}}f\left(  a\right)  =f\left(  \left(
s,x\right)  \right)  $.

The definition of $f^{-}\left(  x\right)  $ yields
\begin{equation}
f^{-}\left(  x\right)  =\sum_{\substack{a\in A\text{ is an arc}\\\text{with
target }x}}f\left(  a\right)  =f\left(  \left(  s,x\right)  \right)  .
\label{pf.prop.hall.lem1.a.f-}%
\end{equation}


Observation 4 shows that the arcs $a\in A$ having source $x$ are exactly the
arcs $\overrightarrow{e}$ for $e\in E$ satisfying $x\in e$. Since these arcs
$\overrightarrow{e}$ are pairwise distinct (by Observation 3), we thus
conclude that $\sum_{\substack{a\in A\text{ is an arc}\\\text{with source }%
x}}f\left(  a\right)  =\sum_{\substack{e\in E;\\x\in e}}f\left(
\overrightarrow{e}\right)  $.

The definition of $f^{+}\left(  x\right)  $ yields%
\begin{equation}
f^{+}\left(  x\right)  =\sum_{\substack{a\in A\text{ is an arc}\\\text{with
source }x}}f\left(  a\right)  =\sum_{\substack{e\in E;\\x\in e}}f\left(
\overrightarrow{e}\right)  . \label{pf.prop.hall.lem1.a.f+}%
\end{equation}


Now, recall that $f^{-}\left(  x\right)  =f^{+}\left(  x\right)  $. In view of
(\ref{pf.prop.hall.lem1.a.f-}) and (\ref{pf.prop.hall.lem1.a.f+}), this
rewrites as $f\left(  \left(  s,x\right)  \right)  =\sum_{\substack{e\in
E;\\x\in e}}f\left(  \overrightarrow{e}\right)  $. This proves Proposition
\ref{prop.hall.lem1} \textbf{(a)}.

\textbf{(b)} The proof of Proposition \ref{prop.hall.lem1} \textbf{(b)} is
analogous to that of Proposition \ref{prop.hall.lem1} \textbf{(a)}.

\textbf{(d)} Clearly, $M=\left\{  e\in E\ \mid\ f\left(  \overrightarrow{e}%
\right)  =1\right\}  $ is a subset of $E$. We must prove that this subset $M$
is a matching in $G$. Since we already know that $M$ is a subset of $E$, we
only need to verify that no two distinct edges in $M$ have a vertex in common
(by the definition of a matching).

Let us assume the contrary. Thus, there exist two distinct edges $g$ and $h$
in $M$ that have a vertex in common. Consider such $g$ and $h$.

The edges $g$ and $h$ have a vertex in common. In other words, there exists a
$w\in W$ such that $w\in g$ and $w\in h$. Consider this $w$.

We have $g\in M\subseteq E$ and $w\in g$. Thus, $g$ is an edge $e\in E$
satisfying $w\in e$. Similarly, $h$ is an edge $e\in E$ satisfying $w\in e$.

From $g\in M=\left\{  e\in E\ \mid\ f\left(  \overrightarrow{e}\right)
=1\right\}  $, we conclude that $f\left(  \overrightarrow{g}\right)  =1$.
Similarly, $f\left(  \overrightarrow{h}\right)  =1$.

We have $w\in W=X\cup Y$ (since $\left(  G;X,Y\right)  $ is a bipartite
graph); thus, we have either $w\in X$ or $w\in Y$. So we are in one of the
following two cases:

\begin{statement}
\textit{Case 1:} We have $w\in X$.
\end{statement}

\begin{statement}
\textit{Case 2:} We have $w\in Y$.
\end{statement}

Let us consider Case 1. In this case, we have $w\in X$. Proposition
\ref{prop.hall.lem1} \textbf{(a)} (applied to $w=x$) thus yields%
\begin{align*}
f\left(  \left(  s,w\right)  \right)   &  =\sum_{\substack{e\in E;\\w\in
e}}f\left(  \overrightarrow{e}\right)  =\underbrace{f\left(
\overrightarrow{g}\right)  }_{=1}+\underbrace{f\left(  \overrightarrow{h}%
\right)  }_{=1}+\sum_{\substack{e\in E;\\w\in e;\\e\notin\left\{  g,h\right\}
}}\underbrace{f\left(  \overrightarrow{e}\right)  }_{\substack{\geq
0\\\text{(since }f\text{ is a map }A\rightarrow\mathbb{Q}_{+}\text{)}}}\\
&  \ \ \ \ \ \ \ \ \ \ \left(
\begin{array}
[c]{c}%
\text{here, we have split off the addends for }e=g\text{ and }e=h\text{,}\\
\text{since }g\text{ and }h\text{ are two distinct edges }e\in E\text{
satisfying }w\in e
\end{array}
\right) \\
&  \geq1+1+\sum_{\substack{e\in E;\\w\in e;\\e\notin\left\{  g,h\right\}
}}0=2.
\end{align*}


But recall that $f$ satisfies the capacity constraints. Thus, $0\leq f\left(
a\right)  \leq c\left(  a\right)  $ for each arc $a\in A$. Applying this to
$a=\left(  s,w\right)  $, we conclude that $0\leq f\left(  \left(  s,w\right)
\right)  \leq c\left(  \left(  s,w\right)  \right)  $. Hence, $f\left(
\left(  s,w\right)  \right)  \leq c\left(  \left(  s,w\right)  \right)  =1$
(by the definition of $c$). This contradicts $f\left(  \left(  s,w\right)
\right)  \geq2>1$. Thus, we have found a contradiction in Case 1.

Similarly, we obtain a contradiction in Case 2 (using Proposition
\ref{prop.hall.lem1} \textbf{(b)} instead of Proposition \ref{prop.hall.lem1}
\textbf{(a)}).

Hence, we have found a contradiction in both Cases 1 and 2. Thus, we always
get a contradiction. This completes our proof that $M$ is a matching in $G$.
Thus, Proposition \ref{prop.hall.lem1} \textbf{(d)} is established.

\textbf{(e)} For any edge $e\in E$, we have either $f\left(
\overrightarrow{e}\right)  =0$ or $f\left(  \overrightarrow{e}\right)
=1$\ \ \ \ \footnote{\textit{Proof.} Let $e\in E$. Then, $\overrightarrow{e}$
is an arc in $A$ (namely, a $G$-arc). But recall that $f$ satisfies the
capacity constraints. Thus, $0\leq f\left(  a\right)  \leq c\left(  a\right)
$ for each arc $a\in A$. Applying this to $a=\overrightarrow{e}$, we conclude
that $0\leq f\left(  \overrightarrow{e}\right)  \leq c\left(
\overrightarrow{e}\right)  $. Hence, $f\left(  \overrightarrow{e}\right)  \leq
c\left(  \overrightarrow{e}\right)  =1$ (by the definition of $c$). Thus,
$0\leq f\left(  \overrightarrow{e}\right)  \leq1$.
\par
But $f$ is an integer flow; thus, $f\left(  \overrightarrow{e}\right)  $ is an
integer. Combining this with $0\leq f\left(  \overrightarrow{e}\right)  \leq
1$, we conclude that $f\left(  \overrightarrow{e}\right)  \in\left\{
0,1\right\}  $. In other words, either $f\left(  \overrightarrow{e}\right)
=0$ or $f\left(  \overrightarrow{e}\right)  =1$. Qed.}.

Observation 8 shows that the arcs $a\in A$ having source $s$ are exactly the
arcs $\left(  s,x\right)  $ for $x\in X$, and these arcs are pairwise
distinct. Hence,%
\begin{align}
\sum_{\substack{a\in A\text{ is an arc}\\\text{with source }s}}f\left(
a\right)   &  =\sum_{x\in X}\underbrace{f\left(  \left(  s,x\right)  \right)
}_{\substack{=\sum_{\substack{e\in E;\\x\in e}}f\left(  \overrightarrow{e}%
\right)  \\\text{(by Proposition \ref{prop.hall.lem1} \textbf{(a)})}%
}}=\underbrace{\sum_{x\in X}\sum_{\substack{e\in E;\\x\in e}}}_{=\sum_{e\in
E}\sum_{\substack{x\in X;\\x\in e}}}f\left(  \overrightarrow{e}\right)
\nonumber\\
&  =\sum_{e\in E}\sum_{\substack{x\in X;\\x\in e}}f\left(  \overrightarrow{e}%
\right)  . \label{pf.prop.hall.lem1.e.1}%
\end{align}
But each $e\in E$ satisfies
\begin{equation}
\sum_{\substack{x\in X;\\x\in e}}f\left(  \overrightarrow{e}\right)  =f\left(
\overrightarrow{e}\right)  \label{pf.prop.hall.lem1.e.2}%
\end{equation}
\footnote{\textit{Proof of (\ref{pf.prop.hall.lem1.e.2}):} Let $e\in E$. Each
edge of $G$ contains exactly one vertex in $X$ and exactly one vertex in $Y$
(since $\left(  G;X,Y\right)  $ is a bipartite graph). Thus, in particular,
each edge of $G$ contains exactly one vertex in $X$. Applying this to the edge
$e$, we conclude that $e$ contains exactly one vertex in $X$. In other words,
there is exactly one $x\in X$ satisfying $x\in e$. Thus, the sum
$\sum_{\substack{x\in X;\\x\in e}}f\left(  \overrightarrow{e}\right)  $ has
exactly one addend. Hence, this sum simplifies as follows: $\sum
_{\substack{x\in X;\\x\in e}}f\left(  \overrightarrow{e}\right)  =f\left(
\overrightarrow{e}\right)  $. This proves (\ref{pf.prop.hall.lem1.e.2}).}.
Hence, (\ref{pf.prop.hall.lem1.e.1}) becomes%
\begin{align}
\sum_{\substack{a\in A\text{ is an arc}\\\text{with source }s}}f\left(
a\right)   &  =\sum_{e\in E}\underbrace{\sum_{\substack{x\in X;\\x\in
e}}f\left(  \overrightarrow{e}\right)  }_{\substack{=f\left(
\overrightarrow{e}\right)  \\\text{(by (\ref{pf.prop.hall.lem1.e.2}))}}%
}=\sum_{e\in E}f\left(  \overrightarrow{e}\right) \nonumber\\
&  =\sum_{\substack{e\in E;\\f\left(  \overrightarrow{e}\right)
=0}}\underbrace{f\left(  \overrightarrow{e}\right)  }_{=0}+\sum
_{\substack{e\in E;\\f\left(  \overrightarrow{e}\right)  =1}%
}\underbrace{f\left(  \overrightarrow{e}\right)  }_{=1}\nonumber\\
&  \ \ \ \ \ \ \ \ \ \ \left(
\begin{array}
[c]{c}%
\text{because for any edge }e\in E\text{, we have}\\
\text{either }f\left(  \overrightarrow{e}\right)  =0\text{ or }f\left(
\overrightarrow{e}\right)  =1\text{ (but not both)}%
\end{array}
\right) \nonumber\\
&  =\underbrace{\sum_{\substack{e\in E;\\f\left(  \overrightarrow{e}\right)
=0}}0}_{=0}+\sum_{\substack{e\in E;\\f\left(  \overrightarrow{e}\right)
=1}}1=\sum_{\substack{e\in E;\\f\left(  \overrightarrow{e}\right)
=1}}1\nonumber\\
&  =\left\vert \underbrace{\left\{  e\in E\ \mid\ f\left(  \overrightarrow{e}%
\right)  =1\right\}  }_{=M}\right\vert \cdot1=\left\vert M\right\vert
\cdot1=\left\vert M\right\vert . \label{pf.prop.hall.lem1.e.4}%
\end{align}


On the other hand, Observation 9 shows that there are no arcs $a\in A$ having
target $s$. Hence,
\begin{equation}
\sum_{\substack{a\in A\text{ is an arc}\\\text{with target }s}}f\left(
a\right)  =\left(  \text{empty sum}\right)  =0. \label{pf.prop.hall.lem1.e.5}%
\end{equation}


But the definition of $\left\vert f\right\vert $ yields%
\begin{align*}
\left\vert f\right\vert  &  =\underbrace{f^{+}\left(  s\right)  }%
_{\substack{=\sum_{\substack{a\in A\text{ is an arc}\\\text{with source }%
s}}f\left(  a\right)  \\\text{(by the definition of }f^{+}\left(  s\right)
\text{)}}}-\underbrace{f^{-}\left(  s\right)  }_{\substack{=\sum
_{\substack{a\in A\text{ is an arc}\\\text{with target }s}}f\left(  a\right)
\\\text{(by the definition of }f^{-}\left(  s\right)  \text{)}}}\\
&  =\underbrace{\sum_{\substack{a\in A\text{ is an arc}\\\text{with source }%
s}}f\left(  a\right)  }_{\substack{=\left\vert M\right\vert \\\text{(by
(\ref{pf.prop.hall.lem1.e.4}))}}}-\underbrace{\sum_{\substack{a\in A\text{ is
an arc}\\\text{with target }s}}f\left(  a\right)  }_{\substack{=0\\\text{(by
(\ref{pf.prop.hall.lem1.e.5}))}}}\\
&  =\left\vert M\right\vert -0=\left\vert M\right\vert .
\end{align*}
In other words, $\left\vert M\right\vert =\left\vert f\right\vert $. This
proves Proposition \ref{prop.hall.lem1} \textbf{(e)}.
\end{proof}

\begin{noncompile}
\begin{proof}
[Proof of Proposition \ref{prop.hall.lem1}.](THIS\ IS\ OUTDATED!) Note that
$f$ is a flow; thus, $f$ satisfies the capacity constraints and the
conservation constraints.

Every $\left(  x,y\right)  \in A_{G}$ satisfies $\left\{  x,y\right\}  \in
E$\ \ \ \ \footnote{\textit{Proof.} Let $\left(  x,y\right)  \in A_{G}$. Then,
$\left(  x,y\right)  \in A_{G}=\left\{  \overrightarrow{e}\ \mid\ e\in
E\right\}  $. In other words, $\left(  x,y\right)  =\overrightarrow{e}$ for
some $e\in E$. Consider this $e$.
\par
We know that $e$ is an edge of $G$ (since $e\in E$). Thus, exactly one
endpoint of $e$ lies in $X$, and exactly one endpoint of $e$ lies in $Y$. Let
$u$ be the endpoint of $e$ lying in $X$, and let $v$ be the endpoint of $e$
lying in $Y$. The definition of $\overrightarrow{e}$ thus shows that
$\overrightarrow{e}=\left(  u,v\right)  $. Hence, $\left(  x,y\right)
=\overrightarrow{e}=\left(  u,v\right)  $. Therefore, $\left\{  x,y\right\}
=\left\{  u,v\right\}  =e$ (since $u$ and $v$ are the endpoints of $e$). Thus,
$\left\{  x,y\right\}  =e\in E$, qed.}. Thus, the set \newline$\left\{
\left\{  x,y\right\}  \ \mid\ \left(  x,y\right)  \in A_{G};\ f\left(  \left(
x,y\right)  \right)  =1\right\}  $ is a subset of $E$. In other words, the set
$M$ is a subset of $E$ (since $M=\left\{  \left\{  x,y\right\}  \ \mid
\ \left(  x,y\right)  \in A_{G};\ f\left(  \left(  x,y\right)  \right)
=1\right\}  $).

In order to prove that this set $M$ is a matching in $G$, we thus only need to
prove that no two distinct edges in $M$ have a vertex in common.

Let us assume the contrary. Thus, there exist two distinct edges $g$ and $h$
in $M$ that have a vertex in common. Consider such $g$ and $h$.

The edges $g$ and $h$ have a vertex in common. In other words, there exists a
$w\in W$ such that $w\in g$ and $w\in h$. Consider this $w$. We have $w\in
W=X\cup Y$ (since $\left(  G;X,Y\right)  $ is a bipartite graph); thus, we
have either $w\in X$ or $w\in Y$. So we are in one of the following two cases:

\begin{statement}
\textit{Case 1:} We have $w\in X$.
\end{statement}

\begin{statement}
\textit{Case 2:} We have $w\in Y$.
\end{statement}

Let us consider Case 1 first. In this case, we have $w\in X$. Thus, there is
only one arc $a\in A$ having target $w$: namely, the $s$-arc $\left(
s,w\right)  $ (because of how the set $A$ is defined). Hence, $f^{-}\left(
w\right)  =f\left(  \left(  s,w\right)  \right)  \leq c\left(  \left(
s,w\right)  \right)  $ (since $f$ satisfies the capacity constraints). But
also, $w\in W=V\setminus\left\{  s,t\right\}  $ (since $V=W\cup\left\{
s,t\right\}  $ and since $s,t\notin W$); therefore, $f^{-}\left(  w\right)
=f^{+}\left(  w\right)  $ (since $f$ satisfies the conservation constraints).
Hence, $f^{+}\left(  w\right)  =f^{-}\left(  w\right)  \leq c\left(  \left(
s,w\right)  \right)  =1$ (by the definition of $c$).

We have $X\cap Y=\varnothing$ (since $\left(  G;X,Y\right)  $ is a bipartite
graph). Thus, from $w\in X$, we obtain $w\notin Y$.

But $g\in M=\left\{  \left\{  x,y\right\}  \ \mid\ \left(  x,y\right)  \in
A_{G};\ f\left(  \left(  x,y\right)  \right)  =1\right\}  $. In other words,
$g=\left\{  x,y\right\}  $ for some $\left(  x,y\right)  \in A_{G}$ satisfying
$f\left(  \left(  x,y\right)  \right)  =1$. Consider this $\left(  x,y\right)
$. We have $\left(  x,y\right)  \in A_{G}=\left\{  \overrightarrow{e}%
\ \mid\ e\in E\right\}  $, so that $\left(  x,y\right)  =\overrightarrow{e}$
for some $e\in E$. Consider this $e$. Clearly, from $\overrightarrow{e}%
=\left(  x,y\right)  $, we obtain $e=\left\{  x,y\right\}  =g$. Hence,
$g=\left\{  x,y\right\}  $. Moreover, $x=w$\ \ \ \ 
\end{proof}

\textit{Proof.} We have $w\in g=\left\{  x,y\right\}  $. Hence, either $w=x$
and $w=y$. Since $w=y$ is impossible (because $w\notin Y$ but $y\in Y$).
\end{noncompile}

We omit the proof of the rest of Proposition \ref{prop.hall.lem1}; it is an
easy exercise on bookkeeping and understanding the definitions of flows and matchings.

TODO: Prove the rest of Proposition \ref{prop.hall.lem1}.

Proposition \ref{prop.hall.lem1} allows us to make the following definition:

\begin{definition}
We define a map%
\begin{align*}
\Phi:\left\{  \text{integer flows on }N\right\}   &  \rightarrow\left\{
\text{matchings in }G\right\}  ,\\
f  &  \mapsto\left\{  e\in E\ \mid\ f\left(  \overrightarrow{e}\right)
=1\right\}  .
\end{align*}
This map is well-defined, because Proposition \ref{prop.hall.lem1}
\textbf{(d)} shows that if $f$ is an integer flow on $N$, then $\left\{  e\in
E\ \mid\ f\left(  \overrightarrow{e}\right)  =1\right\}  $ is a matching in
$G$.
\end{definition}

We aim to show that this map $\Phi$ is a bijection. In order to do so, we will
construct its inverse, which of course will be a map transforming each
matching in $G$ into an integer flow on $N$. This requires the following
lemma:\footnote{We are again using the Iverson bracket notation.}

\begin{lemma}
\label{lem.hall.lem2}Let $M$ be any matching in $G$. Define a map
$f:A\rightarrow\mathbb{Q}_{+}$ by setting%
\begin{align*}
f\left(  a\right)   &  =%
\begin{cases}
\left[  \text{there is an }e\in M\text{ such that }a=\overrightarrow{e}%
\right]  , & \text{if }a\in A_{G};\\
\left[  x\text{ is matched in }M\right]  , & \text{if }a=\left(  s,x\right)
\text{ for some }x\in X;\\
\left[  y\text{ is matched in }M\right]  , & \text{if }a=\left(  y,t\right)
\text{ for some }y\in Y
\end{cases}
\\
&  \ \ \ \ \ \ \ \ \ \ \text{for each }a\in A.
\end{align*}
Then, $f$ is an integer flow on $N$.
\end{lemma}

Again, the proof of Lemma \ref{lem.hall.lem2} is easy (just check that the
capacity and conservation constraints are satisfied).

TODO: Prove Lemma \ref{lem.hall.lem2}.

Lemma \ref{lem.hall.lem2} allows us to define the following:

\begin{definition}
We define a map
\[
\Psi:\left\{  \text{matchings in }G\right\}  \rightarrow\left\{  \text{integer
flows on }N\right\}
\]
by the requirement that $\Psi$ map any matching $M$ in $G$ to the integer flow
$f$ constructed in Lemma \ref{lem.hall.lem2}.
\end{definition}

\begin{proposition}
\label{prop.hall.inverse}\textbf{(a)} The maps $\Phi$ and $\Psi$ are mutually
inverse bijections between $\left\{  \text{integer flows on }N\right\}  $ and
$\left\{  \text{matchings in }G\right\}  $.

\textbf{(b)} For any integer flow $f$ on $N$, we have $\left\vert \Phi\left(
f\right)  \right\vert =\left\vert f\right\vert $.
\end{proposition}

We leave the proof of Proposition \ref{prop.hall.inverse} to the reader again.
(It is fairly simple: Part \textbf{(b)} follows from Proposition
\ref{prop.hall.lem1} \textbf{(e)}. Part \textbf{(a)} requires proving that
$\Phi\circ\Psi=\operatorname*{id}$ and $\Psi\circ\Phi=\operatorname*{id}$. The
proof of $\Phi\circ\Psi=\operatorname*{id}$ is obvious; the proof of
$\Psi\circ\Phi=\operatorname*{id}$ relies on the conservation constraints.)

We now see how to find a matching in $G$ of maximum size. Indeed, we can find
an integer flow on $N$ of maximum value (see Remark \ref{rmk.3balg} for how
this is done). Then, the bijection $\Phi$ transforms this integer flow into a
matching in $G$ of maximum size (because Proposition \ref{prop.hall.inverse}
\textbf{(b)} shows that the value of an integer flow equals the size of the
matching corresponding to this flow under the bijection $\Phi$). Hence, we
obtain a matching in $G$ of maximum size.

We are also close to proving Theorem \ref{thm.hall} now. Before we do this,
let us prove a really simple lemma about matchings:

\begin{lemma}
\label{lem.hall.complete-mat}Let $\left(  G;X,Y\right)  $ be a bipartite
graph. Let $M$ be a matching in $G$.

\textbf{(a)} We have $\left\vert M\right\vert \leq\left\vert X\right\vert $.

\textbf{(b)} If $\left\vert M\right\vert \geq\left\vert X\right\vert $, then
the matching $M$ is $X$-complete.
\end{lemma}

\begin{proof}
[Proof of Lemma \ref{lem.hall.complete-mat} (sketched).]Recall that $\left(
G;X,Y\right)  $ is a bipartite graph. Hence, each edge of $G$ contains exactly
one vertex in $X$ and exactly one vertex in $Y$. Thus, we can define a map
$\mathbf{x}:M\rightarrow X$ that sends each edge $e\in M$ to the unique vertex
in $X$ contained in $e$. Consider this map $\mathbf{x}$.

The map $\mathbf{x}$ is injective\footnote{\textit{Proof.} Let $m$ and $n$ be
two distinct elements of $M$. We want to prove that $\mathbf{x}\left(
m\right)  \neq\mathbf{x}\left(  n\right)  $.
\par
Assume the contrary. Thus, $\mathbf{x}\left(  m\right)  =\mathbf{x}\left(
n\right)  $. Define an $x\in X$ by $x=\mathbf{x}\left(  m\right)
=\mathbf{x}\left(  n\right)  $. Now, $\mathbf{x}\left(  m\right)  $ is the
unique vertex in $X$ contained in $m$ (by the definition of $\mathbf{x}$). In
other words, $x$ is the unique vertex in $X$ contained in $m$ (since
$x=\mathbf{x}\left(  m\right)  $). Thus, $x$ is contained in $m$. In other
words, $x\in m$. Similarly, $x\in n$. The edges $m$ and $n$ are two distinct
edges in $M$, and have the vertex $x$ in common (since $x\in m$ and $x\in n$).
\par
But $M$ is a matching. Hence, no two distinct edges in $M$ have a vertex in
common (by the definition of a matching). This contradicts that the edges $m$
and $n$ are two distinct edges in $M$ that have a vertex in common (namely,
the vertex $x$). This contradiction shows that our assumption was wrong;
hence, $\mathbf{x}\left(  m\right)  \neq\mathbf{x}\left(  n\right)  $ is
proven.
\par
Now, forget that we fixed $m$ and $n$. We thus have shown that if $m$ and $n$
are two distinct elements of $M$, then $\mathbf{x}\left(  m\right)
\neq\mathbf{x}\left(  n\right)  $. In other words, the map $\mathbf{x}$ is
injective.}. Hence, $\left\vert \mathbf{x}\left(  M\right)  \right\vert
=\left\vert M\right\vert $. Thus, $\left\vert M\right\vert =\left\vert
\mathbf{x}\left(  M\right)  \right\vert \leq\left\vert X\right\vert $ (since
$\mathbf{x}\left(  M\right)  \subseteq X$). This proves Lemma
\ref{lem.hall.complete-mat} \textbf{(a)}.

\textbf{(b)} Assume that $\left\vert M\right\vert \geq\left\vert X\right\vert
$. Combining this with $\left\vert M\right\vert \leq\left\vert X\right\vert $,
we obtain $\left\vert M\right\vert =\left\vert X\right\vert $.

Now, $\mathbf{x}\left(  M\right)  $ is a subset of $X$ that has the same size
as $X$ (because $\left\vert \mathbf{x}\left(  M\right)  \right\vert
=\left\vert M\right\vert =\left\vert X\right\vert $). But the only such subset
is $X$ itself (since $X$ is a finite set). Thus, $\mathbf{x}\left(  M\right)
$ must be $X$. In other words, the map $\mathbf{x}$ is surjective. This shows
that the matching $M$ is $X$-complete\footnote{\textit{Proof.} Let $v\in X$ be
any vertex. Then, there exists some $m\in M$ such that $v=\mathbf{x}\left(
m\right)  $ (since the map $\mathbf{x}$ is surjective). Consider this $m$. The
vertex $\mathbf{x}\left(  m\right)  $ is the unique vertex in $X$ contained in
$m$ (by the definition of $\mathbf{x}$). Hence, $\mathbf{x}\left(  m\right)  $
is contained in $m$. In other words, $\mathbf{x}\left(  m\right)  \in m$.
Hence, $v=\mathbf{x}\left(  m\right)  \in m$. In other words, $v$ is an
endpoint of $m$. Hence, there exists an edge $e\in M$ such that $v$ is an
endpoint of $e$ (namely, $e=m$). In other words, the vertex $v$ is matched in
$M$.
\par
Now, forget that we fixed $v$. We thus have shown that each vertex $v\in X$ is
matched in $M$. In other words, the matching $M$ is $X$-complete (by the
definition of \textquotedblleft$X$-complete\textquotedblright).}. This proves
Lemma \ref{lem.hall.complete-mat} \textbf{(b)}.
\end{proof}

Now, we can prove the following crucial lemma:

\begin{lemma}
\label{lem.hall.MU}Let $\left(  G;X,Y\right)  $ be a bipartite graph. Then,
there exist a matching $M$ in $G$ and a subset $U$ of $X$ satisfying
$\left\vert M\right\vert \geq\left\vert N\left(  U\right)  \right\vert
+\left\vert X\right\vert -\left\vert U\right\vert $.
\end{lemma}

\begin{proof}
[Proof of Lemma \ref{lem.hall.MU}.]Consider the network $N$. It clearly has
the property that $c\left(  a\right)  \in\mathbb{N}$ for each $a\in A$ (since
$c\left(  a\right)  =1$ for each $a\in A$). Hence, Lemma \ref{lem.3bat} shows
that there exist an integer flow $f:A\rightarrow\mathbb{Q}_{+}$ and a subset
$S$ of $V$ satisfying $s\in S$ and $t\notin S$ and $c\left(  S,\overline
{S}\right)  =\left\vert f\right\vert $. Consider these $f$ and $S$.

Let $U$ be the subset $X\cap S$ of $X$. Then, $N\left(  U\right)  \subseteq Y$
(by Proposition \ref{prop.bipartite.NXY}), so that $N\left(  U\right)  \cap
S\subseteq Y\cap S$ and thus $\left\vert N\left(  U\right)  \cap S\right\vert
\leq\left\vert Y\cap S\right\vert $. Also, $U=X\cap S\subseteq S$.

However, the set $N\left(  U\right)  $ is clearly the union of its two
disjoint subsets $N\left(  U\right)  \cap S$ and $N\left(  U\right)  \setminus
S$ (indeed, the former contains all the elements of $N\left(  U\right)  $ that
belong to $S$, whereas the latter contains all those that don't). thus,%
\[
\left\vert N\left(  U\right)  \right\vert =\underbrace{\left\vert N\left(
U\right)  \cap S\right\vert }_{\leq\left\vert Y\cap S\right\vert }+\left\vert
N\left(  U\right)  \setminus S\right\vert \leq\left\vert Y\cap S\right\vert
+\left\vert N\left(  U\right)  \setminus S\right\vert .
\]
Hence,%
\begin{equation}
\left\vert Y\cap S\right\vert +\left\vert N\left(  U\right)  \setminus
S\right\vert \geq\left\vert N\left(  U\right)  \right\vert .
\label{pf.thm.hall.hard.3}%
\end{equation}


But the set $X$ is the union of its two disjoint subsets $X\cap S$ and
$X\setminus S$. Hence,
\[
\left\vert X\right\vert =\left\vert \underbrace{X\cap S}_{=U}\right\vert
+\left\vert X\setminus S\right\vert =\left\vert U\right\vert +\left\vert
X\setminus S\right\vert .
\]
Thus,%
\begin{equation}
\left\vert X\setminus S\right\vert =\left\vert X\right\vert -\left\vert
U\right\vert . \label{pf.thm.hall.hard.4}%
\end{equation}


Lemma \ref{lem.hall.cPQ} (applied to $P=S$ and $Q=\overline{S}$) yields
$c\left(  S,\overline{S}\right)  =\left\vert \left[  S,\overline{S}\right]
\right\vert $. Thus,
\begin{equation}
\left\vert \left[  S,\overline{S}\right]  \right\vert =c\left(  S,\overline
{S}\right)  =\left\vert f\right\vert . \label{pf.thm.hall.hard.5}%
\end{equation}


Let us now analyze the set $\left[  S,\overline{S}\right]  $. This set
consists of all arcs $a\in A$ whose source lies in $S$ and whose target lies
in $\overline{S}$. Recall the the digraph $\left(  V,A\right)  $ has three
kinds of arcs: the $G$-arcs, the $s$-arcs and the $t$-arcs. Some of these arcs
belong to $\left[  S,\overline{S}\right]  $:

\begin{itemize}
\item For every vertex $x\in X\setminus S$, the $s$-arc $\left(  s,x\right)  $
belongs to $\left[  S,\overline{S}\right]  $\ \ \ \ \footnote{\textit{Proof.}
Let $x\in X\setminus S$. Then, $x\in X\setminus S\subseteq X$, so that
$\left(  s,x\right)  \in A_{s}$ (by the definition of $A_{s}$). Thus, $\left(
s,x\right)  $ is an $s$-arc. The source of this arc lies in $S$ (since $s\in
S$), but the target of this arc lies in $\overline{S}$ (since $x\in
\underbrace{X}_{\subseteq V}\setminus S\subseteq V\setminus S=\overline{S}$).
In other words, this arc $\left(  s,x\right)  $ belongs to $\left[
S,\overline{S}\right]  $. Qed.}. Of course, these $s$-arcs $\left(
s,x\right)  $ are distinct (since their targets $x$ are distinct). Thus, we
have found a total of $\left\vert X\setminus S\right\vert $ different $s$-arcs
belonging to $\left[  S,\overline{S}\right]  $ (one for each $x\in X\setminus
S$).

\item For every vertex $y\in N\left(  U\right)  \setminus S$, there is at
least one $G$-arc with target $y$ that belongs to $\left[  S,\overline
{S}\right]  $\ \ \ \ \footnote{\textit{Proof.} Let $y\in N\left(  U\right)
\setminus S$. We must prove that there is at least one $G$-arc with target $y$
that belongs to $\left[  S,\overline{S}\right]  $.
\par
We know that $y\in N\left(  U\right)  \setminus S\subseteq N\left(  U\right)
$. In other words, $y$ has a neighbor in $U$ (by the definition of $N\left(
U\right)  $). In other words, there exists some $x\in U$ such that $\left\{
x,y\right\}  $ is an edge of $G$. Consider this $x$. Also, $y\in N\left(
U\right)  \subseteq Y$.
\par
We know that $\left\{  x,y\right\}  $ is an edge of $G$. Let us denote this
edge by $e$. Then, the unique endpoint of $e$ lying in $X$ is $x$ (since $x\in
U\subseteq X$), whereas the unique endpoint of $e$ lying in $Y$ is $y$ (since
$y\in Y$). Hence, the definition of $\overrightarrow{e}$ yields
$\overrightarrow{e}=\left(  x,y\right)  $.
\par
Now, $e\in E$ (since $e$ is an edge of $G$), and thus $\overrightarrow{e}\in
A_{G}$ (by the definition of $A_{G}$). Hence, $\left(  x,y\right)
=\overrightarrow{e}\in A_{G}$. In other words, $\left(  x,y\right)  $ is a
$G$-arc. This $G$-arc $\left(  x,y\right)  $ has target $y$ and belongs to
$\left[  S,\overline{S}\right]  $ (since its source lies in $S$ (since $x\in
U\subseteq S$) and its target lies in $\overline{S}$ (since $y\in
\underbrace{N\left(  U\right)  }_{\subseteq V}\setminus S\subseteq V\setminus
S=\overline{S}$)). Thus, there is at least one $G$-arc with target $y$ that
belongs to $\left[  S,\overline{S}\right]  $ (namely, the $G$-arc $\left(
x,y\right)  $). Qed.}. If we pick one such $G$-arc for each vertex $y\in
N\left(  U\right)  \setminus S$, then we obtain a total of $\left\vert
N\left(  U\right)  \setminus S\right\vert $ different $G$-arcs belonging to
$\left[  S,\overline{S}\right]  $ (indeed, they are distinct because their
targets $y$ are distinct).

\item For every vertex $y\in Y\cap S$, the $t$-arc $\left(  y,t\right)  $
belongs to $\left[  S,\overline{S}\right]  $\ \ \ \ \footnote{\textit{Proof.}
Let $y\in Y\cap S$. Then, $y\in Y\cap S\subseteq Y$, so that $\left(
y,t\right)  \in A_{t}$ (by the definition of $A_{t}$). Thus, $\left(
y,t\right)  $ is a $t$-arc. The source of this arc lies in $S$ (since $y\in
Y\cap S\subseteq S$), but the target of this arc lies in $\overline{S}$ (since
$t\in\overline{S}$ (because $t\notin S$)). In other words, this arc $\left(
y,t\right)  $ belongs to $\left[  S,\overline{S}\right]  $. Qed.}. Of course,
these $t$-arcs $\left(  y,t\right)  $ are distinct (since their sources $y$
are distinct). Thus, we have found a total of $\left\vert Y\cap S\right\vert $
different $t$-arcs belonging to to $\left[  S,\overline{S}\right]  $ (one for
each $y\in N\left(  U\right)  \setminus S$).
\end{itemize}

Now, recall that there is no overlap between the $G$-arcs, the $s$-arcs and
the $t$-arcs. Hence,%
\begin{align*}
\left\vert \left[  S,\overline{S}\right]  \right\vert  &  =\underbrace{\left(
\text{the number of all }G\text{-arcs belonging to }\left[  S,\overline
{S}\right]  \right)  }_{\substack{\geq\left\vert N\left(  U\right)  \setminus
S\right\vert \\\text{(since we have found }\left\vert N\left(  U\right)
\setminus S\right\vert \text{ different }G\text{-arcs belonging to }\left[
S,\overline{S}\right]  \text{)}}}\\
&  \ \ \ \ \ \ \ \ \ \ +\underbrace{\left(  \text{the number of all
}s\text{-arcs belonging to }\left[  S,\overline{S}\right]  \right)
}_{\substack{\geq\left\vert X\setminus S\right\vert \\\text{(since we have
found }\left\vert X\setminus S\right\vert \text{ different }s\text{-arcs
belonging to }\left[  S,\overline{S}\right]  \text{)}}}\\
&  \ \ \ \ \ \ \ \ \ \ +\underbrace{\left(  \text{the number of all
}t\text{-arcs belonging to }\left[  S,\overline{S}\right]  \right)
}_{\substack{\geq\left\vert Y\cap S\right\vert \\\text{(since we have found
}\left\vert Y\cap S\right\vert \text{ different }t\text{-arcs belonging to
}\left[  S,\overline{S}\right]  \text{)}}}\\
&  \geq\left\vert N\left(  U\right)  \setminus S\right\vert +\left\vert
X\setminus S\right\vert +\left\vert Y\cap S\right\vert \\
&  =\underbrace{\left\vert Y\cap S\right\vert +\left\vert N\left(  U\right)
\setminus S\right\vert }_{\substack{\geq\left\vert N\left(  U\right)
\right\vert \\\text{(by (\ref{pf.thm.hall.hard.3}))}}}+\underbrace{\left\vert
X\setminus S\right\vert }_{\substack{=\left\vert X\right\vert -\left\vert
U\right\vert \\\text{(by (\ref{pf.thm.hall.hard.4}))}}}\\
&  \geq\left\vert N\left(  U\right)  \right\vert +\left\vert X\right\vert
-\left\vert U\right\vert .
\end{align*}
Hence, (\ref{pf.thm.hall.hard.5}) yields $\left\vert f\right\vert =\left\vert
\left[  S,\overline{S}\right]  \right\vert \geq\left\vert N\left(  U\right)
\right\vert +\left\vert X\right\vert -\left\vert U\right\vert $.

But let $M$ be the subset $\left\{  e\in E\ \mid\ f\left(  \overrightarrow{e}%
\right)  =1\right\}  $ of $E$. Then, Proposition \ref{prop.hall.lem1}
\textbf{(d)} yields that the set $M=\left\{  e\in E\ \mid\ f\left(
\overrightarrow{e}\right)  =1\right\}  $ is a matching in $G$. Also,
Proposition \ref{prop.hall.lem1} \textbf{(e)} shows that
\[
\left\vert M\right\vert =\left\vert f\right\vert \geq\left\vert N\left(
U\right)  \right\vert +\left\vert X\right\vert -\left\vert U\right\vert .
\]
We have thus found a matching $M$ in $G$ and a subset $U$ of $X$ satisfying
$\left\vert M\right\vert \geq\left\vert N\left(  U\right)  \right\vert
+\left\vert X\right\vert -\left\vert U\right\vert $. This proves Lemma
\ref{lem.hall.MU}.
\end{proof}

We can now easily prove Theorem \ref{thm.hall}:

\begin{proof}
[Proof of Theorem \ref{thm.hall} (sketched).]$\Longrightarrow:$ Assume that
$G$ has an $X$-complete matching. We must prove that each subset $U$ of $X$
satisfies $\left\vert N\left(  U\right)  \right\vert \geq\left\vert
U\right\vert $.

This is the so-called \textquotedblleft easy part\textquotedblright\ of
Theorem \ref{thm.hall}, and can be proven without any reference to flows and
cuts. Just fix an $X$-complete matching $M$ in $G$ (such a matching exists, by
assumption). Let $U$ be a subset of $X$. We must prove that $\left\vert
N\left(  U\right)  \right\vert \geq\left\vert U\right\vert $.

The matching $M$ is $X$-complete; thus, each vertex $x\in X$ has an
$M$-partner. Let $i:X\rightarrow Y$ be the map that sends each vertex $x\in X$
to its $M$-partner. Any two distinct elements of $X$ must have distinct
$M$-partners (because otherwise, the edges joining them to their common
$M$-partner would be two distinct edges in $M$ that have a vertex in common;
but this is not allowed for a matching). In other words, the map $i$ is
injective. Hence, every subset $Z$ of $X$ satisfies $\left\vert i\left(
Z\right)  \right\vert =\left\vert Z\right\vert $. Applying this to $Z=U$, we
obtain $\left\vert i\left(  U\right)  \right\vert =\left\vert U\right\vert $.

But each $x\in U$ satisfies $i\left(  x\right)  \in N\left(  U\right)  $
(because the vertex $i\left(  x\right)  $ has a neighbor in $U$ (namely, the
vertex $x$)). In other words, $i\left(  U\right)  \subseteq N\left(  U\right)
$. Hence, $\left\vert i\left(  U\right)  \right\vert \leq\left\vert N\left(
U\right)  \right\vert $, so that $\left\vert N\left(  U\right)  \right\vert
\geq\left\vert i\left(  U\right)  \right\vert =\left\vert U\right\vert $.

Now, forget that we fixed $U$. We thus have shown that each subset $U$ of $X$
satisfies $\left\vert N\left(  U\right)  \right\vert \geq\left\vert
U\right\vert $. This proves the $\Longrightarrow$ direction of Theorem
\ref{thm.hall}.

$\Longleftarrow:$ Assume that
\begin{equation}
\text{each subset }U\text{ of }X\text{ satisfies }\left\vert N\left(
U\right)  \right\vert \geq\left\vert U\right\vert .
\label{pf.thm.hall.hard.ass}%
\end{equation}
We must prove that $G$ has an $X$-complete matching.

Lemma \ref{lem.hall.MU} shows that there exist a matching $M$ in $G$ and a
subset $U$ of $X$ satisfying $\left\vert M\right\vert \geq\left\vert N\left(
U\right)  \right\vert +\left\vert X\right\vert -\left\vert U\right\vert $.
Consider these $M$ and $U$. Then,%
\[
\left\vert M\right\vert \geq\underbrace{\left\vert N\left(  U\right)
\right\vert }_{\substack{\geq\left\vert U\right\vert \\\text{(by
(\ref{pf.thm.hall.hard.ass}))}}}+\left\vert X\right\vert -\left\vert
U\right\vert \geq\left\vert U\right\vert +\left\vert X\right\vert -\left\vert
U\right\vert =\left\vert X\right\vert .
\]
Hence, Lemma \ref{lem.hall.complete-mat} \textbf{(b)} yields that the matching
$M$ is $X$-complete. Hence, $G$ has an $X$-complete matching (namely, $M$).
This proves the $\Longleftarrow$ direction of Theorem \ref{thm.hall}. Thus,
the proof of Theorem \ref{thm.hall} is complete.
\end{proof}

Lemma \ref{lem.hall.MU} puts us at a vantage point to prove not just Hall's
marriage theorem, but also its close relative, \textit{K\"{o}nig's vertex
cover theorem}. Before we state the latter theorem, we need to define the
concept of a vertex cover:

\begin{definition}
Let $G=\left(  W,E\right)  $ be a simple graph. Then, a \textit{vertex cover}
of $G$ means a subset $C$ of $W$ such that each edge $e\in E$ contains at
least one vertex in $C$.
\end{definition}

For example, if $G=\left(  W,E\right)  $ is the simple graph%
\begin{equation}%
%TCIMACRO{\TeXButton{a triangle}{\xymatrix{
%& 1 \are[dl] \are[dr] \\
%2 \are[rr] & & 3
%}}}%
%BeginExpansion
\xymatrix{
& 1 \are[dl] \are[dr] \\
2 \are[rr] & & 3
}%
%EndExpansion
, \label{eq.hall.vertexcover.ex1}%
\end{equation}
then every $2$-element subset of $W$ is a vertex cover (but no smaller subset
of $W$ is).

Now, \textit{K\"{o}nig's theorem on vertex covers} states the following:

\begin{theorem}
\label{thm.hall.koenig}Let $\left(  G;X,Y\right)  $ be a bipartite graph.
Then, the maximum size of a matching in $G$ equals the minimum size of a
vertex cover of $G$.
\end{theorem}

Notice that the claim of Theorem \ref{thm.hall.koenig} isn't true for
arbitrary graphs $G$; for example, if $G$ is the simple graph in
(\ref{eq.hall.vertexcover.ex1}), then the maximum size of a matching in $G$ is
$1$, but the minimum size of a vertex cover of $G$ is $2$. Nevertheless, the
claim is true when $G$ is part of a bipartite graph $\left(  G;X,Y\right)  $.

What is true for arbitrary graphs $G$ is the following inequality:

\begin{proposition}
\label{prop.hall.koenig.<=}Let $G$ be a simple graph. Then, the maximum size
of a matching in $G$ is $\leq$ to the minimum size of a vertex cover of $G$.
\end{proposition}

\begin{proof}
[Proof of Proposition \ref{prop.hall.koenig.<=}.]Let $m$ be the maximum size
of a matching in $G$. Let $c$ be the minimum size of a vertex cover of $G$. We
must prove that $m\leq c$.

There exists a matching $M$ in $G$ such that $\left\vert M\right\vert =m$
(since $m$ is the size of a matching in $G$). Consider this $M$.

There exists a vertex cover $C$ of $G$ such that $\left\vert C\right\vert =c$
(since $c$ is the size of a vertex cover of $G$). Consider this $C$.

Write the simple graph $G$ in the form $G=\left(  W,E\right)  $. Recall that
$C$ is a vertex cover of $G$. In other words, $C$ is a subset of $W$ such that
each edge $e\in E$ contains at least one vertex in $C$ (by the definition of a
\textquotedblleft vertex cover\textquotedblright).

Each edge $e\in E$ contains at least one vertex in $C$. In other words, for
each edge $e\in E$, there is at least one vertex $v\in C$ such that $v\in e$.
In other words, for each edge $e\in E$, we have%
\begin{equation}
\left(  \text{the number of }v\in C\text{ such that }v\in e\right)  \geq1.
\label{pf.prop.hall.koenig.<=.>=}%
\end{equation}


On the other hand, $M$ is a matching in $G$. In other words, $M$ is a subset
of $E$ such that no two distinct edges in $M$ have a vertex in common (by the
definition of a \textquotedblleft matching\textquotedblright). No two distinct
edges in $M$ have a vertex in common. In other words, no vertex of $G$ belongs
to more than one edge in $M$. In other words, each vertex of $G$ lies in at
most one edge in $M$. In other words, for each vertex $v$ of $G$, there is at
most one edge $e\in M$ such that $v\in e$. In other words, for each vertex $v$
of $G$, we have%
\begin{equation}
\left(  \text{the number of edges }e\in M\text{ such that }v\in e\right)
\leq1. \label{pf.prop.hall.koenig.<=.<=}%
\end{equation}


Now,
\begin{align*}
&  \left(  \text{the number of pairs }\left(  v,e\right)  \in C\times M\text{
such that }v\in e\right) \\
&  =\sum_{v\in C}\underbrace{\left(  \text{the number of edges }e\in M\text{
such that }v\in e\right)  }_{\substack{\leq1\\\text{(by
(\ref{pf.prop.hall.koenig.<=.<=}))}}}\leq\sum_{v\in C}1\\
&  =\left\vert C\right\vert \cdot1=\left\vert C\right\vert =c.
\end{align*}
Thus,%
\begin{align*}
c  &  \geq\left(  \text{the number of pairs }\left(  v,e\right)  \in C\times
M\text{ such that }v\in e\right) \\
&  =\sum_{e\in M}\underbrace{\left(  \text{the number of }v\in C\text{ such
that }v\in e\right)  }_{\substack{\geq1\\\text{(by
(\ref{pf.prop.hall.koenig.<=.>=}))}}}\geq\sum_{e\in M}1\\
&  =\left\vert M\right\vert \cdot1=\left\vert M\right\vert =m.
\end{align*}
In other words, $m\leq c$. This completes our proof of Proposition
\ref{prop.hall.koenig.<=}.
\end{proof}

\begin{proof}
[Proof of Theorem \ref{thm.hall.koenig} (sketched).]Let $m$ be the maximum
size of a matching in $G$. Let $c$ be the minimum size of a vertex cover of
$G$. We must prove that $m=c$.

Proposition \ref{prop.hall.koenig.<=} yields $m\leq c$.

Write the simple graph $G$ in the form $G=\left(  W,E\right)  $.

On the other hand, Lemma \ref{lem.hall.MU} shows that there exist a matching
$M$ in $G$ and a subset $U$ of $X$ satisfying $\left\vert M\right\vert
\geq\left\vert N\left(  U\right)  \right\vert +\left\vert X\right\vert
-\left\vert U\right\vert $. Consider these $M$ and $U$.

The size $\left\vert M\right\vert $ of the matching $M$ is clearly $\leq$ to
the maximum size of a matching in $G$. In other words, $\left\vert
M\right\vert \leq m$ (since $m$ is the maximum size of a matching in $G$).
Thus,%
\begin{equation}
m\geq\left\vert M\right\vert \geq\left\vert N\left(  U\right)  \right\vert
+\left\vert X\right\vert -\left\vert U\right\vert .
\label{pf.thm.hall.koenig.m>=}%
\end{equation}


On the other hand, let $C$ be the subset $\left(  X\setminus U\right)  \cup
N\left(  U\right)  $ of $W$. Then, we have the following:

\begin{statement}
\textit{Observation 1:} The set $C$ is a vertex cover of $G$.
\end{statement}

[\textit{Proof of Observation 1:} Let $e\in E$ be any edge. We shall show that
$e$ contains at least one vertex in $C$.

Indeed, assume the contrary. Thus, $e$ contains no vertex in $C$.

Each edge of $G$ contains exactly one vertex in $X$ and exactly one vertex in
$Y$ (since $\left(  G;X,Y\right)  $ is a bipartite graph). Thus, in
particular, each edge of $G$ contains exactly one vertex in $X$. Applying this
to the edge $e$, we conclude that the edge $e$ contains exactly one vertex in
$X$. Let $x$ be this vertex. Thus, $x\in X$, and the edge $e$ contains the
vertex $x$. If we had $x\in C$, then $e$ would contain a vertex in $C$
(namely, the vertex $x$), which would contradict the fact that $e$ contains no
vertex in $C$. Hence, we cannot have $x\in C$. Thus, we have $x\notin C$.
Hence, $x\in X\setminus C$.

But $C=\left(  X\setminus U\right)  \cup N\left(  U\right)  \supseteq
X\setminus U$, so that $X\setminus\underbrace{C}_{\supseteq X\setminus
U}\subseteq X\setminus\left(  X\setminus U\right)  =U$ (since $U\subseteq X$).
Hence, $x\in X\setminus C\subseteq U$.

Now, let $y$ be the endpoint of the edge $e$ distinct from $x$. (This is
well-defined, since we already know that $e$ contains $x$.) Then, $x$ and $y$
are the two endpoints of the edge $e$. Hence, $e=\left\{  x,y\right\}  $.
Thus, $\left\{  y,x\right\}  =\left\{  x,y\right\}  =e$ is an edge of $G$; in
other words, $x$ is a neighbor of $y$. Hence, the vertex $y$ has a neighbor in
$U$ (namely, the neighbor $x$). In other words, $y\in N\left(  U\right)  $ (by
the definition of $N\left(  U\right)  $). Hence, $y\in N\left(  U\right)
\subseteq\left(  X\setminus U\right)  \cup N\left(  U\right)  =C$. In other
words, $y$ is a vertex in $C$. Thus, the edge $e$ contains at least one vertex
in $C$ (namely, the vertex $y$), because $e$ contains $y$. This contradicts
the fact that $e$ contains no vertex in $C$.

This contradiction proves that our assumption was wrong. Hence, we have shown
that $e$ contains at least one vertex in $C$.

Now, forget that we fixed $e$. We thus have proven that each edge $e\in E$
contains at least one vertex in $C$. In other words, $C$ is a vertex cover of
$G$ (by the definition of \textquotedblleft vertex cover\textquotedblright).
This proves Observation 1.]

Now, $C=\left(  X\setminus U\right)  \cup N\left(  U\right)  $, so that
\begin{align*}
\left\vert C\right\vert  &  =\left\vert \left(  X\setminus U\right)  \cup
N\left(  U\right)  \right\vert \leq\underbrace{\left\vert X\setminus
U\right\vert }_{\substack{=\left\vert X\right\vert -\left\vert U\right\vert
\\\text{(since }U\subseteq X\text{)}}}+\left\vert N\left(  U\right)
\right\vert \\
&  \ \ \ \ \ \ \ \ \ \ \left(  \text{this is in fact an equality, but we don't
need this}\right) \\
&  =\left\vert X\right\vert -\left\vert U\right\vert +\left\vert N\left(
U\right)  \right\vert =\left\vert N\left(  U\right)  \right\vert +\left\vert
X\right\vert -\left\vert U\right\vert \leq m\ \ \ \ \ \ \ \ \ \ \left(
\text{by (\ref{pf.thm.hall.koenig.m>=})}\right)  .
\end{align*}


But the set $C$ is a vertex cover of $G$ (by Observation 1). Hence, its size
$\left\vert C\right\vert $ is $\geq$ to the minimum size of a vertex cover of
$G$. In other words, $\left\vert C\right\vert \geq c$ (since $c$ is the
minimum size of a vertex cover of $G$). Thus, $c\leq\left\vert C\right\vert
\leq m$. Combining this with $m\leq c$, we obtain $m=c$. This completes the
proof of Theorem \ref{thm.hall.koenig}.
\end{proof}

\begin{thebibliography}{99999999}                                                                                         %


\bibitem[ForFul62]{ForFul62}L. R. Ford (Jr.) and D. R. Fulkerson,
\textit{Flows in Networks}, Princeton University Press 1962.\newline A
preprint is freely available at
\url{https://www.rand.org/content/dam/rand/pubs/reports/2007/R375.pdf} .

\bibitem[Goeman17]{Goeman17}Michel X. Goemans, \textit{18.453 Combinatorial
Optimization}, course at MIT, Spring 2017.\newline\url{http://math.mit.edu/~goemans/18453S17/18453.html}

\bibitem[LeLeMe17]{LeLeMe17}Eric Lehman, F. Thomson Leighton, Albert R. Meyer,
\textit{Mathematics for Computer Science}, version 15 November 2017.\newline\url{https://courses.csail.mit.edu/6.042/fall17/mcs.pdf}

\bibitem[Martin17]{Martin17}Jeremy L. Martin, \textit{Lecture Notes on
Algebraic Combinatorics}, 8 June 2017.\newline\url{http://people.ku.edu/~jlmartin/LectureNotes.pdf}

\bibitem[Schrij12]{Schrij12}%
\href{https://www.math.uni-bielefeld.de/documenta/vol-ismp/33_schrijver-alexander-tmf.pdf}{Alexander
Schrijver, \textit{On the History of the Transportation and Maximum Flow
Problems}, Documenta Mathematica, Extra Volume ISMP (2012), pp. 169--180.}

\bibitem[Schrij17]{Schrij17}Alexander (Lex) Schrijver, \textit{A Course in
Combinatorial Optimization}, 23 March 2017.\newline\url{https://homepages.cwi.nl/~lex/files/dict.pdf}

\bibitem[Thalwi08]{Thalwi08}Timon Thalwitzer, \textit{Max-Flow Min-Cut},
diploma thesis at Universit\"{a}t Wien.\newline\url{http://www.mat.univie.ac.at/~kratt/theses/thalwitz.pdf}

\bibitem[Zwick95]{Zwick95}%
\href{https://doi.org/10.1016/0304-3975(95)00022-O}{Uri Zwick, \textit{The
smallest networks on which the Ford-Fulkerson maximum flow procedure may fail
to terminate}, Theoretical Computer Science \textbf{148}, Issue 1, 21 August
1995, pp. 165--170}.
\end{thebibliography}


\end{document}