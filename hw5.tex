\documentclass[numbers=enddot,12pt,final,onecolumn,notitlepage]{scrartcl}%
\usepackage[headsepline,footsepline,manualmark]{scrlayer-scrpage}
\usepackage[all,cmtip]{xy}
\usepackage{amssymb}
\usepackage{amsmath}
\usepackage{amsthm}
\usepackage{framed}
\usepackage{comment}
\usepackage{color}
\usepackage{hyperref}
\usepackage{ifthen}
\usepackage[sc]{mathpazo}
\usepackage[T1]{fontenc}
\usepackage{needspace}
\usepackage{tabls}
\newcounter{exer}
\theoremstyle{definition}
\newtheorem{theo}{Theorem}[section]
\newenvironment{theorem}[1][]
{\begin{theo}[#1]\begin{leftbar}}
{\end{leftbar}\end{theo}}
\newtheorem{lem}[theo]{Lemma}
\newenvironment{lemma}[1][]
{\begin{lem}[#1]\begin{leftbar}}
{\end{leftbar}\end{lem}}
\newtheorem{prop}[theo]{Proposition}
\newenvironment{proposition}[1][]
{\begin{prop}[#1]\begin{leftbar}}
{\end{leftbar}\end{prop}}
\newtheorem{defi}[theo]{Definition}
\newenvironment{definition}[1][]
{\begin{defi}[#1]\begin{leftbar}}
{\end{leftbar}\end{defi}}
\newtheorem{remk}[theo]{Remark}
\newenvironment{remark}[1][]
{\begin{remk}[#1]\begin{leftbar}}
{\end{leftbar}\end{remk}}
\newtheorem{coro}[theo]{Corollary}
\newenvironment{corollary}[1][]
{\begin{coro}[#1]\begin{leftbar}}
{\end{leftbar}\end{coro}}
\newtheorem{conv}[theo]{Convention}
\newenvironment{condition}[1][]
{\begin{conv}[#1]\begin{leftbar}}
{\end{leftbar}\end{conv}}
\newtheorem{quest}[theo]{Question}
\newenvironment{algorithm}[1][]
{\begin{quest}[#1]\begin{leftbar}}
{\end{leftbar}\end{quest}}
\newtheorem{warn}[theo]{Warning}
\newenvironment{conclusion}[1][]
{\begin{warn}[#1]\begin{leftbar}}
{\end{leftbar}\end{warn}}
\newtheorem{conj}[theo]{Conjecture}
\newenvironment{conjecture}[1][]
{\begin{conj}[#1]\begin{leftbar}}
{\end{leftbar}\end{conj}}
\newtheorem{exam}[theo]{Example}
\newenvironment{example}[1][]
{\begin{exam}[#1]\begin{leftbar}}
{\end{leftbar}\end{exam}}
\newtheorem{exmp}[exer]{Exercise}
\newenvironment{exercise}[1][]
{\begin{exmp}[#1]\begin{leftbar}}
{\end{leftbar}\end{exmp}}
\newenvironment{statement}{\begin{quote}}{\end{quote}}
\iffalse
\newenvironment{proof}[1][Proof]{\noindent\textbf{#1.} }{\ \rule{0.5em}{0.5em}}
\fi
\let\sumnonlimits\sum
\let\prodnonlimits\prod
\let\cupnonlimits\bigcup
\let\capnonlimits\bigcap
\renewcommand{\sum}{\sumnonlimits\limits}
\renewcommand{\prod}{\prodnonlimits\limits}
\renewcommand{\bigcup}{\cupnonlimits\limits}
\renewcommand{\bigcap}{\capnonlimits\limits}
\setlength\tablinesep{3pt}
\setlength\arraylinesep{3pt}
\setlength\extrarulesep{3pt}
\voffset=0cm
\hoffset=-0.7cm
\setlength\textheight{22.5cm}
\setlength\textwidth{15.5cm}
\newenvironment{verlong}{}{}
\newenvironment{vershort}{}{}
\newenvironment{noncompile}{}{}
\excludecomment{verlong}
\includecomment{vershort}
\excludecomment{noncompile}
\newcommand{\id}{\operatorname{id}}
\newcommand{\rev}{\operatorname{rev}}
\newcommand{\conncomp}{\operatorname{conncomp}}
\newcommand{\conn}{\operatorname{conn}}
\newcommand{\NN}{\mathbb{N}}
\newcommand{\ZZ}{\mathbb{Z}}
\newcommand{\QQ}{\mathbb{Q}}
\newcommand{\RR}{\mathbb{R}}
\newcommand{\powset}[2][]{\ifthenelse{\equal{#2}{}}{\mathcal{P}\left(#1\right)}{\mathcal{P}_{#1}\left(#2\right)}}
% $\powset[k]{S}$ stands for the set of all $k$-element subsets of
% $S$. The argument $k$ is optional, and if not provided, the result
% is the whole powerset of $S$.
\newcommand{\set}[1]{\left\{ #1 \right\}}
% $\set{...}$ yields $\left\{ ... \right\}$.
\newcommand{\abs}[1]{\left| #1 \right|}
% $\abs{...}$ yields $\left| ... \right|$.
\newcommand{\tup}[1]{\left( #1 \right)}
% $\tup{...}$ yields $\left( ... \right)$.
\newcommand{\ive}[1]{\left[ #1 \right]}
% $\ive{...}$ yields $\left[ ... \right]$.
\newcommand{\verts}[1]{\operatorname{V}\left( #1 \right)}
% $\verts{...}$ yields $\operatorname{V}\left( ... \right)$.
\newcommand{\edges}[1]{\operatorname{E}\left( #1 \right)}
% $\edges{...}$ yields $\operatorname{E}\left( ... \right)$.
\newcommand{\arcs}[1]{\operatorname{A}\left( #1 \right)}
% $\arcs{...}$ yields $\operatorname{A}\left( ... \right)$.
\newcommand{\underbrack}[2]{\underbrace{#1}_{\substack{#2}}}
% $\underbrack{...1}{...2}$ yields
% $\underbrace{...1}_{\substack{...2}}$. This is useful for doing
% local rewriting transformations on mathematical expressions with
% justifications.
\newcommand{\are}{\ar@{-}}
% In an xymatrix environment, $\are$ gives an arrow without
% arrowhead. I use this to represent edges in graphs.
\newcommand{\arebi}[1][]{\ar@{<-}@/_/[#1] \ar@/^/[#1]}
% In an xymatrix environment, $\arebi$ gives two arrows with
% opposite direction. I use this to represent edges in
% bidirected digraphs.
\ihead{Math 5707 Spring 2017 (Darij Grinberg): homework set 5}
\ohead{page \thepage}
\cfoot{}
\begin{document}

\begin{center}
\textbf{Math 5707 Spring 2017 (Darij Grinberg): homework set 5}

\textbf{due: Wed, 26 Apr 2017, in class} or by email
(\texttt{dgrinber@umn.edu}) before class

\textbf{Please hand in solutions to FOUR of the 7 problems.}
\end{center}

%\tableofcontents

\subsection{Reminders}

See the
\href{http://www-users.math.umn.edu/~dgrinber/5707s17/nogra.pdf}{lecture notes}
and also the
\href{http://www-users.math.umn.edu/~dgrinber/5707s17/}{handwritten notes}
for relevant material.
See also
\href{http://www-users.math.umn.edu/~dgrinber/5707s17/hw2s.pdf}{the solutions to homework set 2}
for various conventions and notations that are in use here.

\subsection{Sandpiles: recapitulating definitions and results}

Let me recall the definitions of the basic concepts on chipfiring
done in class.
Various sources on this material are
\cite{BjoLov92} (and, less directly, \cite{BjLoSh91}),
\cite{HLMPPW13}, \cite[Lectures 29--31]{Musike09} and \cite{CorPet16}.
(None of these is as readable as I would like to have it, but the
whole subject is about 30 years old, with most activity very
recent...
Also, be aware of incompatible notations, as well as of the fact that
some of the sources only consider undirected graphs.)
The particular case of the ``integer lattice'' graph has attracted
particular attention due to the mysterious pictures it generates;
see \url{http://www.math.cmu.edu/~wes/sand.html#next-page} for some
of these pictures, as well as
\url{http://www.math.cornell.edu/~levine/apollonian-slides.pdf} for a
talk with various illustrations.

Let me give a brief (proof-less) survey of what we did in class (and
a bit of what we should have done).

Fix a loopless multidigraph $D = \tup{V, A, \phi}$.

\begin{definition}
A \textit{configuration} (on $D$) means a map $f : V \to \NN$.
(Recall that $\NN = \set{0, 1, 2, \ldots}$.)

A configuration is also called a \textit{chip configuration} or
\textit{sandpile}.

We like to think of a configuration as a way to place
a finite number of game chips on the vertices of $D$:
Namely, the configuration $f$ corresponds to placing
$f \tup{v}$ chips on the vertex $v$ for each $v \in V$.
The chips are understood to be undistinguishable, so the
only thing that matters is how many of them are placed on
each given vertex.
Sometimes, we speak of grains of sand instead of chips.
\end{definition}

\begin{definition}
A \textit{$\ZZ$-configuration} (on $D$) means a map
$f : V \to \ZZ$.
We shall regard each configuration as a $\ZZ$-configuration
(since $\NN \subseteq \ZZ$).
\end{definition}

\begin{definition}
Let $f : V \to \ZZ$ be a $\ZZ$-configuration.

\textbf{(a)} A vertex $v \in V$ is said to be
\textit{active} in $f$ if and only if
$f\tup{v} \geq \deg^+ v$.
(Recall that $\deg^+ v$ is the outdegree of $v$.)

\textbf{(b)} The $\ZZ$-configuration $f$ is said to be
\textit{stable} if no vertex $v \in V$ is active in $f$.
\end{definition}

Notice that there are only finitely many stable configurations
(because if $f$ is a stable configuration, then, for each
$v \in V$, the stability of $f$ implies $f \tup{v} \leq \deg^+ v$,
whereas the fact that $f$ is a configuration implies
$f \tup{v} \geq 0$; but these two inequalities combined
leave only finitely many possible values for $f \tup{v}$).

\begin{definition} \label{def.chip.add-configs}
The set $\ZZ^V$ of all $\ZZ$-configurations can be equipped with
operations of addition and subtraction, defined as follows:

\begin{itemize}
 \item For any two $\ZZ$-configurations $f : V \to \ZZ$ and
       $g : V \to \ZZ$, we define a $\ZZ$-configuration
       $f + g : V \to \ZZ$ by setting
       \[
        \tup{f + g} \tup{v}
        = f \tup{v} + g \tup{v}
        \qquad \text{ for each } v \in V .
       \]

 \item For any two $\ZZ$-configurations $f : V \to \ZZ$ and
       $g : V \to \ZZ$, we define a $\ZZ$-configuration
       $f - g : V \to \ZZ$ by setting
       \[
        \tup{f - g} \tup{v}
        = f \tup{v} - g \tup{v}
        \qquad \text{ for each } v \in V .
       \]
\end{itemize}

These operations of addition and subtraction satisfy the
standard rules (e.g., we always have
$\tup{f+g} + h = f + \tup{g+h}$ and
$\tup{f-g} - h = f - \tup{g+h}$).
Hence, we can write terms like $f + g + h$ or $f - g - h$
without having to explicitly place parentheses.

Also, we can define a ``zero configuration'' $0 : V \to \ZZ$,
which is the configuration that sends each $v \in V$ to the
number $0$.
(Hopefully, the dual use of the symbol $0$ for both the number
$0$ and this zero configuration is not too confusing.)

Also, for each $\ZZ$-configuration $f : V \to \ZZ$ and each
integer $N$, we define a $\ZZ$-configuration $Nf : V \to \ZZ$
by
\[
\tup{Nf} \tup{v} = N f \tup{v}
\qquad \text{ for each } v \in V .
\]
\end{definition}

\begin{definition}
Let $f : V \to \ZZ$ be any $\ZZ$-configuration.
Then, $\sum f$ shall denote the integer
$\sum_{v \in V} f \tup{v}$.

This integer $\sum f$ is called the \textit{degree} of $f$.
\end{definition}

If $f$ is a configuration, then $\sum f$ is the total
number of chips in $f$.

\begin{definition}
Let $v \in V$ be a vertex.
Then, a $\ZZ$-configuration $\Delta v$ is defined by setting
\[
 \tup{\Delta v} \tup{w}
 = \begin{cases}
    \deg^+ v , & \text{if } w = v ; \\
    - a_{v, w}, & \text{if } w \neq v
   \end{cases}
  \qquad \text{ for all } w \in V ,
\]
where $a_{v, w}$ denotes the number of all arcs of $D$ having
source $v$ and target $w$.
(Note that $a_{v, w}$ might be $> 1$, since $D$ is a multidigraph.)
\end{definition}

\begin{definition}
Let $v \in V$ be a vertex.
Then, \textit{firing $v$} is the operation on
$\ZZ$-configurations (i.e., formally speaking, the mapping from
$\ZZ^V$ to $\ZZ^V$) that sends each $\ZZ$-configuration
$f : V \to \ZZ$ to $f - \Delta v$.

We sometimes say ``toppling $v$'' instead of ``firing $v$''.
\end{definition}

If $f : V \to \NN$ is a configuration, then the
$\ZZ$-configuration $f - \Delta v$ obtained by firing $v$ can be
described as follows:
The vertex $v$ ``donates'' $\deg^+ v$ of its chips to its
neighbors, by sending one chip along each of its outgoing arcs
(i.e., for each arc having source $v$, the vertex $v$ sends one
chip along this arc to the target of this arc).
Thus, the number of chips on $v$ (weakly) decreases, while the
number of chips on each other vertex (weakly) increases.
Of course, the resulting $\ZZ$-configuration $f - \Delta v$ is
not necessarily a configuration.
(In fact, it is a configuration if and only if the vertex $v$ is
active in $f$.)

Notice that $\sum \tup{\Delta v} = 0$ for each vertex $v$.
Thus,
$\sum \tup{f - \Delta v}
= \sum f - \underbrace{\sum \tup{\Delta v}}_{= 0}
= \sum f$
for each $\ZZ$-configuration $f : V \to \ZZ$ and each vertex $v$.
In other words, firing a vertex $v$ does not change the degree of
a $\ZZ$-configuration.

\begin{definition}
Let $f : V \to \NN$ be a configuration.

Let $\tup{v_1, v_2, \ldots, v_k}$ be a sequence of vertices of
$D$.

\textbf{(a)} The sequence $\tup{v_1, v_2, \ldots, v_k}$ is said
to be \textit{legal} for $f$ if for each
$i \in \set{1, 2, \ldots, k}$, the vertex $v_i$ is active in
the $\ZZ$-configuration
$f - \Delta v_1 - \Delta v_2 - \cdots - \Delta v_{i-1}$.

\textbf{(b)} The sequence $\tup{v_1, v_2, \ldots, v_k}$ is said
to be \textit{stabilizing} for $f$ if the $\ZZ$-configuration
$f - \Delta v_1 - \Delta v_2 - \cdots - \Delta v_k$ is stable.
\end{definition}

What is the rationale behind the notions of ``legal'' and
``stabilizing''?
A sequence of vertices provides a way to modify a
configuration by first firing the first vertex in the sequence,
then firing the second, and so on.
The sequence is said to be \textit{legal} (for $f$) if the
configuration remains a configuration throughout this ordeal
(i.e., at no point does a vertex have a negative number of
chips).
The sequence is said to be \textit{stabilizing} (for $f$) if
the $\ZZ$-configuration resulting from it at the very end is
stable.

We notice some obvious consequences of the definitions:

\begin{itemize}
 \item If a sequence $\tup{v_1, v_2, \ldots, v_k}$ is legal
       for a configuration $f$, then all of the
       $\ZZ$-configurations
       $f - \Delta v_1 - \Delta v_2 - \cdots - \Delta v_i$
       for $i \in \set{0, 1, \ldots, k}$ are actually
       configurations.
 \item If a sequence $\tup{v_1, v_2, \ldots, v_k}$ is legal
       for a configuration $f$, then each prefix of this
       sequence (i.e., each sequence of the form
       $\tup{v_1, v_2, \ldots, v_i}$ for some
       $i \in \set{0, 1, \ldots, k}$) is legal for $f$ as
       well.
 \item If a sequence $\tup{v_1, v_2, \ldots, v_k}$ is
       stabilizing for a configuration $f$, then each
       permutation of this sequence (i.e., each sequence of
       the form
       $\tup{v_{\sigma\tup{1}}, v_{\sigma\tup{2}}, \ldots,
             v_{\sigma\tup{k}}}$ for a permutation
       $\sigma$ of $\set{1, 2, \ldots, k}$) is stabilizing
       for $f$ as well.
 \item If $\tup{v_1, v_2, \ldots, v_k}$ is a legal sequence
       for a configuration $f$, then
       $\tup{v_1, v_2, \ldots, v_k}$ is stabilizing if and
       only if there exist no $v \in V$ such that the
       sequence $\tup{v_1, v_2, \ldots, v_k, v}$ is legal.
\end{itemize}


An important property of chipfiring is the following result
(sometimes called the ``least action principle''):

\begin{theorem} \label{thm.chip.lap}
Let $f : V \to \NN$ be any configuration.
Let $\ell$ and $s$ be two sequences of vertices of $D$ such
that $\ell$ is legal for $f$ while $s$ is stabilizing for $f$.
Then, $\ell$ is a subpermutation of $s$.
\end{theorem}

Here, we are using the following notation:

\begin{definition}
Let $\tup{p_1, p_2, \ldots, p_u}$ and
$\tup{q_1, q_2, \ldots, q_v}$ be two finite sequences.
Then, we say that $\tup{p_1, p_2, \ldots, p_u}$ is
a \textit{subpermutation} of $\tup{q_1, q_2, \ldots, q_v}$
if and only if, for each object $t$, the following holds:
The number of $i \in \set{1, 2, \ldots, u}$ satisfying
$p_i = t$ is less or equal to the number of
$j \in \set{1, 2, \ldots, v}$ satisfying $q_j = t$.

Equivalently, the sequence $\tup{p_1, p_2, \ldots, p_u}$ is
a subpermutation of the sequence $\tup{q_1, q_2, \ldots, q_v}$
if and only if you can obtain the former from the latter by
removing some entries and permuting the remaining entries.
(``Some'' allows for the possibility of ``zero''.)
\end{definition}

\begin{corollary} \label{cor.chip.lap-cor}
Let $f : V \to \NN$ be any configuration.
Let $\ell$ and $\ell'$ be two sequences of vertices of $D$
that are both legal and stabilizing for $f$.
Then:

\textbf{(a)} The sequence $\ell'$ is a permutation of $\ell$.

In particular:

\textbf{(b)} The sequences $\ell$ and $\ell'$ have the
same length.

\textbf{(c)} For each $t \in V$, the number of times $t$
appears in $\ell'$ equals the number of times $t$ appears in
$\ell$.

\textbf{(d)} The configuration obtained from $f$ by firing
all vertices in $\ell$ (one after the other) equals the
configuration obtained from $f$ by firing
all vertices in $\ell'$ (one after the other).
\end{corollary}

Next we state some facts about legal sequences:

\begin{lemma} \label{lem.chip.leg-bound1}
Let $f : V \to \NN$ be a configuration.
Let $h = \sum f$.
Let $\ell$ be a legal sequence for $f$.

Let $a$ be an arc of $D$.
Let $u$ be the source of $a$, and let $v$ be the target of $a$.

\textbf{(a)}
If $u$ appears more than $h$ times in the sequence $\ell$,
then $v$ must appear at least once in the sequence $\ell$.

\textbf{(b)}
Fix $k \in \NN$.
If $u$ appears more than $kh$ times in the sequence $\ell$,
then $v$ must appear at least $k$ times in the sequence $\ell$.
\end{lemma}

\begin{lemma} \label{lem.chip.leg-bound1b}
Let $f : V \to \NN$ be a configuration.
Let $h = \sum f$.
Let $\ell$ be a legal sequence for $f$.

Let $u$ and $v$ be two vertices of $D$ such that there exists
a path of length $d$ from $u$ to $v$.

If $u$ appears at least $\dfrac{h^{d+1}-1}{h-1}$
times in the sequence $\ell$,
then $v$ must appear at least once in the sequence $\ell$.

(The fraction $\dfrac{h^{d+1}-1}{h-1}$ should be interpreted
as $d+1$ when $h = 1$.)
\end{lemma}

\begin{proposition} \label{prop.chip.leg-everyone-fires}
Let $f : V \to \NN$ be a configuration.
Let $h = \sum f$.
Let $\ell$ be a legal sequence for $f$.
Let $n = \abs{V}$.

Let $q$ be a vertex of $D$ such that for each vertex
$u \in V$, there exists a path from $u$ to $q$.

If the length of $\ell$ is
$> \tup{n-1} \tup{ \dfrac{h^n-1}{h-1} - 1 }$,
then $q$ must appear at least once in the sequence $\ell$.

(The fraction $\dfrac{h^n-1}{h-1}$ should be interpreted
as $n$ when $h = 1$.)
\end{proposition}

\begin{proposition} \label{prop.chip.leg-period}
Let $f : V \to \NN$ be a configuration.
Let $h = \sum f$.
Let $\ell = \tup{\ell_1, \ell_2, \ldots, \ell_k}$ be a
legal sequence for $f$.
Let
$g = f - \Delta \ell_1 - \Delta \ell_2 - \cdots - \Delta \ell_k$
be the configuration obtained from $f$ by firing the vertices in
$\ell$ (one after the other).

\textbf{(a)} We have $g \in \set{0, 1, \ldots, h}^V$.
(In other words, $g \tup{v} \in \set{0, 1, \ldots, h}$
for each $v \in V$.)

\textbf{(b)} Let $n = \abs{V}$.
If the sequence $\ell$ has length
$\geq \tup{h+1}^n$, then there exist legal sequences (for $f$)
of arbitrary length.
\end{proposition}

\begin{definition}
Let $f : V \to \NN$ be a configuration.

We say that $f$ is \textit{finitary} if there
exists a sequence of vertices that is stabilizing for $f$.
Otherwise, we say that $f$ is \textit{infinitary}.
\end{definition}

\begin{theorem} \label{thm.chip.dichotomy}
Let $f : V \to \NN$ be a configuration.
Then, exactly one of the following two statements holds:

\begin{itemize}
\item \textit{Statement 1:}
      The configuration $f$ is finitary. \par
      There exists a sequence $s$ of vertices that is both
      legal and stabilizing for $f$. \par
      All such sequences are permutations of $s$.
      \par
      All legal sequences (for $f$) are subpermutations
      of $s$, and in particular are at most as long as $s$.

\item \textit{Statement 2:}
      The configuration $f$ is infinitary. \par
      There exists no stabilizing sequence for $f$. \par
      There exist legal sequences for $f$ of arbitrary length.
      More precisely, each legal sequence for $f$ can be
      extended to a longer legal sequence.
\end{itemize}
\end{theorem}

\begin{definition} \label{def.chip.stabilization}
Let $f : V \to \NN$ be a finitary configuration.
Then, Statement 1 in Theorem~\ref{thm.chip.dichotomy} must
hold.
Therefore, there exists a sequence $s$ of vertices that is both
legal and stabilizing for $f$.
The \textit{stabilization} of $f$ means
the configuration obtained from $f$ by firing
all vertices in $s$ (one after the other).
(This does not depend on the choice of $s$, because of
Corollary~\ref{cor.chip.lap-cor} \textbf{(d)}.)

The stabilization of $f$ is denoted by $f^\circ$.
\end{definition}

Something similar holds if we forbid firing a specific
vertex:

\begin{definition}
Let $q \in V$.

Let $f : V \to \NN$ be a configuration.

Let $\tup{v_1, v_2, \ldots, v_k}$ be a sequence of vertices of
$D$.

\textbf{(a)} The sequence $\tup{v_1, v_2, \ldots, v_k}$ is said
to be \textit{$q$-legal} for $f$ if it is legal and does not
contain the vertex $q$.

\textbf{(b)} The sequence $\tup{v_1, v_2, \ldots, v_k}$ is said
to be \textit{$q$-stabilizing} for $f$ if the $\ZZ$-configuration
$f - \Delta v_1 - \Delta v_2 - \cdots - \Delta v_k$ has no
active vertices except (possibly) $q$.
\end{definition}

We can now define ``$q$-finitary'' and ``$q$-infinitary''
and obtain an analogue of Theorem~\ref{thm.chip.dichotomy}.
But the most commonly considered case is that when $q$ is
a ``global sink'' (a vertex with no outgoing arcs, and which
is reachable from any vertex),
and in this case \textbf{every} configuration is
$q$-finitary.
Let us state this as its own result:

\begin{theorem} \label{thm.chip.dichotomy-q}
Let $f : V \to \NN$ be a configuration.
Let $q \in V$.
Assume that for each vertex $u \in V$, there exists a path
$u \to q$.
Then, there exists a sequence $s$ of vertices that is both
$q$-legal and $q$-stabilizing for $f$.
All such sequences are permutations of $s$.
All $q$-legal sequences (for $f$) are subpermutations
of $s$, and in particular are at most as long as $s$.
\end{theorem}

\begin{definition}
Let $f : V \to \NN$ be a configuration.
Let $q \in V$.
Assume that for each vertex $u \in V$, there exists a path
$u \to q$.
Then, Theorem~\ref{thm.chip.dichotomy-q} shows that
there exists a sequence $s$ of vertices that is both
$q$-legal and $q$-stabilizing for $f$.
The \textit{$q$-stabilization} of $f$ means
the configuration obtained from $f$ by firing
all vertices in $s$ (one after the other).
(This does not depend on the choice of $s$, because of
the analogue of
Corollary~\ref{cor.chip.lap-cor} \textbf{(d)}
for $q$-legal and $q$-stabilizing sequences.)
\end{definition}


\subsection{Exercise \ref{exe.chip.better-bounds}: better
bounds for legal sequences}

The following exercise improves on the bound given in
Proposition~\ref{prop.chip.leg-period} \textbf{(b)} and
also on the one given in
Proposition~\ref{prop.chip.leg-everyone-fires}\footnote{To
  see that Exercise~\ref{exe.chip.better-bounds} \textbf{(b)}
  improves on the bound given in
  Proposition~\ref{prop.chip.leg-everyone-fires},
  we need to check that
  $\tup{n-1} \tup{ \dfrac{h^n-1}{h-1} - 1} + 1
  \geq \dbinom{n+h-1}{n-1}$.
  This is easy for $n \leq 1$ (in fact, the case
  $n = 0$ is impossible due to the existence of a
  $q \in V$, and the case $n = 1$ is an equality
  case).
  In the remaining case $n \geq 2$,
  the stronger inequality
  $\dfrac{h^n-1}{h-1} - 1 + 1
  \geq \dbinom{n+h-1}{n-1}$
  can be proven by a simple induction
  over $n$.}.
I don't know whether the improved bounds can be further
improved.

\begin{exercise} \label{exe.chip.better-bounds}
Fix a loopless multidigraph $D = \tup{V, A, \phi}$.
Let $f : V \to \NN$ be a configuration.
Let $h = \sum f$.
Let $n = \abs{V}$.
Assume that $n > 0$.

Let $\ell = \tup{\ell_1, \ell_2, \ldots, \ell_k}$ be a
legal sequence for $f$ having length
$k \geq \dbinom{n+h-1}{n-1}$.

Prove the following:

\textbf{(a)} There exist legal sequences (for $f$) of
arbitrary length.

\textbf{(b)} Let $q$ be a vertex of $D$ such that for each vertex
$u \in V$, there exists a path from $u$ to $q$.
Then, $q$ must appear at least once in the sequence $\ell$.

[\textbf{Hint:}
For \textbf{(a)}, apply the same pigeonhole-principle argument
as for Proposition~\ref{prop.chip.leg-period} \textbf{(b)}.]
\end{exercise}

In the above exercise,
you are allowed to use the fact\footnote{See, for example,
  \url{https://math.stackexchange.com/questions/36250/number-of-monomials-of-certain-degree}
  for a proof of this fact (in the language of monomials).
  Or see \cite[\S 1.2]{Stanle11} (search for
  ``weak composition'' and read the first paragraph that
  comes up).}
that the number of
$n$-tuples $\tup{a_1, a_2, \ldots, a_n}$ of nonnegative
integers satisfying $a_1 + a_2 + \cdots + a_n = h$
is $\dbinom{n+h-1}{n-1}$.

\subsection{Exercise \ref{exe.chip.examples}: examples of chip-firing}

\Needspace{15cm}
\begin{exercise} \label{exe.chip.examples}
\textbf{(a)} Let $D$ be the following digraph:
\[
\xymatrix{
 u \ar[r] & v \ar[r] & q
}
\]
(i.e., the digraph $D$ with three vertices
$u, v, q$ and two arcs $uv, vq$.)

Let $k$ be a positive integer.
Consider the configuration $g_k$ on $D$ which has $k$ chips at $u$
and $0$ chips at each other vertex.

Find the $q$-stabilization of $g_k$.

\textbf{(b)} Let $D$ be the following digraph:
\[
\xymatrix{
 u \are[r] & v \ar[r] & q
}
\]
where a curve without an arrow stands for one arc in each
direction.
(Thus, formally speaking, the digraph $D$ has three vertices
$u, v, q$ and three arcs $uv, vu, vq$.)

Let $k$ be a positive integer.
Consider the configuration $g_k$ on $D$ which has $k$ chips at $u$
and $0$ chips at each other vertex.

Find the $q$-stabilization of $g_k$.

\textbf{(c)} Let $D$ be the following digraph:
\[
\xymatrix{
 & v \are[dl] \are[dr] \ar[dd] \\
 u \are[rr] \ar[dr] & & w \ar[dl] \\
 & q
}
\]
where a curve without an arrow stands for one arc in each
direction.
(Thus, formally speaking, the digraph $D$ has four vertices
$u, v, w, q$ and nine arcs
$uv, vu, vw, wv, wu, uw, uq, vq, wq$.)

Let $k \geq 2$ be an integer.
Consider the configuration $f_k$ on $D$ which has $k$ chips at each
vertex (i.e., which has $f_k \tup{v} = k$ for each
$v \in \set{u, v, w, q}$).

Find the $q$-stabilization of $f_k$.
\end{exercise}

\subsection{Exercise \ref{exe.chip.adj-cycles}: a lower bound
on the degree of an infinitary configuration}

We shall use the notation
$\operatorname{s}\tup{a}$ for the source of an arc
$a \in A$.
We shall also the notation
$\operatorname{t}\tup{a}$ for the target of an arc
$a \in A$.

\begin{exercise} \label{exe.chip.adj-cycles}
Assume that the multidigraph $D$ is strongly connected.
Let $f : V \to \NN$ be an infinitary configuration.

\textbf{(a)} Prove that $D$ cannot have more than
$\sum f$ vertex-disjoint cycles.
(A set of cycles is said to be \textit{vertex-disjoint}
if no two distinct cycles in the set have a vertex in
common.)

\textbf{(b)} Prove that $D$ cannot have more than
$\sum f$ arc-disjoint cycles.
(A set of cycles is said to be \textit{arc-disjoint}
if no two distinct cycles in the set have an arc in
common.)
\end{exercise}

Exercise~\ref{exe.chip.adj-cycles} \textbf{(b)} is
\cite[Theorem 2.2]{BjoLov92}, but the proof given there
is vague and unrigorous.

\subsection{Exercise \ref{exe.chip.assoc}: an associativity
law for stabilizations}

Recall Definition~\ref{def.chip.stabilization}.

\begin{exercise} \label{exe.chip.assoc}
Let $f : V \to \NN$, $g : V \to \NN$ and $h : V \to \NN$
be three configurations such that both configurations
$f$ and $g + h$ are finitary, and such that the
configuration $f + \tup{g + h}^\circ$ is also finitary.

Prove the following:

\textbf{(a)} The configurations $f + g$ and $h$ are also
finitary.

\textbf{(b)} The configurations $f + g + h$ and
$\tup{f + g}^\circ + h$ are also finitary, and
satisfy
\[
\tup{f + g + h}^\circ
= \tup{f + \tup{g + h}^\circ}^\circ
= \tup{\tup{f + g}^\circ + h}^\circ .
\]

[\textbf{Hint:} The following piece of notation is
useful:
If $k$ and $k'$ are two configurations, then
$k \to k'$ shall mean that there exists a legal
sequence $\ell$ for $k$ such that firing all vertices
in $\ell$ (one after the other) transforms $k$ into
$k'$.
This relation $\to$ is reflexive and transitive.
Show that if $c$, $k$ and $k'$ are three configurations
satisfying $k \to k'$, then $c + k \to c + k'$.]
\end{exercise}

\subsection{Exercise \ref{exe.chip.Z2}: chip-firing on
the integer lattice}

Now, we shall briefly discuss chip-firing on the
integer lattice $\ZZ^2$; this is one of the most famous
cases of chip-firing, leading to some of the pretty
pictures.
For examples and illustrations, check out
\cite{Ellenb15} as well as some of the links above.

We have not defined infinite graphs in class; the
theory of infinite graphs involves some subtleties that
would take us too far.
However, for this particular exercise, we need only a
specific infinite graph, which is fairly simple.

\begin{definition}
\textbf{(a)} A \textit{locally finite multigraph} means
a triple $\tup{V, E, \phi}$, where $V$ and $E$ are sets
and $\phi : E \to \powset[2]{V}$ is a map having the
following property:
\begin{enumerate}
\item[(*)] For each $v \in V$, there exist only finitely
many $e \in E$ satisfying $v \in \phi \tup{e}$.
\end{enumerate}

Most of the concepts defined for (usual) multigraphs still
make sense for locally finite multigraphs.
In particular, the elements of $V$ are called the
vertices, and the elements of $E$ are called the
edges.
The property (*) says that each vertex is contained in
only finitely many edges; this allows us to define
the degree of a vertex.

\textbf{(b)} The \textit{integer lattice} shall mean
the locally finite multigraph defined as follows:
\begin{itemize}
\item The vertices of the integer lattice are the
      pairs $\tup{i, j}$ of two integers $i$ and $j$.
      In other words, the vertex set of the integer
      lattice is
        $\ZZ^2 = \set{ \tup{i, j} \ \mid \ i \in \ZZ
                        \text{ and } j \in \ZZ }$.
      We view these vertices as points in the
      plane, and draw the multigraph accordingly.
\item Two vertices of the integer lattice are
      adjacent if and only if they have distance $1$
      (as points in the plane).
      In other words, a vertex $\tup{i, j}$ is
      adjacent to the four vertices
      $\tup{i+1, j}, \tup{i, j+1}, \tup{i-1, j},
      \tup{i, j-1}$ and no others.
\end{itemize}

\textbf{(c)} You can guess how locally finite
multidigraphs are defined.
Each locally finite multigraph can be regarded as a
locally finite multidigraph by replacing each edge
by a pair of two arcs (directed in both possible
directions).
\end{definition}

Let us show a piece of the integer lattice, viewed as
a locally finite multigraph:
\[
\xymatrix{
& \are[d] & \are[d] & \are[d] & \\
\are[r] & \tup{-1, 1} \are[d] \are[r] & \tup{0, 1} \are[d] \are[r] & \tup{1, 1} \are[d] \are[r] & \\
\are[r] & \tup{-1, 0} \are[d] \are[r] & \tup{0, 0} \are[d] \are[r] & \tup{1, 0} \are[d] \are[r] & \\
\are[r] & \tup{-1, -1} \are[d] \are[r] & \tup{0, -1} \are[d] \are[r] & \tup{1, -1} \are[d] \are[r] & \\
& & & &
}
\]
And here is it again, viewed as a locally finite
multidigraph:
\[
\xymatrix{
& \arebi[d] & \arebi[d] & \arebi[d] & \\
\arebi[r] & \tup{-1, 1} \arebi[d] \arebi[r] & \tup{0, 1} \arebi[d] \arebi[r] & \tup{1, 1} \arebi[d] \arebi[r] & \\
\arebi[r] & \tup{-1, 0} \arebi[d] \arebi[r] & \tup{0, 0} \arebi[d] \arebi[r] & \tup{1, 0} \arebi[d] \arebi[r] & \\
\arebi[r] & \tup{-1, -1} \arebi[d] \arebi[r] & \tup{0, -1} \arebi[d] \arebi[r] & \tup{1, -1} \arebi[d] \arebi[r] & \\
& & & &
}
\]

\begin{exercise} \label{exe.chip.Z2}
Let $f$ be a configuration on the integer lattice
(where we view the integer lattice as a locally
finite multidigraph).
(The notion of a configuration and related notions
are defined in the same way as for usual, finite
multidigraphs.)

Assume that only finitely many vertices $v \in \ZZ^2$
satisfy $f \tup{v} \neq 0$.
(Thus, the total number of chips $\sum f$ is finite.)

An edge $e$ of the integer lattice is
said to be \textit{non-void} in $f$ if and only if
at least one of the endpoints of $e$ has at least one
chip in $f$.

Prove the following:

\textbf{(a)} If an edge of the integer lattice is
non-void in $f$, then this edge remains non-void after
firing any legal sequence of vertices.
(``Firing a sequence'' means firing all the vertices
in the sequence, one after the other.)

\textbf{(b)} The total number of configurations that
can be obtained from $f$ by firing a legal sequence
of vertices is finite.

\textbf{(c)} If we fire any active vertex, then the
sum $\sum_{\tup{i, j} \in \ZZ^2} f \tup{\tup{i, j}}
\cdot \tup{i + j}^2$ increases.

\textbf{(d)} The configuration $f$ is finitary (so its
stabilization is well-defined).
\end{exercise}

This exercise gives the reason why pictures such as the
ones in \cite{Ellenb15} exist (although it does not
explain their shapes and patterns).

\subsection{Exercise \ref{exe.aco.score-vector}: acyclic
orientations are determined by their score vectors}

Now, we leave the chip-firing setting.

Roughly speaking, an \textit{orientation} of
a multigraph $G$ is a way to assign to each edge of $G$
a direction (thus making it an arc).
If the resulting \textbf{digraph} has no cycles, then
this orientation will be called \textit{acyclic}.
A rigorous way to state this definition is the
following:

\begin{definition} \label{def.aco.aco}
Let $G = \tup{V, E, \psi}$ be a multigraph.

\textbf{(a)} An \textit{orientation} of $G$ is a map
$\phi : E \to V \times V$ such that each $e \in E$
has the following property:
If we write $\phi \tup{e}$ in the form
$\phi \tup{e} = \tup{u, v}$, then
$\psi \tup{e} = \set{u, v}$.

\textbf{(b)} An orientation $\phi$ of $G$ is said to
be \textit{acyclic} if and only if the multidigraph
$\tup{V, E, \phi}$ has no cycles.
\end{definition}

\begin{example}
Let $G = \tup{V, E, \psi}$ be the following multigraph:
\[
\xymatrix{
& 2 \are[dl]^a \are@/_2pc/[dl]_b \are[dr]^c \\
1 \are[rr]_d & & 3
}
\]
Then, the following four maps $\phi$ are orientations of $G$:
\begin{itemize}
\item the map sending $a$ to $\tup{1, 2}$, sending $b$ to
      $\tup{1, 2}$, sending $c$ to $\tup{3, 2}$, and
      sending $d$ to $\tup{1, 3}$;
\item the map sending $a$ to $\tup{2, 1}$, sending $b$ to
      $\tup{1, 2}$, sending $c$ to $\tup{3, 2}$, and
      sending $d$ to $\tup{3, 1}$;
\item the map sending $a$ to $\tup{1, 2}$, sending $b$ to
      $\tup{1, 2}$, sending $c$ to $\tup{2, 3}$, and
      sending $d$ to $\tup{1, 3}$;
\item the map sending $a$ to $\tup{1, 2}$, sending $b$ to
      $\tup{1, 2}$, sending $c$ to $\tup{2, 3}$, and
      sending $d$ to $\tup{3, 1}$.
\end{itemize}
Here are the multidigraphs $\tup{V, E, \phi}$ corresponding to
these four maps (in the order mentioned):
\[
\begin{tabular}{|c|c|c|c|}
\xymatrix{
& 2 \ar@{<-}[dl]^a \ar@{<-}@/_2pc/[dl]_b \ar@{<-}[dr]^c \\
1 \ar[rr]_d & & 3
} &
\xymatrix{
& 2 \ar[dl]^a \ar@{<-}@/_2pc/[dl]_b \ar@{<-}[dr]^c \\
1 \ar@{<-}[rr]_d & & 3
} &
\xymatrix{
& 2 \ar@{<-}[dl]^a \ar@{<-}@/_2pc/[dl]_b \ar[dr]^c \\
1 \ar[rr]_d & & 3
} &
\xymatrix{
& 2 \ar@{<-}[dl]^a \ar@{<-}@/_2pc/[dl]_b \ar[dr]^c \\
1 \ar@{<-}[rr]_d & & 3
}
\end{tabular}
\]
Only the first and the third of these orientations
$\phi$ are acyclic (since only the first and the third
of these multidigraphs have no cycles).
\end{example}

\begin{exercise} \label{exe.aco.score-vector}
Let $G = \tup{V, E, \psi}$ be a multigraph.

Prove the following:

\textbf{(a)} If $\phi$ is any acyclic orientation of $G$,
and if $V \neq \varnothing$,
then there exists a $v \in V$ such that no arc of the
multidigraph $\tup{V, E, \phi}$ has target $v$.

\textbf{(b)} If $\phi_1$ and $\phi_2$ are two acyclic orientations
of $G$ such that each $v \in V$ satisfies
\[
\deg^+_{\tup{V, E, \phi_1}} v = \deg^+_{\tup{V, E, \phi_2}} v ,
\]
then $\phi_1 = \phi_2$.
\end{exercise}

\subsection{Exercise \ref{exe.flows-cuts.cut-lattice}: the
lattice structure on minimum cuts}

Let us recall some terminology from
\href{http://www-users.math.umn.edu/~dgrinber/5707s17/5707lec16.pdf}{lecture 16}:

\begin{itemize}
\item A \textit{network} consists of:
      \begin{itemize}
      \item a digraph $\tup{V, A}$;
      \item two distinct vertices $s \in V$ and $t \in V$,
            called the \textit{source} and the \textit{sink},
            respectively (although we do not require $s$ to
            have indegree $0$ or $t$ to have outdegree $0$);
      \item a function $c : A \to \QQ_+$, called the
            \textit{capacity function}.
            (Here, $\QQ_+$ means the set
            $\set{ x \in \QQ \mid x \geq 0 }$.)
      \end{itemize}

\item Given a network consisting of a digraph
      $\tup{V, A}$, a source $s \in V$ and a sink $t \in V$,
      and a capacity function $c : A \to \QQ_+$, we define
      the following notations:
      \begin{itemize}
      \item For any subset $S$ of $V$, we let $\overline{S}$
            denote the subset $V \setminus S$ of $V$.
      \item If $P$ and $Q$ are two subsets of $V$, then
            $\ive{P, Q}$ shall mean the set of all arcs
            $a \in A$ whose source belongs to $P$ and whose
            target belongs to $Q$.
            (In other words,
            $\ive{P, Q} = A \cap \tup{P \times Q}$.)
      \item If $P$ and $Q$ are two subsets of $V$, then the
            number $c \tup{P, Q} \in \QQ_+$ is defined by
            \[
            c \tup{P, Q} = \sum_{a \in \ive{P, Q}} c \tup{a} .
            \]
      \end{itemize}
\end{itemize}

We also refer to lecture 16 for the definition of a flow
(which is not necessary for the following problem, but may
be helpful).

\begin{exercise} \label{exe.flows-cuts.cut-lattice}
Consider a network consisting of a digraph
$\tup{V, A}$, a source $s \in V$ and a sink $t \in V$,
and a capacity function $c : A \to \QQ_+$
such that $s \neq t$.

An \textit{$s$-$t$-cutting subset} shall mean a subset
$S$ of $V$ satisfying $s \in S$ and $t \notin S$.

Let $m$ denote the minimum possible value of
$c \tup{S, \overline{S}}$ where $S$ ranges over the
$s$-$t$-cutting subsets.
(Recall that this is the maximum value of a
flow, according to the
maximum-flow-minimum-cut theorem.)

An $s$-$t$-cutting subset $S$ is said to be
\textit{cut-minimal} if it satisfies
$c \tup{S, \overline{S}} = m$.

Let $X$ and $Y$ be two cut-minimal $s$-$t$-cutting subsets.
Prove that $X \cap Y$ and $X \cup Y$ also are
cut-minimal $s$-$t$-cutting subsets.
\end{exercise}

\begin{thebibliography}{99999999}

\bibitem[BjoLov92]{BjoLov92}
Anders Bj\"orner, L\'aszl\'o Lov\'asz,
\textit{Chip-firing games on directed graphs},
J. Algebraic Combinatorics 1 (1991), pp. 305--328,
revised version, July 1992.
\newline\url{http://www.cs.elte.hu/~lovasz/morepapers/abacus.pdf}

\bibitem[BjLoSh91]{BjLoSh91}
Anders Bj\"orner, L\'aszl\'o Lov\'asz, P. W. Shor,
\textit{Chip-firing games on graphs},
Europ. J. Comb. 12 (1991), pp. 283--291.
\newline\url{http://www.cs.elte.hu/~lovasz/morepapers/chips.pdf}

\bibitem[CorPet16]{CorPet16}
Scott Corry, David Perkinson,
\textit{Divisors and Sandpiles},
draft of 20 November 2016.
\newline\url{http://people.reed.edu/~davidp/divisors_and_sandpiles/}

\bibitem[Ellenb15]{Ellenb15}
Jordan Ellenberg,
\textit{The Amazing, Autotuning Sandpile},
Nautilus, 2 April 2015.
\newline\url{http://nautil.us/issue/23/dominoes/the-amazing-autotuning-sandpile}

\bibitem[HLMPPW13]{HLMPPW13}
Alexander E. Holroyd, Lionel Levine, Karola M\'esz\'aros,
Yuval Peres, James Propp, David B. Wilson,
\textit{Chip-Firing and Rotor-Routing on Directed Graphs},
arXiv:0801.3306v4.
\newline\url{https://arxiv.org/abs/0801.3306v4}

\bibitem[Musike09]{Musike09}
Gregg Musiker,
\textit{18.312: Algebraic Combinatorics, Spring 2009},
MIT OpenCourseWare.
\newline\url{https://ocw.mit.edu/courses/mathematics/18-312-algebraic-combinatorics-spring-2009/readings-and-lecture-notes/}

\bibitem[Stanle11]{Stanle11}Richard P. Stanley, \textit{Enumerative
Combinatorics, volume 1}, Cambridge University Press, 2011. \newline%
\url{http://math.mit.edu/~rstan/ec/ec1/}

\end{thebibliography}

\end{document}