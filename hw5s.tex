\documentclass[numbers=enddot,12pt,final,onecolumn,notitlepage]{scrartcl}%
\usepackage[headsepline,footsepline,manualmark]{scrlayer-scrpage}
\usepackage[all,cmtip]{xy}
\usepackage{amssymb}
\usepackage{amsmath}
\usepackage{amsthm}
\usepackage{framed}
\usepackage{comment}
\usepackage{color}
\usepackage{hyperref}
\usepackage{ifthen}
\usepackage[sc]{mathpazo}
\usepackage[T1]{fontenc}
\usepackage{needspace}
\usepackage{tabls}
\usepackage{epigraph}
\newcounter{exer}
\theoremstyle{definition}
\newtheorem{theo}{Theorem}[section]
\newenvironment{theorem}[1][]
{\begin{theo}[#1]\begin{leftbar}}
{\end{leftbar}\end{theo}}
\newtheorem{lem}[theo]{Lemma}
\newenvironment{lemma}[1][]
{\begin{lem}[#1]\begin{leftbar}}
{\end{leftbar}\end{lem}}
\newtheorem{prop}[theo]{Proposition}
\newenvironment{proposition}[1][]
{\begin{prop}[#1]\begin{leftbar}}
{\end{leftbar}\end{prop}}
\newtheorem{defi}[theo]{Definition}
\newenvironment{definition}[1][]
{\begin{defi}[#1]\begin{leftbar}}
{\end{leftbar}\end{defi}}
\newtheorem{remk}[theo]{Remark}
\newenvironment{remark}[1][]
{\begin{remk}[#1]\begin{leftbar}}
{\end{leftbar}\end{remk}}
\newtheorem{coro}[theo]{Corollary}
\newenvironment{corollary}[1][]
{\begin{coro}[#1]\begin{leftbar}}
{\end{leftbar}\end{coro}}
\newtheorem{conv}[theo]{Convention}
\newenvironment{condition}[1][]
{\begin{conv}[#1]\begin{leftbar}}
{\end{leftbar}\end{conv}}
\newtheorem{quest}[theo]{Question}
\newenvironment{algorithm}[1][]
{\begin{quest}[#1]\begin{leftbar}}
{\end{leftbar}\end{quest}}
\newtheorem{warn}[theo]{Warning}
\newenvironment{conclusion}[1][]
{\begin{warn}[#1]\begin{leftbar}}
{\end{leftbar}\end{warn}}
\newtheorem{conj}[theo]{Conjecture}
\newenvironment{conjecture}[1][]
{\begin{conj}[#1]\begin{leftbar}}
{\end{leftbar}\end{conj}}
\newtheorem{exam}[theo]{Example}
\newenvironment{example}[1][]
{\begin{exam}[#1]\begin{leftbar}}
{\end{leftbar}\end{exam}}
\newtheorem{exmp}[exer]{Exercise}
\newenvironment{exercise}[1][]
{\begin{exmp}[#1]\begin{leftbar}}
{\end{leftbar}\end{exmp}}
\newenvironment{statement}{\begin{quote}}{\end{quote}}
\iffalse
\newenvironment{proof}[1][Proof]{\noindent\textbf{#1.} }{\ \rule{0.5em}{0.5em}}
\fi
\let\sumnonlimits\sum
\let\prodnonlimits\prod
\let\cupnonlimits\bigcup
\let\capnonlimits\bigcap
\renewcommand{\sum}{\sumnonlimits\limits}
\renewcommand{\prod}{\prodnonlimits\limits}
\renewcommand{\bigcup}{\cupnonlimits\limits}
\renewcommand{\bigcap}{\capnonlimits\limits}
\setlength\tablinesep{3pt}
\setlength\arraylinesep{3pt}
\setlength\extrarulesep{3pt}
\voffset=0cm
\hoffset=-0.7cm
\setlength\textheight{22.5cm}
\setlength\textwidth{15.5cm}
\newenvironment{verlong}{}{}
\newenvironment{vershort}{}{}
\newenvironment{noncompile}{}{}
\excludecomment{verlong}
\includecomment{vershort}
\excludecomment{noncompile}
\newcommand{\id}{\operatorname{id}}
\newcommand{\rev}{\operatorname{rev}}
\newcommand{\conncomp}{\operatorname{conncomp}}
\newcommand{\conn}{\operatorname{conn}}
\newcommand{\NN}{\mathbb{N}}
\newcommand{\ZZ}{\mathbb{Z}}
\newcommand{\QQ}{\mathbb{Q}}
\newcommand{\RR}{\mathbb{R}}
\newcommand{\powset}[2][]{\ifthenelse{\equal{#2}{}}{\mathcal{P}\left(#1\right)}{\mathcal{P}_{#1}\left(#2\right)}}
% $\powset[k]{S}$ stands for the set of all $k$-element subsets of
% $S$. The argument $k$ is optional, and if not provided, the result
% is the whole powerset of $S$.
\newcommand{\set}[1]{\left\{ #1 \right\}}
% $\set{...}$ yields $\left\{ ... \right\}$.
\newcommand{\abs}[1]{\left| #1 \right|}
% $\abs{...}$ yields $\left| ... \right|$.
\newcommand{\tup}[1]{\left( #1 \right)}
% $\tup{...}$ yields $\left( ... \right)$.
\newcommand{\ive}[1]{\left[ #1 \right]}
% $\ive{...}$ yields $\left[ ... \right]$.
\newcommand{\verts}[1]{\operatorname{V}\left( #1 \right)}
% $\verts{...}$ yields $\operatorname{V}\left( ... \right)$.
\newcommand{\edges}[1]{\operatorname{E}\left( #1 \right)}
% $\edges{...}$ yields $\operatorname{E}\left( ... \right)$.
\newcommand{\arcs}[1]{\operatorname{A}\left( #1 \right)}
% $\arcs{...}$ yields $\operatorname{A}\left( ... \right)$.
\newcommand{\underbrack}[2]{\underbrace{#1}_{\substack{#2}}}
% $\underbrack{...1}{...2}$ yields
% $\underbrace{...1}_{\substack{...2}}$. This is useful for doing
% local rewriting transformations on mathematical expressions with
% justifications.
\newcommand{\are}{\ar@{-}}
% In an xymatrix environment, $\are$ gives an arrow without
% arrowhead. I use this to represent edges in graphs.
\newcommand{\arebi}[1][]{\ar@{<-}@/_/[#1] \ar@/^/[#1]}
% In an xymatrix environment, $\arebi$ gives two arrows with
% opposite direction. I use this to represent edges in
% bidirected digraphs.
\ihead{Math 5707 Spring 2017 (Darij Grinberg): homework set 5}
\ohead{page \thepage}
\cfoot{}
\begin{document}

\begin{center}
\textbf{Math 5707 Spring 2017 (Darij Grinberg): homework set 5}

\textbf{Solution sketches.}

\textbf{extended and generalized version, with an introduction to sandpile theory (\today).}
\end{center}

\epigraph{This tale grew in the telling.}{J. R. R. Tolkien}

\tableofcontents

\subsection{Reminders}

See the
\href{http://www-users.math.umn.edu/~dgrinber/5707s17/nogra.pdf}{lecture notes}
and also the
\href{http://www-users.math.umn.edu/~dgrinber/5707s17/}{handwritten notes}
for relevant material.
See also
\href{http://www-users.math.umn.edu/~dgrinber/5707s17/hw2s.pdf}{the solutions to homework set 2}
\cite{hw2s}
for various conventions and notations that are in use here.

\subsection{Sandpiles: recapitulating definitions and results}

Let me recall the definitions of the basic concepts on chipfiring
done in class.
Various sources on this material are
\cite{BjoLov92} (and, less directly, \cite{BjLoSh91}),
\cite{HLMPPW13}, \cite[Lectures 29--31]{Musike09} and \cite{CorPet16}.
(None of these is as readable as I would like to have it, but the
whole subject is about 30 years old, with most activity very
recent...
Also, be aware of incompatible notations, as well as of the fact that
some of the sources only consider undirected graphs.)
The particular case of the ``integer lattice'' graph has attracted
particular attention due to the mysterious pictures it generates;
see \url{http://www.math.cmu.edu/~wes/sand.html#next-page} for some
of these pictures, as well as
\url{http://www.math.cornell.edu/~levine/apollonian-slides.pdf} for a
talk with various illustrations.

Let me give a brief (proof-less) survey of what we did in class (and
a bit of what we should have done).

We refer to \cite{hw2s} for the definition of a multidigraph, as well
as for the definitions of walks, paths, circuits and cycles in a
multidigraph.

If $v$ is a vertex of a multidigraph $\tup{V, A, \phi}$, then the
\textit{outdegree} $\deg^+ v$ of $v$ is defined to be the number
of all arcs $a \in A$ whose source is $v$.
Similarly,
if $v$ is a vertex of a multidigraph $\tup{V, A, \phi}$, then the
\textit{indegree} $\deg^- v$ of $v$ is defined to be the number
of all arcs $a \in A$ whose target is $v$.

Fix a loopless multidigraph $D = \tup{V, A, \phi}$.

\begin{definition}
A \textit{configuration} (on $D$) means a map $f : V \to \NN$.
(Recall that $\NN = \set{0, 1, 2, \ldots}$.)

A configuration is also called a \textit{chip configuration} or
\textit{sandpile}.

We like to think of a configuration as a way to place
a finite number of game chips on the vertices of $D$:
Namely, the configuration $f$ corresponds to placing
$f \tup{v}$ chips on the vertex $v$ for each $v \in V$.
The chips are understood to be undistinguishable, so the
only thing that matters is how many of them are placed on
each given vertex.
Sometimes, we speak of grains of sand instead of chips.
\end{definition}

\begin{definition}
A \textit{$\ZZ$-configuration} (on $D$) means a map
$f : V \to \ZZ$.
We shall regard each configuration as a $\ZZ$-configuration
(since $\NN \subseteq \ZZ$).
\end{definition}

\begin{definition}
Let $f : V \to \ZZ$ be a $\ZZ$-configuration.

\textbf{(a)} A vertex $v \in V$ is said to be
\textit{active} in $f$ if and only if
$f\tup{v} \geq \deg^+ v$.
(Recall that $\deg^+ v$ is the outdegree of $v$.)

\textbf{(b)} The $\ZZ$-configuration $f$ is said to be
\textit{stable} if no vertex $v \in V$ is active in $f$.
\end{definition}

Notice that there are only finitely many stable configurations
(because if $f$ is a stable configuration, then, for each
$v \in V$, the stability of $f$ implies $f \tup{v} \leq \deg^+ v$,
whereas the fact that $f$ is a configuration implies
$f \tup{v} \geq 0$; but these two inequalities combined
leave only finitely many possible values for $f \tup{v}$).

\begin{definition} \label{def.chip.add-configs}
The set $\ZZ^V$ of all $\ZZ$-configurations can be equipped with
operations of addition and subtraction, defined as follows:

\begin{itemize}
 \item For any two $\ZZ$-configurations $f : V \to \ZZ$ and
       $g : V \to \ZZ$, we define a $\ZZ$-configuration
       $f + g : V \to \ZZ$ by setting
       \[
        \tup{f + g} \tup{v}
        = f \tup{v} + g \tup{v}
        \qquad \text{ for each } v \in V .
       \]

 \item For any two $\ZZ$-configurations $f : V \to \ZZ$ and
       $g : V \to \ZZ$, we define a $\ZZ$-configuration
       $f - g : V \to \ZZ$ by setting
       \[
        \tup{f - g} \tup{v}
        = f \tup{v} - g \tup{v}
        \qquad \text{ for each } v \in V .
       \]
\end{itemize}

These operations of addition and subtraction satisfy the
standard rules (e.g., we always have
$\tup{f+g} + h = f + \tup{g+h}$ and
$\tup{f-g} - h = f - \tup{g+h}$).
Hence, we can write terms like $f + g + h$ or $f - g - h$
without having to explicitly place parentheses.

Also, we can define a ``zero configuration'' $0 : V \to \ZZ$,
which is the configuration that sends each $v \in V$ to the
number $0$.
(Hopefully, the dual use of the symbol $0$ for both the number
$0$ and this zero configuration is not too confusing.)

Also, for each $\ZZ$-configuration $f : V \to \ZZ$ and each
integer $N$, we define a $\ZZ$-configuration $Nf : V \to \ZZ$
by
\[
\tup{Nf} \tup{v} = N f \tup{v}
\qquad \text{ for each } v \in V .
\]
\end{definition}

\begin{definition}
Let $f : V \to \ZZ$ be any $\ZZ$-configuration.
Then, $\sum f$ shall denote the integer
$\sum_{v \in V} f \tup{v}$.

This integer $\sum f$ is called the \textit{degree} of $f$.
\end{definition}

If $f$ is a configuration, then $\sum f$ is the total
number of chips in $f$.

It is easy to see that any two $\ZZ$-configurations $f$ and
$g$ satisfy $\sum \tup{f+g} = \sum f + \sum g$ and
$\sum \tup{f-g} = \sum f - \sum g$.
Also, any $N \in \ZZ$ and any $\ZZ$-configuration $f$
satisfy $\sum \tup{Nf} = N \sum f$.

\begin{definition}
Let $v \in V$ be a vertex.
Then, a $\ZZ$-configuration $\Delta v$ is defined by setting
\[
 \tup{\Delta v} \tup{w}
 = \begin{cases}
    \deg^+ v , & \text{if } w = v ; \\
    - a_{v, w}, & \text{if } w \neq v
   \end{cases}
  \qquad \text{ for all } w \in V ,
\]
where $a_{v, w}$ denotes the number of all arcs of $D$ having
source $v$ and target $w$.
(Note that $a_{v, w}$ might be $> 1$, since $D$ is a multidigraph.)
\end{definition}

\begin{definition}
Let $v \in V$ be a vertex.
Then, \textit{firing $v$} is the operation on
$\ZZ$-configurations (i.e., formally speaking, the mapping from
$\ZZ^V$ to $\ZZ^V$) that sends each $\ZZ$-configuration
$f : V \to \ZZ$ to $f - \Delta v$.

We sometimes say ``toppling $v$'' instead of ``firing $v$''.
\end{definition}

If $f : V \to \NN$ is a configuration, then the
$\ZZ$-configuration $f - \Delta v$ obtained by firing $v$ can be
described as follows:
The vertex $v$ ``donates'' $\deg^+ v$ of its chips to its
neighbors, by sending one chip along each of its outgoing arcs
(i.e., for each arc having source $v$, the vertex $v$ sends one
chip along this arc to the target of this arc).
Thus, the number of chips on $v$ (weakly) decreases, while the
number of chips on each other vertex (weakly) increases.
Of course, the resulting $\ZZ$-configuration $f - \Delta v$ is
not necessarily a configuration.
(In fact, it is a configuration if and only if the vertex $v$ is
active in $f$.)

Notice that $\sum \tup{\Delta v} = 0$ for each vertex $v$.
Thus,
$\sum \tup{f - \Delta v}
= \sum f - \underbrace{\sum \tup{\Delta v}}_{= 0}
= \sum f$
for each $\ZZ$-configuration $f : V \to \ZZ$ and each vertex $v$.
In other words, firing a vertex $v$ does not change the degree of
a $\ZZ$-configuration.

We shall now study the effects of repeatedly firing vertices
of $D$ (that is, firing several vertices one after the other).
If $v_1, v_2, \ldots, v_k$ are finitely many elements of $V$,
and if $f : V \to \ZZ$ is any $\ZZ$-configuration,
then firing the vertices $v_1, v_2, \ldots, v_k$
(one after the other, in this order) results in the
$\ZZ$-configuration
$f - \Delta v_1 - \Delta v_2 - \cdots - \Delta v_k$.
This shows, in particular, that the order in which we fire
the vertices does not matter for the final result
(i.e., firing them in any other order would yield the same
resulting $\ZZ$-configuration);
however, the intermediate configurations of course do depend
on the order.

\begin{definition}
Let $f : V \to \NN$ be a configuration.

Let $\tup{v_1, v_2, \ldots, v_k}$ be a sequence of vertices of
$D$.

\textbf{(a)} The sequence $\tup{v_1, v_2, \ldots, v_k}$ is said
to be \textit{legal} for $f$ if for each
$i \in \set{1, 2, \ldots, k}$, the vertex $v_i$ is active in
the $\ZZ$-configuration
$f - \Delta v_1 - \Delta v_2 - \cdots - \Delta v_{i-1}$.

\textbf{(b)} The sequence $\tup{v_1, v_2, \ldots, v_k}$ is said
to be \textit{stabilizing} for $f$ if the $\ZZ$-configuration
$f - \Delta v_1 - \Delta v_2 - \cdots - \Delta v_k$ is stable.
\end{definition}

What is the rationale behind the notions of ``legal'' and
``stabilizing''?
A sequence of vertices provides a way to modify a
configuration by first firing the first vertex in the sequence,
then firing the second, and so on.
The sequence is said to be \textit{legal} (for $f$) if the
configuration remains a configuration throughout this ordeal
(i.e., at no point does a vertex have a negative number of
chips).
The sequence is said to be \textit{stabilizing} (for $f$) if
the $\ZZ$-configuration resulting from it at the very end is
stable.

We notice some obvious consequences of the definitions:

\begin{itemize}
 \item If a sequence $\tup{v_1, v_2, \ldots, v_k}$ is legal
       for a configuration $f$, then all of the
       $\ZZ$-configurations
       $f - \Delta v_1 - \Delta v_2 - \cdots - \Delta v_i$
       for $i \in \set{0, 1, \ldots, k}$ are actually
       configurations.
 \item If a sequence $\tup{v_1, v_2, \ldots, v_k}$ is legal
       for a configuration $f$, then each prefix of this
       sequence (i.e., each sequence of the form
       $\tup{v_1, v_2, \ldots, v_i}$ for some
       $i \in \set{0, 1, \ldots, k}$) is legal for $f$ as
       well.
 \item If a sequence $\tup{v_1, v_2, \ldots, v_k}$ is
       stabilizing for a configuration $f$, then each
       permutation of this sequence (i.e., each sequence of
       the form
       $\tup{v_{\sigma\tup{1}}, v_{\sigma\tup{2}}, \ldots,
             v_{\sigma\tup{k}}}$ for a permutation
       $\sigma$ of $\set{1, 2, \ldots, k}$) is stabilizing
       for $f$ as well.
 \item If $\tup{v_1, v_2, \ldots, v_k}$ is a legal sequence
       for a configuration $f$, then
       $\tup{v_1, v_2, \ldots, v_k}$ is stabilizing if and
       only if there exist no $v \in V$ such that the
       sequence $\tup{v_1, v_2, \ldots, v_k, v}$ is legal.
\end{itemize}


An important property of chipfiring is the following result
(sometimes called the ``least action principle''):

\begin{theorem} \label{thm.chip.lap}
Let $f : V \to \NN$ be any configuration.
Let $\ell$ and $s$ be two sequences of vertices of $D$ such
that $\ell$ is legal for $f$ while $s$ is stabilizing for $f$.
Then, $\ell$ is a subpermutation of $s$.
\end{theorem}

Here, we are using the following notation:

\begin{definition}
Let $\tup{p_1, p_2, \ldots, p_u}$ and
$\tup{q_1, q_2, \ldots, q_v}$ be two finite sequences.
Then, we say that $\tup{p_1, p_2, \ldots, p_u}$ is
a \textit{subpermutation} of $\tup{q_1, q_2, \ldots, q_v}$
if and only if, for each object $t$, the following holds:
The number of $i \in \set{1, 2, \ldots, u}$ satisfying
$p_i = t$ is less or equal to the number of
$j \in \set{1, 2, \ldots, v}$ satisfying $q_j = t$.

Equivalently, the sequence $\tup{p_1, p_2, \ldots, p_u}$ is
a subpermutation of the sequence $\tup{q_1, q_2, \ldots, q_v}$
if and only if you can obtain the former from the latter by
removing some entries and permuting the remaining entries.
(``Some'' allows for the possibility of ``zero''.)
\end{definition}

\begin{corollary} \label{cor.chip.lap-cor}
Let $f : V \to \NN$ be any configuration.
Let $\ell$ and $\ell'$ be two sequences of vertices of $D$
that are both legal and stabilizing for $f$.
Then:

\textbf{(a)} The sequence $\ell'$ is a permutation of $\ell$.

In particular:

\textbf{(b)} The sequences $\ell$ and $\ell'$ have the
same length.

\textbf{(c)} For each $t \in V$, the number of times $t$
appears in $\ell'$ equals the number of times $t$ appears in
$\ell$.

\textbf{(d)} The configuration obtained from $f$ by firing
all vertices in $\ell$ (one after the other) equals the
configuration obtained from $f$ by firing
all vertices in $\ell'$ (one after the other).
\end{corollary}

Next we state some facts about legal sequences:

\begin{lemma} \label{lem.chip.leg-bound1}
Let $f : V \to \NN$ be a configuration.
Let $h = \sum f$.
Let $\ell$ be a legal sequence for $f$.

Let $a$ be an arc of $D$.
Let $u$ be the source of $a$, and let $v$ be the target of $a$.

\textbf{(a)}
If $u$ appears more than $h$ times in the sequence $\ell$,
then $v$ must appear at least once in the sequence $\ell$.

\textbf{(b)}
Fix $k \in \NN$.
If $u$ appears more than $kh$ times in the sequence $\ell$,
then $v$ must appear at least $k$ times in the sequence $\ell$.
\end{lemma}

\begin{lemma} \label{lem.chip.leg-bound1b}
Let $f : V \to \NN$ be a configuration.
Let $h = \sum f$.
Let $\ell$ be a legal sequence for $f$.

Let $u$ and $v$ be two vertices of $D$ such that there exists
a path of length $d$ from $u$ to $v$.

If $u$ appears at least $\dfrac{h^{d+1}-1}{h-1}$
times in the sequence $\ell$,
then $v$ must appear at least once in the sequence $\ell$.
\end{lemma}

\begin{proposition} \label{prop.chip.leg-everyone-fires}
Let $f : V \to \NN$ be a configuration.
Let $h = \sum f$.
Let $\ell$ be a legal sequence for $f$.
Let $n = \abs{V}$.

Let $q$ be a vertex of $D$ such that for each vertex
$u \in V$, there exists a path from $u$ to $q$.

If the length of $\ell$ is
$> \tup{n-1} \tup{ \dfrac{h^n-1}{h-1} - 1 }$,
then $q$ must appear at least once in the sequence $\ell$.
\end{proposition}

\begin{proposition} \label{prop.chip.leg-period}
Let $f : V \to \NN$ be a configuration.
Let $h = \sum f$.
Let $\ell = \tup{\ell_1, \ell_2, \ldots, \ell_k}$ be a
legal sequence for $f$.
Let
$g = f - \Delta \ell_1 - \Delta \ell_2 - \cdots - \Delta \ell_k$
be the configuration obtained from $f$ by firing the vertices in
$\ell$ (one after the other).

\textbf{(a)} We have $g \in \set{0, 1, \ldots, h}^V$.
(In other words, $g \tup{v} \in \set{0, 1, \ldots, h}$
for each $v \in V$.)

\textbf{(b)} Let $n = \abs{V}$.
If the sequence $\ell$ has length
$\geq \tup{h+1}^n$, then there exist legal sequences (for $f$)
of arbitrary length.
\end{proposition}

\begin{definition}
Let $f : V \to \NN$ be a configuration.

We say that $f$ is \textit{finitary} if there
exists a sequence of vertices that is stabilizing for $f$.
Otherwise, we say that $f$ is \textit{infinitary}.
\end{definition}

\begin{theorem} \label{thm.chip.dichotomy}
Let $f : V \to \NN$ be a configuration.
Then, exactly one of the following two statements holds:

\begin{itemize}
\item \textit{Statement 1:}
      The configuration $f$ is finitary. \par
      There exists a sequence $s$ of vertices that is both
      legal and stabilizing for $f$. \par
      All such sequences are permutations of $s$.
      \par
      All legal sequences (for $f$) are subpermutations
      of $s$, and in particular are at most as long as $s$.

\item \textit{Statement 2:}
      The configuration $f$ is infinitary. \par
      There exists no stabilizing sequence for $f$. \par
      There exist legal sequences for $f$ of arbitrary length.
      More precisely, each legal sequence for $f$ can be
      extended to a longer legal sequence.
\end{itemize}
\end{theorem}

How do we actually tell whether a given configuration $f$
is finitary or infinitary?
Proposition~\ref{prop.chip.leg-period} \textbf{(b)} suggests
an algorithm:
Keep firing active vertices over and over until either no
more active vertices remain, or you have fired at least
$\tup{h+1}^n$ many times.
In the former case, $f$ is finitary (and the sequence
of vertices you have fired is a legal stabilizing sequence
for $f$).
In the latter case, $f$ is infinitary (by
Proposition~\ref{prop.chip.leg-period} \textbf{(b)},
since the sequence of vertices you have fired is a
legal sequence of length $\geq \tup{h+1}^n$).
Needless to say, this algorithm is extremely slow and
inefficient in practice, but to some extent this slowness
is unavoidable:
A finitary configuration may require a huge number of firings
before it stabilizes.
Nevertheless, there are faster algorithms in many particular
cases; this is a subject of ongoing research.

There are also criteria which, in certain cases, guarantee
that a configuration is finitary or infinitary.
For example, it is not hard to show that if $f$ is a
configuration with $\sum f > \abs{A} - \abs{V}$,
then $f$ is infinitary\footnote{The proof is easy:
Let $f$ be a configuration with
$\sum f > \abs{A} - \abs{V}$.
Thus, $\sum_{v \in V} f \tup{v} = \sum f > \abs{A} - \abs{V}
= \sum_{v \in V} \tup{\deg^+ v - 1}$.
Hence, there exists at least one $v \in V$ satisfying
$f \tup{v} > \deg^+ v - 1$.
In other words, there exists at least one $v \in V$ satisfying
$f \tup{v} \geq \deg^+ v$.
In other words, there exists at least one active vertex
$v \in V$ in $f$.
Choose such an active vertex and fire it.
The resulting configuration $f'$ satisfies
$\sum f' = \sum f > \abs{A} - \abs{V}$,
so the very same argument can be applied to $f'$ instead
of $f$, and shows that there exists at least one active
vertex $v' \in V$ in $f'$.
Choose such an active vertex and fire it.
Keep doing so.
You can thus keep firing active vertices for
as long as you wish.
That is, $f$ has legal sequences of arbitrary length.
Hence, $f$ is infinitary.}.
Conversely, the results of
Exercise~\ref{exe.chip.adj-cycles} can be viewed as
bounds on $\sum f$ that guarantee that $f$ must be
finitary.

\begin{definition} \label{def.chip.stabilization}
Let $f : V \to \NN$ be a finitary configuration.
Then, Statement 1 in Theorem~\ref{thm.chip.dichotomy} must
hold.
Therefore, there exists a sequence $s$ of vertices that is both
legal and stabilizing for $f$.
The \textit{stabilization} of $f$ means
the configuration obtained from $f$ by firing
all vertices in $s$ (one after the other).
(This does not depend on the choice of $s$, because of
Corollary~\ref{cor.chip.lap-cor} \textbf{(d)}.)

The stabilization of $f$ is denoted by $f^\circ$.
\end{definition}

Something similar holds if we forbid firing a specific
vertex:

\begin{definition}
Let $q \in V$.

Let $f : V \to \NN$ be a configuration.

Let $\tup{v_1, v_2, \ldots, v_k}$ be a sequence of vertices of
$D$.

\textbf{(a)} The sequence $\tup{v_1, v_2, \ldots, v_k}$ is said
to be \textit{$q$-legal} for $f$ if it is legal and does not
contain the vertex $q$.

\textbf{(b)} The sequence $\tup{v_1, v_2, \ldots, v_k}$ is said
to be \textit{$q$-stabilizing} for $f$ if the $\ZZ$-configuration
$f - \Delta v_1 - \Delta v_2 - \cdots - \Delta v_k$ has no
active vertices except (possibly) $q$.
\end{definition}

We can now define ``$q$-finitary'' and ``$q$-infinitary''
and obtain an analogue of Theorem~\ref{thm.chip.dichotomy}.
But the most commonly considered case is that when $q$ is
a ``global sink'' (a vertex with no outgoing arcs, and which
is reachable from any vertex),
and in this case \textbf{every} configuration is
$q$-finitary.
Let us state this as its own result:

\begin{theorem} \label{thm.chip.dichotomy-q}
Let $f : V \to \NN$ be a configuration.
Let $q \in V$.
Assume that for each vertex $u \in V$, there exists a path
$u \to q$.
Then, there exists a sequence $s$ of vertices that is both
$q$-legal and $q$-stabilizing for $f$.
All such sequences are permutations of $s$.
All $q$-legal sequences (for $f$) are subpermutations
of $s$, and in particular are at most as long as $s$.
\end{theorem}

\begin{definition}
Let $f : V \to \NN$ be a configuration.
Let $q \in V$.
Assume that for each vertex $u \in V$, there exists a path
$u \to q$.
Then, Theorem~\ref{thm.chip.dichotomy-q} shows that
there exists a sequence $s$ of vertices that is both
$q$-legal and $q$-stabilizing for $f$.
The \textit{$q$-stabilization} of $f$ means
the configuration obtained from $f$ by firing
all vertices in $s$ (one after the other).
(This does not depend on the choice of $s$, because of
the analogue of
Corollary~\ref{cor.chip.lap-cor} \textbf{(d)}
for $q$-legal and $q$-stabilizing sequences.)
\end{definition}


\subsection{Exercise \ref{exe.chip.better-bounds}: better
bounds for legal sequences}

The following exercise improves on the bound given in
Proposition~\ref{prop.chip.leg-period} \textbf{(b)} and
also on the one given in
Proposition~\ref{prop.chip.leg-everyone-fires}\footnote{To
  see that Exercise~\ref{exe.chip.better-bounds} \textbf{(b)}
  improves on the bound given in
  Proposition~\ref{prop.chip.leg-everyone-fires},
  we need to check that
  $\tup{n-1} \tup{ \dfrac{h^n-1}{h-1} - 1} + 1
  \geq \dbinom{n+h-1}{n-1}$.
  This is easy for $n \leq 1$ (in fact, the case
  $n = 0$ is impossible due to the existence of a
  $q \in V$, and the case $n = 1$ is an equality
  case).
  In the remaining case $n \geq 2$,
  the stronger inequality
  $\dfrac{h^n-1}{h-1} - 1 + 1
  \geq \dbinom{n+h-1}{n-1}$
  can be proven by a simple induction
  over $n$.}.
I don't know whether the improved bounds can be further
improved.

\begin{exercise} \label{exe.chip.better-bounds}
Fix a loopless multidigraph $D = \tup{V, A, \phi}$.
Let $f : V \to \NN$ be a configuration.
Let $h = \sum f$.
Let $n = \abs{V}$.
Assume that $n > 0$.

Let $\ell = \tup{\ell_1, \ell_2, \ldots, \ell_k}$ be a
legal sequence for $f$ having length
$k \geq \dbinom{n+h-1}{n-1}$.

Prove the following:

\textbf{(a)} There exist legal sequences (for $f$) of
arbitrary length.

\textbf{(b)} Let $q$ be a vertex of $D$ such that for each vertex
$u \in V$, there exists a path from $u$ to $q$.
Then, $q$ must appear at least once in the sequence $\ell$.

[\textbf{Hint:}
For \textbf{(a)}, apply the same pigeonhole-principle argument
as for Proposition~\ref{prop.chip.leg-period} \textbf{(b)}.]
\end{exercise}

In the solution of Exercise \ref{exe.chip.better-bounds}, the following
classical fact will turn out useful:

\begin{proposition}
\label{prop.multichoose}Let $n\in\mathbb{N}$ and $h\in\mathbb{N}$ be such that
$n>0$. The number of $n$-tuples $\left(  a_{1},a_{2},\ldots,a_{n}\right)  $ of
nonnegative integers satisfying $a_{1}+a_{2}+\cdots+a_{n}=h$ is $\dbinom
{n+h-1}{n-1}$.
\end{proposition}

\begin{proof}
[Proof of Proposition \ref{prop.multichoose}.]See, for example,
\url{https://math.stackexchange.com/questions/36250/number-of-monomials-of-certain-degree}
for a proof of this fact (in the language of monomials). Or see \cite[\S 1.2]%
{Stanle11} (search for \textquotedblleft weak composition\textquotedblright%
\ and read the first paragraph that comes up).
Or see \cite[Proposition 13.3]{Galvin} (in the case $n > 0$; but the case
$n = 0$ is trivial).
\end{proof}

\begin{proof}
[Hints to Exercise \ref{exe.chip.better-bounds}.]For each $i\in\left\{
0,1,\ldots,k\right\}  $, the $\mathbb{Z}$-configuration $f-\Delta\ell
_{1}-\Delta\ell_{2}-\cdots-\Delta\ell_{i}$ is a configuration (since the
sequence $\left(  \ell_{1},\ell_{2},\ldots,\ell_{k}\right)  $ is legal for
$f$). Moreover, each of the $k+1$ configurations $f-\Delta\ell_{1}-\Delta
\ell_{2}-\cdots-\Delta\ell_{i}$ (with $i\in\left\{  0,1,\ldots,k\right\}  $)
has exactly $h$ chips\footnote{\textit{Proof.} Let $i\in\left\{
0,1,\ldots,k\right\}  $. Then,%
\begin{align*}
\sum\underbrace{\left(  f-\Delta\ell_{1}-\Delta\ell_{2}-\cdots-\Delta\ell
_{i}\right)  }_{=f-\sum_{j=1}^{i}\Delta\ell_{j}}  &  =\sum\left(  f-\sum
_{j=1}^{i}\Delta\ell_{j}\right)  =\underbrace{\sum f}_{=h}-\sum_{j=1}%
^{i}\underbrace{\sum\left(  \Delta\ell_{j}\right)  }%
_{\substack{=0\\\text{(since }\sum\left(  \Delta v\right)  =0\\\text{for each
vertex }v\text{)}}}\\
&  =h-\underbrace{\sum_{j=1}^{i}0}_{=0}=h.
\end{align*}
In other words, the configuration $f-\Delta\ell_{1}-\Delta\ell_{2}%
-\cdots-\Delta\ell_{i}$ has exactly $h$ chips.}.

There exist precisely $\dbinom{n+h-1}{n-1}$ configurations having exactly $h$
chips\footnote{\textit{Proof.} Recall that $\mathbb{N}$ denotes the set of
nonnegative integers.
\par
Let $v_{1},v_{2},\ldots,v_{n}$ be the vertices of $D$ (listed with no
repetitions). Then, there is a bijection%
\begin{align*}
\left\{  \text{configurations on }D\right\}   &  \rightarrow\mathbb{N}^{n},\\
f  &  \mapsto\left(  f\left(  v_{1}\right)  ,f\left(  v_{2}\right)
,\ldots,f\left(  v_{n}\right)  \right)  .
\end{align*}
This bijection restricts to a bijection%
\begin{align*}
\left\{  \text{configurations on }D\text{ having exactly }h\text{
chips}\right\}   &  \rightarrow\left\{  \left(  a_{1},a_{2},\ldots
,a_{n}\right)  \in\mathbb{N}^{n}\ \mid\ a_{1}+a_{2}+\cdots+a_{n}=h\right\}
,\\
f  &  \mapsto\left(  f\left(  v_{1}\right)  ,f\left(  v_{2}\right)
,\ldots,f\left(  v_{n}\right)  \right)
\end{align*}
(since the number of chips in a configuration $f$ on $D$ is $\sum f=f\left(
v_{1}\right)  +f\left(  v_{2}\right)  +\cdots+f\left(  v_{n}\right)  $).
Hence,%
\begin{align*}
&  \left\vert \left\{  \text{configurations on }D\text{ having exactly
}h\text{ chips}\right\}  \right\vert \\
&  =\left\vert \left\{  \left(  a_{1},a_{2},\ldots,a_{n}\right)  \in
\mathbb{N}^{n}\ \mid\ a_{1}+a_{2}+\cdots+a_{n}=h\right\}  \right\vert \\
&  =\dbinom{n+h-1}{n-1}\ \ \ \ \ \ \ \ \ \ \left(  \text{by Proposition
\ref{prop.multichoose}}\right)  .
\end{align*}
In other words, there exist precisely $\dbinom{n+h-1}{n-1}$ configurations
having exactly $h$ chips.}. Since $k+1>k\geq\dbinom{n+h-1}{n-1}$, we thus
conclude (by the pigeonhole principle) that two of the $k+1$ configurations
$f-\Delta\ell_{1}-\Delta\ell_{2}-\cdots-\Delta\ell_{i}$ (with $i\in\left\{
0,1,\ldots,k\right\}  $) must be equal (because all these $k+1$ configurations
are configurations having exactly $h$ chips). In other words, there exist two
elements $i$ and $j$ of $\left\{  0,1,\ldots,k\right\}  $ satisfying $i<j$ and%
\begin{equation}
f-\Delta\ell_{1}-\Delta\ell_{2}-\cdots-\Delta\ell_{i}=f-\Delta\ell_{1}%
-\Delta\ell_{2}-\cdots-\Delta\ell_{j}. \label{sol.chip.better-bounds.ivsj}%
\end{equation}
Consider these $i$ and $j$.

The equality (\ref{sol.chip.better-bounds.ivsj}) means that the result of
firing the vertices $\ell_{1},\ell_{2},\ldots,\ell_{i}$ in the configuration
$f$ is the same as the result of firing the vertices $\ell_{1},\ell_{2}%
,\ldots,\ell_{j}$ in the configuration $f$. Thus, after having firing the
vertices $\ell_{1},\ell_{2},\ldots,\ell_{i}$ in the configuration $f$,
additionally firing the vertices $\ell_{i+1},\ell_{i+2},\ldots,\ell_{j}$ does
not change the configuration. Therefore, we can fire these vertices
$\ell_{i+1},\ell_{i+2},\ldots,\ell_{j}$ over and over again, periodically
coming back to the same configuration and never \textquotedblleft getting into
debt\textquotedblright\ (i.e., after each firing we still have a
configuration). Thus, for every $k\in\mathbb{N}$, the sequence%
\begin{equation}
\left(  \ell_{1},\ell_{2},\ldots,\ell_{i},\underbrace{\ell_{i+1},\ell
_{i+2},\ldots,\ell_{j}}_{\text{repeated }k\text{ times}}\right)
\label{sol.chip.better-bound.recurseq}%
\end{equation}
is legal for $f$. Therefore, for each $k\in\mathbb{N}$, there exists a legal
sequence (for $f$): Namely, we construct such a sequence by taking the first
$k$ entries of the legal sequence $\left(  \ell_{1},\ell_{2},\ldots,\ell
_{i},\underbrace{\ell_{i+1},\ell_{i+2},\ldots,\ell_{j}}_{\text{repeated
}k\text{ times}}\right)  $. Thus, we have shown that there exist legal
sequences (for $f$) of arbitrary length. This solves Exercise
\ref{exe.chip.better-bounds} \textbf{(a)}.

\textbf{(b)} Let $\ell^{\prime}$ be the legal sequence
(\ref{sol.chip.better-bound.recurseq}) defined for $k=\left(  n-1\right)
\left(  \dfrac{h^{n}-1}{h-1}-1\right)  +1$. Then, the length of $\ell^{\prime
}$ is%
\begin{align*}
\underbrace{i}_{\geq0}+k\underbrace{\left(  j-i\right)  }_{\substack{\geq
1\\\text{(since }i<j\text{)}}}  &  \geq0+k\cdot1=k=\left(  n-1\right)  \left(
\dfrac{h^{n}-1}{h-1}-1\right)  +1\\
&  >\left(  n-1\right)  \left(  \dfrac{h^{n}-1}{h-1}-1\right)  .
\end{align*}
Hence, Proposition \ref{prop.chip.leg-everyone-fires} (applied to
$\ell^{\prime}$ instead of $\ell$) yields that $q$ must appear at least once
in the sequence $\ell^{\prime}$. Therefore, $q$ must also appear at least once
in the sequence $\ell$ (since each entry of $\ell^{\prime}$ is an entry of
$\ell$). This solves Exercise \ref{exe.chip.better-bounds} \textbf{(b)}.
\end{proof}

\subsection{Exercise \ref{exe.chip.examples}: examples of chip-firing}

\Needspace{15cm}
\begin{exercise} \label{exe.chip.examples}
\textbf{(a)} Let $D$ be the following digraph:
\[
\xymatrix{
 u \ar[r] & v \ar[r] & q
}
\]
(i.e., the digraph $D$ with three vertices
$u, v, q$ and two arcs $uv, vq$.)

Let $k$ be a positive integer.
Consider the configuration $g_k$ on $D$ which has $k$ chips at $u$
and $0$ chips at each other vertex.

Find the $q$-stabilization of $g_k$.

\textbf{(b)} Let $D$ be the following digraph:
\[
\xymatrix{
 u \are[r] & v \ar[r] & q
}
\]
where a curve without an arrow stands for one arc in each
direction.
(Thus, formally speaking, the digraph $D$ has three vertices
$u, v, q$ and three arcs $uv, vu, vq$.)

Let $k$ be a positive integer.
Consider the configuration $g_k$ on $D$ which has $k$ chips at $u$
and $0$ chips at each other vertex.

Find the $q$-stabilization of $g_k$.

\textbf{(c)} Let $D$ be the following digraph:
\[
\xymatrix{
 & v \are[dl] \are[dr] \ar[dd] \\
 u \are[rr] \ar[dr] & & w \ar[dl] \\
 & q
}
\]
where a curve without an arrow stands for one arc in each
direction.
(Thus, formally speaking, the digraph $D$ has four vertices
$u, v, w, q$ and nine arcs
$uv, vu, vw, wv, wu, uw, uq, vq, wq$.)

Let $k \geq 2$ be an integer.
Consider the configuration $f_k$ on $D$ which has $k$ chips at each
vertex (i.e., which has $f_k \tup{v} = k$ for each
$v \in \set{u, v, w, q}$).

Find the $q$-stabilization of $f_k$.
\end{exercise}

Before we approach this exercise, let us introduce a simple notation:

\begin{definition}
\label{def.firearrow}If $k$ and $k^{\prime}$ are two configurations on $D$,
then $k\overset{\ast}{\rightarrow}k^{\prime}$ shall mean that there exists a
legal sequence $\ell$ for $k$ such that firing all vertices in $\ell$ (one
after the other) transforms $k$ into $k^{\prime}$. Thus, we have defined a
binary relation $\overset{\ast}{\rightarrow}$ on the set of all configurations
of $D$.
\end{definition}

\begin{proposition}
\label{prop.firearrow.1}\textbf{(a)} The binary relation $\overset{\ast
}{\rightarrow}$ is reflexive (i.e., each configuration $k$ satisfies
$k\overset{\ast}{\rightarrow}k$) and transitive (i.e., every three
configurations $k_{1}$, $k_{2}$ and $k_{3}$ satisfying $k_{1}\overset{\ast
}{\rightarrow}k_{2}$ and $k_{2}\overset{\ast}{\rightarrow}k_{3}$ satisfy
$k_{1}\overset{\ast}{\rightarrow}k_{3}$).

\textbf{(b)} If $c$, $k$ and $k^{\prime}$ are three configurations satisfying
$k\overset{\ast}{\rightarrow}k^{\prime}$, then $c+k\overset{\ast}{\rightarrow
}c+k^{\prime}$.
\end{proposition}

\begin{proof}
[Proof of Proposition \ref{prop.firearrow.1} (sketched).]\textbf{(a)} Each
configuration $k$ satisfies $k\overset{\ast}{\rightarrow}k$ (because the empty
sequence $\left(  {}\right)  $ is legal for $k$, and clearly firing all
vertices in the empty sequence $\left(  {}\right)  $ transforms $k$ into $k$).
Thus, the relation $\overset{\ast}{\rightarrow}$ is reflexive. It now remains
to prove that the relation $\overset{\ast}{\rightarrow}$ is transitive.

Let $k_{1}$, $k_{2}$ and $k_{3}$ be three configurations satisfying
$k_{1}\overset{\ast}{\rightarrow}k_{2}$ and $k_{2}\overset{\ast}{\rightarrow
}k_{3}$. We shall show that $k_{1}\overset{\ast}{\rightarrow}k_{3}$.

From $k_{1}\overset{\ast}{\rightarrow}k_{2}$, we conclude that there exists a
legal sequence $\ell_{1}$ for $k_{1}$ such that firing all vertices in
$\ell_{1}$ transforms $k_{1}$ into $k_{2}$. Consider this $\ell_{1}$.

From $k_{2}\overset{\ast}{\rightarrow}k_{3}$, we conclude that there exists a
legal sequence $\ell_{2}$ for $k_{2}$ such that firing all vertices in
$\ell_{2}$ transforms $k_{2}$ into $k_{3}$. Consider this $\ell_{2}$.

Write the sequences $\ell_{1}$ and $\ell_{2}$ as $\ell_{1}=\left(  v_{1}%
,v_{2},\ldots,v_{p}\right)  $ and $\ell_{2}=\left(  w_{1},w_{2},\ldots
,w_{q}\right)  $. Let $\ell$ be the sequence $\left(  v_{1},v_{2},\ldots
,v_{p},w_{1},w_{2},\ldots,w_{q}\right)  $. (This is called the concatenation
of the sequences $\ell_{1}$ and $\ell_{2}$.) Then, firing all vertices in
$\ell$ transforms $k_{1}$ into $k_{3}$ (because the first $p$ firings
transform $k_{1}$ into $k_{2}$, and from there on the remaining $q$ firings
take us to $k_{3}$). Moreover, all the $\mathbb{Z}$-configurations obtained
during this firing process are configurations\footnote{In fact, all the
$\mathbb{Z}$-configurations obtained during the first $p$ firings are
configurations (because $\left(  v_{1},v_{2},\ldots,v_{p}\right)  =\ell_{1}$
is a legal sequence for $k_{1}$), and all the $\mathbb{Z}$-configurations
obtained during the remaining $q$ firings are configurations as well (since
$\left(  w_{1},w_{2},\ldots,w_{q}\right)  =\ell_{2}$ is a legal sequence for
$k_{2}$).}. Thus, the sequence $\ell$ is legal for $k_{1}$.

We thus have found a legal sequence $\ell$ for $k_{1}$ such that firing all
vertices in $\ell$ (one after the other) transforms $k_{1}$ into $k_{3}$.
Hence, such a legal sequence $\ell$ exists. In other words, $k_{1}%
\overset{\ast}{\rightarrow}k_{3}$ (by the definition of the relation
$\overset{\ast}{\rightarrow}$).

We thus have shown that every three configurations $k_{1}$, $k_{2}$ and
$k_{3}$ satisfying $k_{1}\overset{\ast}{\rightarrow}k_{2}$ and $k_{2}%
\overset{\ast}{\rightarrow}k_{3}$ satisfy $k_{1}\overset{\ast}{\rightarrow
}k_{3}$. In other words, the relation $\overset{\ast}{\rightarrow}$ is
transitive. This proves Proposition \ref{prop.firearrow.1} \textbf{(a)}.

\textbf{(b)} Let $c$, $k$ and $k^{\prime}$ be three configurations satisfying
$k\overset{\ast}{\rightarrow}k^{\prime}$.

From $k\overset{\ast}{\rightarrow}k^{\prime}$, we conclude that there exists a
legal sequence $\ell$ for $k$ such that firing all vertices in $\ell$
transforms $k$ into $k^{\prime}$. Consider this $\ell$.

The sequence $\ell$ is legal for $k$, and thus also for $c+k$ (since $\left(
c+k\right)  \left(  v\right)  =\underbrace{c\left(  v\right)  }_{\geq
0}+k\left(  v\right)  \geq k\left(  v\right)  $ for each vertex $v$ of $D$).
Moreover, firing all vertices in $\ell$ transforms $c+k$ into $c+k^{\prime}$
(since firing all vertices in $\ell$ transforms $k$ into $k^{\prime}$). Hence,
we have found a legal sequence $\ell$ for $c+k$ such that firing all vertices
in $\ell$ (one after the other) transforms $c+k$ into $c+k^{\prime}$. Hence,
such a legal sequence $\ell$ exists. In other words, $c+k\overset{\ast
}{\rightarrow}c+k^{\prime}$ (by the definition of the relation $\overset{\ast
}{\rightarrow}$). This proves Proposition \ref{prop.firearrow.1} \textbf{(b)}.
\end{proof}

Similar facts hold for $q$-stabilization:

\begin{definition}
\label{def.firearrow-q}Let $q$ be a vertex of $D$. If $k$ and $k^{\prime}$ are
two configurations on $D$, then $k\underset{q}{\overset{\ast}{\rightarrow}%
}k^{\prime}$ shall mean that there exists a $q$-legal sequence $\ell$ for $k$
such that firing all vertices in $\ell$ (one after the other) transforms $k$
into $k^{\prime}$. Thus, we have defined a binary relation
$\underset{q}{\overset{\ast}{\rightarrow}}$ on the set of all configurations
of $D$.
\end{definition}

\begin{proposition}
\label{prop.firearrow-q.1}Let $q$ be a vertex of $D$.

\textbf{(a)} The binary relation $\underset{q}{\overset{\ast}{\rightarrow}}$
is reflexive (i.e., each configuration $k$ satisfies
$k\underset{q}{\overset{\ast}{\rightarrow}}k$) and transitive (i.e., every
three configurations $k_{1}$, $k_{2}$ and $k_{3}$ satisfying $k_{1}%
\underset{q}{\overset{\ast}{\rightarrow}}k_{2}$ and $k_{2}%
\underset{q}{\overset{\ast}{\rightarrow}}k_{3}$ satisfy $k_{1}%
\underset{q}{\overset{\ast}{\rightarrow}}k_{3}$).

\textbf{(b)} If $c$, $k$ and $k^{\prime}$ are three configurations satisfying
$k\underset{q}{\overset{\ast}{\rightarrow}}k^{\prime}$, then
$c+k\underset{q}{\overset{\ast}{\rightarrow}}c+k^{\prime}$.
\end{proposition}

\begin{proof}
[Proof of Proposition \ref{prop.firearrow-q.1} (sketched).]Analogous to the
proof of Proposition \ref{prop.firearrow.1}.
\end{proof}

\begin{proof}
[Hints to Exercise \ref{exe.chip.examples}.]\textbf{(a)} We shall write each
configuration $f$ on $D$ as the triple $\left(  f\left(  u\right)  ,f\left(
v\right)  ,f\left(  q\right)  \right)  $. Thus, $g_{k}=\left(  k,0,0\right)  $.

For each $h\in\left\{  0,1,\ldots,k\right\}  $, define a configuration $z_{h}$
by $z_{h}=\left(  k-h,0,h\right)  $.

Now, for each $h\in\left\{  0,1,\ldots,k-1\right\}  $, we have%
\[
z_{h}\underset{q}{\overset{\ast}{\rightarrow}}z_{h+1}%
\]
\footnote{\textit{Proof.} Let $h\in\left\{  0,1,\ldots,k-1\right\}  $. The
sequence $\left(  u,v\right)  $ is $q$-legal for $z_{h}$, and firing all
vertices in this sequence transforms $z_{h}$ into $z_{h+1}$ (because this
firing process looks as follows:%
\[
z_{h}=\left(  k-h,0,h\right)  \overset{\text{fire }u}{\longrightarrow}\left(
k-h-1,1,h\right)  \overset{\text{fire }v}{\longrightarrow}\left(
k-h-1,0,h+1\right)  =\left(  k-\left(  h+1\right)  ,0,h+1\right)  =z_{h+1}%
\]
). Hence, $z_{h}\underset{q}{\overset{\ast}{\rightarrow}}z_{h+1}$.}. Hence,%
\[
z_{0}\underset{q}{\overset{\ast}{\rightarrow}}z_{1}\underset{q}{\overset{\ast
}{\rightarrow}}z_{2}\underset{q}{\overset{\ast}{\rightarrow}}z_{3}%
\underset{q}{\overset{\ast}{\rightarrow}}\cdots\underset{q}{\overset{\ast
}{\rightarrow}}z_{k-1}\underset{q}{\overset{\ast}{\rightarrow}}z_{k}.
\]
Since the relation $\underset{q}{\overset{\ast}{\rightarrow}}$ is transitive,
we thus have $z_{0}\underset{q}{\overset{\ast}{\rightarrow}}z_{k}$. In other
words, $g_{k}\underset{q}{\overset{\ast}{\rightarrow}}\left(  0,0,k\right)  $
(since $z_{0}=\left(  k-0,0,0\right)  =\left(  k,0,0\right)  =g_{k}$ and
$z_{k}=\left(  k-k,0,k\right)  =\left(  0,0,k\right)  $). Since the
configuration $\left(  0,0,k\right)  $ is $q$-stable, this shows that $\left(
0,0,k\right)  $ is the $q$-stabilization of $g_{k}$.

\textbf{(b)} We shall write each configuration $f$ on $D$ as the triple
$\left(  f\left(  u\right)  ,f\left(  v\right)  ,f\left(  q\right)  \right)
$. Thus, $g_{k}=\left(  k,0,0\right)  $.

For each $h\in\left\{  0,1,\ldots,k\right\}  $, define a configuration $z_{h}$
by $z_{h}=\left(  k-h,0,h\right)  $.

Now, for each $h\in\left\{  0,1,\ldots,k-2\right\}  $, we have%
\[
z_{h}\underset{q}{\overset{\ast}{\rightarrow}}z_{h+1}%
\]
\footnote{\textit{Proof.} Let $h\in\left\{  0,1,\ldots,k-2\right\}  $. The
sequence $\left(  u,u,v\right)  $ is $q$-legal for $z_{h}$, and firing all
vertices in this sequence transforms $z_{h}$ into $z_{h+1}$ (because this
firing process looks as follows:%
\begin{align*}
&  z_{h}=\left(  k-h,0,h\right)  \overset{\text{fire }u}{\longrightarrow
}\left(  k-h-1,1,h\right)  \overset{\text{fire }u}{\longrightarrow}\left(
k-h-2,2,h\right) \\
&  \overset{\text{fire }v}{\longrightarrow}\left(  k-h-1,0,h+1\right)
=\left(  k-\left(  h+1\right)  ,0,h+1\right)  =z_{h+1}%
\end{align*}
). Hence, $z_{h}\underset{q}{\overset{\ast}{\rightarrow}}z_{h+1}$.}. Hence,%
\[
z_{0}\underset{q}{\overset{\ast}{\rightarrow}}z_{1}\underset{q}{\overset{\ast
}{\rightarrow}}z_{2}\underset{q}{\overset{\ast}{\rightarrow}}z_{3}%
\underset{q}{\overset{\ast}{\rightarrow}}\cdots\underset{q}{\overset{\ast
}{\rightarrow}}z_{k-2}\underset{q}{\overset{\ast}{\rightarrow}}z_{k-1}.
\]
\footnote{This chain of relations can consist of a single configuration (and
$0$ relation signs) when $k=1$. There is nothing wrong about this!} Since the
relation $\underset{q}{\overset{\ast}{\rightarrow}}$ is transitive and
reflexive, we thus have $z_{0}\underset{q}{\overset{\ast}{\rightarrow}}%
z_{k-1}$. In other words, $g_{k}\underset{q}{\overset{\ast}{\rightarrow}%
}\left(  1,0,k-1\right)  $ (since $z_{0}=\left(  k-0,0,0\right)  =\left(
k,0,0\right)  =g_{k}$ and $z_{k-1}=\left(  k-\left(  k-1\right)
,0,k-1\right)  =\left(  1,0,k-1\right)  $).

Combining this with $\left(  1,0,k-1\right)  \underset{q}{\overset{\ast
}{\rightarrow}}\left(  0,1,k-1\right)  $ (because firing the legal sequence
$\left(  u\right)  $ transforms the configuration $\left(  1,0,k-1\right)  $
into $\left(  0,1,k-1\right)  $), we obtain $g_{k}\underset{q}{\overset{\ast
}{\rightarrow}}\left(  0,1,k-1\right)  $. Since the configuration $\left(
0,1,k-1\right)  $ is $q$-stable, this shows that $\left(  0,1,k-1\right)  $ is
the $q$-stabilization of $g_{k}$.

\textbf{(c)} We shall write each configuration $f$ on $D$ as the $4$-tuple
$\left(  f\left(  u\right)  ,f\left(  v\right)  ,f\left(  w\right)  ,f\left(
q\right)  \right)  $. Thus, $f_{k}=\left(  k,k,k,k\right)  $.

For each $h\in\left\{  0,1,\ldots,k\right\}  $, define a configuration $z_{h}$
by $z_{h}=\left(  k-h,k-h,k-h,k+3h\right)  $.

Now, for each $h\in\left\{  0,1,\ldots,k-3\right\}  $, we have%
\[
z_{h}\underset{q}{\overset{\ast}{\rightarrow}}z_{h+1}%
\]
\footnote{\textit{Proof.} Let $h\in\left\{  0,1,\ldots,k-3\right\}  $. The
sequence $\left(  u,v,w\right)  $ is $q$-legal for $z_{h}$, and firing all
vertices in this sequence transforms $z_{h}$ into $z_{h+1}$ (because this
firing process looks as follows:%
\begin{align*}
z_{h}  &  =\left(  k-h,k-h,k-h,k+3h\right) \\
&  \overset{\text{fire }u}{\longrightarrow}\left(
k-h-3,k-h+1,k-h+1,k+3h+1\right)  \overset{\text{fire }v}{\longrightarrow
}\left(  k-h-2,k-h-2,k-h+2,k+3h+2\right) \\
&  \overset{\text{fire }w}{\longrightarrow}\left(
k-h-1,k-h-1,k-h-1,k+3h+3\right) \\
&  =\left(  k-\left(  h+1\right)  ,k-\left(  h+1\right)  ,k-\left(
h+1\right)  ,k+3\left(  h+1\right)  \right)  =z_{h+1}%
\end{align*}
). Hence, $z_{h}\underset{q}{\overset{\ast}{\rightarrow}}z_{h+1}$.}. Hence,%
\[
z_{0}\underset{q}{\overset{\ast}{\rightarrow}}z_{1}\underset{q}{\overset{\ast
}{\rightarrow}}z_{2}\underset{q}{\overset{\ast}{\rightarrow}}z_{3}%
\underset{q}{\overset{\ast}{\rightarrow}}\cdots\underset{q}{\overset{\ast
}{\rightarrow}}z_{k-3}\underset{q}{\overset{\ast}{\rightarrow}}z_{k-2}.
\]
\footnote{This chain of relations can consist of a single configuration (and
$0$ relation signs) when $k=2$. There is nothing wrong about this!} Since the
relation $\underset{q}{\overset{\ast}{\rightarrow}}$ is transitive and
reflexive, we thus have $z_{0}\underset{q}{\overset{\ast}{\rightarrow}}%
z_{k-2}$. In other words, $f_{k}\underset{q}{\overset{\ast}{\rightarrow}%
}\left(  2,2,2,4k-6\right)  $ (since $z_{0}=\left(  k-0,k-0,k-0,k+3\cdot
0\right)  =\left(  k,k,k,k\right)  =f_{k}$ and $z_{k-2}=\left(  k-\left(
k-2\right)  ,k-\left(  k-2\right)  ,k-\left(  k-2\right)  ,k+3\left(
k-2\right)  \right)  =\left(  2,2,2,4k-6\right)  $)). Since the configuration
$\left(  2,2,2,4k-6\right)  $ is $q$-stable, this shows that $\left(
2,2,2,4k-6\right)  $ is the $q$-stabilization of $f_{k}$.
\end{proof}

\subsection{Exercise \ref{exe.chip.adj-cycles}: a lower bound
on the degree of an infinitary configuration}

\subsubsection{The exercise}

We shall use the notation
$\operatorname{s}\tup{a}$ for the source of an arc
$a \in A$.
We shall also the notation
$\operatorname{t}\tup{a}$ for the target of an arc
$a \in A$.

\begin{exercise} \label{exe.chip.adj-cycles}
Assume that the multidigraph $D$ is strongly connected
and satisfies $\abs{V} > 1$.
Let $f : V \to \NN$ be an infinitary configuration.

\textbf{(a)} Prove that $D$ cannot have more than
$\sum f$ vertex-disjoint cycles.
(A set of cycles is said to be \textit{vertex-disjoint}
if no two distinct cycles in the set have a vertex in
common.)

\textbf{(b)} Prove that $D$ cannot have more than
$\sum f$ arc-disjoint cycles.
(A set of cycles is said to be \textit{arc-disjoint}
if no two distinct cycles in the set have an arc in
common.)
\end{exercise}

Exercise~\ref{exe.chip.adj-cycles} \textbf{(b)} is
\cite[Theorem 2.2]{BjoLov92}, but the proof given there
is vague and unrigorous.

\subsubsection{Solution to part \textbf{(a)}}

We shall first solve Exercise \ref{exe.chip.adj-cycles} \textbf{(a)}.
First, let us introduce a simple notation:

\begin{definition}
\label{def.chip.cycle-non-void}Let $g:V\rightarrow\mathbb{N}$ be a
configuration. Let $c$ be a cycle of $D$. We say that the cycle $c$ is
\textit{non-void} in $g$ if and only if there exists at least one vertex $v$
on $c$ satisfying $g\left(  v\right)  \geq1$. (In other words, the cycle $c$
is non-void in $g$ if and only if at least one vertex of $c$ has at least one
chip in $g$.)
\end{definition}

In the following lemma, we shall use the notation from Definition
\ref{def.firearrow}.

\begin{lemma}
\label{lem.chip.adj-cycles.cycles-preserve}Let $g:V\rightarrow\mathbb{N}$ and
$g^{\prime}:V\rightarrow\mathbb{N}$ be two configurations such that
$g\overset{\ast}{\rightarrow}g^{\prime}$. Let $c$ be a cycle of $D$ such that
$c$ is non-void in $g$. Then, $c$ is non-void in $g^{\prime}$.
\end{lemma}

\begin{proof}
[Proof of Lemma \ref{lem.chip.adj-cycles.cycles-preserve} (sketched).]We must
merely show that whenever we fire a vertex $w$ in $g$, the cycle $c$ remains
non-void. But this is easy:

We know that there exists at least one vertex $v$ on $c$ satisfying $g\left(
v\right)  \geq1$ (since $c$ is non-void in $g$). Consider this $v$. Now:

\begin{itemize}
\item If $w=v$, then the vertex that follows $v$ on the cycle $c$ gains at
least one chip when we fire $w$; thus, the cycle $c$ remains non-void in this case.

\item If $w\neq v$, then the vertex $v$ does not lose any chips when we fire
$w$; thus, the cycle $c$ remains non-void in this case as well.
\end{itemize}

This concludes the proof of Lemma \ref{lem.chip.adj-cycles.cycles-preserve}.
\end{proof}

\begin{proof}
[Hints to Exercise \ref{exe.chip.adj-cycles}.]\textbf{(a)}
Let $h=\sum f$. Thus, we must prove that $D$ cannot have more than $h$
vertex-disjoint cycles.

The configuration
$f$ is infinitary. Thus, there exist arbitrarily long legal sequences for $f$.
In particular, there exists a legal sequence $\ell=\left(  \ell_{1},\ell
_{2},\ldots,\ell_{k}\right)  $ for $f$ of length $k>\left(  n-1\right)
\left(  \dfrac{h^{n}-1}{h-1}-1\right)  $. Consider this $\ell$.

Each vertex of $D$ appears at least once in the sequence $\ell$%
\ \ \ \ \footnote{\textit{Proof.} Let $q$ be a vertex of $D$. We must show
that $q$ appears at least once in the sequence $\ell$.
\par
For each vertex $u\in V$, there exists a path from $u$ to $q$ (since $D$ is
strongly connected). Hence, Proposition \ref{prop.chip.leg-everyone-fires}
shows that $q$ appears at least once in the sequence $\ell$.}.

Let $f^{\prime}$ be the configuration obtained from $f$ by firing the vertices
in $\ell$ (one after the other). Each cycle of $D$ is non-void in $f^{\prime}%
$\ \ \ \ \footnote{\textit{Proof.} Let $c$ be a cycle of $D$. We must then
show that $c$ is non-void in $f^{\prime}$.
\par
For each $i\in\left\{  0,1,\ldots,k\right\}  $, we define $f_{i}$ to be the
configuration obtained from $f$ by firing the vertices $\ell_{1},\ell
_{2},\ldots,\ell_{i}$ (in this order). (This is indeed a configuration, since
the sequence $\ell$ is legal.) Thus, $f_{0}\overset{\ast}{\rightarrow}%
f_{1}\overset{\ast}{\rightarrow}\cdots\overset{\ast}{\rightarrow}f_{k}$.
Notice that $f_{0}=f$ and $f_{k}=f^{\prime}$ (by the definition of $f^{\prime
}$).
\par
Pick a vertex $v$ on the cycle $c$. Then, $v$ appears at least once in the
sequence $\ell$ (since each vertex of $D$ appears at least once in the
sequence $\ell$). In other words, there exists some $i\in\left\{
1,2,\ldots,k\right\}  $ satisfying $\ell_{i}=v$. Consider this $i$.
\par
The vertex $\ell_{i}$ is active in the configuration $f_{i-1}$ (because the
configuration $f_{i}$ is obtained by firing $\ell_{i}$ from the configuration
$f_{i-1}$). In other words, $f_{i-1}\left(  \ell_{i}\right)  \geq\deg
^{-}\left(  \ell_{i}\right)  \geq1$ (since $D$ is strongly connected, but
$\left\vert V\right\vert >1$). In other words, $f_{i-1}\left(  v\right)
\geq1$ (since $\ell_{i}=v$). Thus, the cycle $c$ is non-void in $f_{i-1}$. But
$f_{i-1}\overset{\ast}{\rightarrow}f_{k}$ (since $f_{0}\overset{\ast
}{\rightarrow}f_{1}\overset{\ast}{\rightarrow}\cdots\overset{\ast
}{\rightarrow}f_{k}$, and since the relation $\overset{\ast}{\rightarrow}$ is
transitive). Hence, the cycle $c$ is non-void in $f_{k}$ (by Lemma
\ref{lem.chip.adj-cycles.cycles-preserve}, applied to $g=f_{i-1}$ and
$g^{\prime}=f_{k}$). In other words, the cycle $c$ is non-void in $f^{\prime}$
(since $f_{k}=f^{\prime}$). Qed.}.

Recall that we must prove that $D$ cannot have more than $h$
vertex-disjoint cycles.

Assume the contrary. Thus, $D$ has more than $h$ vertex-disjoint cycles. In
other words, there exist $h+1$ vertex-disjoint cycles $c_{1},c_{2}%
,\ldots,c_{h+1}$. Consider these cycles $c_{1},c_{2},\ldots,c_{h+1}$. All
these cycles are non-void in $f^{\prime}$ (since each cycle of $D$ is non-void
in $f^{\prime}$), and of course are vertex-disjoint. Thus, the configuration
$f^{\prime}$ has $h+1$ vertex-disjoint non-void cycles. Thus, it has a chip on
each of these cycles. Consequently, it has at least $h+1$ chips. In other
words, $\sum f^{\prime}\geq h+1$.

But $f^{\prime}$ is obtained from $f$ by firing vertices. Hence, $\sum
f^{\prime}=\sum f$ (since firing vertices does not change the total number of
chips). Thus, $\sum f=\sum f^{\prime}\geq h+1$, which contradicts $\sum
f=h<h+1$. This contradiction proves that our assumption was wrong, qed.
\end{proof}

\subsubsection{Solution to part \textbf{(b)}}

Our next goal is to solve Exercise \ref{exe.chip.adj-cycles} \textbf{(b)}.
Our solution is going to be a rigorous version of
\cite[proof of Theorem 2.2]{BjoLov92}.\footnote{This is one of
  those cases where
  making a proof rigorous is difficult and makes the proof
  much longer.}
We first need to prepare by introducing notations and showing lemmas.

We begin by introducing some notations:

\begin{definition}
Let $f:V\rightarrow\mathbb{Z}$ and $g:V\rightarrow\mathbb{Z}$ be two
$\mathbb{Z}$-configurations on a loopless multidigraph $D=\left(
V,A,\phi\right)  $. Then, we say that $f\geq g$ if and only if each $v\in V$
satisfies $f\left(  v\right)  \geq g\left(  v\right)  $.
\end{definition}

Thus, a $\mathbb{Z}$-configuration $f:V\rightarrow\mathbb{Z}$ is a
configuration if and only if $f\geq0$ (where $0$ means the \textquotedblleft
zero configuration\textquotedblright\ $0:V\rightarrow\mathbb{Z}$).

\begin{definition}
Let $v\in V$ be a vertex of a loopless multidigraph $D=\left(  V,A,\phi
\right)  $. Then, a configuration $\delta_{v}:V\rightarrow\mathbb{N}$ is
defined by setting
\[
\delta_{v}\left(  w\right)  =%
\begin{cases}
1, & \text{if }w=v;\\
0, & \text{if }w\neq v
\end{cases}
\qquad\text{ for all }w\in V.
\]
(Roughly speaking, $\delta_{v}$ is the configuration having $1$ chip at vertex
$v$ and no further chips.)
\end{definition}

Thus, each $\mathbb{Z}$-configuration $f:V\rightarrow\mathbb{Z}$ satisfies
$f=\sum_{v\in V}f\left(  v\right)  \delta_{v}$. Notice that $\sum\delta_{v}=1$
for each $v\in V$.

\begin{definition}
\label{def.chip.adj-cycles.K-captured}Consider a loopless multidigraph
$D=\left(  V,A,\phi\right)  $.

\textbf{(a)} If $c$ is a cycle of $D$, then $\operatorname*{V}\left(
c\right)  $ shall mean the set of all vertices on $c$.

\textbf{(b)} Let $K$ be a set of cycles of $D$. Let $f:V\rightarrow\mathbb{Z}$
be a $\mathbb{Z}$-configuration on $D$. We say that $f$ is $K$%
\textit{-captured} if and only if we can choose a vertex $v_{c}\in
\operatorname*{V}\left(  c\right)  $ for each cycle $c\in K$ such that
$f\geq\sum_{c\in K}\delta_{v_{c}}$.
\end{definition}

The notion of \textquotedblleft$K$-captured\textquotedblright\ $\mathbb{Z}%
$-configurations is somewhat subtle. Intuitively speaking, a $\mathbb{Z}%
$-configuration $f$ is $K$-captured if and only if (pretending that the chips
in $f$ are distinguishable) we can assign to each cycle $c\in K$ a specific
chip in $f$ that lies on a vertex of $c$, without having to assign the same
chip to two different cycles. Notice that the vertices $v_{c}$ in Definition
\ref{def.chip.adj-cycles.K-captured} \textbf{(b)} need not be distinct.
(Intuitively speaking, this means that the chips assigned to different cycles
may lie on the same vertices -- but they must not be the same chip.)

We notice that if a $\mathbb{Z}$-configuration $f:V\rightarrow\mathbb{Z}$ is
$K$-captured for some set $K$ of cycles of $D$, then $f$ is a configuration
(since the vertices $v_{c}$ from Definition
\ref{def.chip.adj-cycles.K-captured} \textbf{(b)} satisfy $f\geq\sum_{c\in
K}\delta_{v_{c}}\geq0$). Also, we notice that a $\mathbb{Z}$-configuration
$f:V\rightarrow\mathbb{Z}$ is $\varnothing$-captured if and only if $f$ is a configuration.

\begin{lemma}
\label{lem.chip.adj-cycles.capture-size}Let $f:V\rightarrow\mathbb{Z}$ be a
configuration on a loopless multidigraph $D=\left(  V,A,\phi\right)  $. Let
$K\subseteq C$ be such that $f$ is $K$-captured. Then, $\sum f\geq\left\vert
K\right\vert $.
\end{lemma}

\begin{proof}
[Proof of Lemma \ref{lem.chip.adj-cycles.capture-size}.] We know that $f$ is
$K$-captured. In other words, we can choose a vertex $v_{c}\in
\operatorname*{V}\left(  c\right)  $ for each cycle $c\in K$ such that
$f\geq\sum_{c\in K}\delta_{v_{c}}$. Consider these vertices $v_{c}$.

Now,%
\begin{align*}
\sum f  & =\sum_{p\in V}\underbrace{f\left(  p\right)  }_{\substack{\geq
\left(  \sum_{c\in K}\delta_{v_{c}}\right)  \left(  p\right)  \\\text{(since
}f\geq\sum_{c\in K}\delta_{v_{c}}\text{)}}}\geq\sum_{p\in V}\left(  \sum_{c\in
K}\delta_{v_{c}}\right)  \left(  p\right)  =\sum_{p\in V}\sum_{c\in K}%
\delta_{v_{c}}\left(  p\right)  \\
& =\sum_{c\in K}\underbrace{\sum_{p\in V}\delta_{v_{c}}\left(  p\right)
}_{\substack{=\sum\delta_{v_{c}}=1\\\text{(since }\sum\delta_{v}=1\text{ for
each }v\in V\text{)}}}=\sum_{c\in K}1=\left\vert K\right\vert .
\end{align*}
This proves Lemma \ref{lem.chip.adj-cycles.capture-size}.
\end{proof}

The following lemma will be crucial:

\begin{lemma}
\label{lem.chip.adj-cycles.capture}Let $f:V\rightarrow\mathbb{Z}$ be a
configuration on a loopless multidigraph $D=\left(  V,A,\phi\right)  $. Let
$w\in V$ be a vertex such that $w$ is active in $f$. Let $C$ be a set of
arc-disjoint cycles of $D$. Let $C_{w}=\left\{  c\in C\ \mid\ w\in
\operatorname*{V}\left(  c\right)  \right\}  $. Let $K\subseteq C$ be such
that $f$ is $K$-captured. Then, the configuration $f-\Delta w$ is $K\cup
C_{w}$-captured.
\end{lemma}

\begin{proof}
[Proof of Lemma \ref{lem.chip.adj-cycles.capture}.] We know that $f$ is
$K$-captured. In other words, we can choose a vertex $v_{c}\in
\operatorname*{V}\left(  c\right)  $ for each cycle $c\in K$ such that
$f\geq\sum_{c\in K}\delta_{v_{c}}$. Consider these vertices $v_{c}$.

Notice that $K\cup C_{w}\subseteq C$ (since $K\subseteq C$ and $C_{w}\subseteq
C$).

\begin{noncompile}
Let $L=\left\{  c\in K\ \mid\ v_{c}\neq w\right\}  $ and $M=\left\{  c\in
K\ \mid\ v_{c}=w\right\}  $. Notice that $M\subseteq C_{w}$%
\ \ \ \ \footnote{\textit{Proof.} Let $c\in M$. Thus, $v_{c}=w$ (by the
definition of $M$) and thus $w=v_{c}\in\operatorname*{V}\left(  c\right)  $.
Hence, $c\in C_{w}$ (by the definition of $C_{w}$). Since we have proven this
for each $c\in M$, we thus conclude that $M\subseteq C_{w}$.}.
\end{noncompile}

For each $c\in K\cup C_{w}$, we define a vertex $u_{c}\in\operatorname*{V}%
\left(  c\right)  $ by%
\[
u_{c}=%
\begin{cases}
v_{c}, & \text{if }w\notin\operatorname*{V}\left(  c\right)  ;\\
\left(  \text{the vertex succeeding }w\text{ on the cycle }c\right)  , &
\text{if }w\in\operatorname*{V}\left(  c\right)
\end{cases}
.
\]
(This definition makes sense, because $v_{c}$ is a well-defined element of
$\operatorname*{V}\left(  c\right)  $ whenever $w\notin\operatorname*{V}%
\left(  c\right)  $\ \ \ \ \footnote{\textit{Proof.} Assume that
$w\notin\operatorname*{V}\left(  c\right)  $. Then, $c\notin C_{w}$ (by the
definition of $C_{w}$). Combining this with $c\in K\cup C_{w}$, we obtain
$c\in\left(  K\cup C_{w}\right)  \setminus C_{w}\subseteq K$. Hence, $v_{c}$
is a well-defined element of $\operatorname*{V}\left(  c\right)  $.}.) Notice
that%
\begin{equation}
w\neq u_{c}\ \ \ \ \ \ \ \ \ \ \text{for each }c\in K\cup C_{w}%
\label{pf.lem.chip.adj-cycles.capture.ucneq}%
\end{equation}
(because if we had $w=u_{c}$, then we would have $w=u_{c}\in\operatorname*{V}%
\left(  c\right)  $, whence the definition of $u_{c}$ would yield
$u_{c}=\left(  \text{the vertex succeeding }w\text{ on the cycle }c\right)
\neq w=u_{c}$, which is absurd).

Now, we claim that%
\begin{equation}
\left(  f-\Delta w\right)  \left(  p\right)  \geq\left(  \sum_{c\in K\cup
C_{w}}\delta_{u_{c}}\right)  \left(  p\right)  \ \ \ \ \ \ \ \ \ \ \text{for
every }p\in V.\label{pf.lem.chip.adj-cycles.capture.ineq}%
\end{equation}


[\textit{Proof of (\ref{pf.lem.chip.adj-cycles.capture.ineq}):} Let $p\in V$.
We must prove the inequality (\ref{pf.lem.chip.adj-cycles.capture.ineq}).

Notice that%
\begin{align}
\left(  \sum_{c\in K\cup C_{w}}\delta_{u_{c}}\right)  \left(  p\right)    &
=\sum_{c\in K\cup C_{w}}\underbrace{\delta_{u_{c}}\left(  p\right)  }_{=%
\begin{cases}
1, & \text{if }p=u_{c};\\
0, & \text{if }p\neq u_{c}%
\end{cases}
}=\sum_{c\in K\cup C_{w}}%
\begin{cases}
1, & \text{if }p=u_{c};\\
0, & \text{if }p\neq u_{c}%
\end{cases}
\nonumber\\
& =\left\vert \left\{  c\in K\cup C_{w}\ \mid\ p=u_{c}\right\}  \right\vert
.\label{pf.lem.chip.adj-cycles.capture.ineq.rhs}%
\end{align}


We are in one of the following two cases:

\begin{statement}
\textit{Case 1:} We have $p=w$.
\end{statement}

\begin{statement}
\textit{Case 2:} We have $p\neq w$.
\end{statement}

First, let us consider Case 1. In this case, we have $p=w$.

\begin{noncompile}
Hence, $\left(  \Delta w\right)  \left(  p\right)  =\left(  \Delta w\right)
\left(  w\right)  =\deg^+ w$ (by the definition of $\Delta w$).
\end{noncompile}

Now, (\ref{pf.lem.chip.adj-cycles.capture.ineq.rhs}) becomes
\begin{align*}
\left(  \sum_{c\in K\cup C_{w}}\delta_{u_{c}}\right)  \left(  p\right)    &
=\left\vert \left\{  c\in K\cup C_{w}\ \mid\ \underbrace{p}_{=w}%
=u_{c}\right\}  \right\vert =\left\vert \underbrace{\left\{  c\in K\cup
C_{w}\ \mid\ w=u_{c}\right\}  }_{\substack{=\varnothing\\\text{(by
(\ref{pf.lem.chip.adj-cycles.capture.ucneq}))}}}\right\vert \\
& =\left\vert \varnothing\right\vert =0.
\end{align*}
But the vertex $w$ is active in $f$. Hence, $\left(  f-\Delta w\right)
\left(  w\right)  \geq0$. Hence, $\left(  f-\Delta w\right)  \left(
\underbrace{p}_{=w}\right)  =\left(  f-\Delta w\right)  \left(  w\right)
\geq0=\left(  \sum_{c\in K\cup C_{w}}\delta_{u_{c}}\right)  \left(  p\right)
$. Thus, (\ref{pf.lem.chip.adj-cycles.capture.ineq}) is proven in Case 1.

Let us now consider Case 2. In this case, we have $p\neq w$. Hence, the
definition of $\Delta w$ yields $\left(  \Delta w\right)  \left(  p\right)
=-a_{w,p}$, where $a_{w,p}$ denotes the number of all arcs of $D$ having
source $w$ and target $p$.

Recall that $f\geq\sum_{c\in K}\delta_{v_{c}}$. Hence,%
\begin{align}
f\left(  p\right)    & \geq\left(  \sum_{c\in K}\delta_{v_{c}}\right)  \left(
p\right)  =\sum_{c\in K}\underbrace{\delta_{v_{c}}\left(  p\right)  }_{=%
\begin{cases}
1, & \text{if }p=v_{c};\\
0, & \text{if }p\neq v_{c}%
\end{cases}
}=\sum_{c\in K}%
\begin{cases}
1, & \text{if }p=v_{c};\\
0, & \text{if }p\neq v_{c}%
\end{cases}
\nonumber\\
& =\left\vert \left\{  c\in K\ \mid\ p=v_{c}\right\}  \right\vert
.\label{pf.lem.chip.adj-cycles.capture.ineq.c2.1}%
\end{align}


Let $Q=\left\{  c\in K\ \mid\ p=v_{c}\right\}  $. Thus,
(\ref{pf.lem.chip.adj-cycles.capture.ineq.c2.1}) rewrites as
\begin{equation}
f\left(  p\right)  \geq\left\vert Q\right\vert
.\label{pf.lem.chip.adj-cycles.capture.ineq.c2.1b}%
\end{equation}


On the other hand, let $R=\left\{  c\in K\cup C_{w}\ \mid\ p=u_{c}\right\}
\setminus Q$. Then, the sets $Q$ and $R$ are disjoint, and satisfy $\left\{
c\in K\cup C_{w}\ \mid\ p=u_{c}\right\}  \subseteq Q\cup R$. Hence,%
\begin{equation}
\left\vert \left\{  c\in K\cup C_{w}\ \mid\ p=u_{c}\right\}  \right\vert
\leq\left\vert Q\cup R\right\vert =\left\vert Q\right\vert +\left\vert
R\right\vert \label{pf.lem.chip.adj-cycles.capture.ineq.c2.4}%
\end{equation}
(since $Q$ and $R$ are disjoint). Moreover, $R\subseteq C$%
\ \ \ \ \footnote{\textit{Proof.} We have $R=\left\{  c\in K\cup C_{w}%
\ \mid\ p=u_{c}\right\}  \setminus Q\subseteq\left\{  c\in K\cup C_{w}%
\ \mid\ p=u_{c}\right\}  \subseteq K\cup C_{w}\subseteq C$.}. Thus, any two
cycles in $R$ are arc-disjoint\footnote{\textit{Proof.} We have $R\subseteq
C$. Hence, any two cycles in $R$ are arc-disjoint (since $C$ is a set of
arc-disjoint cycles).}.

But if $c$ is an element of $R$, then the arc with source $w$ that belongs to
the cycle $c$ is a well-defined element of $\left\{  \text{arcs of }D\text{
having source }w\text{ and target }p\right\}  $%
\ \ \ \ \footnote{\textit{Proof.} Recall that%
\begin{equation}
R=\left\{  c\in K\cup C_{w}\ \mid\ p=u_{c}\right\}  \setminus Q=\left\{  c\in
K\cup C_{w}\ \mid\ p=u_{c}\text{ and }c\notin Q\right\}
.\label{pf.lem.chip.adj-cycles.capture.ineq.c2.fn3.pf.1}%
\end{equation}
\par
Now, let $c$ be an element of $R$. Thus, $c$ is an element of $K\cup C_{w}$
satisfying $p=u_{c}$ and $c\notin Q$ (by
(\ref{pf.lem.chip.adj-cycles.capture.ineq.c2.fn3.pf.1})).
\par
Assume (for the sake of contradiction) that $w\notin\operatorname*{V}\left(
c\right)  $. Hence, $c\notin C_{w}$ (by the definition of $C_{w}$, since $c\in
K\cup C_{w}\subseteq C$). Combining this with $c\in K\cup C_{w}$, we obtain
$c\in\left(  K\cup C_{w}\right)  \setminus C_{w}\subseteq K$.
\par
But the definition of $u_{c}$ yields $u_{c}=v_{c}$ (since $w\notin%
\operatorname*{V}\left(  c\right)  $). Thus, $p=u_{c}=v_{c}$. Thus, $c\in K$
and $p=v_{c}$. Hence, $c\in Q$ (by the definition of $Q$). This contradicts
$c\notin Q$. This contradiction shows that our assumption (that $w\notin%
\operatorname*{V}\left(  c\right)  $) was false. Hence, we must have
$w\in\operatorname*{V}\left(  c\right)  $. In other words, the vertex $w$ lies
on the cycle $c$. Thus, the arc with source $w$ that belongs to the cycle $c$
is well-defined.
\par
It remains to prove that this arc is an element of $\left\{  \text{arcs of
}D\text{ having source }w\text{ and target }p\right\}  $. In other words, it
remains to prove that this arc has source $w$ and target $p$. Since it clearly
has source $w$, we thus only need to show that it has target $p$.
\par
We have $w\in\operatorname*{V}\left(  c\right)  $. The definition of $u_{c}$
thus yields $u_{c}=\left(  \text{the vertex succeeding }w\text{ on the cycle
}c\right)  $. Hence, $p=u_{c}=\left(  \text{the vertex succeeding }w\text{ on
the cycle }c\right)  $. Therefore, the arc with source $w$ that belongs to the
cycle $c$ has target $p$. This completes our proof.}. Hence, the map%
\begin{align*}
R  & \rightarrow\left\{  \text{arcs of }D\text{ having source }w\text{ and
target }p\right\}  ,\\
c  & \mapsto\left(  \text{the arc with source }w\text{ that belongs to the
cycle }c\right)
\end{align*}
is well-defined. This map is furthermore injective\footnote{\textit{Proof.}
This is because any two cycles in $R$ are arc-disjoint.}. Hence,%
\[
\left\vert R\right\vert \leq\left\vert \left\{  \text{arcs of }D\text{ having
source }w\text{ and target }p\right\}  \right\vert =a_{w,p}%
\]
(by the definition of $a_{w,p}$). Thus,
(\ref{pf.lem.chip.adj-cycles.capture.ineq.c2.4}) becomes%
\[
\left\vert \left\{  c\in K\cup C_{w}\ \mid\ p=u_{c}\right\}  \right\vert
\leq\underbrace{\left\vert Q\right\vert }_{\substack{\leq f\left(  p\right)
\\\text{(by (\ref{pf.lem.chip.adj-cycles.capture.ineq.c2.1b}))}}%
}+\underbrace{\left\vert R\right\vert }_{\leq a_{w,p}}\leq f\left(  p\right)
+a_{w,p}.
\]
Hence,%
\[
f\left(  p\right)  +a_{w,p}\geq\left\vert \left\{  c\in K\cup C_{w}%
\ \mid\ p=u_{c}\right\}  \right\vert =\left(  \sum_{c\in K\cup C_{w}}%
\delta_{u_{c}}\right)  \left(  p\right)
\]
(by (\ref{pf.lem.chip.adj-cycles.capture.ineq.rhs})). Since%
\[
\left(  f-\Delta w\right)  \left(  p\right)  =f\left(  p\right)
-\underbrace{\left(  \Delta w\right)  \left(  p\right)  }_{=-a_{w,p}}=f\left(
p\right)  +a_{w,p},
\]
this rewrites as $\left(  f-\Delta w\right)  \left(  p\right)  \geq\left(
\sum_{c\in K\cup C_{w}}\delta_{u_{c}}\right)  \left(  p\right)  $. Thus,
(\ref{pf.lem.chip.adj-cycles.capture.ineq}) is proven in Case 2.

We have now proven (\ref{pf.lem.chip.adj-cycles.capture.ineq}) in each of the
two Cases 1 and 2. Hence, (\ref{pf.lem.chip.adj-cycles.capture.ineq}) always holds.]

From (\ref{pf.lem.chip.adj-cycles.capture.ineq}), we obtain $f-\Delta
w\geq\sum_{c\in K\cup C_{w}}\delta_{u_{c}}$. 

Thus, the configuration $f-\Delta w$ is $K\cup C_{w}$-captured (since
$u_{c}\in\operatorname*{V}\left(  c\right)  $ for each $c\in K\cup C_{w}$).
This proves Lemma \ref{lem.chip.adj-cycles.capture}.
\end{proof}

\begin{corollary}
\label{cor.chip.adj-cycles.captures}Let $f:V\rightarrow\mathbb{Z}$ be a
configuration on a loopless multidigraph $D=\left(  V,A,\phi\right)  $. Let
$\left(  \ell_{1},\ell_{2},\ldots,\ell_{k}\right)  $ be a legal sequence for
$f$. Let $C$ be a set of arc-disjoint cycles of $D$. Let $C_{w}=\left\{  c\in
C\ \mid\ w\in\operatorname*{V}\left(  c\right)  \right\}  $ for each $w\in
V$.  Let $K\subseteq C$ be such that $f$ is $K$-captured. Then, the
configuration $f-\Delta\ell_{1}-\Delta\ell_{2}-\cdots-\Delta\ell_{k}$ is
$K\cup C_{\ell_{1}}\cup C_{\ell_{2}}\cup\cdots\cup C_{\ell_{k}}$-captured.
\end{corollary}

\begin{proof}
[Proof of Corollary \ref{cor.chip.adj-cycles.captures}.] We will show that for
each $i\in\left\{  0,1,\ldots,k\right\}  $,
\begin{equation}
\text{the configuration }f-\Delta\ell_{1}-\Delta\ell_{2}-\cdots-\Delta\ell
_{i}\text{ is }K\cup C_{\ell_{1}}\cup C_{\ell_{2}}\cup\cdots\cup C_{\ell_{i}%
}\text{-captured.}\label{pf.cor.chip.adj-cycles.captures.i}%
\end{equation}


[\textit{Proof of (\ref{pf.cor.chip.adj-cycles.captures.i}):} We shall prove
(\ref{pf.cor.chip.adj-cycles.captures.i}) by induction over $i$:

\textit{Induction base:} If $i=0$, then
(\ref{pf.cor.chip.adj-cycles.captures.i}) simply states that the configuration
$f$ is $K$-captured. This is true, because it was an assumption. This
completes the induction base.

\textit{Induction step:} Let $j\in\left\{  0,1,\ldots,k\right\}  $ be
positive. Assume that (\ref{pf.cor.chip.adj-cycles.captures.i}) holds for
$i=j-1$. We must show that (\ref{pf.cor.chip.adj-cycles.captures.i}) holds for
$i=j$.

We have assumed that (\ref{pf.cor.chip.adj-cycles.captures.i}) holds for
$i=j-1$. In other words,
\[
\text{the configuration }f-\Delta\ell_{1}-\Delta\ell_{2}-\cdots-\Delta
\ell_{j-1}\text{ is }K\cup C_{\ell_{1}}\cup C_{\ell_{2}}\cup\cdots\cup
C_{\ell_{j-1}}\text{-captured.}%
\]
But the vertex $\ell_{j}$ is active in the configuration $f-\Delta\ell
_{1}-\Delta\ell_{2}-\cdots-\Delta\ell_{j-1}$ (since the sequence $\left(
\ell_{1},\ell_{2},\ldots,\ell_{k}\right)  $ is legal for $f$). Hence, Lemma
\ref{lem.chip.adj-cycles.capture} (applied to $f-\Delta\ell_{1}-\Delta\ell
_{2}-\cdots-\Delta\ell_{j-1}$, $\ell_{j}$ and $K\cup C_{\ell_{1}}\cup
C_{\ell_{2}}\cup\cdots\cup C_{\ell_{j-1}}$ instead of $f$, $w$ and $K$) shows
that the configuration $\left(  f-\Delta\ell_{1}-\Delta\ell_{2}-\cdots
-\Delta\ell_{j-1}\right)  -\Delta\ell_{j}$ is $\left(  K\cup C_{\ell_{1}}\cup
C_{\ell_{2}}\cup\cdots\cup C_{\ell_{j-1}}\right)  \cup C_{\ell_{j}}%
$-captured\footnote{since $K\cup C_{\ell_{1}}\cup C_{\ell_{2}}\cup\cdots\cup
C_{\ell_{j-1}}\subseteq C$ (since both $K$ and each $C_{w}$ are subsets of
$C$)}. In other words, the configuration $f-\Delta\ell_{1}-\Delta\ell
_{2}-\cdots-\Delta\ell_{j}$ is $K\cup C_{\ell_{1}}\cup C_{\ell_{2}}\cup
\cdots\cup C_{\ell_{j}}$-captured. In other words,
(\ref{pf.cor.chip.adj-cycles.captures.i}) holds for $i=j$. This completes the
induction step. Thus, (\ref{pf.cor.chip.adj-cycles.captures.i}) is proven by induction.]

Now, applying (\ref{pf.cor.chip.adj-cycles.captures.i}) to $i=k$, we obtain
the claim of Corollary \ref{cor.chip.adj-cycles.captures}.
\end{proof}

\begin{proof}
[Solution to Exercise \ref{exe.chip.adj-cycles} \textbf{(b)} (sketched).]
Assume the contrary. Thus, $D$ has more than $\sum f$ arc-disjoint cycles. In
other words, there exists a set $C$ of arc-disjoint cycles of $D$ such that
$\left\vert C\right\vert >\sum f$. Consider this $C$.

Set $h=\sum f$. Thus, $\left\vert C\right\vert >\sum f=h$.

Let $C_{w}=\left\{  c\in C\ \mid\ w\in\operatorname*{V}\left(  c\right)
\right\}  $ for each $w\in V$.

The configuration $f$ is infinitary. Hence, there exist legal sequences (for
$f$) of arbitrary length. In particular, this shows that there exists a legal
sequence $\ell=\left(  \ell_{1},\ell_{2},\ldots,\ell_{k}\right)  $ for $f$
having length $k\geq\dbinom{n+h-1}{n-1}$. Consider such an $\ell$.

Each vertex $q$ of $D$ must appear  at least once in the sequence $\ell
$\ \ \ \ \footnote{\textit{Proof.} Let $q$ be a vertex of $D$. Then, for each
vertex $u\in V$, there exists a path from $u$ to $q$ (since the multidigraph
$D$ is strongly connected). Hence, Exercise \ref{exe.chip.better-bounds}
\textbf{(b)} shows that $q$ must appear at least once in the sequence $\ell$.
Qed.}. Thus, $C_{\ell_{1}}\cup C_{\ell_{2}}\cup\cdots\cup C_{\ell_{k}}%
=C$\ \ \ \ \footnote{\textit{Proof.} Clearly, $C_{w}\subseteq C$ for each
$w\in V$. Thus, $C_{\ell_{1}}\cup C_{\ell_{2}}\cup\cdots\cup C_{\ell_{k}%
}\subseteq C$.
\par
On the other hand, let $c\in C$. Thus, $c$ is a cycle of $D$. Hence, there
exists some $v\in\operatorname*{V}\left(  c\right)  $. Consider this $v$.
Then, $v$ must appear at least once in the sequence $\ell$ (since each vertex
$q$ of $D$ must appear at least once in the sequence $\ell$). In other words,
there exists some $i\in\left\{  1,2,\ldots,k\right\}  $ such that $v=\ell_{i}%
$. Consider this $i$. Now, from $\ell_{i}=v\in\operatorname*{V}\left(
c\right)  $, we obtain $c\in C_{\ell_{i}}$ (by the definition of $C_{\ell_{i}%
}$). Therefore, $c\in C_{\ell_{i}}\subseteq C_{\ell_{1}}\cup C_{\ell_{2}}%
\cup\cdots\cup C_{\ell_{k}}$.
\par
Now, forget that we fixed $c$. We thus have proven that $c\in C_{\ell_{1}}\cup
C_{\ell_{2}}\cup\cdots\cup C_{\ell_{k}}$ for each $c\in C$. In other words,
$C\subseteq C_{\ell_{1}}\cup C_{\ell_{2}}\cup\cdots\cup C_{\ell_{k}}$.
Combining this with $C_{\ell_{1}}\cup C_{\ell_{2}}\cup\cdots\cup C_{\ell_{k}%
}\subseteq C$, we obtain $C_{\ell_{1}}\cup C_{\ell_{2}}\cup\cdots\cup
C_{\ell_{k}}=C$.}.

But the configuration $f$ is $\varnothing$-captured (since any configuration
is $\varnothing$-captured). Hence, Corollary
\ref{cor.chip.adj-cycles.captures} (applied to $K=\varnothing$) shows that the
configuration $f-\Delta\ell_{1}-\Delta\ell_{2}-\cdots-\Delta\ell_{k}$ is
$\varnothing\cup C_{\ell_{1}}\cup C_{\ell_{2}}\cup\cdots\cup C_{\ell_{k}}%
$-captured. In other words, the configuration $f-\Delta\ell_{1}-\Delta\ell
_{2}-\cdots-\Delta\ell_{k}$ is $C$-captured (since $\varnothing\cup
C_{\ell_{1}}\cup C_{\ell_{2}}\cup\cdots\cup C_{\ell_{k}}=C_{\ell_{1}}\cup
C_{\ell_{2}}\cup\cdots\cup C_{\ell_{k}}=C$). Hence, Lemma
\ref{lem.chip.adj-cycles.capture-size} (applied to $f-\Delta\ell_{1}-\Delta\ell
_{2}-\cdots-\Delta\ell_{k}$ and $C$ instead of $f$ and $K$) yields%
\[
\sum\left(  f-\Delta\ell_{1}-\Delta\ell_{2}-\cdots-\Delta\ell_{k}\right)
\geq\left\vert C\right\vert >h.
\]
Since%
\begin{align*}
& \sum\underbrace{\left(  f-\Delta\ell_{1}-\Delta\ell_{2}-\cdots-\Delta
\ell_{k}\right)  }_{=f-\sum_{i=1}^{k}\Delta\ell_{i}}\\
& =\sum\left(  f-\sum_{i=1}^{k}\Delta\ell_{i}\right)  =\sum f-\sum_{i=1}%
^{k}\underbrace{\sum\Delta\ell_{i}}_{\substack{=0\\\text{(since }\sum\Delta
v=0\text{ for each }v\in V\text{)}}}=\sum f-\underbrace{\sum_{i=1}^{k}0}%
_{=0}=\sum f,
\end{align*}
this rewrites as $\sum f>h$. This contradicts $h=\sum f$. This contradiction
completes the solution of Exercise \ref{exe.chip.adj-cycles} \textbf{(b)}.
\end{proof}

Now that we have solved Exercise \ref{exe.chip.adj-cycles} \textbf{(b)}, we
can obtain a second solution to Exercise \ref{exe.chip.adj-cycles}
\textbf{(a)}:

\begin{proof}
[Second solution to Exercise \ref{exe.chip.adj-cycles} \textbf{(a)} (sketched).]
Assume the contrary. Thus, $D$ has more than $\sum f$ vertex-disjoint cycles.
Hence, $D$ has more than $\sum f$ arc-disjoint cycles (since vertex-disjoint
cycles are automatically arc-disjoint). But this contradicts Exercise
\ref{exe.chip.adj-cycles} \textbf{(b)}.  This contradiction completes the
solution of Exercise \ref{exe.chip.adj-cycles} \textbf{(a)}.
\end{proof}

% Here is an old, INCORRECT hint to Exercise \ref{exe.chip.adj-cycles} \textbf{(b)}:
%
% [\textbf{Hint:} Reduce part \textbf{(b)} to part
% \textbf{(a)} by constructing the following
% multidigraph $D'$:
%
% \begin{itemize}
% \item The vertices of $D'$ shall be the vertices of
      % $D$ as well as $\abs{A}$ new vertices.
      % Namely, we add a new vertex $v_a$ for each
      % arc $a \in A$ of $D$.
% \item The arcs of $D'$ are the following:
      % For each arc $a \in A$ of $D$, we build two
      % new arcs $a_1$ and $a_2$ with
      % \begin{align*}
      % \operatorname{s} \tup{a_1} = \operatorname{s} \tup{a};
      % &\qquad
      % \operatorname{t} \tup{a_1} = v_a;
      % \\
      % \operatorname{s} \tup{a_2} = v_a;
      % &\qquad
      % \operatorname{t} \tup{a_2} = \operatorname{t} \tup{a} .
      % \end{align*}
      % The arcs of $D'$ shall be these new arcs $a_1$
      % and $a_2$ for all $a \in A$ (but not the original
      % arcs of $D$).
% \end{itemize}
%
% For each configuration $f$ on $D$, construct a
% configuration $f'$ on $D'$ such that any legal
% sequence for $f$ can be translated into a (longer)
% legal sequence for $f'$.]

\subsection{Exercise \ref{exe.chip.assoc}: an associativity
law for stabilizations}

Recall Definition~\ref{def.chip.stabilization}.

\begin{exercise} \label{exe.chip.assoc}
Let $f : V \to \NN$, $g : V \to \NN$ and $h : V \to \NN$
be three configurations such that both configurations
$f$ and $g + h$ are finitary, and such that the
configuration $f + \tup{g + h}^\circ$ is also finitary.

Prove the following:

\textbf{(a)} The configurations $f + g$ and $h$ are also
finitary.

\textbf{(b)} The configurations $f + g + h$ and
$\tup{f + g}^\circ + h$ are also finitary, and
satisfy
\[
\tup{f + g + h}^\circ
= \tup{f + \tup{g + h}^\circ}^\circ
= \tup{\tup{f + g}^\circ + h}^\circ .
\]
\end{exercise}

In order to solve Exercise \ref{exe.chip.assoc}, we shall use the notations
from Definition \ref{def.firearrow}. We also state a few simple lemmas:

\begin{lemma}
\label{lem.chip.finitary-less}Let $c:V\rightarrow\mathbb{N}$ and
$d:V\rightarrow\mathbb{N}$ be two configurations such that $c+d$ is finitary.
Then, $c$ and $d$ are finitary.
\end{lemma}

\begin{proof}
[Proof of Lemma \ref{lem.chip.finitary-less} (sketched).]The configuration
$c+d$ is finitary. Thus, there exists a stabilizing sequence $s$ for $c+d$.
Consider this $s$.

Firing the vertices in $s$ (one after the other) transforms the configuration
$c+d$ into a stable $\mathbb{Z}$-configuration $z$ (since $s$ is stabilizing).
Consider this $z$.

The $\mathbb{Z}$-configuration $z$ is stable. In other words, no vertex of $V$
is active in $z$. In other words, $z\left(  v\right)  <\deg^+ v$ for each
$v\in V$. Hence,
\[
\left(  z-d\right)  \left(  v\right)  =z\left(  v\right)
-\underbrace{d\left(  v\right)  }_{\geq0}\leq z\left(  v\right)  < \deg^+ v
\]
for each $v\in V$. In other words, no vertex of $V$ is active in $z-d$. In
other words, the $\mathbb{Z}$-configuration $z-d$ is stable.

But firing the vertices in $s$ (one after the other) transforms the
configuration $c$ into the $\mathbb{Z}$-configuration $z-d$ (since firing the
vertices in $s$ (one after the other) transforms the configuration $c+d$ into
the $\mathbb{Z}$-configuration $z$). Thus, the sequence $s$ is stabilizing for
$c$ (since the $\mathbb{Z}$-configuration $z-d$ is stable). Hence, there
exists a stabilizing sequence for $c$. In other words, $c$ is finitary.
Similarly, $d$ is finitary. This proves Lemma \ref{lem.chip.finitary-less}.
\end{proof}

\begin{lemma}
\label{lem.chip.finitary-arrow}Let $c:V\rightarrow\mathbb{N}$ and
$d:V\rightarrow\mathbb{N}$ be two configurations such that $c\overset{\ast
}{\rightarrow}d$.

\textbf{(a)} If $c$ is finitary, then $d$ is finitary and satisfies $c^{\circ
}=d^{\circ}$.

\textbf{(b)} If $d$ is finitary, then $c$ is finitary and satisfies $c^{\circ
}=d^{\circ}$.
\end{lemma}

\begin{proof}
[Proof of Lemma \ref{lem.chip.finitary-arrow} (sketched).]From $c\overset{\ast
}{\rightarrow}d$, we conclude that there exists a legal sequence $\ell_{1}$
for $c$ such that firing all vertices in $\ell_{1}$ transforms $c$ into $d$.
Consider this $\ell_{1}$.

\textbf{(b)} Assume that $d$ is finitary. Hence, the stabilization $d^{\circ}$
is well-defined, and is a stable configuration (by the definition of
stabilization). Also, $d^{\circ}$ is obtained from $d$ by firing vertices from
a legal sequence (by the definition of $d^{\circ}$). Hence, $d\overset{\ast
}{\rightarrow}d^{\circ}$. From $c\overset{\ast}{\rightarrow}d\overset{\ast
}{\rightarrow}d^{\circ}$, we obtain $c\overset{\ast}{\rightarrow}d^{\circ}$
(since the relation $\overset{\ast}{\rightarrow}$ is transitive). Since
$d^{\circ}$ is stable, this shows that $c$ is finitary, and that $d^{\circ}$
is the stabilization of $c$.

Now, $c^{\circ}$ is well-defined (since $c$ is finitary) and satisfies
$d^{\circ}=c^{\circ}$ (since $d^{\circ}$ is the stabilization of $c$). In
other words, $c^{\circ}=d^{\circ}$. This proves Lemma
\ref{lem.chip.finitary-arrow} \textbf{(b)}.

\textbf{(a)} Assume that $c$ is finitary.

Let us first show that $d$ is finitary.

Indeed, assume the contrary. Thus, $d$ is infinitary. Hence, there exist
arbitrarily long legal sequences for $d$. In other words, for each
$N\in\mathbb{N}$, there exists a legal sequence for $d$ having length $\geq N$.

Fix $N\in\mathbb{N}$. Then, there exists a legal sequence for $d$ having
length $\geq N$ (as we have just seen). Fix such a legal sequence, and denote
it by $\ell_{2}$.

Write the sequences $\ell_{1}$ and $\ell_{2}$ as $\ell_{1}=\left(  v_{1}%
,v_{2},\ldots,v_{p}\right)  $ and $\ell_{2}=\left(  w_{1},w_{2},\ldots
,w_{q}\right)  $. Let $\ell$ be the sequence $\left(  v_{1},v_{2},\ldots
,v_{p},w_{1},w_{2},\ldots,w_{q}\right)  $. (This is called the concatenation
of the sequences $\ell_{1}$ and $\ell_{2}$.) Then, consider the process of
firing all vertices in $\ell$ (one after the other) in $c$. All the
$\mathbb{Z}$-configurations obtained during this firing process are
configurations\footnote{In fact, all the $\mathbb{Z}$-configurations obtained
during the first $p$ firings are configurations (because $\left(  v_{1}%
,v_{2},\ldots,v_{p}\right)  =\ell_{1}$ is a legal sequence for $c$), and all
the $\mathbb{Z}$-configurations obtained during the remaining $q$ firings are
configurations as well (since the first $p$ firings have turned the
configuration $c$ into $d$, but $\left(  w_{1},w_{2},\ldots,w_{q}\right)
=\ell_{2}$ is a legal sequence for $d$).}. Thus, the sequence $\ell$ is legal
for $c$. Moreover, the length of $\ell$ is $\underbrace{p}_{\geq0}+q\geq
q=\left(  \text{the length of }\ell_{2}\right)  \geq N$ (since $\ell_{2}$ has
length $\geq N$). Thus, there exists a legal sequence for $c$ having length
$\geq N$ (namely, $\ell$).

Now, forget that we fixed $N$. We thus have showed that for each
$N\in\mathbb{N}$, there exists a legal sequence for $c$ having length $\geq N$
(namely, $\ell$). In other words, there exist arbitrarily long legal sequences
for $c$. In other words, $c$ is infinitary. This contradicts the fact that $c$
is finitary. This contradiction shows that our assumption was false. Thus, we
have proven that $d$ is finitary.

Lemma \ref{lem.chip.finitary-arrow} \textbf{(b)} thus yields $c^{\circ
}=d^{\circ}$. This proves Lemma \ref{lem.chip.finitary-arrow} \textbf{(a)}.
\end{proof}

\begin{proof}
[Hints to Exercise \ref{exe.chip.assoc}.]For each finitary configuration $k$,
we have%
\begin{equation}
k\overset{\ast}{\rightarrow}k^{\circ} \label{sol.chip.assoc.1}%
\end{equation}
(because $k^{\circ}$ is obtained from $k$ by firing vertices from a legal
sequence). Hence, $g+h\overset{\ast}{\rightarrow}\left(  g+h\right)  ^{\circ}%
$. Thus, Proposition \ref{prop.firearrow.1} \textbf{(b)} (applied to $c=f$ and
$k=g+h$ and $k^{\prime}=\left(  g+h\right)  ^{\circ}$) yields
$f+g+h\overset{\ast}{\rightarrow}f+\left(  g+h\right)  ^{\circ}$.

But the configuration $f+\left(  g+h\right)  ^{\circ}$ is also finitary. Thus,
(\ref{sol.chip.assoc.1}) (applied to $k=f+\left(  g+h\right)  ^{\circ}$)
yields $f+\left(  g+h\right)  ^{\circ}\overset{\ast}{\rightarrow}\left(
f+\left(  g+h\right)  ^{\circ}\right)  ^{\circ}$. Thus,%
\[
f+g+h\overset{\ast}{\rightarrow}f+\left(  g+h\right)  ^{\circ}\overset{\ast
}{\rightarrow}\left(  f+\left(  g+h\right)  ^{\circ}\right)  ^{\circ}.
\]
Since the configuration $\left(  f+\left(  g+h\right)  ^{\circ}\right)
^{\circ}$ is stable, we thus conclude that $f+g+h$ is finitary, and
furthermore that the configuration $\left(  f+\left(  g+h\right)  ^{\circ
}\right)  ^{\circ}$ is the stabilization of $f+g+h$. Thus,
\begin{equation}
\left(  f+g+h\right)  ^{\circ}=\left(  f+\left(  g+h\right)  ^{\circ}\right)
^{\circ}. \label{sol.chip.assoc.4}%
\end{equation}


The configuration $f+g+h$ is finitary. Hence, Lemma
\ref{lem.chip.finitary-less} (applied to $c=f+g$ and $d=h$) shows that the
configurations $f+g$ and $h$ are also finitary. This solves Exercise
\ref{exe.chip.assoc} \textbf{(a)}.

\textbf{(b)} Applying (\ref{sol.chip.assoc.1}) to $k=f+g$, we obtain
$f+g\overset{\ast}{\rightarrow}\left(  f+g\right)  ^{\circ}$. Thus,
Proposition \ref{prop.firearrow.1} \textbf{(b)} (applied to $c=h$ and $k=f+g$
and $k^{\prime}=\left(  f+g\right)  ^{\circ}$) yields $h+f+g\overset{\ast
}{\rightarrow}h+\left(  f+g\right)  ^{\circ}$. Thus, $f+g+h=h+f+g\overset{\ast
}{\rightarrow}h+\left(  f+g\right)  ^{\circ}=\left(  f+g\right)  ^{\circ}+h$.
Hence, Lemma \ref{lem.chip.finitary-arrow} \textbf{(a)} (applied to $c=f+g+h$
and $d=\left(  f+g\right)  ^{\circ}+h$) shows that the configuration $\left(
f+g\right)  ^{\circ}+h$ is finitary and satisfies
\begin{equation}
\left(  f+g+h\right)  ^{\circ}=\left(  \left(  f+g\right)  ^{\circ}+h\right)
^{\circ} \label{sol.chip.assoc.5}%
\end{equation}
(since the configuration $f+g+h$ is finitary). Combining
(\ref{sol.chip.assoc.4}) and (\ref{sol.chip.assoc.5}), we obtain%
\[
\left(  f+g+h\right)  ^{\circ}=\left(  f+\left(  g+h\right)  ^{\circ}\right)
^{\circ}=\left(  \left(  f+g\right)  ^{\circ}+h\right)  ^{\circ}.
\]
This completes the solution of Exercise \ref{exe.chip.assoc} \textbf{(b)}.
\end{proof}

\subsection{Exercise \ref{exe.chip.Z2}: chip-firing on
the integer lattice}

Now, we shall briefly discuss chip-firing on the
integer lattice $\ZZ^2$; this is one of the most famous
cases of chip-firing, leading to some of the pretty
pictures.
For examples and illustrations, check out
\cite{Ellenb15} as well as some of the links above.

We have not defined infinite graphs in class; the
theory of infinite graphs involves some subtleties that
would take us too far.
However, for this particular exercise, we need only a
specific infinite graph, which is fairly simple.

\begin{definition}
\textbf{(a)} A \textit{locally finite multigraph} means
a triple $\tup{V, E, \phi}$, where $V$ and $E$ are sets
and $\phi : E \to \powset[2]{V}$ is a map having the
following property:
\begin{enumerate}
\item[(*)] For each $v \in V$, there exist only finitely
many $e \in E$ satisfying $v \in \phi \tup{e}$.
\end{enumerate}

Most of the concepts defined for (usual) multigraphs still
make sense for locally finite multigraphs.
In particular, the elements of $V$ are called the
vertices, and the elements of $E$ are called the
edges.
The property (*) says that each vertex is contained in
only finitely many edges; this allows us to define
the degree of a vertex.

\textbf{(b)} The \textit{integer lattice} shall mean
the locally finite multigraph defined as follows:
\begin{itemize}
\item The vertices of the integer lattice are the
      pairs $\tup{i, j}$ of two integers $i$ and $j$.
      In other words, the vertex set of the integer
      lattice is
        $\ZZ^2 = \set{ \tup{i, j} \ \mid \ i \in \ZZ
                        \text{ and } j \in \ZZ }$.
      We view these vertices as points in the
      plane, and draw the multigraph accordingly.
\item Two vertices of the integer lattice are
      adjacent if and only if they have distance $1$
      (as points in the plane).
      In other words, a vertex $\tup{i, j}$ is
      adjacent to the four vertices
      $\tup{i+1, j}, \tup{i, j+1}, \tup{i-1, j},
      \tup{i, j-1}$ and no others.
\end{itemize}

\textbf{(c)} You can guess how locally finite
multidigraphs are defined.
Each locally finite multigraph can be regarded as a
locally finite multidigraph by replacing each edge
by a pair of two arcs (directed in both possible
directions).
\end{definition}

Let us show a piece of the integer lattice, viewed as
a locally finite multigraph:
\[
\xymatrix{
& \are[d] & \are[d] & \are[d] & \\
\are[r] & \tup{-1, 1} \are[d] \are[r] & \tup{0, 1} \are[d] \are[r] & \tup{1, 1} \are[d] \are[r] & \\
\are[r] & \tup{-1, 0} \are[d] \are[r] & \tup{0, 0} \are[d] \are[r] & \tup{1, 0} \are[d] \are[r] & \\
\are[r] & \tup{-1, -1} \are[d] \are[r] & \tup{0, -1} \are[d] \are[r] & \tup{1, -1} \are[d] \are[r] & \\
& & & &
}
\]
And here is it again, viewed as a locally finite
multidigraph:
\[
\xymatrix{
& \arebi[d] & \arebi[d] & \arebi[d] & \\
\arebi[r] & \tup{-1, 1} \arebi[d] \arebi[r] & \tup{0, 1} \arebi[d] \arebi[r] & \tup{1, 1} \arebi[d] \arebi[r] & \\
\arebi[r] & \tup{-1, 0} \arebi[d] \arebi[r] & \tup{0, 0} \arebi[d] \arebi[r] & \tup{1, 0} \arebi[d] \arebi[r] & \\
\arebi[r] & \tup{-1, -1} \arebi[d] \arebi[r] & \tup{0, -1} \arebi[d] \arebi[r] & \tup{1, -1} \arebi[d] \arebi[r] & \\
& & & &
}
\]

\begin{exercise} \label{exe.chip.Z2}
Let $f$ be a configuration on the integer lattice
(where we view the integer lattice as a locally
finite multidigraph).
(The notion of a configuration and related notions
are defined in the same way as for usual, finite
multidigraphs.)

Assume that only finitely many vertices $v \in \ZZ^2$
satisfy $f \tup{v} \neq 0$.
(Thus, the total number of chips $\sum f$ is finite.)

An edge $e$ of the integer lattice is
said to be \textit{non-void} in $f$ if and only if
at least one of the endpoints of $e$ has at least one
chip in $f$.

Prove the following:

\textbf{(a)} If an edge of the integer lattice is
non-void in $f$, then this edge remains non-void after
firing any legal sequence of vertices.
(``Firing a sequence'' means firing all the vertices
in the sequence, one after the other.)

\textbf{(b)} The total number of configurations that
can be obtained from $f$ by firing a legal sequence
of vertices is finite.

\textbf{(c)} If we fire any active vertex, then the
sum $\sum_{\tup{i, j} \in \ZZ^2} f \tup{\tup{i, j}}
\cdot \tup{i + j}^2$ increases.

\textbf{(d)} The configuration $f$ is finitary (so its
stabilization is well-defined).
\end{exercise}

This exercise gives the reason why pictures such as the
ones in \cite{Ellenb15} exist (although it does not
explain their shapes and patterns).

\begin{proof}
[Hints to Exercise \ref{exe.chip.Z2}.]\textbf{(a)} Recall that we view the
integer lattice as a locally finite multidigraph. Thus, each edge $e$ of the
integer lattice is not an actual edge, but really is two arcs $\left(
u,v\right)  $ and $\left(  v,u\right)  $ (where $u$ and $v$ are its
endpoints). These two arcs form a cycle, which we denote by $c_{e}$ (we choose
arbitrarily which vertex to begin this cycle at). Clearly, $e$ is non-void in
a configuration $f$ if and only if the cycle $c_{e}$ is non-void in $f$ (where
the notion of \textquotedblleft non-void\textquotedblright\ for a cycle is
understood as in Definition \ref{def.chip.cycle-non-void}). Thus, Exercise
\ref{exe.chip.Z2} \textbf{(a)} follows from Lemma
\ref{lem.chip.adj-cycles.cycles-preserve} (at least if we extend the latter
lemma to locally finite multidigraphs).

\textbf{(b)} For each configuration $g$, we shall denote by $S_{g}$ the set
$\left\{  v\in\mathbb{Z}^{2}\ \mid\ g\left(  v\right)  \neq0\right\}  $.
Clearly, if $\sum g$ is finite, then the set $S_{g}$ is finite. Thus, in
particular, the set $S_{g}$ is finite whenever the configuration $g$ is
obtained from $f$ by repeatedly firing vertices.

Now, we make the following observation:

\begin{statement}
\textit{Observation 1:} If a configuration $g^{\prime}$ is obtained from a
configuration $g$ by firing an active vertex, then each $v\in S_{g^{\prime}}$
either belongs to $S_{g}$ or has at least one neighbor in $S_{g}$.
\end{statement}

Observation 1 is easy to check.

Using Observation 1, Exercise \ref{exe.chip.Z2} \textbf{(a)}, and
straightforward induction (over the length of the legal sequence), we can
argue the following observation:

\begin{statement}
\textit{Observation 2:} If a configuration $g^{\prime}$ is obtained from a
configuration $g$ by firing a legal sequence of vertices, then for each $v\in
S_{g^{\prime}}$, there exists a vertex $w\in S_{g}$ and a walk $p$ from $w$ to
$v$ such that each edge of $p$ is non-void in $g^{\prime}$.
\end{statement}

Now, let $h=\sum f$. Define a subset $T$ of $\mathbb{Z}^{2}$ by%
\[
T=\left\{  v\in\mathbb{Z}^{2}\ \mid\ \text{there exists some }w\in S_{f}\text{
such that }d\left(  v,w\right)  \leq4h\right\}  .
\]
We call this set $T$ the $4h$\textit{-neighborhood of }$S_{f}$. This set $T$
is finite (since $S_{f}$ is finite).

Let $f^{\prime}$ be a configuration obtained from the configuration $f$ by
firing a legal sequence of vertices. Let $v\in S_{f^{\prime}}$. Then,
Observation 2 (applied to $g=f$ and $g^{\prime}=f^{\prime}$) shows that there
exists a vertex $w\in S_{f}$ and a walk $p$ from $w$ to $v$ such that each
edge of $p$ is non-void in $f^{\prime}$. Consider these $w$ and $p$. We WLOG
assume that the walk $p$ is a path (since otherwise, we can simply remove
cycles from $p$ until $p$ becomes a path). Since $\sum f^{\prime}=\sum f=h$,
there cannot be more than $4h$ edges that are non-void in $f^{\prime}$ (since
each chip makes only $4$ edges non-void, and there can be overlap). Thus, the
path $p$ cannot have more than $4h$ edges (since each edge of $p$ is non-void
in $f^{\prime}$). In other words, the length of the path $p$ is $\leq4h$.
Thus, there exists a path of length $\leq4h$ from $w$ to $v$ (namely, the path
$p$). Hence, $d\left(  w,v\right)  \leq4h$. In other words,
$d \tup{v, w} \leq 4h$.
Hence, $v\in T$ (by the definition of $T$, because $w\in S_{f}$).

Now, forget that we fixed $v$. We thus have shown that $v\in T$ for each $v\in
S_{f^{\prime}}$. In other words, $S_{f^{\prime}}\subseteq T$. Hence,
$f^{\prime}\left(  v\right)  =0$ for each $v\in\mathbb{Z}^{2}\setminus T$.
Moreover, $f^{\prime}\left(  v\right)  \in\left\{  0,1,\ldots,h\right\}  $ for
each $v\in T$ (since $f^{\prime}\left(  v\right)  \leq\sum f^{\prime}=h$).

Now, forget that we fixed $f^{\prime}$. We thus have proven that each
configuration $f^{\prime}$ obtained from the configuration $f$ by firing a
legal sequence of vertices satisfies
\begin{align*}
f^{\prime}\left(  v\right)   &  =0\ \ \ \ \ \ \ \ \ \ \text{for each }%
v\in\mathbb{Z}^{2}\setminus T,\ \ \ \ \ \ \ \ \ \ \text{and}\\
f^{\prime}\left(  v\right)   &  \in\left\{  0,1,\ldots,h\right\}
\ \ \ \ \ \ \ \ \ \ \text{for each }v\in T.
\end{align*}
Thus, all the configurations $f^{\prime}$ obtained from the configuration $f$ by
firing a legal sequence of vertices can be regarded as maps from $T$ to
$\left\{  0,1,\ldots,h\right\}  $. Of course, there are only finitely many
such maps; thus, there are only finitely many configurations that can be
obtained from $f$ by firing a legal sequence of vertices. This solves Exercise
\ref{exe.chip.Z2} \textbf{(b)}.

\textbf{(c)} Firing a given vertex $\left(  p,q\right)  $ results in four
chips disappearing from this vertex and reappearing on its four adjacent
vertices $\left(  p+1,q\right)  ,\left(  p-1,q\right)  ,\left(  p,q+1\right)
,\left(  p,q-1\right)  $. Thus, the value of $\sum_{\left(  i,j\right)
\in\mathbb{Z}^{2}}f\left(  \left(  i,j\right)  \right)  \cdot\left(
i+j\right)  ^{2}$ increases by%
\[
\left(  \left(  p+1\right)  +q\right)  ^{2}+\left(  \left(  p-1\right)
+q\right)  ^{2}+\left(  p+\left(  q+1\right)  \right)  ^{2}+\left(  p+\left(
q-1\right)  \right)  ^{2}-4\left(  p+q\right)  ^{2}=4.
\]
In particular, it increases. This solves Exercise \ref{exe.chip.Z2}
\textbf{(c)}.

\textbf{(d)} Let $\ell=\left(  \ell_{1},\ell_{2},\ldots,\ell_{k}\right)  $ be
a legal sequence for $f$. Then, for each $i\in\left\{  0,1,\ldots,k\right\}
$, the $\mathbb{Z}$-configuration $f-\Delta\ell_{1}-\Delta\ell_{2}%
-\cdots-\Delta\ell_{i}$ is a configuration; denote this configuration by
$f_{i}$. For each $i\in\left\{  1,2,\ldots,k\right\}  $, the configuration
$f_{i}$ is obtained from $f$ by firing a legal sequence of vertices.

But each $p\in\left\{  1,2,\ldots,k\right\}  $ satisfies
\[
\sum_{\left(  i,j\right)  \in\mathbb{Z}^{2}}f_{p-1}\left(  \left(  i,j\right)
\right)  \cdot\left(  i+j\right)  ^{2}<\sum_{\left(  i,j\right)  \in
\mathbb{Z}^{2}}f_{p}\left(  \left(  i,j\right)  \right)  \cdot\left(
i+j\right)  ^{2}%
\]
(by Exercise \ref{exe.chip.Z2} \textbf{(c)}, because the configuration $f_{p}$
is obtained from $f_{p-1}$ by firing the vertex $\ell_{p}$). Thus, the
sequence%
\[
\left(  \sum_{\left(  i,j\right)  \in\mathbb{Z}^{2}}f_{p}\left(  \left(
i,j\right)  \right)  \cdot\left(  i+j\right)  ^{2}\right)  _{p\in\left\{
0,1,\ldots,k\right\}  }%
\]
is strictly increasing. Hence, its entries are distinct. Thus, the
configurations $f_{0},f_{1},\ldots,f_{k}$ are distinct.

Recall that for each $i\in\left\{  1,2,\ldots,k\right\}  $, the configuration
$f_{i}$ is obtained from $f$ by firing a legal sequence of vertices. But
Exercise \ref{exe.chip.Z2} \textbf{(b)} shows that the total number of such
configurations is finite. Hence, if $k$ is large enough, then two of the
configurations $f_{0},f_{1},\ldots,f_{k}$ are equal. This contradicts the fact
that the configurations $f_{0},f_{1},\ldots,f_{k}$ are distinct. Hence, $k$
cannot be large enough.

Now, forget that we fixed $k$. We thus have shown that if $\ell=\left(
\ell_{1},\ell_{2},\ldots,\ell_{k}\right)  $ is a legal sequence for $f$, then
$k$ cannot be arbitrarily large. In other words, $f$ does not have arbitrarily
long legal sequences. In other words, $f$ is finitary. This solves Exercise
\ref{exe.chip.Z2} \textbf{(d)}.
\end{proof}

\subsection{Exercise \ref{exe.aco.score-vector}: acyclic
orientations are determined by their score vectors}

Now, we leave the chip-firing setting.

Roughly speaking, an \textit{orientation} of
a multigraph $G$ is a way to assign to each edge of $G$
a direction (thus making it an arc).
If the resulting \textbf{digraph} has no cycles, then
this orientation will be called \textit{acyclic}.
A rigorous way to state this definition is the
following:

\begin{definition} \label{def.aco.aco}
Let $G = \tup{V, E, \psi}$ be a multigraph.

\textbf{(a)} An \textit{orientation} of $G$ is a map
$\phi : E \to V \times V$ such that each $e \in E$
has the following property:
If we write $\phi \tup{e}$ in the form
$\phi \tup{e} = \tup{u, v}$, then
$\psi \tup{e} = \set{u, v}$.

\textbf{(b)} An orientation $\phi$ of $G$ is said to
be \textit{acyclic} if and only if the multidigraph
$\tup{V, E, \phi}$ has no cycles.
\end{definition}

\begin{example}
Let $G = \tup{V, E, \psi}$ be the following multigraph:
\[
\xymatrix{
& 2 \are[dl]^a \are@/_2pc/[dl]_b \are[dr]^c \\
1 \are[rr]_d & & 3
}
\]
Then, the following four maps $\phi$ are orientations of $G$:
\begin{itemize}
\item the map sending $a$ to $\tup{1, 2}$, sending $b$ to
      $\tup{1, 2}$, sending $c$ to $\tup{3, 2}$, and
      sending $d$ to $\tup{1, 3}$;
\item the map sending $a$ to $\tup{2, 1}$, sending $b$ to
      $\tup{1, 2}$, sending $c$ to $\tup{3, 2}$, and
      sending $d$ to $\tup{3, 1}$;
\item the map sending $a$ to $\tup{1, 2}$, sending $b$ to
      $\tup{1, 2}$, sending $c$ to $\tup{2, 3}$, and
      sending $d$ to $\tup{1, 3}$;
\item the map sending $a$ to $\tup{1, 2}$, sending $b$ to
      $\tup{1, 2}$, sending $c$ to $\tup{2, 3}$, and
      sending $d$ to $\tup{3, 1}$.
\end{itemize}
Here are the multidigraphs $\tup{V, E, \phi}$ corresponding to
these four maps (in the order mentioned):
\[
\begin{tabular}{|c|c|c|c|}
\xymatrix{
& 2 \ar@{<-}[dl]^a \ar@{<-}@/_2pc/[dl]_b \ar@{<-}[dr]^c \\
1 \ar[rr]_d & & 3
} &
\xymatrix{
& 2 \ar[dl]^a \ar@{<-}@/_2pc/[dl]_b \ar@{<-}[dr]^c \\
1 \ar@{<-}[rr]_d & & 3
} &
\xymatrix{
& 2 \ar@{<-}[dl]^a \ar@{<-}@/_2pc/[dl]_b \ar[dr]^c \\
1 \ar[rr]_d & & 3
} &
\xymatrix{
& 2 \ar@{<-}[dl]^a \ar@{<-}@/_2pc/[dl]_b \ar[dr]^c \\
1 \ar@{<-}[rr]_d & & 3
}
\end{tabular}
\]
Only the first and the third of these orientations
$\phi$ are acyclic (since only the first and the third
of these multidigraphs have no cycles).
\end{example}

\begin{exercise} \label{exe.aco.score-vector}
Let $G = \tup{V, E, \psi}$ be a multigraph.

Prove the following:

\textbf{(a)} If $\phi$ is any acyclic orientation of $G$,
and if $V \neq \varnothing$,
then there exists a $v \in V$ such that no arc of the
multidigraph $\tup{V, E, \phi}$ has target $v$.

\textbf{(b)} If $\phi_1$ and $\phi_2$ are two acyclic orientations
of $G$ such that each $v \in V$ satisfies
\[
\deg^+_{\tup{V, E, \phi_1}} v = \deg^+_{\tup{V, E, \phi_2}} v ,
\]
then $\phi_1 = \phi_2$.
\end{exercise}

Our solution to Exercise \ref{exe.aco.score-vector} uses the
following fact:

\begin{proposition} \label{prop.mt3.deg-cycle-muldir}
Let $D = \tup{V, A, \phi}$ be a multidigraph with $\abs{V} > 0$.
Assume that each vertex $v \in V$ satisfies $\deg^- v > 0$. 
Then, $D$ has at least one cycle.
\end{proposition}

\begin{proof}[Proof of Proposition~\ref{prop.mt3.deg-cycle-muldir}.]
Proposition~\ref{prop.mt3.deg-cycle-muldir} is simply the
obvious generalization of
\href{http://www-users.math.umn.edu/~dgrinber/5707s17/mt1s.pdf}{Exercise 3 on
Midterm 1} to multidigraphs;
it is proven in precisely the same way.
\end{proof}

\begin{proof}
[Hints to Exercise \ref{exe.aco.score-vector}.]\textbf{(a)} Let $\phi$ be any
acyclic orientation of $G$. Thus, the multidigraph $\left(  V,E,\phi\right)  $
has no cycles.

Assume that $V\neq\varnothing$. Then, $\left\vert V\right\vert >0$.

If each vertex $v\in V$ satisfies $\deg_{\left(  V,E,\phi\right)  }^{-}v>0$,
then the multidigraph $\left(  V,E,\phi\right)  $ has at least one
cycle\footnote{This follows from
Proposition~\ref{prop.mt3.deg-cycle-muldir} (applied to
$\tup{V, E, \phi}$ and $E$ instead of $D$ and $A$).}.
Therefore, not each vertex $v\in V$ satisfies $\deg_{\left(
V,E,\phi\right)  }^{-}v>0$ (because the multidigraph $\left(  V,E,\phi\right)
$ has no cycles). In other words, there exists a vertex $v\in V$ satisfying
$\deg_{\left(  V,E,\phi\right)  }^{-}v=0$. In other words, there exists a
$v\in V$ such that no arc of the multidigraph $\left(  V,E,\phi\right)  $ has
target $v$. This solves Exercise \ref{exe.aco.score-vector} \textbf{(a)}.

\textbf{(b)} We shall solve Exercise \ref{exe.aco.score-vector} \textbf{(b)}
by induction over $\left\vert V\right\vert $.

The \textit{induction base} (the case when $\left\vert V\right\vert =0$) is
obvious (because in this case there are no edges, and thus there is only one
possible orientation).

Now, we come to the \textit{induction step}: Let $N\in\mathbb{N}$ be positive.
Assume (as the induction hypothesis) that Exercise \ref{exe.aco.score-vector}
\textbf{(b)} holds whenever $\left\vert V\right\vert =N-1$. We must now prove
that Exercise \ref{exe.aco.score-vector} \textbf{(b)} also holds whenever
$\left\vert V\right\vert =N$.

So let $G$, $V$, $E$ and $\psi$ be as in Exercise \ref{exe.aco.score-vector}
\textbf{(b)}, and assume that $\left\vert V\right\vert =N$. Let $\phi_{1}$ and
$\phi_{2}$ be two acyclic orientations of $G$ such that each $v\in V$
satisfies
\begin{equation}
\deg_{\left(  V,E,\phi_{1}\right)  }^{+}v=\deg_{\left(  V,E,\phi_{2}\right)
}^{+}v. \label{sol.aco.score-vector.b.1}%
\end{equation}
We must then prove that $\phi_{1}=\phi_{2}$.

We have $\left\vert V\right\vert =N>0$, thus $V\neq\varnothing$. Hence,
Exercise \ref{exe.aco.score-vector} \textbf{(a)} (applied to $\phi=\phi_{1}$)
shows that there exists a $v\in V$ such that no arc of the multidigraph
$\left(  V,E,\phi_{1}\right)  $ has target $v$. Fix such a $v$, and denote it
by $w$. Thus, no arc of the multidigraph $\left(  V,E,\phi_{1}\right)  $ has
target $w$. Hence, $\deg_{\left(  V,E,\phi_{1}\right)  }^{-}w=0$.

But $\phi_{1}$ is an orientation of $G$. Hence, some of the edges of $G$ that
contain $w$ become arcs with source $w$ in $\phi_{1}$, whereas the remaining
edges of $G$ that contain $w$ become arcs with target $w$ in $\phi_{1}$.
Therefore,
\begin{equation}
\deg_{\left(  V,E,\phi_{1}\right)  }^{-}w+\deg_{\left(  V,E,\phi_{1}\right)
}^{+}w=\deg_{G}w. \label{sol.aco.score-vector.b.deg+deg1}%
\end{equation}
Similarly,%
\begin{equation}
\deg_{\left(  V,E,\phi_{2}\right)  }^{-}w+\deg_{\left(  V,E,\phi_{2}\right)
}^{+}w=\deg_{G}w. \label{sol.aco.score-vector.b.deg+deg2}%
\end{equation}


Now, (\ref{sol.aco.score-vector.b.deg+deg1}) yields%
\[
\deg_{G}w=\underbrace{\deg_{\left(  V,E,\phi_{1}\right)  }^{-}w}_{=0}%
+\deg_{\left(  V,E,\phi_{1}\right)  }^{+}w=\deg_{\left(  V,E,\phi_{1}\right)
}^{+}w=\deg_{\left(  V,E,\phi_{2}\right)  }^{+}w
\]
(by (\ref{sol.aco.score-vector.b.1}), applied to $v=w$). Hence,%
\[
\deg_{\left(  V,E,\phi_{2}\right)  }^{+}w=\deg_{G}w=\deg_{\left(  V,E,\phi
_{2}\right)  }^{-}w+\deg_{\left(  V,E,\phi_{2}\right)  }^{+}%
w\ \ \ \ \ \ \ \ \ \ \left(  \text{by (\ref{sol.aco.score-vector.b.deg+deg2}%
)}\right)  .
\]
Subtracting $\deg_{\left(  V,E,\phi_{2}\right)  }^{+}w$ from both sides of
this equality, we find $0=\deg_{\left(  V,E,\phi_{2}\right)  }^{-}w$. Hence,
no arc of the multidigraph $\left(  V,E,\phi_{2}\right)  $ has target $w$.

Now, we see that
\begin{equation}
\text{each edge }e\text{ of }G\text{ satisfying }w\in\psi\left(  e\right)
\text{ satisfies }\phi_{1}\left(  e\right)  =\phi_{2}\left(  e\right)
\label{sol.aco.score-vector.b.wine}%
\end{equation}
\footnote{\textit{Proof of (\ref{sol.aco.score-vector.b.wine}):} Let $e$ be an
edge of $G$ satisfying $w\in\psi\left(  e\right)  $. Write $\psi\left(
e\right)  $ in the form $\psi\left(  e\right)  =\left\{  u,w\right\}  $ for
some $u\in V$. Then, $\phi_{1}\left(  e\right)  $ is either $\left(
u,w\right)  $ or $\left(  w,u\right)  $. But since no arc of the multidigraph
$\left(  V,E,\phi_{1}\right)  $ has target $w$, we cannot have $\phi
_{1}\left(  e\right)  =\left(  u,w\right)  $. Hence, we have $\phi_{1}\left(
e\right)  =\left(  w,u\right)  $. Similarly, $\phi_{2}\left(  e\right)
=\left(  w,u\right)  $ (since no arc of the multidigraph $\left(  V,E,\phi
_{2}\right)  $ has target $w$). Hence, $\phi_{1}\left(  e\right)  =\left(
w,u\right)  =\phi_{2}\left(  e\right)  $. This proves
(\ref{sol.aco.score-vector.b.wine}).}.

On the other hand, let $G^{\prime}=\left(  V^{\prime},E^{\prime},\psi^{\prime
}\right)  $ be the multigraph obtained from $G$ by removing the vertex $w$ and
all edges containing $w$. (Thus, $V^{\prime}=V\setminus\left\{  w\right\}  $,
$E^{\prime}=\left\{  e\in E\ \mid\ w\notin\psi\left(  e\right)  \right\}  $
and $\psi^{\prime}=\psi\mid_{E^{\prime}}$.) Notice that $\left\vert V^{\prime
}\right\vert =\left\vert V\setminus\left\{  w\right\}  \right\vert
=\underbrace{\left\vert V\right\vert }_{=N}-1=N-1$.

Let $\phi_{1}^{\prime}$ and $\phi_{2}^{\prime}$ be the two orientations of the
multigraph $G^{\prime}$ obtained by restricting $\phi_{1}$ and $\phi_{2}$.
(Thus, $\phi_{1}^{\prime}=\phi_{1}\mid_{E^{\prime}}$ and $\phi_{2}^{\prime
}=\phi_{2}\mid_{E^{\prime}}$.) Clearly, the orientation $\phi_{1}^{\prime}$ is
acyclic (since each cycle of the multidigraph $\left(  V^{\prime},E^{\prime
},\phi_{1}^{\prime}\right)  $ would be a cycle of the multidigraph $\left(
V,E,\phi_{1}\right)  $, but the latter multidigraph has no cycles since
$\phi_{1}$ is acyclic). Similarly, the orientation $\phi_{2}^{\prime}$ is acyclic.

Each $v\in V^{\prime}$ satisfies%
\[
\deg_{\left(  V^{\prime},E^{\prime},\phi_{1}^{\prime}\right)  }^{+}%
v=\deg_{\left(  V^{\prime},E^{\prime},\phi_{2}^{\prime}\right)  }^{+}v
\]
\footnote{\textit{Proof.} Let $v\in V^{\prime}$. Thus, $v\in V^{\prime
}=V\setminus\left\{  w\right\}  $, so that $v\in V$ and $v\neq w$.
\par
Now, recall that no arc of the multidigraph $\left(  V,E,\phi_{1}\right)  $
has target $w$. Hence, each arc of the multidigraph $\left(  V,E,\phi
_{1}\right)  $ having source $v$ is also an arc of the multidigraph $\left(
V^{\prime},E^{\prime},\phi_{1}^{\prime}\right)  $. Therefore, the arcs of the
multidigraph $\left(  V,E,\phi_{1}\right)  $ having source $v$ are precisely
the arcs of the multidigraph $\left(  V^{\prime},E^{\prime},\phi_{1}^{\prime
}\right)  $ having source $v$. Therefore, $\deg_{\left(  V^{\prime},E^{\prime
},\phi_{1}^{\prime}\right)  }^{+}v=\deg_{\left(  V,E,\phi_{1}\right)  }^{+}v$.
Similarly, $\deg_{\left(  V^{\prime},E^{\prime},\phi_{2}^{\prime}\right)
}^{+}v=\deg_{\left(  V,E,\phi_{2}\right)  }^{+}v$. Now,%
\begin{align*}
\deg_{\left(  V^{\prime},E^{\prime},\phi_{1}^{\prime}\right)  }^{+}v  &
=\deg_{\left(  V,E,\phi_{1}\right)  }^{+}v=\deg_{\left(  V,E,\phi_{2}\right)
}^{+}v\ \ \ \ \ \ \ \ \ \ \left(  \text{by (\ref{sol.aco.score-vector.b.1}%
)}\right) \\
&  =\deg_{\left(  V^{\prime},E^{\prime},\phi_{2}^{\prime}\right)  }^{+}v,
\end{align*}
qed.}. Hence, the induction hypothesis (applied to $G^{\prime}$, $V^{\prime}$,
$E^{\prime}$, $\psi^{\prime}$, $\phi_{1}^{\prime}$ and $\phi_{2}^{\prime}$
instead of $G$, $V$, $E$, $\psi$, $\phi_{1}$ and $\phi_{2}$) shows that
$\phi_{1}^{\prime}=\phi_{2}^{\prime}$ (since $\left\vert V^{\prime}\right\vert
=N-1$).

Thus,%
\begin{equation}
\text{each edge }e\text{ of }G\text{ satisfying }w\notin\psi\left(  e\right)
\text{ satisfies }\phi_{1}\left(  e\right)  =\phi_{2}\left(  e\right)
\label{sol.aco.score-vector.b.wnine}%
\end{equation}
\footnote{\textit{Proof of (\ref{sol.aco.score-vector.b.wnine}):} Let $e$ be
an edge of $G$ satisfying $w\notin\psi\left(  e\right)  $. Then, $e$ is an
edge of the multigraph $G^{\prime}$ (by the definition of $G^{\prime}$).
Hence, $\phi_{1}^{\prime}\left(  e\right)  $ and $\phi_{2}^{\prime}\left(
e\right)  $ are well-defined. Moreover, $\phi_{1}^{\prime}$ is a restriction
of $\phi_{1}$; hence, $\phi_{1}^{\prime}\left(  e\right)  =\phi_{1}\left(
e\right)  $. Similarly, $\phi_{2}^{\prime}\left(  e\right)  =\phi_{2}\left(
e\right)  $. Now, $\phi_{1}\left(  e\right)  =\underbrace{\phi_{1}^{\prime}%
}_{=\phi_{2}^{\prime}}\left(  e\right)  =\phi_{2}^{\prime}\left(  e\right)
=\phi_{2}\left(  e\right)  $. This proves (\ref{sol.aco.score-vector.b.wnine}%
).}. Combining this with (\ref{sol.aco.score-vector.b.wine}), we conclude that
each edge $e$ of $G$ satisfies $\phi_{1}\left(  e\right)  =\phi_{2}\left(
e\right)  $. In other words, we have $\phi_{1}=\phi_{2}$.

Thus, we have proven that Exercise \ref{exe.aco.score-vector} \textbf{(b)}
also holds whenever $\left\vert V\right\vert =N$. This completes the induction
step. Hence, Exercise \ref{exe.aco.score-vector} \textbf{(b)} is solved.
\end{proof}

\subsection{Exercise \ref{exe.flows-cuts.cut-lattice}: the
lattice structure on minimum cuts}

Let us recall some terminology from
\cite{lec16}:

\begin{itemize}
\item A \textit{network} consists of:
      \begin{itemize}
      \item a digraph $\tup{V, A}$;
      \item two distinct vertices $s \in V$ and $t \in V$,
            called the \textit{source} and the \textit{sink},
            respectively (although we do not require $s$ to
            have indegree $0$ or $t$ to have outdegree $0$);
      \item a function $c : A \to \QQ_+$, called the
            \textit{capacity function}.
            (Here, $\QQ_+$ means the set
            $\set{ x \in \QQ \mid x \geq 0 }$.)
      \end{itemize}

\item Given a network consisting of a digraph
      $\tup{V, A}$, a source $s \in V$ and a sink $t \in V$,
      and a capacity function $c : A \to \QQ_+$, we define
      the following notations:
      \begin{itemize}
      \item For any subset $S$ of $V$, we let $\overline{S}$
            denote the subset $V \setminus S$ of $V$.
      \item If $P$ and $Q$ are two subsets of $V$, then
            $\ive{P, Q}$ shall mean the set of all arcs
            $a \in A$ whose source belongs to $P$ and whose
            target belongs to $Q$.
            (In other words,
            $\ive{P, Q} = A \cap \tup{P \times Q}$.)
      \item If $P$ and $Q$ are two subsets of $V$, then the
            number $c \tup{P, Q} \in \QQ_+$ is defined by
            \[
            c \tup{P, Q} = \sum_{a \in \ive{P, Q}} c \tup{a} .
            \]
      \end{itemize}
\end{itemize}

We also refer to lecture 16 \cite{lec16} for the definition of a flow
(which is not necessary for the following problem, but may
be helpful).

\begin{exercise} \label{exe.flows-cuts.cut-lattice}
Consider a network consisting of a digraph
$\tup{V, A}$, a source $s \in V$ and a sink $t \in V$,
and a capacity function $c : A \to \QQ_+$
such that $s \neq t$.

An \textit{$s$-$t$-cutting subset} shall mean a subset
$S$ of $V$ satisfying $s \in S$ and $t \notin S$.

Let $m$ denote the minimum possible value of
$c \tup{S, \overline{S}}$ where $S$ ranges over the
$s$-$t$-cutting subsets.
(Recall that this is the maximum value of a
flow, according to the
maximum-flow-minimum-cut theorem.)

An $s$-$t$-cutting subset $S$ is said to be
\textit{cut-minimal} if it satisfies
$c \tup{S, \overline{S}} = m$.

Let $X$ and $Y$ be two cut-minimal $s$-$t$-cutting subsets.
Prove that $X \cap Y$ and $X \cup Y$ also are
cut-minimal $s$-$t$-cutting subsets.
\end{exercise}

Exercise~\ref{exe.flows-cuts.cut-lattice} is not new;
it appears, e.g., in \cite[Corollary 3]{PicQue80}\footnote{To
  be fully precise, the version in
  \cite[Corollary 3]{PicQue80} differs by having $c$ be
  a map $A \to \RR_{> 0}$ (where $\RR_{> 0}$ is the set of
  all positive reals) instead of having $c$ be a map
  $A \to \QQ_+$. But the difference is unsubstantial,
  and roughly the same proofs work for both results.}.

In our solution of Exercise \ref{exe.flows-cuts.cut-lattice}, we shall use
some further material from
\cite{lec16}. Namely, we shall use the concept of flows, and the following fact:

\begin{lemma}
\label{lem.flows-cuts.flow-estimate}Consider a network consisting of a digraph
$\left(  V,A\right)  $, a source $s\in V$ and a sink $t\in V$, and a capacity
function $c:A\rightarrow\mathbb{Q}_{+}$ such that $s\neq t$. Let $f$ be a flow
on this network.

Let $S$ be a subset of $V$ satisfying $s\in S$ and $t\notin S$.

\textbf{(a)} We have $\left\vert f\right\vert \leq c\left(  S,\overline
{S}\right)  $.

\textbf{(b)} We have $\left\vert f\right\vert =c\left(  S,\overline{S}\right)
$ if and only if
\begin{equation}
\left(  \text{each }a\in\left[  \overline{S},S\right]  \text{ satisfies
}f\left(  a\right)  =0\right)  \label{eq.lem.flows-cuts.flow-estimate.b.1}
\end{equation}
and
\begin{equation}
\left(  \text{each }a\in\left[  S,\overline{S}\right]  \text{ satisfies
}f\left(  a\right)  =c\left(  a\right)  \right)
.\label{eq.lem.flows-cuts.flow-estimate.b.2}%
\end{equation}

\end{lemma}

\begin{proof}
[Proof of Lemma \ref{lem.flows-cuts.flow-estimate} (sketched).]\textbf{(a)}
This is precisely \cite[Proposition 1.7 \textbf{(c)}]{lec16}.

\textbf{(b)} This is precisely \cite[Proposition 1.7 \textbf{(d)}]{lec16}.
\end{proof}

\begin{proof}
[Hints to Exercise \ref{exe.flows-cuts.cut-lattice}.]From the
max-flow-min-cut-theorem (specifically, \cite[Theorem 1.10]{lec16}), we know
that%
\begin{align*}
&  \max\left\{  \left\vert f\right\vert \ \mid\ f\text{ is a flow}\right\} \\
&  =\min\left\{  c\left(  S,\overline{S}\right)  \ \mid
\ \underbrace{S\subseteq V\text{; }s\in S\text{; }t\notin S}%
_{\Longleftrightarrow\ \left(  S\text{ is an }s\text{-}t\text{-cutting subset
of }V\right)  }\right\} \\
&  =\min\left\{  c\left(  S,\overline{S}\right)  \ \mid\ S\text{ is an
}s\text{-}t\text{-cutting subset of }V\right\} \\
&  =m\ \ \ \ \ \ \ \ \ \ \left(  \text{by the definition of }m\right)  .
\end{align*}
Hence, there exists a flow $f$ with $\left\vert f\right\vert =m$. Consider
such an $f$.

The subset $X$ is $s$-$t$-cutting; thus, $X$ is a subset of $V$ satisfying
$s\in X$ and $t\notin X$.

The subset $Y$ is $s$-$t$-cutting; thus, $Y$ is a subset of $V$ satisfying
$s\in Y$ and $t\notin Y$.

We have $c\left(  X,\overline{X}\right)  =m$ (since $X$ is cut-minimal) and
$c\left(  Y,\overline{Y}\right)  =m$ (since $Y$ is cut-minimal). Thus,
$\left\vert f\right\vert =m=c\left(  X,\overline{X}\right)  $ and $\left\vert
f\right\vert =m=c\left(  Y,\overline{Y}\right)  $.

We have $s\in X\cap Y$ (since $s\in X$ and $s\in Y$) and $s\in X\cup Y$
(likewise). We have $t\notin X\cup Y$ (since $t\notin X$ and $t\notin Y$) and
$t\notin X\cap Y$ (likewise). The subset $X\cap Y$ of $V$ is $s$-$t$-cutting
(since $s\in X\cap Y$ and $t\notin X\cap Y$). Similarly, the subset $X\cup Y$
of $V$ is $s$-$t$-cutting.

Lemma \ref{lem.flows-cuts.flow-estimate} \textbf{(b)} (applied to $S=X$) shows
that $\left\vert f\right\vert =c\left(  X,\overline{X}\right)  $ if and only
if
\begin{equation}
\left(  \text{each }a\in\left[  \overline{X},X\right]  \text{ satisfies
}f\left(  a\right)  =0\right)  \label{sol.flows-cuts.cut-lattice.X1}%
\end{equation}
and%
\begin{equation}
\left(  \text{each }a\in\left[  X,\overline{X}\right]  \text{ satisfies
}f\left(  a\right)  =c\left(  a\right)  \right)
.\label{sol.flows-cuts.cut-lattice.X2}%
\end{equation}
Therefore, (\ref{sol.flows-cuts.cut-lattice.X1}) and
(\ref{sol.flows-cuts.cut-lattice.X2}) hold (since $\left\vert f\right\vert
=c\left(  X,\overline{X}\right)  $).

Lemma \ref{lem.flows-cuts.flow-estimate} \textbf{(b)} (applied to $S=Y$) shows
that $\left\vert f\right\vert =c\left(  Y,\overline{Y}\right)  $ if and only
if
\begin{equation}
\left(  \text{each }a\in\left[  \overline{Y},Y\right]  \text{ satisfies
}f\left(  a\right)  =0\right)  \label{sol.flows-cuts.cut-lattice.Y1}%
\end{equation}
and%
\begin{equation}
\left(  \text{each }a\in\left[  Y,\overline{Y}\right]  \text{ satisfies
}f\left(  a\right)  =c\left(  a\right)  \right)
.\label{sol.flows-cuts.cut-lattice.Y2}%
\end{equation}
Therefore, (\ref{sol.flows-cuts.cut-lattice.Y1}) and
(\ref{sol.flows-cuts.cut-lattice.Y2}) hold (since $\left\vert f\right\vert
=c\left(  Y,\overline{Y}\right)  $).

Now,
\begin{equation}
\left(  \text{each }a\in\left[  \overline{X\cap Y},X\cap Y\right]  \text{
satisfies }f\left(  a\right)  =0\right)
\label{sol.flows-cuts.cut-lattice.cap1}%
\end{equation}
\footnote{\textit{Proof of (\ref{sol.flows-cuts.cut-lattice.cap1}):} Let
$a\in\left[  \overline{X\cap Y},X\cap Y\right]  $. We must prove that
$f\left(  a\right)  =0$.
\par
We have $a\in\left[  \overline{X\cap Y},X\cap Y\right]  $. Thus, $a=\left(
u,v\right)  $ for some $u\notin X\cap Y$ and $v\in X\cap Y$. Consider these
$u$ and $v$.
\par
We have $v\in X\cap Y\subseteq Y$. Hence, if $u\notin Y$, then $\left(
u,v\right)  \in\left[  \overline{Y},Y\right]  $. Thus, if $u\notin Y$, then
$a=\left(  u,v\right)  \in\left[  \overline{Y},Y\right]  $. Therefore, if
$u\notin Y$, then (\ref{sol.flows-cuts.cut-lattice.Y1}) shows that $f\left(
a\right)  =0$. Thus, for the rest of this proof, we WLOG assume that $u\notin
Y$ does not hold. Hence, $u\in Y$. Now, $u\in Y$ but $u\notin X\cap Y$. Hence,
$u\in Y\setminus\left(  X\cap Y\right)  =Y\setminus X$, so that $u\notin X$.
But $v\in X\cap Y\subseteq X$. Hence, $\left(  u,v\right)  \in\left[
\overline{X},X\right]  $, so that $a=\left(  u,v\right)  \in\left[
\overline{X},X\right]  $. Thus, (\ref{sol.flows-cuts.cut-lattice.X1}) shows
that $f\left(  a\right)  =0$. This proves
(\ref{sol.flows-cuts.cut-lattice.cap1}).} and%
\begin{equation}
\left(  \text{each }a\in\left[  X\cap Y,\overline{X\cap Y}\right]  \text{
satisfies }f\left(  a\right)  =c\left(  a\right)  \right)
\label{sol.flows-cuts.cut-lattice.cap2}%
\end{equation}
\footnote{\textit{Proof of (\ref{sol.flows-cuts.cut-lattice.cap2}):} Let
$a\in\left[  X\cap Y,\overline{X\cap Y}\right]  $. We must prove that
$f\left(  a\right)  =c\left(  a\right)  $.
\par
We have $a\in\left[  X\cap Y,\overline{X\cap Y}\right]  $. Thus, $a=\left(
u,v\right)  $ for some $u\in X\cap Y$ and $v\notin X\cap Y$. Consider these
$u$ and $v$.
\par
We have $u\in X\cap Y\subseteq Y$. Hence, if $v\notin Y$, then $\left(
u,v\right)  \in\left[  Y,\overline{Y}\right]  $. Thus, if $v\notin Y$, then
$a=\left(  u,v\right)  \in\left[  Y,\overline{Y}\right]  $. Therefore, if
$v\notin Y$, then $f\left(  a\right)  =c\left(  a\right)  $ follows
immediately from (\ref{sol.flows-cuts.cut-lattice.Y2}). Thus, for the rest of
this proof, we WLOG assume that $v\notin Y$ does not hold. Hence, $v\in Y$.
Now, $v\in Y$ but $v\notin X\cap Y$. Hence, $v\in Y\setminus\left(  X\cap
Y\right)  =Y\setminus X$, so that $v\notin X$. But $u\in X\cap Y\subseteq X$.
Hence, $\left(  u,v\right)  \in\left[  X,\overline{X}\right]  $. Thus,
$a=\left(  u,v\right)  \in\left[  X,\overline{X}\right]  $. Hence,
(\ref{sol.flows-cuts.cut-lattice.X2}) shows that $f\left(  a\right)  =c\left(
a\right)  $. This proves (\ref{sol.flows-cuts.cut-lattice.cap2}).}.

But Lemma \ref{lem.flows-cuts.flow-estimate} \textbf{(b)} (applied to $S=X\cap
Y$) shows that $\left\vert f\right\vert =c\left(  X\cap Y,\overline{X\cap
Y}\right)  $ if and only if (\ref{sol.flows-cuts.cut-lattice.cap1}) and
(\ref{sol.flows-cuts.cut-lattice.cap2}) hold. Hence, $\left\vert f\right\vert
=c\left(  X\cap Y,\overline{X\cap Y}\right)  $ (since
(\ref{sol.flows-cuts.cut-lattice.cap1}) and
(\ref{sol.flows-cuts.cut-lattice.cap2}) hold). Thus, $c\left(  X\cap
Y,\overline{X\cap Y}\right)  =\left\vert f\right\vert =m$. In other words, the
$s$-$t$-cutting subset $X\cap Y$ of $V$ is cut-minimal.

Furthermore,
\begin{equation}
\left(  \text{each }a\in\left[  \overline{X\cup Y},X\cup Y\right]  \text{
satisfies }f\left(  a\right)  =0\right)
\label{sol.flows-cuts.cut-lattice.cup1}%
\end{equation}
\footnote{\textit{Proof of (\ref{sol.flows-cuts.cut-lattice.cup1}):} Let
$a\in\left[  \overline{X\cup Y},X\cup Y\right]  $. We must prove that
$f\left(  a\right)  =0$.
\par
We have $a\in\left[  \overline{X\cup Y},X\cup Y\right]  $. In other words,
$a=\left(  u,v\right)  $ for some $u\notin X\cup Y$ and $v\in X\cup Y$.
Consider these $u$ and $v$.
\par
From $u\notin X\cup Y$, we obtain $u\notin X$ and $u\notin Y$. Hence, if $v\in
X$, then $\left(  u,v\right)  \in\left[  \overline{X},X\right]  $. Thus, if
$v\in X$, then $a=\left(  u,v\right)  \in\left[  \overline{X},X\right]  $.
Hence, if $v\in X$, then $f\left(  a\right)  =0$ follows immediately from
(\ref{sol.flows-cuts.cut-lattice.X1}). Thus, for the rest of this proof, we
WLOG assume that $v\in X$ does not hold. Hence, $v\notin X$. Now, $v\in X\cup
Y$ but $v\notin X$. Hence, $v\in\left(  X\cup Y\right)  \setminus X=Y\setminus
X\subseteq Y$. But $u\notin Y$. Thus, $\left(  u,v\right)  \in\left[
\overline{Y},Y\right]  $. Therefore, $a=\left(  u,v\right)  \in\left[
\overline{Y},Y\right]  $. Hence, (\ref{sol.flows-cuts.cut-lattice.Y1}) shows
that $f\left(  a\right)  =0$. This proves
(\ref{sol.flows-cuts.cut-lattice.cup1}).} and%
\begin{equation}
\left(  \text{each }a\in\left[  X\cup Y,\overline{X\cup Y}\right]  \text{
satisfies }f\left(  a\right)  =c\left(  a\right)  \right)
\label{sol.flows-cuts.cut-lattice.cup2}%
\end{equation}
\footnote{\textit{Proof of (\ref{sol.flows-cuts.cut-lattice.cup2}):} Let
$a\in\left[  X\cup Y,\overline{X\cup Y}\right]  $. We must prove that
$f\left(  a\right)  =c\left(  a\right)  $.
\par
We have $a\in\left[  X\cup Y,\overline{X\cup Y}\right]  $. In other words,
$a=\left(  u,v\right)  $ for some $u\in X\cup Y$ and $v\notin X\cup Y$.
Consider these $u$ and $v$.
\par
From $v\notin X\cup Y$, we obtain $v\notin X$ and $v\notin Y$. Hence, if $u\in
X$, then $\left(  u,v\right)  \in\left[  X,\overline{X}\right]  $. Thus, if
$u\in X$, then $a=\left(  u,v\right)  \in\left[  X,\overline{X}\right]  $.
Therefore, if $u\in X$, then $f\left(  a\right)  =c\left(  a\right)  $ follows
immediately from (\ref{sol.flows-cuts.cut-lattice.X2}). Thus, for the rest of
this proof, we WLOG assume that $u\in X$ does not hold. Hence, $u\notin X$.
Now, $u\in X\cup Y$ but $u\notin X$. Hence, $u\in\left(  X\cup Y\right)
\setminus X=Y\setminus X\subseteq Y$. But $v\notin Y$. Thus, $\left(
u,v\right)  \in\left[  Y,\overline{Y}\right]  $. Now, $a=\left(  u,v\right)
\in\left[  Y,\overline{Y}\right]  $. Hence,
(\ref{sol.flows-cuts.cut-lattice.Y2}) shows that $f\left(  a\right)  =c\left(
a\right)  $. This proves (\ref{sol.flows-cuts.cut-lattice.cup2}).}.

But Lemma \ref{lem.flows-cuts.flow-estimate} \textbf{(b)} (applied to $S=X\cup
Y$) shows that $\left\vert f\right\vert =c\left(  X\cup Y,\overline{X\cup
Y}\right)  $ if and only if (\ref{sol.flows-cuts.cut-lattice.cup1}) and
(\ref{sol.flows-cuts.cut-lattice.cup2}) hold. Hence, $\left\vert f\right\vert
=c\left(  X\cup Y,\overline{X\cup Y}\right)  $ (since
(\ref{sol.flows-cuts.cut-lattice.cup1}) and
(\ref{sol.flows-cuts.cut-lattice.cup2}) hold). Thus, $c\left(  X\cup
Y,\overline{X\cup Y}\right)  =\left\vert f\right\vert =m$. In other words, the
$s$-$t$-cutting subset $X\cup Y$ of $V$ is cut-minimal. Exercise
\ref{exe.flows-cuts.cut-lattice} is thus shown.
\end{proof}

\begin{thebibliography}{99999999}

\bibitem[BjoLov92]{BjoLov92}
Anders Bj\"orner, L\'aszl\'o Lov\'asz,
\textit{Chip-firing games on directed graphs},
J. Algebraic Combinatorics 1 (1991), pp. 305--328,
revised version, July 1992.
\newline\url{http://www.cs.elte.hu/~lovasz/morepapers/abacus.pdf}

\bibitem[BjLoSh91]{BjLoSh91}
Anders Bj\"orner, L\'aszl\'o Lov\'asz, P. W. Shor,
\textit{Chip-firing games on graphs},
Europ. J. Comb. 12 (1991), pp. 283--291.
\newline\url{http://www.cs.elte.hu/~lovasz/morepapers/chips.pdf}

\bibitem[CorPet16]{CorPet16}
Scott Corry, David Perkinson,
\textit{Divisors and Sandpiles},
draft of 20 November 2016.
\newline\url{http://people.reed.edu/~davidp/divisors_and_sandpiles/}

\bibitem[Ellenb15]{Ellenb15}
Jordan Ellenberg,
\textit{The Amazing, Autotuning Sandpile},
Nautilus, 2 April 2015.
\newline\url{http://nautil.us/issue/23/dominoes/the-amazing-autotuning-sandpile}

\bibitem[ForFul74]{ForFul74}
L. R. Ford, D. R. Fulkerson,
\textit{Flows in Networks},
Princeton University Press,
sixth printing 1974.

\bibitem[Galvin17]{Galvin}David Galvin, \textit{Basic discrete mathematics}, 30
November 2017.\newline
\url{http://www-users.math.umn.edu/~dgrinber/comb/60610lectures2017-Galvin.pdf}

\bibitem[Grinbe17a]{hw2s}
Darij Grinberg,
\textit{UMN, Spring 2017, Math 5707: homework set \#2 with solutions},
\newline\url{http://www-users.math.umn.edu/~dgrinber/5707s17/hw2s.pdf} .

\bibitem[Grinbe17b]{lec16}
Darij Grinberg,
\textit{UMN, Spring 2017, Math 5707: Lecture 16 (flows and cuts in networks)},
\newline\url{http://www-users.math.umn.edu/~dgrinber/5707s17/5707lec16.pdf} .

\bibitem[HLMPPW13]{HLMPPW13}
Alexander E. Holroyd, Lionel Levine, Karola M\'esz\'aros,
Yuval Peres, James Propp, David B. Wilson,
\textit{Chip-Firing and Rotor-Routing on Directed Graphs},
arXiv:0801.3306v4.
\newline\url{https://arxiv.org/abs/0801.3306v4}

\bibitem[Musike09]{Musike09}
Gregg Musiker,
\textit{18.312: Algebraic Combinatorics, Spring 2009},
MIT OpenCourseWare.
\newline\url{https://ocw.mit.edu/courses/mathematics/18-312-algebraic-combinatorics-spring-2009/readings-and-lecture-notes/}

\bibitem[PicQue80]{PicQue80}
Jean-Claude Picard, Maurice Queyranne,
\textit{On the structure of all minimum cuts in a network and applications},
Mathematical Programming,
December 1982, Volume 22, Issue 1, pp. 121--121.
\newline\url{https://link.springer.com/article/10.1007/BF01581031}

\bibitem[Stanle11]{Stanle11}Richard P. Stanley, \textit{Enumerative
Combinatorics, volume 1}, Cambridge University Press, 2011. \newline%
\url{http://math.mit.edu/~rstan/ec/ec1/}

\end{thebibliography}

\end{document}