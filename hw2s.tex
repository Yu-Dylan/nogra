\documentclass[numbers=enddot,12pt,final,onecolumn,notitlepage]{scrartcl}%
\usepackage[headsepline,footsepline,manualmark]{scrlayer-scrpage}
\usepackage[all,cmtip]{xy}
\usepackage{amssymb}
\usepackage{amsmath}
\usepackage{amsthm}
\usepackage{framed}
\usepackage{comment}
\usepackage{color}
\usepackage{hyperref}
\usepackage{ifthen}
\usepackage[sc]{mathpazo}
\usepackage[T1]{fontenc}
\usepackage{needspace}
\usepackage{tabls}
%TCIDATA{OutputFilter=latex2.dll}
%TCIDATA{Version=5.50.0.2960}
%TCIDATA{LastRevised=Friday, September 16, 2016 20:39:00}
%TCIDATA{SuppressPackageManagement}
%TCIDATA{<META NAME="GraphicsSave" CONTENT="32">}
%TCIDATA{<META NAME="SaveForMode" CONTENT="1">}
%TCIDATA{BibliographyScheme=Manual}
%TCIDATA{Language=American English}
%BeginMSIPreambleData
\providecommand{\U}[1]{\protect\rule{.1in}{.1in}}
%EndMSIPreambleData
\newcounter{exer}
\theoremstyle{definition}
\newtheorem{theo}{Theorem}[section]
\newenvironment{theorem}[1][]
{\begin{theo}[#1]\begin{leftbar}}
{\end{leftbar}\end{theo}}
\newtheorem{lem}[theo]{Lemma}
\newenvironment{lemma}[1][]
{\begin{lem}[#1]\begin{leftbar}}
{\end{leftbar}\end{lem}}
\newtheorem{prop}[theo]{Proposition}
\newenvironment{proposition}[1][]
{\begin{prop}[#1]\begin{leftbar}}
{\end{leftbar}\end{prop}}
\newtheorem{defi}[theo]{Definition}
\newenvironment{definition}[1][]
{\begin{defi}[#1]\begin{leftbar}}
{\end{leftbar}\end{defi}}
\newtheorem{remk}[theo]{Remark}
\newenvironment{remark}[1][]
{\begin{remk}[#1]\begin{leftbar}}
{\end{leftbar}\end{remk}}
\newtheorem{coro}[theo]{Corollary}
\newenvironment{corollary}[1][]
{\begin{coro}[#1]\begin{leftbar}}
{\end{leftbar}\end{coro}}
\newtheorem{conv}[theo]{Convention}
\newenvironment{condition}[1][]
{\begin{conv}[#1]\begin{leftbar}}
{\end{leftbar}\end{conv}}
\newtheorem{quest}[theo]{Question}
\newenvironment{algorithm}[1][]
{\begin{quest}[#1]\begin{leftbar}}
{\end{leftbar}\end{quest}}
\newtheorem{warn}[theo]{Warning}
\newenvironment{conclusion}[1][]
{\begin{warn}[#1]\begin{leftbar}}
{\end{leftbar}\end{warn}}
\newtheorem{conj}[theo]{Conjecture}
\newenvironment{conjecture}[1][]
{\begin{conj}[#1]\begin{leftbar}}
{\end{leftbar}\end{conj}}
\newtheorem{exam}[theo]{Example}
\newenvironment{example}[1][]
{\begin{exam}[#1]\begin{leftbar}}
{\end{leftbar}\end{exam}}
\newtheorem{exmp}[exer]{Exercise}
\newenvironment{exercise}[1][]
{\begin{exmp}[#1]\begin{leftbar}}
{\end{leftbar}\end{exmp}}
\newenvironment{statement}{\begin{quote}}{\end{quote}}
\iffalse
\newenvironment{proof}[1][Proof]{\noindent\textbf{#1.} }{\ \rule{0.5em}{0.5em}}
\fi
\let\sumnonlimits\sum
\let\prodnonlimits\prod
\let\cupnonlimits\bigcup
\let\capnonlimits\bigcap
\renewcommand{\sum}{\sumnonlimits\limits}
\renewcommand{\prod}{\prodnonlimits\limits}
\renewcommand{\bigcup}{\cupnonlimits\limits}
\renewcommand{\bigcap}{\capnonlimits\limits}
\setlength\tablinesep{3pt}
\setlength\arraylinesep{3pt}
\setlength\extrarulesep{3pt}
\voffset=0cm
\hoffset=-0.7cm
\setlength\textheight{22.5cm}
\setlength\textwidth{15.5cm}
\newenvironment{verlong}{}{}
\newenvironment{vershort}{}{}
\newenvironment{noncompile}{}{}
\excludecomment{verlong}
\includecomment{vershort}
\excludecomment{noncompile}
\newcommand{\id}{\operatorname{id}}
\newcommand{\rev}{\operatorname{rev}}
\newcommand{\conncomp}{\operatorname{conncomp}}
\newcommand{\NN}{\mathbb{N}}
\newcommand{\ZZ}{\mathbb{Z}}
\newcommand{\QQ}{\mathbb{Q}}
\newcommand{\RR}{\mathbb{R}}
\newcommand{\powset}[2][]{\ifthenelse{\equal{#2}{}}{\mathcal{P}\left(#1\right)}{\mathcal{P}_{#1}\left(#2\right)}}
% $\powset[k]{S}$ stands for the set of all $k$-element subsets of
% $S$. The argument $k$ is optional, and if not provided, the result
% is the whole powerset of $S$.
\newcommand{\set}[1]{\left\{ #1 \right\}}
% $\set{...}$ yields $\left\{ ... \right\}$.
\newcommand{\abs}[1]{\left| #1 \right|}
% $\abs{...}$ yields $\left| ... \right|$.
\newcommand{\tup}[1]{\left( #1 \right)}
% $\tup{...}$ yields $\left( ... \right)$.
\newcommand{\ive}[1]{\left[ #1 \right]}
% $\ive{...}$ yields $\left[ ... \right]$.
\newcommand{\verts}[1]{\operatorname{V}\left( #1 \right)}
% $\verts{...}$ yields $\operatorname{V}\left( ... \right)$.
\newcommand{\edges}[1]{\operatorname{E}\left( #1 \right)}
% $\edges{...}$ yields $\operatorname{E}\left( ... \right)$.
\newcommand{\arcs}[1]{\operatorname{A}\left( #1 \right)}
% $\arcs{...}$ yields $\operatorname{A}\left( ... \right)$.
\newcommand{\underbrack}[2]{\underbrace{#1}_{\substack{#2}}}
% $\underbrack{...1}{...2}$ yields
% $\underbrace{...1}_{\substack{...2}}$. This is useful for doing
% local rewriting transformations on mathematical expressions with
% justifications.
\ihead{Math 5707 Spring 2017 (Darij Grinberg): homework set 2}
\ohead{page \thepage}
\cfoot{}
\begin{document}

\begin{center}
\textbf{Math 5707 Spring 2017 (Darij Grinberg): homework set 2}

\textbf{Solution sketches.}
\end{center}

\tableofcontents

\subsection{Notations and conventions}

See the
\href{http://www-users.math.umn.edu/~dgrinber/5707s17/nogra.pdf}{lecture notes}
and also the
\href{http://www-users.math.umn.edu/~dgrinber/5707s17/}{handwritten notes}
for relevant material.
If you reference results from the lecture notes, please \textbf{mention the date and time} of the version of the notes you are using (as the numbering changes during updates).

Let me recall a few notations:
\begin{itemize}
\item A \textit{simple graph} is a pair $\tup{V, E}$, where $V$ is a
      finite set and where $E$ is a subset of $\powset[2]{V}$.
      The set $V$ is called the \textit{vertex set} of the simple
      graph (and the elements of $V$ are called the \textit{vertices}
      of the simple graph),
      whereas the set $E$ is called the \textit{edge set} of the
      simple graph (and the elements of $E$ are called the
      \textit{edges} of the simple graph).
\item A \textit{multigraph} is a triple $\tup{V, E, \phi}$, where $V$
      and $E$ are finite sets and where $\phi$ is a map from $E$ to
      $\powset[2]{V}$. The set $V$ is called the \textit{vertex set}
      of the multigraph (and the elements of $V$ are called the
      \textit{vertices} of the multigraph); the set $E$ is called the
      \textit{edge set} of the multigraph (and the elements of $E$ are
      called the \textit{edges} of the multigraph); the map $\phi$ is
      called the \textit{endpoint map} of the multigraph. If $e$ is an
      edge of the multigraph, then the two elements of $\phi\tup{e}$
      are called the \textit{endpoints} of $e$.
\item A \textit{digraph} is a pair $\tup{V, A}$, where $V$ is a
      finite set and where $A$ is a subset of $V \times V$. The set
      $V$ is called the \textit{vertex set} of the digraph (and the
      elements of $V$ are called the \textit{vertices} of the
      digraph); the set $A$ is called the \textit{arc set} of the
      digraph (and the elements of $A$ are called the \textit{arcs}
      of the digraph). If $\tup{v, w}$ is an arc of the digraph, then
      $v$ is called the \textit{source} of this arc, and $w$ is called
      the \textit{target} of this arc. If the source of an arc equals
      its target, then this arc is said to be a \textit{loop}.
\item A \textit{multidigraph} is a triple $\tup{V, A, \phi}$, where
      $V$ and $A$ are finite sets and where $\phi$ is a map from $A$
      to $V \times V$. The set $V$ is called the \textit{vertex set}
      of the multidigraph (and the elements of $V$ are called the
      \textit{vertices} of the multidigraph); the set $A$ is called
      the \textit{arc set} of the multidigraph (and the elements of
      $A$ are called the \textit{arcs} of the multidigraph). If $a$
      is an arc of the multidigraph, and if
      $\tup{v, w} = \phi\tup{a}$, then $v$ is called the
      \textit{source} of this arc, and $w$ is called the
      \textit{target} of this arc.
\item If $u$ and $v$ are two vertices of a simple graph $G$, then we
      use the shorthand notation $uv$ for the set $\set{u, v}$ (even
      if this set is not an arc of $G$, and even if it is not a
      two-element set). If $u$ and $v$ are two vertices of a digraph
      $G$, then we use the shorthand notation $uv$ for the pair
      $\tup{u, v}$. We hope the two notations will not be confused.
\item A \textit{walk} in a simple graph $G$ (or in a digraph $G$) is
      defined to be a list of the form $\tup{v_0, v_1, \ldots, v_k}$,
      where $v_0, v_1, \ldots, v_k$ are vertices of $G$, and where
      $v_i v_{i+1}$ is an edge of $G$ (or an arc of $G$, respectively)
      for each $i \in \set{0, 1, \ldots, k-1}$. Here, the meaning of
      $v_i v_{i+1}$ depends on whether $G$ is a simple graph or a
      digraph (namely, it means a set in the former case,
      and a pair in the latter). \\
      A \textit{walk} in a multigraph $G$ is defined to be a list of
      the form
      $\tup{v_0, e_1, v_1, e_2, v_2, \ldots, e_k, v_k}$, where
      $v_0, v_1, \ldots, v_k$ are vertices of $G$, and where $e_i$ is
      an edge of $G$ having endpoints $v_{i-1}$ and $v_i$ for each
      $i \in \set{1, 2, \ldots, k}$. \\
      A \textit{walk} in a multidigraph $G$ is defined to be a list of
      the form $\tup{v_0, a_1, v_1, a_2, v_2, \ldots, a_k, v_k}$,
      where $v_0, v_1, \ldots, v_k$ are vertices of $G$, and where
      $a_i$ is an arc of $G$ having source $v_{i-1}$ and target $v_i$
      for each $i \in \set{1, 2, \ldots, k}$. \\
      In each of these cases, the \textit{vertices} of the walk are
      defined to be $v_0, v_1, \ldots, v_k$. Moreover, $v_0$ is called
      the \textit{starting point} of the walk, and $v_k$ is called the
      \textit{ending point} of the walk. The \textit{edges} of the
      walk are defined to be $v_0 v_1, v_1 v_2, \ldots, v_{k-1} v_k$
      (if $G$ is a simple graph) or
      $e_1, e_2, \ldots, e_k$ (if $G$ is a multigraph). The
      \textit{arcs} of the walk are defined to be
      $v_0 v_1, v_1 v_2, \ldots, v_{k-1} v_k$
      (if $G$ is a digraph) or
      $a_1, a_2, \ldots, a_k$ (if $G$ is a multidigraph).
\item A \textit{path} in a simple graph $G$ (or in a digraph $G$, or
      in a multigraph $G$, or in a multidigraph $G$) means a walk in
      $G$ such that the vertices of the walk are distinct.
\item A \textit{circuit} in a simple graph $G$ (or in a digraph $G$,
      or in a multigraph $G$, or in a multidigraph $G$) means a walk
      in $G$ such that the starting point of the walk equals the
      ending point of the walk. \\
      If $v_0, v_1, \ldots, v_k$ are the vertices of a circuit
      $\mathbf{c}$, then $v_0, v_1, \ldots, v_{k-1}$ are called the
      \textit{non-ultimate vertices} of $\mathbf{c}$.
\item A \textit{cycle} in a simple graph $G$ means a circuit
      $\tup{v_0, v_1, \ldots, v_k}$ in $G$ such that the vertices
      $v_0, v_1, \ldots, v_{k-1}$ are distinct and such that
      $k \geq 3$. \\
      A \textit{cycle} in a digraph $G$ means a circuit
      $\tup{v_0, v_1, \ldots, v_k}$ in $G$ such that the vertices
      $v_0, v_1, \ldots, v_{k-1}$ are distinct and such that
      $k \geq 1$. (In particular, each loop $\tup{v, v}$ gives rise to
      a cycle.) \\
      A \textit{cycle} in a multigraph $G$ means a circuit
      $\tup{v_0, e_1, v_1, e_2, v_2, \ldots, e_k, v_k}$ such that the
      vertices $v_0, v_1, \ldots, v_{k-1}$ are distinct, the edges
      $e_1, e_2, \ldots, e_k$ are also distinct, and such that
      $k \geq 2$. \\
      A \textit{cycle} in a multidigraph $G$ means a circuit
      $\tup{v_0, a_1, v_1, a_2, v_2, \ldots, a_k, v_k}$ in $G$ such
      that the vertices $v_0, v_1, \ldots, v_{k-1}$ are distinct and
      such that $k \geq 1$.
\item A simple graph $G$ (or multigraph $G$, or digraph $G$, or
      multidigraph $G$) is said to be \textit{strongly connected} if
      its vertex set is nonempty\footnote{Do not forget this
      requirement!} and it has the property that for any two vertices
      $u$ and $v$ of $G$, there exists at least one walk from $u$ to
      $v$ in $G$. (Note that if there exists a walk from $u$ to $v$ in
      $G$, then there also exists a path from $u$ to $v$ in $G$.)
      \par
      When $G$ is a simple graph or a multigraph, we usually say
      ``$G$ is connected'' instead of ``$G$ is strongly connected''.
\item A \textit{Hamiltonian path} in a simple graph $G$ (or multigraph
      $G$, or digraph $G$, or multidigraph $G$) means a path
      $\mathbf{p}$ in $G$ such that each vertex of $G$ appears
      exactly once among the vertices of $\mathbf{p}$.
\item A \textit{Hamiltonian cycle} in a simple graph $G$ (or
      multigraph $G$, or digraph $G$, or multidigraph $G$) means a
      cycle $\mathbf{c}$ of $G$ such that each vertex of $G$ appears
      exactly once among the non-ultimate vertices of $\mathbf{c}$.
\item A walk $\mathbf{w}$ in a simple graph $G$ (or multigraph $G$) is
      said to be \textit{Eulerian} if each edge of $G$ appears exactly
      once among the edges of $\mathbf{w}$. \\
      A walk $\mathbf{w}$ in a digraph $G$ (or multidigraph $G$) is
      said to be \textit{Eulerian} if each arc of $G$ appears exactly
      once among the arcs of $\mathbf{w}$. \\
      Notice that this automatically defines the notion of an Eulerian
      circuit (namely, an Eulerian circuit is just a circuit that is
      Eulerian).
\end{itemize}

\subsection{Exercise \ref{exe.hw2.hamilGxH}: Hamiltonian paths in
Cartesian product graphs}

\begin{exercise} \label{exe.hw2.hamilGxH}
Let $G$ and $H$ be two simple graphs. The \textit{Cartesian product} of $G$
and $H$ is a new simple graph, denoted $G \times H$, which is defined as
follows:
\begin{itemize}
\item The vertex set $\verts{G \times H}$ of $G \times H$ is the
Cartesian product $\verts{G} \times \verts{H}$.

\item A vertex $\tup{g, h}$ of $G \times H$ is adjacent to a vertex
$\tup{g', h'}$ of $G \times H$ if and only if we have
\begin{itemize}
\item \textbf{either} $g = g'$ and $hh' \in \edges{H}$,
\item \textbf{or} $h = h'$ and $gg' \in \edges{G}$.
\end{itemize}
(In particular, exactly one of the two equalities $g = g'$ and $h = h'$
has to hold when $\tup{g, h}$ is adjacent to $\tup{g', h'}$.)
\end{itemize}

\textbf{(a)} Recall the $n$-dimensional cube graph $Q_n$ defined for
each $n \in \NN$. (Its vertices are $n$-tuples $\tup{a_1, a_2, \ldots,
a_n} \in \set{0, 1}^n$, and two such vertices are adjacent if and only
if they differ in exactly one entry.) Prove that $Q_n \cong
Q_{n-1} \times Q_1$ for each positive integer $n$. (Thus, $Q_n$ can
be obtained from $Q_1$ by repeatedly forming Cartesian products; i.e.,
it is a ``Cartesian power'' of $Q_1$.)

\textbf{(b)} Assume that each of the graphs $G$ and $H$ has a
Hamiltonian path. Prove that $G \times H$ has a Hamiltonian path.

\textbf{(c)} Assume that both numbers $\abs{\verts{G}}$ and
$\abs{\verts{H}}$ are $> 1$, and that at least one of them is even.
Assume again that each of the graphs $G$ and $H$ has a Hamiltonian
path. Prove that $G \times H$ has a Hamiltonian cycle.

\end{exercise}

\begin{proof}[Solution sketch to Exercise~\ref{exe.hw2.hamilGxH}.]
\textbf{(a)} It suffices to check that the map
\[
\set{0, 1}^n \to \set{0, 1}^{n-1} \times \set{0, 1}, \qquad
\tup{a_1, a_2, \ldots, a_n}
 \mapsto \tup{ \tup{a_1, a_2, \ldots, a_{n-1}}, a_n}
\]
is a graph isomorphism from $Q_n$ to $Q_{n-1} \times Q_1$. The proof
of this is straightforward; the main step is to check that two
$n$-tuples
$\tup{a_1, a_2, \ldots, a_n}$ and $\tup{b_1, b_2, \ldots, b_n}$ in
$\set{0, 1}^n$ differ in exactly one entry (i.e., are adjacent as
vertices of $Q_n$) if and only if
\begin{itemize}
\item \textbf{either} we have
$\tup{a_1, a_2, \ldots, a_{n-1}} = \tup{b_1, b_2, \ldots, b_{n-1}}$ 
and $a_n \neq b_n$,
\item \textbf{or} the $\tup{n-1}$-tuples
$\tup{a_1, a_2, \ldots, a_{n-1}}$ and
$\tup{b_1, b_2, \ldots, b_{n-1}}$ 
differ in exactly one entry (i.e., are adjacent as vertices of
$Q_{n-1}$) and we have $a_n = b_n$.
\end{itemize}
This is obvious.

\textbf{(b)} By assumption, there exists a Hamiltonian path
$\tup{g_1, g_2, \ldots, g_n}$ of $G$, and there exists a Hamiltonian
path $\tup{h_1, h_2, \ldots, h_m}$ of $H$. Use these two paths to
construct the Hamiltonian path
\begin{align}
& \left(
            \tup{g_1, h_1}, \tup{g_1, h_2}, \ldots, \tup{g_1, h_m}, \right. \nonumber \\
&    \left. \  \tup{g_2, h_m}, \tup{g_2, h_{m-1}}, \ldots, \tup{g_2, h_1}, \right. \nonumber \\
&    \left. \  \tup{g_3, h_1}, \tup{g_3, h_2}, \ldots, \tup{g_3, h_m}, \right. \nonumber \\
&    \left. \  \tup{g_4, h_m}, \tup{g_4, h_{m-1}}, \ldots, \tup{g_4, h_1}, \right. \nonumber \\
&    \left. \  \cdots \right)
    \label{sol.hw2.hamilGxH.1}
\end{align}
in $G \times H$. (This Hamiltonian path first traverses all vertices
of the form $\tup{g_1, h_i}$ in the order of increasing $i$, then
traverses all vertices of the form $\tup{g_2, h_i}$ in the order of
decreasing $i$, then traverses all vertices of the form
$\tup{g_3, h_i}$ in the order of increasing $i$, and so on, always
alternating between increasing and decreasing $i$. It ends at the
vertex $\tup{g_n, h_m}$ if $n$ is odd, and at the vertex
$\tup{g_n, h_1}$ if $n$ is even. Here is how it looks like:
\[
\xymatrix{
\tup{g_1, h_1} \ar[r] & \tup{g_1, h_2} \ar[r] & \cdots \ar[r] & \tup{g_1, h_m} \ar[d] \\
\tup{g_2, h_1} \ar[d] & \tup{g_2, h_2} \ar[l] & \cdots \ar[l] & \tup{g_2, h_m} \ar[l] \\
\tup{g_3, h_1} \ar[r] & \tup{g_3, h_2} \ar[r] & \cdots \ar[r] & \tup{g_3, h_m} \ar[d] \\
\tup{g_4, h_1} \ar[d] & \tup{g_4, h_2} \ar[l] & \cdots \ar[l] & \tup{g_4, h_m} \ar[l] \\
\vdots
}
\]
where the arrows merely signify the order in which the vertices are
traversed by the path (the edges are still undirected).)

\textbf{(c)} At least one of the integers $\abs{\verts{G}}$ and
$\abs{\verts{H}}$ is even.
Since
$G \times H \cong H \times G$ (in fact, there is a graph isomorphism
$G \times H \to H \times G$ sending each vertex $\tup{v, w}$ of
$G \times H$ to $\tup{w, v}$), we can WLOG assume that
$\abs{\verts{G}}$ is even (because otherwise we can simply switch
$G$ with $H$).
Assume this, and recall furthermore that $\abs{\verts{H}} > 1$.

By assumption, there exists a Hamiltonian path
$\tup{g_1, g_2, \ldots, g_n}$ of $G$, and there exists a Hamiltonian
path $\tup{h_0, h_1, \ldots, h_m}$ of $H$. (Note that I am indexing
the vertices of the former path from $1$, but indexing the vertices
of the latter path from $0$.) Thus, $n = \abs{\verts{G}}$ is even.
Also, $n = \abs{\verts{G}} > 1$.
Also, $m + 1 = \abs{\verts{H}} > 1$, so that $m > 0$.

Now, consider the path \eqref{sol.hw2.hamilGxH.1}. It is not a
Hamiltonian path, since it misses the vertices of the form
$\tup{g_i, h_0}$. But it is a path from $\tup{g_1, h_1}$ to
$\tup{g_n, h_1}$ (since $n$ is even) that uses each vertex
\textbf{not} of this form exactly once; thus, we can extend it to
a Hamiltonian cycle of $G \times H$ by appending the following
vertices at its end:
\[
 \tup{g_n, h_0}, \tup{g_{n-1}, h_0}, \ldots, \tup{g_1, h_0},
 \tup{g_1, h_1}.
\]
(In other words, we close the path by going back to its starting
point along the missing vertices $\tup{g_i, h_0}$.) Hence, we have
found a Hamiltonian cycle of $G \times H$.

(Here is how this Hamiltonian cycle looks like:
\[
\xymatrix{
\tup{g_1, h_0} \ar[r] & \tup{g_1, h_1} \ar[r] & \tup{g_1, h_2} \ar[r] & \cdots \ar[r] & \tup{g_1, h_m} \ar[d] \\
\tup{g_2, h_0} \ar[u] & \tup{g_2, h_1} \ar[d] & \tup{g_2, h_2} \ar[l] & \cdots \ar[l] & \tup{g_2, h_m} \ar[l] \\
\tup{g_3, h_0} \ar[u] & \tup{g_3, h_1} \ar[r] & \tup{g_3, h_2} \ar[r] & \cdots \ar[r] & \tup{g_3, h_m} \ar[d] \\
\tup{g_4, h_0} \ar[u] & \tup{g_4, h_1} \ar[d] & \tup{g_4, h_2} \ar[l] & \cdots \ar[l] & \tup{g_4, h_m} \ar[l] \\
\vdots \ar[u] & \vdots \ar[d] \\
\tup{g_{n-1}, h_0} \ar[u] & \tup{g_{n-1}, h_1} \ar[r] & \tup{g_{n-1}, h_2} \ar[r] & \cdots \ar[r] & \tup{g_{n-1}, h_m} \ar[d] \\
\tup{g_n, h_0} \ar[u] & \tup{g_n, h_1} \ar[l] & \tup{g_n, h_2} \ar[l] & \cdots \ar[l] & \tup{g_n, h_m} \ar[l]
}
\]
where the arrows merely signify the order in which the vertices are
traversed by the cycle (the edges are still undirected).)
\end{proof}

\subsection{Exercise~\ref{exe.eulertrails.Kn}: Eulerian circuits in
$K_3$, $K_5$ and $K_7$}

\begin{exercise} \label{exe.eulertrails.Kn}
Let $n$ be a positive integer. Recall that $K_n$ denotes the complete
graph on $n$ vertices. This is the graph with vertex set $V =
\set{1, 2, \ldots, n}$ and edge set $\mathcal{P}_2\tup{V}$ (so each two
distinct vertices are connected).

Find Eulerian circuits for the graphs $K_3$, $K_5$, and $K_7$.
\end{exercise}

\begin{proof}[Solution sketch to Exercise~\ref{exe.eulertrails.Kn}.]
An Eulerian circuit of $K_3$ is $\tup{1, 2, 3, 1}$.

An Eulerian circuit of $K_5$ is
$\tup{1, 2, 3, 4, 5, 1, 3, 5, 2, 4, 1}$.

An Eulerian circuit of $K_7$ is
$\tup{1, 2, 3, 4, 5, 6, 7,
      1, 3, 5, 7, 2, 4, 6,
      1, 4, 7, 3, 6, 2, 5,
      1}$.

[\textit{Remark:} Of course, other choices are possible. For each
odd positive integer $n$, the complete graph $K_n$ has an Eulerian
circuit (because it is connected, and each of its vertices has even
degree $n-1$), and so it has at least one Eulerian circuit; but in
truth, there are many. (How many? See
\href{http://oeis.org/A007082}{OEIS entry A007082}. There doesn't seem
to be an explicit formula.)

When $n$ is an odd prime\footnote{Everyone gets confused by the notion
of an ``odd prime'' at least once in their life. But it means exactly
what it says: a prime that is odd. In other words, a prime that is
distinct from $2$.}, there is actually a simple way to construct
an Eulerian circuit in $K_n$: For each $k \in \set{1, 2, \ldots,
\tup{n-1}/2}$, let $c_k$ be the cycle $\tup{a_0, a_1, \ldots, a_n}$,
where $a_i$ denotes the unique element of $\tup{1, 2, \ldots, n}$ that
is congruent to $ki + 1$ modulo $n$. Then, the cycles $c_1, c_2,
\ldots, c_{\tup{n-1}/2}$ can be combined to a single Eulerian circuit.
Finding Eulerian circuits on $K_n$ for non-prime $n$ is harder, but
of course the algorithm done in class still works.]
\end{proof}

\subsection{Exercise~\ref{exe.debruijn.1}: de Bruijn sequences exist}

\begin{exercise} \label{exe.debruijn.1}
Let $n$ be a positive integer, and $K$ be a nonempty finite set.
Let $k = \abs{K}$.
A \textit{de Bruijn sequence} of order $n$ on $K$ means a list
$\tup{c_0, c_1, \ldots, c_{k^n-1}}$ of $k^n$ elements of $K$
such that

\begin{enumerate}
\item[(1)] for each
$n$-tuple $\tup{a_1, a_2, \ldots, a_n} \in K^n$ of elements of $K$,
there exists a \textbf{unique} $r \in \set{0, 1, \ldots, k^n-1}$ such
that
$\tup{a_1, a_2, \ldots, a_n} = \tup{c_r, c_{r+1}, \ldots, c_{r+n-1}}$.
\end{enumerate}

Here, the indices are understood to be cyclic modulo $k^n$; that is,
$c_q$ (for $q \geq k^n$) is defined to be $c_{q \% k^n}$, where
$q \% k^n$ denotes the remainder of $q$ modulo $k^n$.

(Note that the condition (1) can be restated as follows: If we
write the elements $c_0, c_1, \ldots, c_{k^n-1}$ on a circular
necklace (in this order), so that the last element $c_{k^n-1}$ is
followed by the first one, then each $n$-tuple of elements of $K$
appears as a string of $n$ consecutive elements on the necklace, and
the position at which it appears on the necklace is unique.)

For example, $\tup{c_0, c_1, c_2, c_3, c_4, c_5, c_6, c_7, c_8}
= \tup{1, 1, 2, 2, 3, 3, 1, 3, 2}$ is a de Bruijn sequence
of order $2$ on the set $\set{1, 2, 3}$, because for each $2$-tuple
$\tup{a_1, a_2} \in \set{1, 2, 3}^2$, there exists a unique $r \in \set{0, 1,
\ldots, 8}$ such that $\tup{a_1, a_2} = \tup{c_r, c_{r+1}}$. Namely:
\begin{align*}
\tup{1, 1} = \tup{c_0, c_1}; \qquad
\tup{1, 2} = \tup{c_1, c_2}; \qquad
\tup{1, 3} = \tup{c_6, c_7}; \\
\tup{2, 1} = \tup{c_8, c_9}; \qquad
\tup{2, 2} = \tup{c_2, c_3}; \qquad
\tup{2, 3} = \tup{c_3, c_4}; \\
\tup{3, 1} = \tup{c_5, c_6}; \qquad
\tup{3, 2} = \tup{c_7, c_8}; \qquad
\tup{3, 3} = \tup{c_4, c_5}.
\end{align*}

Prove that there exists a de Bruijn sequence of order $n$ on $K$
(no matter what $n$ and $K$ are).

\textbf{Hint:} Let $D$ be the digraph with vertex set $K^{n-1}$ and
an arc from $\tup{a_1, a_2, \ldots, a_{n-1}}$ to
$\tup{a_2, a_3, \ldots, a_n}$ for each
$\tup{a_1, a_2, \ldots, a_n} \in K^n$ (and no other arcs). Prove that
$D$ has an Eulerian circuit.
\end{exercise}

\begin{proof}[Solution sketch to Exercise \ref{exe.debruijn.1}.]
The hint suggests defining a digraph. I shall use a multidigraph
instead, as this is slightly simpler and cleaner.

Recall that a \textit{multidigraph} means a triple
$\tup{V, A, \phi}$, where $V$ and $A$ are two finite sets and where
$\phi$ is a map from $A$ to $V \times V$. We define a multidigraph
$D$ to be $\tup{K^{n-1}, K^n, f}$, where the map
$f : K^n \to K^{n-1} \times K^{n-1}$ is given by the formula
\[
f\tup{a_1, a_2, \ldots, a_n}
= \tup{ \tup{a_1, a_2, \ldots, a_{n-1}},
        \tup{a_2, a_3, \ldots, a_n} }.
\]
(As usual, we write $f\tup{a_1, a_2, \ldots, a_n}$ for
$f\tup{\tup{a_1, a_2, \ldots, a_n}}$, since the extra parentheses do
not add any clarity.) Thus, in the multidigraph $D$, there is an arc
from $\tup{a_1, a_2, \ldots, a_{n-1}}$ to
$\tup{a_2, a_3, \ldots, a_n}$ for each
$\tup{a_1, a_2, \ldots, a_n} \in K^n$. (Note that if $n = 1$, then
there is only one vertex, namely the empty $0$-tuple $\tup{}$, but
there are $\abs{K}$ many arcs from it to itself. This is why we are
using a multidigraph instead of a digraph. Of course, you are free to
throw the $n = 1$ case aside, seeing how easy it is to handle
separately.)

Recall that the \textit{indegree} of a vertex $v$ of a multidigraph
$\tup{V, A, \phi}$ is defined to be the number of all arcs $a \in A$
whose target is $v$ (that is, which satisfy $\phi\tup{a} = \tup{x, v}$
for some $x \in V$). This indegree is denoted by $\deg^- v$.
Also, the \textit{outdegree} of a vertex $v$ of a multidigraph
$\tup{V, A, \phi}$ is defined to be the number of all arcs $a \in A$
whose source is $v$ (that is, which satisfy $\phi\tup{a} = \tup{v, x}$
for some $x \in V$). This outdegree is denoted by $\deg^+ v$.

Recall that a multidigraph $\tup{V, A, \phi}$ is said to be
\textit{strongly connected} if $V \neq \varnothing$ and if, for any
$u \in V$ and $v \in V$, there is at least one walk from $u$ to $v$
in the multidigraph.
The multidigraph $D$ is strongly connected\footnote{\textit{Proof.}
Let $u \in K^{n-1}$ and $v \in K^{n-1}$. We must prove that there is
at least one walk from $u$ to $v$ in $D$. Write the $\tup{n-1}$-tuples
$u$ and $v$ as $u = \tup{u_1, u_2, \ldots, u_{n-1}}$ and
$v = \tup{v_1, v_2, \ldots, v_{n-1}}$, respectively.
Then, the walk
\begin{align*}
& \left( \tup{u_1, u_2, u_3, u_4, \ldots, u_{n-1}} , \tup{u_1, u_2, u_3, u_4, \ldots, u_{n-1}, v_1} , \right. \\
& \left. \phantom{u_1, } \tup{u_2, u_3, u_4, \ldots, u_{n-1}, v_1} , \tup{u_2, u_3, u_4, \ldots, u_{n-1}, v_1, v_2} , \right. \\
& \left. \phantom{u_1, u_2, } \tup{u_3, u_4, \ldots, u_{n-1}, v_1, v_2} , \tup{u_3, u_4, \ldots, u_{n-1}, v_1, v_2, v_3} , \right. \\
& \left. \phantom{u_1, u_2, u_3, } \cdots , \right. \\
& \left. \phantom{u_1, u_2, u_3, u_4, \ldots, } \tup{u_{n-1}, v_1, v_2, \ldots, v_{n-2}} , \tup{u_{n-1}, v_1, v_2, \ldots, v_{n-2}, v_{n-1}} , \right. \\
& \left. \phantom{u_1, u_2, u_3, u_4, \ldots, u_{n-1}, } \tup{v_1, v_2, \ldots, v_{n-2}, v_{n-1}} \right)
\end{align*}
is a walk from $u$ to $v$ in $D$. Hence, such a walk exists.}.
Moreover, each vertex $v \in K^{n-1}$ satisfies
$\deg^- v = \deg^+ v$ (where both indegree and outdegree are taken
respective to the multidigraph $D$)\ \ \ \ \footnote{\textit{Proof.}
Let $v \in K^{n-1}$. Write the $\tup{n-1}$-tuple $v$ in the form
$v = \tup{v_1, v_2, \ldots, v_{n-1}}$.
Now, $\deg^- v$ is the number of arcs of $D$
with target $v$. But these arcs are exactly the $n$-tuples of the
form $\tup{k, v_1, v_2, \ldots, v_{n-1}} \in K^n$ with $k \in K$
(by the definition of the multidigraph $D$). Hence, there are
exactly $\abs{K}$ of them. Therefore, $\deg^- v = \abs{K}$ (since
$\deg^- v$ is the number of these arcs). Similarly,
$\deg^+ v = \abs{K}$. Thus, $\deg^- v = \abs{K} = \deg^+ v$.}.

Recall that a multidigraph has an Eulerian circuit if and only if it
is strongly connected and each vertex $v$ satisfies
$\deg^- v = \deg^+ v$. Hence, the multidigraph $D$ has an Eulerian
circuit (since it is strongly connected and each vertex $v$ satisfies
$\deg^- v = \deg^+ v$). Consider such an Eulerian circuit
$\mathbf{c}$. It contains
each arc of $D$ exactly once, and thus has $k^n$ arcs (since
the number of arcs of $D$ is $\abs{K^n} = \abs{K}^n = k^n$).
Let $p_0, p_1, \ldots, p_{k^n-1}$ be these arcs (listed in the order
in which they appear on the Eulerian circuit). We extend the
indexing of these arcs modulo $k^n$; in other words, we set
$p_i = p_{i \% k^n}$ for each $i \in \ZZ$. (This, of course, does not
conflict with the already introduced notations 
$p_0, p_1, \ldots, p_{k^n-1}$, since each
$i \in \set{0, 1, \ldots, k^n-1}$ satisfies $i \% k^n = i$.)

Let me now explain what I intend to do before I go into the
technical details. We want to construct a de Bruijn sequence
$\tup{c_0, c_1, \ldots, c_{k^n-1}}$. I claim that the sequence
of the first entries of the $n$-tuples $p_0, p_1, \ldots, p_{k^n-1}$
is such a de Bruijn sequence. Once this is proven, the exercise
will clearly be solved.

For each $n$-tuple $h$ and each $j \in \set{1, 2, \ldots, n}$,
we will denote the $j$-th entry of $h$ by $h\ive{j}$. So each
$n$-tuple $h$ has the form
$h = \tup{h\ive{1}, h\ive{2}, \ldots, h\ive{n}}$.
Now, I claim that
\begin{equation}
p_i\ive{j+1} = p_{i+1}\ive{j}
\qquad \text{ for each } i \in \ZZ
\text{ and } j \in \set{1, 2, \ldots, n-1}
\label{pf.debruijn.1.rec.one-step}
\end{equation}
\footnote{\textit{Proof of \eqref{pf.debruijn.1.rec.one-step}.}
Let $i \in \ZZ$. The arc $p_{i+1}$ immediately follows the arc
$p_i$ on the Eulerian circuit $\mathbf{c}$. Hence, the target of
the arc $p_i$ is the source of the arc $p_{i+1}$. But by the
definition of the multidigraph $D$, the former target is
$\tup{p_i\ive{2}, p_i\ive{3}, \ldots, p_i\ive{n}}$, whereas the
latter source is
$\tup{p_{i+1}\ive{1}, p_{i+1}\ive{2}, \ldots, p_{i+1}\ive{n-1}}$.
Hence, we have shown that
$\tup{p_i\ive{2}, p_i\ive{3}, \ldots, p_i\ive{n}}
= \tup{p_{i+1}\ive{1}, p_{i+1}\ive{2}, \ldots, p_{i+1}\ive{n-1}}$.
In other words, $p_i\ive{j+1} = p_{i+1}\ive{j}$ for each
$j \in \set{1, 2, \ldots, n-1}$. This proves
\eqref{pf.debruijn.1.rec.one-step}.}. Hence,
\begin{equation}
p_i\ive{j+g} = p_{i+g}\ive{j}
\qquad \text{ for each } i \in \ZZ \text{ and } g \in \NN
\text{ and } j \in \set{1, 2, \ldots, n-g}
\label{pf.debruijn.1.rec.g-steps}
\end{equation}
\footnote{Indeed, \eqref{pf.debruijn.1.rec.g-steps} can easily
be proven by induction on $g$, using
\eqref{pf.debruijn.1.rec.one-step} in the induction step.}.
From this, we can easily obtain
\begin{equation}
\tup{p_i\ive{1}, p_{i+1}\ive{1}, \ldots, p_{i+n-1}\ive{1}}
= p_i
\qquad \text{ for each } i \in \ZZ
\label{pf.debruijn.1.=pi}
\end{equation}
\footnote{\textit{Proof.} Let $i \in \ZZ$. Then,
\eqref{pf.debruijn.1.rec.g-steps} (applied to $j=1$) shows that
$p_i\ive{1+g} = p_{i+g}\ive{1}$ for each
$g \in \set{0,1,\ldots,n-1}$. In other words,
$\tup{p_i\ive{1}, p_i\ive{2}, \ldots, p_i\ive{n}}
= \tup{p_i\ive{1}, p_{i+1}\ive{1}, \ldots, p_{i+n-1}\ive{1}}$.
Therefore,
$\tup{p_i\ive{1}, p_{i+1}\ive{1}, \ldots, p_{i+n-1}\ive{1}}
= \tup{p_i\ive{1}, p_i\ive{2}, \ldots, p_i\ive{n}}
= p_i$.}.

Now, recall that $p_0, p_1, \ldots, p_{k^n-1}$ are the arcs
of an Eulerian circuit of $D$. Hence, each arc of $D$ appears
exactly once in the list $\tup{p_0, p_1, \ldots, p_{k^n-1}}$.
In other words, for each arc $a$ of $D$, there exists a unique
$r \in \set{0, 1, \ldots, k^n-1}$ such that $a = p_i$.
Since the arcs of $D$ are the $n$-tuples in $K^n$, we can
rewrite this as follows: For each $n$-tuple $a \in K^n$,
there exists a unique
$r \in \set{0, 1, \ldots, k^n-1}$ such that $a = p_r$. Since
each $r \in \ZZ$ satisfies
$\tup{p_r\ive{1}, p_{r+1}\ive{1}, \ldots, p_{r+n-1}\ive{1}}
= p_r$ (by \eqref{pf.debruijn.1.=pi}, applied to $i=r$),
this further rewrites as follows:
For each $n$-tuple $a \in K^n$,
there exists a unique $r \in \set{0, 1, \ldots, k^n-1}$
such that
$a = \tup{p_r\ive{1}, p_{r+1}\ive{1}, \ldots, p_{r+n-1}\ive{1}}$.
Renaming $a$ as $\tup{a_1, a_2, \ldots, a_n}$ in this result,
we obtain the following:
For each
$n$-tuple $\tup{a_1, a_2, \ldots, a_n} \in K^n$ of elements of $K$,
there exists a unique $r \in \set{0, 1, \ldots, k^n-1}$ such
that
$\tup{a_1, a_2, \ldots, a_n}
= \tup{p_r\ive{1}, p_{r+1}\ive{1}, \ldots, p_{r+n-1}\ive{1}}$.
In other words, the list
$\tup{p_0\ive{1}, p_1\ive{1}, \ldots, p_{k^n-1}\ive{1}}$ is a
de Bruijn sequence of order $n$ on $K$ (because the indices are
cyclic modulo $k^n$, so the result in the previous sentence is
precisely what is required in the definition of a de Bruijn
sequence). Therefore, there exists a de Bruijn sequence of order
$n$ on $K$.

[\textit{Remark:} The underlying philosophy of this solution was
to reduce the question of the existence of a de Bruijn sequence
to the existence of an Eulerian circuit in a multidigraph. At a
first glance, this appears unexpected, since it seems more natural
to model a de Bruijn sequence by a Hamiltonian cycle (in a different
multidigraph) instead. However, there are few good criteria for the
existence of a Hamiltonian cycle, whereas the existence of an
Eulerian circuit is easy to check. This is why modelling
de Bruijn sequences by Eulerian circuits proves to be the more
useful approach.

This exercise is merely the starting point of a theory. For example,
it is generalized by the following theorem:
\begin{theorem} \label{sol.debruijn.1.gen}
Let $d$ and $m$ be two positive integers such that $d \mid m$ and
$d > 1$. Then, there exists a permutation
$\tup{x_1, x_2, \ldots, x_m}$ of the list $\tup{0, 1, \ldots, m-1}$
with the following property:
For each $i \in \set{1, 2, \ldots, m}$, we have
$x_{i+1} \equiv d x_i + r_i \mod m$ for some
$r_i \in \set{0, 1, \ldots, d-1}$. (Here, $x_{m+1}$ should be
understood as $x_1$.)
\end{theorem}

Why does Theorem~\ref{sol.debruijn.1.gen} generalize
Exercise~\ref{exe.debruijn.1}? Well, if $m = d^n$ is a power of $d$,
then we can identify
the integers $0, 1, \ldots, d-1$ with $n$-tuples of elements of
$\set{0, 1, \ldots, d-1}$ (by representing them in base $d$, including
just enough leading zeroes to ensure that they all have $n$ digits).
Thus, Theorem~\ref{sol.debruijn.1.gen} turns into
Exercise~\ref{exe.debruijn.1} in this
case. With some work, the solution of Exercise~\ref{exe.debruijn.1}
can be extended to a proof of Theorem~\ref{sol.debruijn.1.gen}.
(Some work is required to prove that the digraph is still strongly
connected.) Note that
\href{https://anhngq.files.wordpress.com/2010/07/imo-2002-shortlist.pdf}{IMO Shortlist 2002}
problem C6 is equivalent to the $d = 2$ particular case of
Theorem~\ref{sol.debruijn.1.gen}.

For more variations on the notion of a de Bruijn sequence, see
\cite{ChDiGr92}. There are several
questions left open in that paper, some of which are apparently still
unsolved.

On the other hand, we can also ask ourselves: How many de Bruijn
sequences of order $n$ exist for a given $n$ and $K$ ?
Interestingly, the answer is very explicit: The number of all
de Bruijn sequences of order $n$ is
\[
 k!^{k^{n-1}} ,
 \qquad \text{ where } k = \abs{K} .
\]
\footnote{Some authors treat two de Bruijn sequences as equal
if one of them is obtained from the other by cyclic rotation.
With that convention, the number has to be divided by $k^n$.}
This is proven in the case of $K = \set{0, 1}$ in
\cite[Corollary 10.11]{Stanley13}. The general case can be proven
along the same lines.]
\end{proof}

\subsection{Exercise~\ref{exe.digraph.indeg=outdeg}: Indegrees and
outdegrees in digraphs}

Recall that the \textit{indegree} of a vertex $v$ of a digraph
$D = \tup{V, A}$ is defined to be the number of all arcs $a \in A$
whose target is $v$. This indegree is denoted by
$\deg^-\tup{v}$ or by $\deg^-_D\tup{v}$ (whenever the graph $D$ is not
clear from the context).

Likewise, the \textit{outdegree} of a vertex $v$ of a digraph
$D = \tup{V, A}$ is defined to be the number of all arcs $a \in A$
whose source is $v$. This outdegree is denoted by
$\deg^+\tup{v}$ or by $\deg^+_D\tup{v}$ (whenever the graph $D$ is not
clear from the context).

\begin{exercise} \label{exe.digraph.indeg=outdeg}
Let $D$ be a digraph. Show that
$\sum_{v \in \verts{D}} \deg^-\tup{v}
= \sum_{v \in \verts{D}} \deg^+\tup{v}$.
\end{exercise}

\begin{proof}[Solution to Exercise~\ref{exe.digraph.indeg=outdeg}.]
Let us actually prove a somewhat more general fact:
\begin{statement}
\textit{Fact 1.} Let $\tup{V, A, \phi}$ be a
multidigraph.
%\footnote{See the solution to
%Exercise~\ref{exe.debruijn.1} above for the definition of
%a multidigraph, and for the definitions of indegrees and
%outdegrees of vertices of a multidigraph.}.
Then,
$\sum_{v \in V} \deg^- v = \sum_{v \in V} \deg^+ v$.
\end{statement}

Fact 1 generalizes Exercise~\ref{exe.digraph.indeg=outdeg},
because each digraph $D = \tup{V, A}$ gives rise to a
multidigraph $\tup{V, A, \id_A}$, and the indegrees and
the outdegrees of the vertices of the former digraph are
exactly the same as in the latter multidigraph. Hence,
proving Fact 1 will suffice.

\begin{proof}[Proof of Fact 1.] For each arc $a \in A$, let
$s \tup{a}$ denote the source of $a$, and let $t \tup{a}$ denote the
target of $a$. (Thus, each $a \in A$ satisfies
$\phi \tup{a} = \tup{ s \tup{a}, t \tup{a} }$.)

Now, let us count the number of arcs $a \in A$ of our multidigraph in
two different ways:

\begin{itemize}
\item Each arc $a \in A$ has a unique source $s \tup{a}$. Thus, we can
obtain the number $\abs{A}$ of all arcs $a \in A$ by computing,
for each $v \in V$, the number of all arcs $a \in A$ satisfying
$s \tup{a} = v$, and then adding up these
numbers over all $v \in V$. Thus, we obtain
\begin{equation}
\abs{A}
= \sum_{v \in V}
  \underbrace{\left(\text{the number of all } a \in A
                \text{ satisfying } s \tup{a} = v \right)}_{
              \substack{= \deg^+ v \\
                \text{(by the definition of the outdegree } \deg^+ v
                \text{ of } v \text{})}}
= \sum_{v \in V} \deg^+ v .
\label{sol.digraph.indeg=outdeg.2}
\end{equation}

\item Each arc $a \in A$ has a unique target $t \tup{a}$. Thus, we can
obtain the number $\abs{A}$ of all arcs $a \in A$ by computing,
for each $v \in V$, the number of all arcs $a \in A$ satisfying
$t \tup{a} = v$, and then adding up these
numbers over all $v \in V$. Thus, we obtain
\begin{equation}
\abs{A}
= \sum_{v \in V}
  \underbrace{\left(\text{the number of all } a \in A
                \text{ satisfying } t \tup{a} = v \right)}_{
              \substack{= \deg^- v \\
                \text{(by the definition of the indegree } \deg^- v
                \text{ of } v \text{})}}
= \sum_{v \in V} \deg^- v .
\label{sol.digraph.indeg=outdeg.1}
\end{equation}
\end{itemize}

Comparing \eqref{sol.digraph.indeg=outdeg.1} with
\eqref{sol.digraph.indeg=outdeg.2}, we obtain
$\sum_{v \in V} \deg^- v = \sum_{v \in V} \deg^+ v$.
This proves Fact 1.
\end{proof}
\end{proof}

\subsection{Exercise~\ref{exe.tourn.num3cycs}: Counting $3$-cycles in
tournaments}

The next few exercises are about \textit{tournaments}. A
\textit{tournament} is a loopless\footnote{A digraph $\tup{V, A}$ is
said to be \textit{loopless} if it has no loops. (A \textit{loop}
means an arc of the form $\tup{v, v}$ for some $v \in V$.)} digraph
$D = \tup{V, A}$ with the following
property: For any two distinct vertices $u \in V$ and $v \in V$,
\textbf{precisely} one of the two pairs $\tup{u, v}$ and $\tup{v, u}$
belongs to $A$. (In other words, any two distinct vertices are
connected by an arc in one direction, but not in both.)

A \textit{$3$-cycle}\footnote{Note that our
notions of $3$-cycles and of cycles are somewhat different in nature:
A $3$-cycle is a triple of distinct vertices, whereas a cycle of
length $k$ is a $\tup{k+1}$-tuple of vertices with its first and its
last entry being the same vertex. Thus, a $3$-cycle $\tup{u, v, w}$ is
not in itself a cycle, but rather corresponds to the cycle
$\tup{u, v, w, u}$. But, of course, the $3$-cycles are in bijection
with the cycles of length $3$; thus, the difference between these two
notions is merely notational.} in a tournament $D = \tup{V, A}$ means
a triple $\tup{u, v, w}$ of vertices in $V$ such that all three pairs
$\tup{u, v}$, $\tup{v, w}$ and $\tup{w, u}$ belong to $A$.

\begin{exercise} \label{exe.tourn.num3cycs}
Let $D = \tup{V, A}$ be a tournament.
Set $n = \abs{V}$ and $m = \sum_{v \in V} \dbinom{\deg^-\tup{v}}{2}$.

\textbf{(a)} Show that $m = \sum_{v \in V} \dbinom{\deg^+\tup{v}}{2}$.

\textbf{(b)} Show that the number of $3$-cycles in $D$ is
$3\tup{\dbinom{n}{3} - m}$.
\end{exercise}

\begin{proof}[Solution sketch to Exercise~\ref{exe.tourn.num3cycs}.]
Let us introduce some convenient notations for this exercise:
\begin{itemize}
\item A \textit{3-set} shall mean a $3$-element subset of $V$.
Clearly, the number of all 3-sets is $\dbinom{n}{3}$ (since
$\abs{V} = n$).
\item A 3-set $\set{u, v, w}$ is said to be \textit{cyclic} if its
elements in \textbf{some} order form a $3$-cycle (i.e., if one of the
triples $\tup{u, v, w}$, $\tup{u, w, v}$, $\tup{v, u, w}$,
$\tup{v, w, u}$, $\tup{w, u, v}$ and $\tup{w, v, u}$ is a $3$-cycle).
\item A 3-set $\set{u, v, w}$ is said to be \textit{acyclic} if it is
not cyclic.
\item We say that a 3-set $S$ is \textit{sourced} at a vertex
$u \in V$ if this vertex $u$ belongs to $S$ and if the arcs $uv$ and
$uw$ are in $A$, where $v$ and $w$ are the two vertices of $S$
distinct from $u$.
\item We say that a 3-set $S$ is \textit{targeted} at a vertex
$u \in V$ if this vertex $u$ belongs to $S$ and if the arcs $vu$ and
$wu$ are in $A$, where $v$ and $w$ are the two vertices of $S$
distinct from $u$.
\end{itemize}

\textbf{(a)} Let us count the number of all acyclic 3-sets in two
different ways.

First, we make a few observations:

\begin{statement}
\textit{Observation 1:} Let $u$, $v$ and $w$ be three distinct
vertices in $V$ such that the arcs $uv$ and $uw$ are in $A$.
Then, $\set{u, v, w}$ is an acyclic 3-set sourced at $u$.
\end{statement}
\begin{proof}[Proof of Observation 1.]
Trivial.
\end{proof}

\begin{statement}
\textit{Observation 2:} Let $u \in V$ be a vertex. Then, the number
of all acyclic 3-sets sourced at $u$ is $\dbinom{\deg^+ u}{2}$.
\end{statement}
\begin{proof}[Proof of Observation 2.]
First of all, we introduce one more notion:
An \textit{out-neighbor} of $u$ shall mean a vertex $x \in V$
such that the arc $ux$ is in $A$.
Clearly, the out-neighbors of $u$ are in bijection with the arcs of
$D$ whose source is $u$\ \ \ \ \footnote{\textit{Proof.} The two maps
\begin{align*}
\tup{\text{the set of all out-neighbors of } u}
&\to
\tup{\text{the set of all arcs of } D \text{ whose source is } u}, \\
x &\mapsto ux
\end{align*}
and
\begin{align*}
\tup{\text{the set of all arcs of } D \text{ whose source is } u}
&\to
\tup{\text{the set of all out-neighbors of } u}, \\
a &\mapsto \tup{\text{the target of } a}
\end{align*}
are mutually inverse (this can be checked in a straightforward
manner). Note that we are here relying on the fact that $D$ is a
digraph, not a multidigraph!}. Hence, the number of the former
out-neighbors equals the number of the latter arcs.
Since the number of the latter arcs
is $\deg^+ u$ (indeed, this is how $\deg^+ u$ was defined), we can
thus conclude that the number of the former out-neighbors is
$\deg^+ u$. In other words, the number of all out-neighbors of $u$
is $\deg^+ u$.

Observation 2 now easily follows: A 3-set is an acyclic 3-set sourced
at $u$ if and only if it consists of $u$ and two distinct
out-neighbors of $u$. Hence, choosing an acyclic 3-set sourced at $u$
is tantamount to choosing two distinct out-neighbors of $u$ (without
specifying the order\footnote{See the next footnote for a more
rigorous way to write up this argument.}.
But the latter can be done in exactly $\dbinom{\deg^+ u}{2}$
ways (since the number of all out-neighbors of $u$ is $\deg^+ u$).
Hence, the number of all acyclic 3-sets sourced at $u$ is
$\dbinom{\deg^+ u}{2}$.\ \ \ \ \footnote{Here is a more rigorous way
to present this argument:

Let $U$ be the set of all out-neighbors of $u$. Let $G$ be the set of
all acyclic 3-sets sourced at $u$. Then, the maps
\begin{align*}
G \to \powset[2]{U},
\qquad S \mapsto S \setminus \set{u}
\end{align*}
and
\begin{align*}
\powset[2]{U} \to G,
\qquad T \mapsto T \cup \set{u}
\end{align*}
are well-defined (this is easy to check: e.g., you have to apply
Observation 1, and you have to argue that
if $S$ is a 3-set sourced at $u$, then the two elements of
$S \setminus \set{u}$ are two distinct out-neighbors of $u$) and
mutually inverse (this is essentially obvious). Hence, they provide
bijections between $G$ and $\powset[2]{U}$. Thus,
$\abs{G} = \abs{\powset[2]{U}} = \dbinom{\abs{U}}{2}$ (since every
finite set $Q$ and each $k \in \NN$ satisfy
$\abs{\powset[k]{Q}} = \dbinom{\abs{Q}}{k}$). But $U$ is the set of
all out-neighbors of $u$, and thus has size $\abs{U} = \deg^+ u$
(since the number of all out-neighbors of $u$ is $\deg^+ u$). Hence,
$\abs{G} = \dbinom{\abs{U}}{2}$ rewrites as
$\abs{G} = \dbinom{\deg^+ u}{2}$. Since $G$ is the set of all acyclic
3-sets sourced at $u$, this means precisely that
the number of all acyclic 3-sets sourced at $u$ is
$\dbinom{\deg^+ u}{2}$.} This proves Observation 2.
\end{proof}

\begin{statement}
\textit{Observation 3:} Let $S$ be an acyclic 3-set. Then, there is
a \textbf{unique} vertex $u \in V$ such that $S$ is sourced at $u$.
\end{statement}
\begin{proof}[Proof of Observation 3.]
This can be straightforwardly verified: Write $S$ in the form
$\set{a, b, c}$. We want to know which of the six pairs $ab$, $ba$,
$bc$, $cb$, $ca$ and $ac$ belong to $A$ (i.e., are arcs of $D$).
We know that
exactly one of the two arcs $ab$ and $ba$
belongs to $A$ (since $D$ is a tournament);
exactly one of the two arcs $bc$ and $cb$
belongs to $A$ (since $D$ is a tournament);
exactly one of the two arcs $ca$ and $ac$
belongs to $A$ (since $D$ is a tournament).
Hence, a total of $2\cdot 2\cdot 2 = 8$ cases are possible
regarding the question which of the six pairs $ab$, $ba$,
$bc$, $cb$, $ca$ and $ac$ belong to $A$
(for example, one case is that $ab$, $cb$ and $ca$ belong to $A$, but
$ba$, $bc$ and $ac$ do not). Two of these cases are impossible due to
the requirement that $S$ be acyclic\footnote{Indeed, the two
impossible cases are ``$ab$, $bc$ and $ca$ belong to $A$, but $ba$,
$cb$ and $ac$ do not'' (because $\tup{a, b, c}$ would be a
$3$-cycle in this case) and
``$ba$, $cb$ and $ac$ belong to $A$, but $ab$, $bc$ and $ca$ do not''
(because $\tup{a, c, b}$ would be a $3$-cycle in this case).}. In the
remaining six cases, it is easy to check that Observation 3 holds.
(For example, if $ab$, $cb$ and $ca$ belong to $A$, then there is a
\textbf{unique} vertex $u \in V$ such that $S$ is sourced at $u$;
namely, this $u$ is $c$. It is unique because clearly, if $S$ is
sourced at $u$, then $u$ has to be an element of $S$, and the only
element of $S$ that works is $c$.) Thus, Observation 3 is proven.
\end{proof}

Now, Observation 3 yields
\begin{align}
& \tup{\text{the number of all acyclic 3-sets}} \nonumber \\
&= \sum_{u \in V}
     \underbrack{\tup{
                 \text{the number of all acyclic 3-sets sourced at }
                 u}}{
                     = \dbinom{\deg^+ u}{2} \\
                     \text{(by Observation 2)}
                    }
 = \sum_{u \in V} \dbinom{\deg^+ u}{2}
  \nonumber \\
&= \sum_{v \in V} \dbinom{\deg^+ v}{2}
\label{sol.tourn.num3cycs.1}
\end{align}
(here, we renamed the summation index $u$ as $v$).

On the other hand, we have the following observations, which mimic
the Observations 1, 2 and 3 above (but with sources replaced by
targets, and arcs changing directions), and whose proofs are
analogous to those of the latter:

\begin{statement}
\textit{Observation 4:} Let $u$, $v$ and $w$ be three distinct
vertices in $V$ such that the arcs $vu$ and $wu$ are in $A$.
Then, $\set{u, v, w}$ is an acyclic 3-set targeted at $u$.
\end{statement}

\begin{statement}
\textit{Observation 5:} Let $u \in V$ be a vertex. Then, the number
of all acyclic 3-sets targeted at $u$ is $\dbinom{\deg^- u}{2}$.
\end{statement}

\begin{statement}
\textit{Observation 6:} Let $S$ be an acyclic 3-set. Then, there is
a \textbf{unique} vertex $u \in V$ such that $S$ is targeted at $u$.
\end{statement}

As I said, the proofs of Observations 4, 5 and 6 are analogous to the
proofs of Observations 1, 2 and 3, and so are omitted. Now, similarly
to how we proved \eqref{sol.tourn.num3cycs.1} using Observations 1, 2
and 3, we can now prove the equality
\begin{equation}
\tup{\text{the number of all acyclic 3-sets}}
= \sum_{v \in V} \dbinom{\deg^- v}{2}
\label{sol.tourn.num3cycs.2}
\end{equation}
using Observations 4, 5 and 6. Comparing this equality with
\eqref{sol.tourn.num3cycs.1}, we find \newline
$ \sum_{v \in V} \dbinom{\deg^+ v}{2}
= \sum_{v \in V} \dbinom{\deg^- v}{2} = m$. This solves part
\textbf{(a)} of the exercise.

\textbf{(b)} Let us make one more observation:

\begin{statement}
\textit{Observation 7:} Let $S$ be a cyclic 3-set. Then, there exist
exactly three $3$-cycles $\tup{u, v, w}$ satisfying
$\set{u, v, w} = S$.
\end{statement}
\begin{proof}[Proof of Observation 7.]
We know that $S$ is a cyclic 3-set. In other words, $S$ is a
$3$-element subset of $V$ whose elements in \textbf{some} order form a
$3$-cycle (because this is how a ``cyclic 3-set'' was defined). In
other words, $S = \set{a, b, c}$ for some $3$-cycle $\tup{a, b, c}$.
Consider this $\tup{a, b, c}$. Hence, $ab$, $bc$ and $ca$ are arcs of
$D$; therefore, $ba$, $cb$ and $ac$ are not arcs of $D$ (since $D$ is
a tournament).

There are exactly six triples $\tup{u, v, w} \in V^3$ satisfying
$\set{u, v, w} = S$ (namely, the triples $\tup{a, b, c}$,
$\tup{a, c, b}$, $\tup{b, a, c}$, $\tup{b, c, a}$, $\tup{c, a, b}$ and
$\tup{c, b, a}$). Among these six triples, exactly three are
$3$-cycles (in fact,
all of the three triples $\tup{a, b, c}$, $\tup{b, c, a}$ and
$\tup{c, a, b}$ are $3$-cycles, whereas none of the three triples
$\tup{a, c, b}$, $\tup{b, a, c}$ and $\tup{c, b, a}$ is a
$3$-cycle). Hence, there exist
exactly three $3$-cycles $\tup{u, v, w}$ satisfying
$\set{u, v, w} = S$. This proves Observation 7.
\end{proof}

Now, it is obvious that for each $3$-cycle $\tup{u, v, w}$, the set
$\set{u, v, w}$ is a cyclic 3-set. Hence, we can count the
number of all $3$-cycles as follows:
\begin{align}
&  \tup{\text{the number of all } 3\text{-cycles}} \nonumber \\
&= \sum_{S \text{ is a cyclic 3-set}}
      \underbrack{\tup{\text{the number of all } 3\text{-cycles}
                       \tup{u, v, w} \text{ satisfying }
                       \set{u, v, w} = S}
                 }{= 3 \\ \text{(by Observation 7)}} \nonumber \\
&= \sum_{S \text{ is a cyclic 3-set}} 3 \nonumber \\
&= 3 \tup{\text{the number of all cyclic 3-sets}} .
\label{sol.tourn.num3cycs.b.1}
\end{align}

But recall that the number of all 3-sets is $\dbinom{n}{3}$. Each of
these 3-sets is either cyclic or acyclic (but not both). Hence,
\begin{align*}
\tup{\text{the number of all cyclic 3-sets}}
&= \dbinom{n}{3}
   - \underbrack{\tup{\text{the number of all acyclic 3-sets}}
                }{= \sum_{v \in V} \dbinom{\deg^- v}{2} \\
                  \text{(by \eqref{sol.tourn.num3cycs.2})}}
\\
&= \dbinom{n}{3}
   - \underbrack{\sum_{v \in V} \dbinom{\deg^- v}{2}
                }{= m}
= \dbinom{n}{3} - m .
\end{align*}
Hence, \eqref{sol.tourn.num3cycs.b.1} rewrites as follows:
\[
\tup{\text{the number of all } 3\text{-cycles}}
= 3 \tup{\dbinom{n}{3} - m}.
\]
This solves part \textbf{(b)} of the exercise.

[\textit{Remark:} There is a simpler argument for \textbf{(a)}; let me
briefly outline it:
\begin{align*}
& \sum_{v \in V} \dbinom{\deg^- v}{2}
  - \sum_{v \in V} \dbinom{\deg^+ v}{2}
= \sum_{v \in V}
    \underbrack{\tup{\dbinom{\deg^- v}{2} - \dbinom{\deg^+ v}{2}}
               }{= \dfrac{1}{2} \tup{\deg^- v - \deg^+ v}
                   \tup{\deg^- v + \deg^+ v + 1}} \\
&= \sum_{v \in V}
    \dfrac{1}{2} \tup{\deg^- v - \deg^+ v}
    \underbrack{\tup{\deg^- v + \deg^+ v + 1}}{= n \\ \text{(why?)}}
= \dfrac{n}{2}
    \underbrack{\sum_{v \in V} \tup{\deg^- v - \deg^+ v}
               }{= \sum_{v \in V} \deg^- v - \sum_{v \in V} \deg^+ v
                 = 0 \\
                 \text{(by Exercise~\ref{exe.digraph.indeg=outdeg})}}
= 0.
\end{align*}
However, this is of little help in proving part \textbf{(b)}.]
\end{proof}

\subsection{Some lemmas}

Before the next exercise, we prove a few simple facts that will
eventually prove useful:

\begin{lemma} \label{lem.digraph.balanced-cycle}
Let $D = \tup{V, A, \phi}$ be a multidigraph such that each vertex
$v \in V$ satisfies $\deg^- v = \deg^+ v$. Assume that the set $A$ is
nonempty (i.e., the multidigraph $D$ has at least one arc). Then, $D$
has at least one cycle. (This cycle may be a one-vertex cycle, i.e.,
it may be of the form $\tup{v, v}$ for a vertex $v \in V$, provided
that there is an arc from $v$ to $v$.)
\end{lemma}

\begin{proof}[Proof of Lemma~\ref{lem.digraph.balanced-cycle}.]
We know that the set $A$ is nonempty.
In other words, the multidigraph $D$ has at least one arc.
Thus, there exists at least one path of
length $\geq 1$ in $D$ (namely, the path consisting of this arc).

The set of paths of $D$ is finite\footnote{\textit{Proof.} Recall that
the vertices of a path in $D$ must be distinct. Hence, a path in $D$
cannot have length larger than $\abs{V}$ (since $D$ has only $\abs{V}$
many vertices). Therefore, there are only finitely many paths in $D$
(since there are only finitely many paths in $D$ of any given
length).} and nonempty\footnote{since we just have shown that
there exists at least one path of length $\geq 1$ in $D$}. Hence,
there exists a longest path in $D$ (that is, a path in $D$ having the
maximum length). Fix such a path, and denote it by
$\tup{v_0, a_1, v_1, a_2, v_2, \ldots, a_k, v_k}$. (As usual, this
means that the vertices on this path are $v_0, v_1, \ldots, v_k$, and
the arcs along this path are $a_1, a_2, \ldots, a_k$.) The vertices
$v_0, v_1, \ldots, v_k$ are distinct (since
$\tup{v_0, a_1, v_1, a_2, v_2, \ldots, a_k, v_k}$ is a path).

We have $k \geq 1$\ \ \ \ \footnote{\textit{Proof.}
  We know that there exists at least one path of
  length $\geq 1$ in $D$.
  Hence, any longest path in $D$ has length $\geq 1$.
  In particular, this shows that the path
  $\tup{v_0, a_1, v_1, a_2, v_2, \ldots, a_k, v_k}$ has length
  $\geq 1$ (since this path is a longest path in $D$).
  In other words, $k \geq 1$.}.
Thus, the arc $a_k$ is well-defined.
This arc $a_k$ has target $v_k$. Thus, at least one arc has target
$v_k$. Hence, $\deg^- \tup{v_k} > 0$. But recall that
each vertex $v \in V$ satisfies $\deg^- v = \deg^+ v$. Applying this
to $v = v_k$, we obtain $\deg^- \tup{v_k} = \deg^+ \tup{v_k}$.
Hence, $\deg^+ \tup{v_k} = \deg^- \tup{v_k} > 0$. Hence, there exists
some arc with source $v_k$. Fix such an arc, and denote it by
$a_{k+1}$ (this is allowed, since so far we have only defined $a_i$
for $i \in \set{1, 2, \ldots, k}$).
Let $v_{k+1}$ be the target of this arc (this is allowed, since so far
we have only defined $v_i$ for $i \in \set{0, 1, \ldots, k}$).

Now,
$\tup{v_0, a_1, v_1, a_2, v_2, \ldots, a_k, v_k, a_{k+1}, v_{k+1}}$ is
clearly a walk in $D$. If this walk was a path, then it would be a
longer path than $\tup{v_0, a_1, v_1, a_2, v_2, \ldots, a_k, v_k}$,
which is absurd (since
$\tup{v_0, a_1, v_1, a_2, v_2, \ldots, a_k, v_k}$
was chosen to be a longest path in $D$). Hence, it is not a path.
Therefore, the vertices $v_0, v_1, \ldots, v_k, v_{k+1}$ are not
distinct (because if they were distinct, then the walk
$\tup{v_0, a_1, v_1, a_2, v_2, \ldots, a_k, v_k, a_{k+1}, v_{k+1}}$
would be a path). In other words, two of these vertices are equal. In
other words, there exist two elements $i$ and $j$ of
$\set{0, 1, \ldots, k+1}$ such that $i < j$ and $v_i = v_j$. Consider
these $i$ and $j$.
Recall that the vertices $v_0, v_1, \ldots, v_k$ are distinct.
Hence, $j$ must be $k+1$ (since otherwise, $v_i = v_j$ would
contradict the distinctness of $v_0, v_1, \ldots, v_k$). Thus,
$v_j = v_{k+1}$. Therefore, $v_i = v_j = v_{k+1}$. Hence,
$\tup{v_i, a_{i+1}, v_{i+1}, a_{i+2}, v_{i+2}, \ldots,
      a_{k+1}, v_{k+1}}$
is a circuit in $D$. This circuit is furthermore a cycle (since the
vertices $v_i, v_{i+1}, \ldots, v_k$ are distinct (because
$v_0, v_1, \ldots, v_k$ are distinct), and since $i < j = k+1$).
Hence, there exists a cycle in $D$.
This proves Lemma~\ref{lem.digraph.balanced-cycle}.
\end{proof}

\begin{corollary} \label{cor.digraph.balanced-cycle.digraph}
Let $D = \tup{V, A}$ be a digraph such that each vertex
$v \in V$ satisfies $\deg^- v = \deg^+ v$. Assume that the set $A$ is
nonempty (i.e., the digraph $D$ has at least one arc). Then, $D$
has at least one cycle.
\end{corollary}

\begin{proof}[Proof of
Corollary~\ref{cor.digraph.balanced-cycle.digraph}.]
The digraph $D = \tup{V, A}$ gives rise to a multidigraph
$D' = \tup{V, A, \id}$. The vertices in $V$ have the same
indegrees with respect to the latter multidigraph $D'$ as they
have with respect to the former digraph $D$; in other words,
each $v \in V$ satisfies $\deg^-_{D'} v = \deg^-_D v$.
Thus, we do not need to distinguish between $\deg^-_D v$ and
$\deg^-_{D'} v$; we can use the notation $\deg^- v$ for both of
these numbers.
Applying Lemma~\ref{lem.digraph.balanced-cycle} to
$D'$ and $\id$ instead of $D$ and $\phi$, we therefore
conclude that $D'$ has at least one cycle. If we denote this
cycle by $\tup{v_0, a_1, v_1, a_2, v_2, \ldots, a_k, v_k}$,
then $\tup{v_0, v_1, \ldots, v_k}$ is a cycle of the digraph
$D$. Thus, the digraph $D$ has at least one cycle.
This proves
Corollary~\ref{cor.digraph.balanced-cycle.digraph}.
\end{proof}

\begin{lemma} \label{lem.tourn.AvsB}
Let $V$ be a finite set. Let $E = \tup{V, A}$ and $F = \tup{V, B}$ be
two tournaments with vertex set $V$. Let $u \in V$ and $v \in V$.
Then, we have the following logical equivalence:
\begin{equation}
\tup{ \tup{u, v} \in B \setminus A }
\Longleftrightarrow
\tup{ \tup{v, u} \in A \setminus B } .
\label{sol.tourn.3-cyc-rev.o4.pf.equiv}
\end{equation}
\end{lemma}

\begin{proof}[Proof of Lemma~\ref{lem.tourn.AvsB}.]
Let us first prove the implication
\begin{equation}
\tup{ \tup{u, v} \in B \setminus A }
\Longrightarrow
\tup{ \tup{v, u} \in A \setminus B }.
\label{sol.tourn.3-cyc-rev.o4.pf.equiv.pf.1}
\end{equation}
Indeed, assume that $\tup{u, v} \in B \setminus A$ holds.
Thus, $\tup{u, v} \in B$ and $\tup{u, v} \notin A$. The pair
$\tup{u, v}$ belongs to $B$, thus is an arc of the tournament $F$.
Therefore, $\tup{u, v}$ is not a loop (since tournaments have no
loops). In other words, $u \neq v$. Hence, exactly one of the pairs
$\tup{u, v}$ and $\tup{v, u}$ is an arc of $E$ (since $E$ is a
tournament). In other words, exactly one of the pairs
$\tup{u, v}$ and $\tup{v, u}$ belongs to $A$ (since $A$ is the set of
the arcs of $E$). Since $\tup{u, v} \notin A$, we thus have
$\tup{v, u} \in A$.
On the other hand, recall again that $u \neq v$. Thus, exactly one of
the pairs $\tup{u, v}$ and $\tup{v, u}$ is an arc of $F$ (since $F$ is
a tournament). In other words, exactly one of the pairs
$\tup{u, v}$ and $\tup{v, u}$ belongs to $B$ (since $B$ is the set of
the arcs of $F$). Since $\tup{u, v} \in B$, we thus have
$\tup{v, u} \notin B$. Combining $\tup{v, u} \in A$ with
$\tup{v, u} \notin B$, we obtain $\tup{v, u} \in A \setminus B$.

Now, forget that we assumed that
$\tup{u, v} \in B \setminus A$ holds. We thus have proven that
$\tup{v, u} \in A \setminus B$ under the assumption that
$\tup{u, v} \in B \setminus A$. In other words, we have proven the
implication \eqref{sol.tourn.3-cyc-rev.o4.pf.equiv.pf.1}.

But we can also apply the implication
\eqref{sol.tourn.3-cyc-rev.o4.pf.equiv.pf.1} to $v$, $u$, $B$, $A$,
$F$ and $E$ instead of $u$, $v$, $A$, $B$, $E$ and $F$. Thus, we
obtain the implication
\begin{equation}
\tup{ \tup{v, u} \in A \setminus B }
\Longrightarrow
\tup{ \tup{u, v} \in B \setminus A }.
\end{equation}
Combining this implication with
\eqref{sol.tourn.3-cyc-rev.o4.pf.equiv.pf.1}, we obtain the
equivalence \eqref{sol.tourn.3-cyc-rev.o4.pf.equiv}. Thus,
Lemma~\ref{lem.tourn.AvsB} is proven.
\end{proof}

\subsection{Exercise~\ref{exe.tourn.3-cyc-rev}: Transforming
tournaments by reversing $3$-cycles}

\begin{exercise} \label{exe.tourn.3-cyc-rev}
If a tournament $D$ has a $3$-cycle $\tup{u, v, w}$, then
we can define a new tournament $D'_{u, v, w}$ as follows: The vertices
of $D'_{u, v, w}$ shall be the same as those of $D$. The arcs of
$D'_{u, v, w}$ shall be the same as those of $D$, except that the
three arcs $\tup{u, v}$, $\tup{v, w}$ and $\tup{w, u}$ are replaced
by the three new arcs $\tup{v, u}$, $\tup{w, v}$ and $\tup{u, w}$.
(Visually speaking, $D'_{u, v, w}$ is obtained from $D$ by turning the
arrows on the arcs $\tup{u, v}$, $\tup{v, w}$ and $\tup{w, u}$
around.) We say that the new tournament $D'_{u, v, w}$ is obtained
from the old tournament $D$ by a
\textit{$3$-cycle reversal operation}.

Now, let $V$ be a finite set, and let $E$ and $F$ be two tournaments
with vertex set $V$. Prove that $F$ can be obtained from $E$ by a
sequence of $3$-cycle reversal operations if and only if each
$v \in V$ satisfies $\deg^-_E \tup{v} = \deg^-_F \tup{v}$.
% (Here, $\deg^-_D \tup{v}$
% denotes the indegree of $v$ with respect to a given digraph $D$.
(Note that a sequence may be empty, which allows handling the case
$E = F$ even if $E$ has no $3$-cycles to reverse.)
\end{exercise}

\begin{proof}[Solution sketch to Exercise~\ref{exe.tourn.3-cyc-rev}.]
Exercise~\ref{exe.tourn.3-cyc-rev} is
\cite[Theorem 35]{Moon13}.

Here is another solution:

Let us forget about $V$, $E$ and $F$. Instead, we first introduce
some more terminology:
\begin{itemize}
\item If $\tup{u, v}$ is an arc
of a tournament $D$, then \textit{reversing} this arc $\tup{u, v}$
means
replacing it by the arc $\tup{v, u}$ (in other words, removing the arc
$\tup{u, v}$, and adding a new arc $\tup{v, u}$ instead). The digraph
that results from this operation is again a tournament.
(Visually speaking, reversing an arc in a tournament means turning the
arrow on this arc around.)
\end{itemize}

Using this terminology, our concept of ``$3$-cycle reversal
operation'' can be reformulated as follows: A tournament $D'$ is
obtained from $D$ by a
\textit{$3$-cycle reversal operation} if and only if there exists a
$3$-cycle $\tup{u, v, w}$ such that $D'$ is obtained from $D$ by
reversing the arcs $\tup{u, v}$, $\tup{v, w}$ and $\tup{w, u}$.
If this is the case, we shall also say (more concretely) that $D'$ is
obtained from $D$ by \textit{reversing the $3$-cycle $\tup{u, v, w}$}.

Let us introduce a new operation:
\begin{itemize}
\item If a tournament $D$ has a
cycle $\mathbf{c} = \tup{v_0, v_1, \ldots, v_k}$, then we let
$D''_{\mathbf{c}}$ be the tournament obtained from $D$ by reversing
the arcs $\tup{v_0, v_1}, \tup{v_1, v_2}, \ldots, \tup{v_{k-1}, v_k}$.
In other words, we let $D''_{\mathbf{c}}$ be the tournament obtained
from $D$ by reversing all arcs of the cycle $\mathbf{c}$.
% The vertices of $D''_{\mathbf{c}}$ shall be the same as those of $D$.
% The arcs of $D''_{\mathbf{c}}$ shall be the same as those of $D$,
% except that the $k$ arcs
% replaced by the $k$ arcs
% $\tup{v_1, v_0}, \tup{v_2, v_1}, \ldots, \tup{v_k, v_{k-1}}$.
% (Visually speaking, $D''_{\mathbf{c}}$ is obtained from $D$ by turning
% the arrows on the arcs
% $\tup{v_0, v_1}, \tup{v_1, v_2}, \ldots, \tup{v_{k-1}, v_k}$
% around.)
We say that the new tournament $D''_{\mathbf{c}}$ is obtained
from the old tournament $D$ by
\textit{reversing the cycle $\mathbf{c}$}.
\end{itemize}

We now claim the following facts:

\begin{statement}
\textit{Observation 1:} Let $D$ be a tournament. Let $\mathbf{c}$ be a
cycle of $D$. Then, $\mathbf{c}$ has length $\geq 3$.
\end{statement}
\begin{proof}[Proof of Observation 1.]
Assume the contrary. Then, $\mathbf{c}$ has length $< 3$. In other
words, $\mathbf{c}$ has length $1$ or $2$.

But $D$ is a tournament, and thus has no loops.

If the cycle $\mathbf{c}$ had length $1$, then it would have the
form $\tup{v, v}$ for some vertex $v$ of $D$. Therefore, $\tup{v, v}$
would be an arc of $D$; this would imply that $D$ has a loop; but this
contradicts the fact that $D$ has no loops. Hence, the cycle
$\mathbf{c}$ cannot have length $1$. Therefore, this cycle must have
length $2$ (since we know that $\mathbf{c}$ has length $1$ or $2$).
Hence, this cycle $\mathbf{c}$ has the form $\tup{u, v, u}$ for some
vertices $u$ and $v$ of $D$. Consider these $u$ and $v$. Thus, both
$\tup{u, v}$ and $\tup{v, u}$ are arcs of $D$. But this contradicts
the fact that \textbf{exactly} one of the pairs
$\tup{u, v}$ and $\tup{v, u}$ is an arc of $D$ (since $D$ is a
tournament). This contradiction proves that our assumption was wrong.
Hence, Observation 1 is proven.
\end{proof}

\begin{statement}
\textit{Observation 2:} Let $D$ be a tournament. Let $\mathbf{c}$ be a
cycle of $D$. Let $D''$ be the tournament obtained from $D$ by
reversing the cycle $\mathbf{c}$. Then, $D''$ can also be obtained
from $D$ by a sequence of $3$-cycle reversal operations.
\end{statement}

\begin{proof}[Proof of Observation 2.]
We shall prove Observation 2 by strong induction over the length of
$\mathbf{c}$. Thus, we fix an integer $k$, and we assume (as the
induction hypothesis) that Observation 2 is proven in the case when
the cycle $\mathbf{c}$ has length $< k$. We must now prove
Observation 2 in the case when the cycle $\mathbf{c}$ has length $k$.

So let us consider the situation of Observation 2, and assume that
the cycle $\mathbf{c}$ has length $k$. Write this cycle $\mathbf{c}$
in the form $\tup{v_0, v_1, \ldots, v_k}$; thus, the vertices
$v_0, v_1, \ldots, v_{k-1}$ of $D$ are distinct, but $v_0 = v_k$.
Observation 1 shows that $\mathbf{c}$ has length $\geq 3$; in other
words, we have $k \geq 3$ (since $k$ is the length of $\mathbf{c}$).

We want to prove that $D''$ can be obtained from $D$ by a sequence of
$3$-cycle reversal operations.
% If $k = 3$, then this is
% easy\footnote{\textit{Proof.} Assume that $k = 3$. Hence, the cycle
% $\mathbf{c}$ has the form $\tup{u, v, w, u}$ for some three distinct
% vertices $u$, $v$ and $w$ of $D$ (since the length of $\mathbf{c}$ is
% $k = 3$). Consider these $u$, $v$ and $w$. Since
% $\tup{u, v, w, u} = \mathbf{c}$ is a cycle, we conclude that
% $\tup{u, v}$, $\tup{v, w}$ and $\tup{w, u}$ are arcs of $D$. Hence,
% $\tup{u, v, w}$ is a $3$-cycle of $D$.

% Thus, the tournament $D'_{u, v, w}$ defined in
% Exercise~\ref{exe.tourn.3-cyc-rev} is obtained from $D$ by reversing
% the arcs $\tup{u, v}$, $\tup{v, w}$ and $\tup{w, u}$. Meanwhile, the
% tournament $D''$ is also obtained from $D$ by reversing
% the arcs $\tup{u, v}$, $\tup{v, w}$ and $\tup{w, u}$ (because $D''$
% is obtained from $D$ by reversing the cycle $\mathbf{c}$, but this
% cycle is $\tup{u, v, w, u}$). Hence, the two tournaments
% $D'_{u, v, w}$ and $D''$ are obtained from $D$ by reversing the same
% arcs. Therefore, these two tournaments are equal. In other words,
% $D'' = D'_{u, v, w}$. Hence, $D''$ can be obtained from $D$ by a
% $3$-cycle reversal (since $D'_{u, v, w}$ can be obtained from $D$ by a
% $3$-cycle reversal). Therefore, $D''$ can also be obtained
% from $D$ by a sequence of $3$-cycle reversals (namely, by the sequence
% consisting only of this one $3$-cycle reversal). Qed.}. Hence, we
% WLOG assume that $k \neq 3$. Thus, $k > 3$ (since $k \geq 3$).

Recall that $v_0, v_1, \ldots, v_{k-1}$ are distinct. Since
$k \geq 3$, this shows that $v_0$ and $v_2$ are distinct. Hence,
exactly one of $\tup{v_0, v_2}$ and $\tup{v_2, v_0}$ is an arc of $D$
(since $D$ is a tournament). We thus are in one of the following two
cases:

\begin{itemize}
\item \textit{Case 1:} The pair $\tup{v_0, v_2}$ is an arc of $D$.
\item \textit{Case 2:} The pair $\tup{v_2, v_0}$ is an arc of $D$.
\end{itemize}

We consider each of these two cases separately:

\begin{itemize}
\item Let us consider Case 1 first. In this case, the pair
      $\tup{v_0, v_2}$ is an arc of $D$. Hence,
      $\tup{v_0, v_2, v_3, \ldots, v_k}$ (this is just the list
      $\tup{v_0, v_1, v_2, \ldots, v_k}$ with the vertex $v_1$
      removed) is a circuit of $D$ (since
      $\tup{v_0, v_1, v_2, \ldots, v_k} = \mathbf{c}$ is a cycle of
      $D$), and furthermore is a cycle (since the vertices
      $v_0, v_2, v_3, \ldots, v_{k-1}$ are pairwise distinct, and
      since $k \geq 3$) and has
      length $k-1 < k$. Denote this cycle by $\mathbf{c}'$.
      Let $D_1$ be the tournament obtained from $D$ by
      reversing the cycle $\mathbf{c}'$. \\
      Recall that (by the induction hypothesis) Observation 2 is
      proven in the case when the cycle $\mathbf{c}$ has length $< k$.
      Hence, we can apply Observation 2 to $\mathbf{c}'$ and $D_1$
      instead of $\mathbf{c}$ and $D''$ (since the cycle $\mathbf{c}'$
      has length $< k$). Thus, we conclude that $D_1$ can also be
      obtained from $D$ by a sequence of $3$-cycle reversal
      operations. \\
      Next, we observe that the arc $\tup{v_0, v_2}$ of $D$ has been
      reversed when we reversed the cycle $\mathbf{c}'$. Therefore,
      the tournament $D_1$ (unlike $D$) has no arc $\tup{v_0, v_2}$,
      but instead has the arc $\tup{v_2, v_0}$. On the other hand, the
      two arcs $\tup{v_0, v_1}$ and $\tup{v_1, v_2}$ have not been
      modified when we reversed the cycle $\mathbf{c}'$ (since these
      arcs are not part of the cycle $\mathbf{c}'$). Hence, these two
      arcs are arcs of $D_1$ as well. Thus, we know that
      $\tup{v_0, v_1}$, $\tup{v_1, v_2}$ and $\tup{v_2, v_0}$ are arcs
      of $D_1$. Therefore, $\tup{v_0, v_1, v_2}$ is a $3$-cycle of
      $D_1$. Let $D_2$ be the tournament obtained from $D_1$ by
      reversing this $3$-cycle $\tup{v_0, v_1, v_2}$. Thus, $D_2$ is
      obtained from $D_1$ by a $3$-cycle reversal operation.
      Since $D_1$ (in turn) is obtained from $D$ by a sequence of
      $3$-cycle reversal operations,
      we thus conclude that $D_2$ is obtained from $D$ by a sequence
      of $3$-cycle reversal operations.\\
      But $D_2 = D''$\ \ \ \ \footnote{\textit{Proof.} Recall how
      $D_2$ was obtained from $D$:
      \begin{itemize}
      \item First, we obtained $D_1$ from $D$ by reversing the cycle
            $\mathbf{c}' = \tup{v_0, v_2, v_3, \ldots, v_k}$ in $D$.
            This amounts to reversing the arcs
            $\tup{v_0, v_2}, \tup{v_2, v_3}, \tup{v_3, v_4},
            \ldots, \tup{v_{k-1}, v_k}$.
      \item Then, we obtained $D_2$ from $D_1$ by reversing the
            $3$-cycle $\tup{v_0, v_1, v_2}$.
            This amounts to reversing the arcs
            $\tup{v_0, v_1}$, $\tup{v_1, v_2}$ and $\tup{v_2, v_0}$.
      \end{itemize}
      Thus, in total, we have reversed the arcs
      $\tup{v_0, v_2}, \tup{v_2, v_3}, \tup{v_3, v_4},
      \ldots, \tup{v_{k-1}, v_k}$ and then the three arcs
      $\tup{v_0, v_1}, \tup{v_1, v_2}, \tup{v_2, v_0}$
      to obtain $D_2$ from $D$.
      Clearly, the reversal of the arc
      $\tup{v_0, v_2}$ was undone by the (later) reversal of the
      arc $\tup{v_2, v_0}$; therefore, we can forget about these two
      reversals. Hence, $D_2$ is obtained from $D$ by reversing the
      arcs $\tup{v_2, v_3}, \tup{v_3, v_4},
      \ldots, \tup{v_{k-1}, v_k}, \tup{v_0, v_1}, \tup{v_1, v_2}$.
      But this is tantamount to reversing the cycle $\mathbf{c}$
      (since $\mathbf{c} = \tup{v_0, v_1, v_2, \ldots, v_k}$).
      Hence, $D_2$ is obtained from $D$ by reversing the cycle
      $\mathbf{c}$. But $D''$ is obtained from $D$ in exactly the same
      way (i.e., by reversing the cycle $\mathbf{c}$). Hence,
      $D_2 = D''$.}.
      But we know that $D_2$ is obtained from $D$ by a sequence of
      $3$-cycle reversal operations.
      In other words, $D''$ is obtained from $D$ by a sequence of
      $3$-cycle reversal operations. This proves what we wanted
      to prove in Case 1.
\item The argument in Case 2 is closely similar to the one we gave for
      Case 1. The only difference is the following: In Case 1, we have
      \textbf{first} reversed the cycle
      $\mathbf{c}' = \tup{v_0, v_2, v_3, \ldots, v_k}$, thus obtaining
      a tournament $D_1$, and
      \textbf{then} reversed the $3$-cycle $\tup{v_0, v_1, v_2}$ in
      $D_1$, thus obtaining a new tournament $D_2$ which was equal to
      $D''$. In contrast, this time, we have to proceed the other way
      round: We have to
      \textbf{first} reverse the $3$-cycle $\tup{v_0, v_1, v_2}$,
      thus obtaining a tournament $D_1$, and
      \textbf{then} reverse the cycle
      $\mathbf{c}' = \tup{v_0, v_2, v_3, \ldots, v_k}$ in $D_1$, thus
      obtaining a new tournament $D_2$ which again is equal to $D''$.
      Apart from this, nothing changes.
\end{itemize}

Hence, in either case, we have shown that
$D''$ can be obtained from $D$ by a sequence of $3$-cycle reversal
operations.
In other words, Observation 2 holds for our $D$ and $\mathbf{c}$. This
completes the induction step. Hence, Observation 2 is proven by
strong induction.
\end{proof}

\begin{statement}
\textit{Observation 3:} Let $V$ be a finite set. Let $E$ and $F$ be
two tournaments with vertex set $V$. Assume that $F$ can be obtained
from $E$ by a sequence of $3$-cycle reversal operations.
Then, each $v \in V$ satisfies $\deg^-_E \tup{v} = \deg^-_F \tup{v}$.
\end{statement}

\begin{proof}[Proof of Observation 3.]
Fix $x \in V$. We shall prove that
\begin{equation}
\deg^-_E \tup{x} = \deg^-_F \tup{x}
.
\label{sol.tourn.3-cyc-rev.o3.pf.1}
\end{equation}
It is clearly enough to prove
\eqref{sol.tourn.3-cyc-rev.o3.pf.1} in the case when $F$ can
be obtained from $E$ by \textbf{one} $3$-cycle reversal operation
(because then, the validity of \eqref{sol.tourn.3-cyc-rev.o3.pf.1}
in the general case would follow by induction). So we
WLOG assume that $F$ can
be obtained from $E$ by \textbf{one} $3$-cycle reversal operation.
In other words, there exists a $3$-cycle $\tup{u, v, w}$ of $E$
such that $F$ can be obtained from $E$ by reversing the $3$-cycle
$\tup{u, v, w}$. Consider this $\tup{u, v, w}$.

If $x \notin \set{u, v, w}$, then the arcs of $E$ having target
$x$ are precisely the arcs of $F$ having target $x$ (since $F$
can be obtained from $E$ by reversing the $3$-cycle
$\tup{u, v, w}$, but this reversal clearly does not affect the
arcs having target $x$). Therefore, if $x \notin \set{u, v, w}$,
then $\deg^-_E \tup{x} = \deg^-_F \tup{x}$. In other words,
\eqref{sol.tourn.3-cyc-rev.o3.pf.1} is proven in the case when
$x \notin \set{u, v, w}$.
Hence, we WLOG assume that we don't have
$x \notin \set{u, v, w}$. In other words, we have
$x \in \set{u, v, w}$. In other words, either $x=u$ or $x=v$ or
$x=w$. We WLOG assume that $x=u$ (since the other two cases are
similar).

Reversing the $3$-cycle $\tup{u, v, w}$ removes the arcs
$\tup{u, v}, \tup{v, w}, \tup{w, u}$ from the digraph $E$ while
adding the arcs $\tup{v, u}, \tup{w, v}, \tup{u, w}$. Therefore, in
total, one arc having target $u$ is removed (namely, the arc
$\tup{w, u}$), and one arc having target $u$ is added (namely, the
arc $\tup{v, u}$). As a consequence, the number of arcs of $F$
having target $u$ is obtained from the number of arcs of $E$ having
target $u$ by subtracting $1$ and then adding $1$ back. In other
words,
$\deg^-_F \tup{u} = \deg^-_E \tup{u} - 1 + 1 = \deg^-_E \tup{u}$.
Hence, $\deg^-_E \tup{u} = \deg^-_F \tup{u}$. Since $x = u$, this
rewrites as $\deg^-_E \tup{x} = \deg^-_F \tup{x}$.
Thus, \eqref{sol.tourn.3-cyc-rev.o3.pf.1} is proven.

Now, forget that we fixed $x$. We thus have shown that
each $x \in V$ satisfies $\deg^-_E \tup{x} = \deg^-_F \tup{x}$.
Renaming $x$ as $v$ in this statement, we conclude that
each $v \in V$ satisfies $\deg^-_E \tup{v} = \deg^-_F \tup{v}$.
This proves Observation 3.
\end{proof}

\begin{statement}
\textit{Observation 4:} Let $V$ be a finite set. Let
$E = \tup{V, A}$ and $F = \tup{V, B}$ be
two digraphs with vertex set $V$. Let $v \in V$.

\textbf{(a)} We have
$\deg^-_{\tup{V, A\setminus B}} \tup{v}
- \deg^-_{\tup{V, B\setminus A}} \tup{v}
= \deg^-_E \tup{v} - \deg^-_F \tup{v}$.

\textbf{(b)} Assume that $E$ and $F$ are tournaments.
Assume furthermore that
$\deg^-_E \tup{v} = \deg^-_F \tup{v}$. Then,
$\deg^-_{\tup{V, A\setminus B}} \tup{v}
= \deg^+_{\tup{V, A\setminus B}} \tup{v}$.
\end{statement}

\begin{proof}[Proof of Observation 4.]
Recall first that
\begin{align}
\tup{\text{the number of all } u \in V \text{ satisfying }
      \tup{u, v} \in A}
 = \deg^-_E \tup{v}
\label{sol.tourn.3-cyc-rev.o4.pf.deg-}
\end{align}
\footnote{\textit{Proof of
\eqref{sol.tourn.3-cyc-rev.o4.pf.deg-}.} Recall
that $E = \tup{V, A}$ is a digraph (not a multidigraph);
therefore, the arcs of $E$ are pairs of vertices in $V$.
Hence, it is easy to check that the two maps
\begin{align*}
\tup{\text{the set of all } u \in V \text{ satisfying }
     \tup{u, v} \in A}
&\to
\tup{\text{the set of all arcs of } E \text{ whose target is } v}, \\
u &\mapsto \tup{u, v}
\end{align*}
and
\begin{align*}
\tup{\text{the set of all arcs of } E \text{ whose target is } v}
&\to
\tup{\text{the set of all } u \in V \text{ satisfying }
     \tup{u, v} \in A}
, \\
a &\mapsto \tup{\text{the source of } a}
\end{align*}
are mutually inverse. Thus, they are inverse bijections between
the set of all arcs of $E$ whose target is $v$ and the set of all
$u \in V$ satisfying $\tup{u, v} \in A$. Consequently, the size
of the latter set equals the size of the former set. In other
words, the number of all $u \in V$ satisfying $\tup{u, v} \in A$
equals the number of all arcs of $E$ whose target is $v$. But
since the latter number is $\deg^-_E \tup{v}$ (in fact, this is how
$\deg^-_E \tup{v}$ is defined), this rewrites as follows: The number
of all $u \in V$ satisfying $\tup{u, v} \in A$ equals
$\deg^-_E \tup{v}$. This proves
\eqref{sol.tourn.3-cyc-rev.o4.pf.deg-}.}. Hence,
\begin{align}
\sum_{\substack{u \in V;\\ \tup{u, v} \in A}} 1
&= \underbrack{\tup{\text{the number of all } u \in V \text{ satisfying }
                \tup{u, v} \in A}}{= \deg^-_E \tup{v}} \cdot 1
\nonumber \\
&= \deg^-_E \tup{v} \cdot 1
= \deg^-_E \tup{v} .
\label{sol.tourn.3-cyc-rev.o4.pf.A}
\end{align}
The same argument (applied to $F$ and $B$ instead of $E$ and $A$) shows
that
\begin{align}
\sum_{\substack{u \in V;\\ \tup{u, v} \in B}} 1
= \deg^-_F \tup{v} .
\label{sol.tourn.3-cyc-rev.o4.pf.B}
\end{align}
Furthermore, the same argument that we used to prove
\eqref{sol.tourn.3-cyc-rev.o4.pf.A} can be applied to
$\tup{V, A \setminus B}$ and $A \setminus B$ instead of $E$ and $A$.
As a result, we find that
\begin{align}
\sum_{\substack{u \in V;\\ \tup{u, v} \in A \setminus B}} 1
= \deg^-_{\tup{V, A\setminus B}} \tup{v} .
\label{sol.tourn.3-cyc-rev.o4.pf.A-B}
\end{align}
Finally, the same argument that we used to prove
\eqref{sol.tourn.3-cyc-rev.o4.pf.A} can be applied to
$\tup{V, B \setminus A}$ and $B \setminus A$ instead of $E$ and $A$.
As a result, we find that
\begin{align}
\sum_{\substack{u \in V;\\ \tup{u, v} \in B \setminus A}} 1
= \deg^-_{\tup{V, B\setminus A}} \tup{v} .
\label{sol.tourn.3-cyc-rev.o4.pf.B-A}
\end{align}

Now, \eqref{sol.tourn.3-cyc-rev.o4.pf.A} yields
\begin{align}
& \deg^-_E \tup{v} \nonumber \\
&= \sum_{\substack{u \in V;\\ \tup{u, v} \in A}} 1
=
\underbrack{\sum_{\substack{u \in V;\\ \tup{u, v} \in A;\\
                                \tup{u, v} \in B}} 1}
           {= \sum_{\substack{u \in V;\\ \tup{u, v} \in A \cap B}} 1}
+
\underbrack{\sum_{\substack{u \in V;\\ \tup{u, v} \in A;\\
                                \tup{u, v} \notin B}} 1}
           {= \sum_{\substack{u \in V;\\ \tup{u, v} \in A \setminus
                                B}} 1} \nonumber \\
& \qquad \left(\text{since each } u \in V \text{ satisfies either }
                \tup{u, v} \in B \text{ or } \tup{u, v} \notin B
                \text{, but not both}\right) \nonumber \\
&= \sum_{\substack{u \in V;\\ \tup{u, v} \in A \cap B}} 1
   + \sum_{\substack{u \in V;\\ \tup{u, v} \in A \setminus B}} 1 .
\label{sol.tourn.3-cyc-rev.o4.pf.A2}
\end{align}
Meanwhile, \eqref{sol.tourn.3-cyc-rev.o4.pf.B} yields
\begin{align}
& \deg^-_F \tup{v} \nonumber \\
&= \sum_{\substack{u \in V;\\ \tup{u, v} \in B}} 1
=
\underbrack{\sum_{\substack{u \in V;\\ \tup{u, v} \in B;\\
                                \tup{u, v} \in A}} 1}
           {= \sum_{\substack{u \in V;\\ \tup{u, v} \in A \cap B}} 1}
+
\underbrack{\sum_{\substack{u \in V;\\ \tup{u, v} \in B;\\
                                \tup{u, v} \notin A}} 1}
           {= \sum_{\substack{u \in V;\\ \tup{u, v} \in B \setminus
                                A}} 1} \nonumber \\
& \qquad \left(\text{since each } u \in V \text{ satisfies either }
                \tup{u, v} \in A \text{ or } \tup{u, v} \notin A
                \text{, but not both}\right) \nonumber \\
&= \sum_{\substack{u \in V;\\ \tup{u, v} \in A \cap B}} 1
   + \sum_{\substack{u \in V;\\ \tup{u, v} \in B \setminus A}} 1 .
\label{sol.tourn.3-cyc-rev.o4.pf.B2}
\end{align}
Subtracting \eqref{sol.tourn.3-cyc-rev.o4.pf.B2} from
\eqref{sol.tourn.3-cyc-rev.o4.pf.A2}, we find
\begin{align*}
& \deg^-_E \tup{v} - \deg^-_F \tup{v} \\
&= \tup{\sum_{\substack{u \in V;\\ \tup{u, v} \in A \cap B}} 1
        + \sum_{\substack{u \in V;\\ \tup{u, v} \in A \setminus B}} 1}
 - \tup{\sum_{\substack{u \in V;\\ \tup{u, v} \in A \cap B}} 1
        + \sum_{\substack{u \in V;\\ \tup{u, v} \in B \setminus A}} 1}
\\
&= \underbrack{\sum_{\substack{u \in V;\\ \tup{u, v} \in A
                        \setminus B}} 1}
              {= \deg^-_{\tup{V, A\setminus B}} \tup{v} \\
               \text{(by \eqref{sol.tourn.3-cyc-rev.o4.pf.A-B})}}
 - \underbrack{\sum_{\substack{u \in V;\\ \tup{u, v} \in B
                        \setminus A}} 1}
              {= \deg^-_{\tup{V, B\setminus A}} \tup{v} \\
               \text{(by \eqref{sol.tourn.3-cyc-rev.o4.pf.B-A})}}
= \deg^-_{\tup{V, A\setminus B}} \tup{v}
- \deg^-_{\tup{V, B\setminus A}} \tup{v}.
\end{align*}
This proves Observation 4 \textbf{(a)}.

\textbf{(b)} The summation sign
$\sum_{\substack{u \in V;\\ \tup{u, v} \in B \setminus A}}$ in
\eqref{sol.tourn.3-cyc-rev.o4.pf.B-A} can be rewritten as
$\sum_{\substack{u \in V;\\ \tup{v, u} \in A \setminus B}}$
(because of the equivalence \eqref{sol.tourn.3-cyc-rev.o4.pf.equiv}).
Thus, \eqref{sol.tourn.3-cyc-rev.o4.pf.B-A} rewrites as follows:
\begin{align}
\sum_{\substack{u \in V;\\ \tup{v, u} \in A \setminus B}} 1
= \deg^-_{\tup{V, B\setminus A}} \tup{v} .
\label{sol.tourn.3-cyc-rev.o4.pf.B-A2}
\end{align}

On the other hand,
\begin{align}
\tup{\text{the number of all } u \in V \text{ satisfying }
      \tup{v, u} \in A}
 = \deg^+_E \tup{v}
\label{sol.tourn.3-cyc-rev.o4.pf.deg+}
\end{align}
\footnote{The proof of
\eqref{sol.tourn.3-cyc-rev.o4.pf.deg+} is analogous to that of
\eqref{sol.tourn.3-cyc-rev.o4.pf.deg-}, and so is left to the
reader.}. Hence,
\begin{align}
\sum_{\substack{u \in V;\\ \tup{v, u} \in A}} 1
&= \underbrack{\tup{\text{the number of all } u \in V \text{ satisfying }
                \tup{v, u} \in A}}{= \deg^+_E \tup{v}} \cdot 1
\nonumber \\
&= \deg^+_E \tup{v} \cdot 1
= \deg^+_E \tup{v} .
\end{align}
The same argument (applied to $\tup{V, A \setminus B}$ and
$A \setminus B$ instead of $E$ and $A$) shows
that
\[
\sum_{\substack{u \in V;\\ \tup{v, u} \in A \setminus B}} 1
= \deg^+_{\tup{V, A \setminus B}} \tup{v} .
\]
Comparing this with \eqref{sol.tourn.3-cyc-rev.o4.pf.B-A2}, we find
$\deg^-_{\tup{V, B\setminus A}} \tup{v}
= \deg^+_{\tup{V, A \setminus B}} \tup{v}$. Now,
Observation 4 \textbf{(a)} yields
$\deg^-_{\tup{V, A\setminus B}} \tup{v}
- \deg^-_{\tup{V, B\setminus A}} \tup{v}
= \deg^-_E \tup{v} - \deg^-_F \tup{v} = 0$ (since
$\deg^-_E \tup{v} = \deg^-_F \tup{v}$). Hence,
$\deg^-_{\tup{V, A\setminus B}} \tup{v}
= \deg^-_{\tup{V, B\setminus A}} \tup{v}
= \deg^+_{\tup{V, A \setminus B}} \tup{v}$.
This proves Observation 4 \textbf{(b)}.
\end{proof}

\begin{statement}
\textit{Observation 5:} Let $V$ be a finite set. Let
$E = \tup{V, A}$ and $F = \tup{V, B}$ be
two tournaments with vertex set $V$ such that $A \neq B$.
Assume that each
$v \in V$ satisfies $\deg^-_E \tup{v} = \deg^-_F \tup{v}$.
Then:

\textbf{(a)} The digraph $\tup{V, A \setminus B}$ has at least one
cycle.

\textbf{(b)} Each cycle of the digraph $\tup{V, A \setminus B}$
is also a cycle of $E$.

\textbf{(c)} Let $\tup{v_0, v_1, \ldots, v_k}$ be a cycle of the
digraph $\tup{V, A \setminus B}$. Then,
\begin{align*}
& \set{\tup{v_0, v_1}, \tup{v_1, v_2}, \ldots,
       \tup{v_{k-1}, v_k}} \subseteq A \setminus B
\qquad \text{and} \\
& \set{\tup{v_1, v_0}, \tup{v_2, v_1}, \ldots,
       \tup{v_k, v_{k-1}}} \subseteq B \setminus A .
\end{align*}
\end{statement}

\begin{proof}[Proof of Observation 5.]
\textbf{(a)} The set
$A \setminus B$ is nonempty\ \ \ \ \footnote{\textit{Proof.}
Assume the contrary. Thus, $A \setminus B = \varnothing$.
Hence, $A \subseteq B$. Since $A \neq B$, we thus conclude that
$A$ is a proper subset of $B$. Hence, there exists some
$\tup{u, v} \in B \setminus A$. Consider this $\tup{u, v}$.
From $\tup{u, v} \in B \setminus A$, we obtain
$\tup{v, u} \in A \setminus B$ (by the equivalence
\eqref{sol.tourn.3-cyc-rev.o4.pf.equiv}). Hence,
$\tup{v, u} \in A \setminus B = \varnothing$, which is absurd.
Hence, we have obtained a contradiction. Therefore, our
assumption was false, qed.}.
Furthermore, each vertex $v \in V$ satisfies
$\deg^-_{\tup{V, A \setminus B}} v
= \deg^+_{\tup{V, A \setminus B}} v$\ \ \ \ \footnote{\textit{Proof.}
Let $v \in V$. The hypothesis of Observation 5 yields
$\deg^-_E \tup{v} = \deg^-_F \tup{v}$. Hence,
Observation 4 \textbf{(b)} shows that
$\deg^-_{\tup{V, A\setminus B}} \tup{v}
= \deg^+_{\tup{V, A\setminus B}} \tup{v}$.
Qed.}. Thus, Corollary~\ref{cor.digraph.balanced-cycle.digraph}
(applied to $\tup{V, A \setminus B}$ and
$A \setminus B$ instead of $D$ and $A$) yields that the
digraph $\tup{V, A \setminus B}$ has at least one cycle.
This proves Observation 5 \textbf{(a)}.

\textbf{(b)} Each arc of the digraph $\tup{V, A \setminus B}$ is
also an arc of $\tup{V, A}$ (since $A \setminus B \subseteq A$). In
other words, each arc of the digraph $\tup{V, A \setminus B}$ is
also an arc of $E$ (since $E = \tup{V, A}$). Therefore, each cycle
of the digraph $\tup{V, A \setminus B}$ is also a cycle of $E$.
This proves Observation 5 \textbf{(b)}.

\textbf{(c)} The arcs
$\tup{v_0, v_1}, \tup{v_1, v_2}, \ldots, \tup{v_{k-1}, v_k}$ are the
arcs of the cycle $\tup{v_0, v_1, \ldots, v_k}$ of the
digraph $\tup{V, A \setminus B}$, and thus are arcs of the digraph
$\tup{V, A \setminus B}$. In other words, they belong to
$A \setminus B$. In other words,
$\set{\tup{v_0, v_1}, \tup{v_1, v_2}, \ldots,
       \tup{v_{k-1}, v_k}} \subseteq A \setminus B$.

On the other hand, each $i \in \set{0, 1, \ldots, k-1}$ satisfies
$\tup{v_{i+1}, v_i} \in B \setminus A$
\ \ \ \ \footnote{\textit{Proof.} Let $i \in \set{0, 1, \ldots, k-1}$.
We know that the arcs
$\tup{v_0, v_1}, \tup{v_1, v_2}, \ldots, \tup{v_{k-1}, v_k}$ belong
to $A \setminus B$. In particular, $\tup{v_i, v_{i+1}}$ belongs to
$A \setminus B$. In other words, $\tup{v_i, v_{i+1}} \in
A \setminus B$. But \eqref{sol.tourn.3-cyc-rev.o4.pf.equiv}
(applied to $v_{i+1}$ and $v_i$ instead of $u$ and $v$) yields that
we have the logical equivalence
\[
\tup{ \tup{v_{i+1}, v_i} \in B \setminus A }
\Longleftrightarrow
\tup{ \tup{v_i, v_{i+1}} \in A \setminus B } .
\]
Hence, we have $\tup{v_{i+1}, v_i} \in B \setminus A$ (since we know
that $\tup{v_i, v_{i+1}} \in A \setminus B$). Qed.}. In other words,
$\set{\tup{v_1, v_0}, \tup{v_2, v_1}, \ldots,
       \tup{v_k, v_{k-1}}} \subseteq B \setminus A$.
The proof of Observation 5 \textbf{(c)} is now complete.
\end{proof}

\begin{statement}
\textit{Observation 6:} Let $V$ be a finite set. Let
$E = \tup{V, A}$ and $F = \tup{V, B}$ be
two tournaments with vertex set $V$ such that $A \neq B$.
Assume that each
$v \in V$ satisfies $\deg^-_E \tup{v} = \deg^-_F \tup{v}$.

Let $\mathbf{c}$ be a cycle of the digraph $\tup{V, A \setminus B}$.
Thus, $\mathbf{c}$ is also a cycle of $E$ (by
Observation 5 \textbf{(b)}).
Let $E'' = \tup{V, A''}$ be the tournament obtained from $E$ by
reversing the cycle $\mathbf{c}$. (This is well-defined,
since $\mathbf{c}$ is a cycle of $E$.) Then:

\textbf{(a)} We have $\abs{A'' \setminus B} < \abs{A \setminus B}$.

\textbf{(b)} Each $v \in V$ satisfies
$\deg^-_{E''} \tup{v} = \deg^-_F \tup{v}$.

\textbf{(c)} The tournament $E''$ can be obtained from $E$ by a
sequence of $3$-cycle reversal operations.
\end{statement}

\begin{proof}[Proof of Observation 6.]
Write the cycle $\mathbf{c}$ in the form
$\tup{v_0, v_1, \ldots, v_k}$ (with $v_0, v_1, \ldots, v_k \in V$).
Then, Observation 5 \textbf{(c)} yields
\begin{align}
& \set{\tup{v_0, v_1}, \tup{v_1, v_2}, \ldots,
       \tup{v_{k-1}, v_k}} \subseteq A \setminus B
\qquad \text{and}
\label{sol.tourn.3-cyc-rev.o6.pf.1}\\
& \set{\tup{v_1, v_0}, \tup{v_2, v_1}, \ldots,
       \tup{v_k, v_{k-1}}} \subseteq B \setminus A .
\label{sol.tourn.3-cyc-rev.o6.pf.2}
\end{align}

Set $X = \set{\tup{v_0, v_1}, \tup{v_1, v_2}, \ldots,
              \tup{v_{k-1}, v_k}}$
and $Y = \set{\tup{v_1, v_0}, \tup{v_2, v_1}, \ldots,
              \tup{v_k, v_{k-1}}}$.
Thus, the relations \eqref{sol.tourn.3-cyc-rev.o6.pf.1}
and \eqref{sol.tourn.3-cyc-rev.o6.pf.2} rewrite as
$X \subseteq A \setminus B$ and
$Y \subseteq B \setminus A$, respectively.
In particular, no element of $X$ belongs to $B$ (since
$X \subseteq A \setminus B$); thus, $X \setminus B = X$.
Also, $Y \subseteq B \setminus A \subseteq B$, so that
$Y \setminus B = \varnothing$.

Also, clearly, $k \geq 1$ (since
$\tup{v_0, v_1, \ldots, v_k} = \mathbf{c}$ is a cycle),
and thus the set $X$ is nonempty. Hence, $\abs{X} > 0$.

\textbf{(a)} Recall that the tournament $E''$ is obtained
from $E$ by reversing the cycle
$\mathbf{c} = \tup{v_0, v_1, \ldots, v_k}$. In other words, the
tournament $E''$ is obtained from $E$ by reversing the arcs
$\tup{v_0, v_1}, \tup{v_1, v_2}, \ldots, \tup{v_{k-1}, v_k}$
(because this is how ``reversing the cycle
$\tup{v_0, v_1, \ldots, v_k}$'' was defined). In other words, the
tournament $E''$ is obtained from $E$ by removing the arcs
$\tup{v_0, v_1}, \tup{v_1, v_2}, \ldots, \tup{v_{k-1}, v_k}$
and adding the new arcs
$\tup{v_1, v_0}, \tup{v_2, v_1}, \ldots, \tup{v_k, v_{k-1}}$.
Since the arc set of $E''$ is $A''$, whereas the arc set of $E$
is $A$, we therefore have
\begin{align*}
A''
&= \tup{A \setminus
   \underbrack{\set{\tup{v_0, v_1}, \tup{v_1, v_2},
                    \ldots, \tup{v_{k-1}, v_k}}}
              {= X}}
   \cup
   \underbrack{\set{\tup{v_1, v_0}, \tup{v_2, v_1},
                    \ldots, \tup{v_k, v_{k-1}}}}
              {= Y} \\
&= \tup{A \setminus X} \cup Y .
\end{align*}
Hence,
\begin{align*}
A'' \setminus B
&= \tup{\tup{A \setminus X} \cup Y} \setminus B
= \tup{\tup{A \setminus B} \setminus
  \underbrack{\tup{X \setminus B}}{= X}} \cup
  \underbrack{\tup{Y \setminus B}}{= \varnothing}
= \tup{\tup{A \setminus B} \setminus X} \cup \varnothing \\
&= \tup{A \setminus B} \setminus X,
\end{align*}
and therefore
\begin{align*}
\abs{A'' \setminus B}
&= \abs{\tup{A \setminus B} \setminus X}
= \abs{A \setminus B} - \abs{X}
\qquad \left(\text{since } X \subseteq A \setminus B \right) \\
&< \abs{A \setminus B}
\qquad \left(\text{since } \abs{X} > 0\right) .
\end{align*}
This proves Observation 6 \textbf{(a)}.

\textbf{(c)} Observation 2 (applied to $E$ and $E''$ instead of
$D$ and $D''$) shows that $E''$ can also be obtained
from $E$ by a sequence of $3$-cycle reversal
operations. This proves Observation 6 \textbf{(c)}.

\textbf{(b)} Let $v \in V$.
Then, $\deg^-_E \tup{v} = \deg^-_F \tup{v}$ (by the hypothesis
of Observation 6). But we know (from Observation 6 \textbf{(c)})
that the tournament $E''$ can also be obtained
from $E$ by a sequence of $3$-cycle reversal operations. Hence,
Observation 3 (applied to $E''$ instead of $F$) shows that
$\deg^-_E \tup{v} = \deg^-_{E''} \tup{v}$. Therefore,
$\deg^-_{E''} \tup{v} = \deg^-_E \tup{v} = \deg^-_F \tup{v}$.
This proves Observation 6 \textbf{(b)}.
\end{proof}

\begin{statement}
\textit{Observation 7:} Let $V$ be a finite set. Let $E$ and $F$ be
two tournaments with vertex set $V$. Assume that
each $v \in V$ satisfies $\deg^-_E \tup{v} = \deg^-_F \tup{v}$.
Then, $F$ can be obtained
from $E$ by a sequence of $3$-cycle reversal operations.
\end{statement}

\begin{proof}[Proof of Observation 7.]
We shall prove Observation 7 by strong induction over
$\abs{\arcs{E} \setminus \arcs{F}}$. Thus, we fix some $N \in \NN$,
and we assume (as the induction hypothesis) that Observation 7 is
already proven in the case when
$\abs{\arcs{E} \setminus \arcs{F}} < N$. In other words, if $E$ and
$F$ are two tournaments with vertex set $V$, if each $v \in V$
satisfies $\deg^-_E \tup{v} = \deg^-_F \tup{v}$, and if we have
$\abs{\arcs{E} \setminus \arcs{F}} < N$, then
\begin{equation}
F \text{ can be obtained from } E
\text{ by a sequence of }3\text{-cycle reversal operations}.
\label{sol.tourn.3-cyc-rev.o7.pf.IH}
\end{equation}

Now, we need to prove Observation 7 in the case when
$\abs{\arcs{E} \setminus \arcs{F}} = N$.
So let $E$ and $F$ be two tournaments wirth vertex set $V$, and
assume that each $v \in V$ satisfies
$\deg^-_E \tup{v} = \deg^-_F \tup{v}$. Assume furthermore that
$\abs{\arcs{E} \setminus \arcs{F}} = N$.
Our goal is to prove that $F$ can be obtained
from $E$ by a sequence of $3$-cycle reversal operations.

Write the tournaments $E$ and $F$ in
the forms $E = \tup{V, A}$ and $F = \tup{V, B}$. (This is possible,
since both $E$ and $F$ have vertex set $V$.) Thus, $\arcs{E} = A$
and $\arcs{F} = B$.
If $A = B$, then
our claim is obvious\footnote{\textit{Proof.} Assume that
$A = B$. Thus, $\tup{V, A} = \tup{V, B}$, so that
$E = \tup{V, A} = \tup{V, B} = F$. Hence, $F$ can be obtained
from $E$ by a sequence of $3$-cycle reversal operations
(namely, by the empty sequence). But this is exactly what we have
to prove.}. Hence, we WLOG assume that we don't have $A = B$.
Thus, $A \neq B$.

At this point, all we need to do is combining observations that
we already have proven. Observation 5 \textbf{(a)} shows that
the digraph $\tup{V, A \setminus B}$ has at least one
cycle. Fix such a cycle, and denote it by $\mathbf{c}$. Then,
$\mathbf{c}$ is also a cycle of $E$ (by Observation 5 \textbf{(b)}).
Let $E'' = \tup{V, A''}$ be the tournament obtained from $E$ by
reversing the cycle $\mathbf{c}$. (This is well-defined,
since $\mathbf{c}$ is a cycle of $E$.) Thus,
$\arcs{E''} = A''$. Observation 6 \textbf{(b)}
shows that each $v \in V$ satisfies
$\deg^-_{E''} \tup{v} = \deg^-_F \tup{v}$.
Observation 6 \textbf{(a)} shows that
$\abs{A'' \setminus B} < \abs{A \setminus B}$.
Since $A'' = \arcs{E''}$, $A = \arcs{E}$ and $B = \arcs{F}$,
this rewrites as
$\abs{\arcs{E''} \setminus \arcs{F}}
< \abs{\arcs{E} \setminus \arcs{F}}$.
Since $\abs{\arcs{E} \setminus \arcs{F}} = N$, this furthermore
rewrites as
$\abs{\arcs{E''} \setminus \arcs{F}} < N$. Hence, we can apply
\eqref{sol.tourn.3-cyc-rev.o7.pf.IH} to $E''$ instead of $E$ (since
we have also shown that each $v \in V$ satisfies
$\deg^-_{E''} \tup{v} = \deg^-_F \tup{v}$). As a result, we conclude
that $F$ can be obtained from $E''$ by a sequence of
$3$-cycle reversal operations. But the tournament $E''$ can (in
turn) be obtained from $E$ by a sequence of $3$-cycle reversal
operations (by Observation 6 \textbf{(c)}). Combining the previous
two sentences, we conclude that $F$ can be obtained from $E$ by
a sequence of $3$-cycle reversal operations (indeed, we first apply
the sequence of $3$-cycle reversal operations that lets us obtain
$E''$ from $E$, and then apply the
the sequence of $3$-cycle reversal operations that lets us obtain
$F$ from $E''$). But this is exactly the claim that we wanted to
prove. Hence, we have proven Observation 7 in the case when
$\abs{\arcs{E} \setminus \arcs{F}} = N$.
Thus, the proof of Observation 7 (by strong induction) is complete.
\end{proof}

Exercise~\ref{exe.tourn.3-cyc-rev} now follows from Observation 3
and Observation 7. (Indeed, the claim of
Exercise~\ref{exe.tourn.3-cyc-rev} is an ``if and only if''
statement. The ``if'' part of this statement follows from
Observation 7, whereas the ``only if'' part follows from
Observation 3.)
\end{proof}

\subsection{Exercise~\ref{exe.tourn.2-path-rev}: Transforming
tournaments by reversing $2$-paths}

A tournament $D = \tup{V, A}$ is called \textit{transitive} if it
has no $3$-cycles.

\begin{exercise} \label{exe.tourn.2-path-rev}
If a tournament $D = \tup{V, A}$ has three distinct vertices
$u$, $v$ and $w$ satisfying $\tup{u, v} \in A$ and $\tup{v, w} \in A$,
then we can define a new tournament $D''_{u, v, w}$ as follows:
The vertices of $D''_{u, v, w}$ shall be the same as those of $D$. The
arcs of $D''_{u, v, w}$ shall be the same as those of $D$, except that
the two arcs $\tup{u, v}$ and $\tup{v, w}$ are replaced
by the two new arcs $\tup{v, u}$ and $\tup{w, v}$.
We say that the new tournament $D''_{u, v, w}$ is obtained
from the old tournament $D$ by a \textit{$2$-path reversal operation}.

Let $D$ be any tournament. Prove that there is a sequence of
$2$-path reversal operations that transforms $D$ into a transitive
tournament.
\end{exercise}

\begin{proof}[Solution sketch to Exercise~\ref{exe.tourn.2-path-rev}.]
We shall solve Exercise~\ref{exe.tourn.2-path-rev} by induction on
$\abs{V}$, where $V$ denotes the vertex set of $D$.

The \textit{induction base} (i.e., the case $\abs{V} = 0$)
is obvious (because in this case, $D$ is already transitive, and thus
the empty sequence of $2$-path reversal operations transforms $D$ into
a transitive tournament).

Now, to the \textit{induction step}. Fix a positive
integer $N$, and assume (as the induction hypothesis) that
Exercise~\ref{exe.tourn.2-path-rev} is already solved in the case when
$\abs{V} = N-1$. Now, we must solve
Exercise~\ref{exe.tourn.2-path-rev} in the case when
$\abs{V} = N$. So let us fix a tournament $D$ with vertex set $V$
satisfying $\abs{V} = N$.
Write $D$ in the form $D = \tup{V, A}$.

We say that a tournament is \textit{sinkless} if it has no vertex that
has outdegree $0$.

We shall now prove the following observation:

\begin{statement}
\textit{Observation 1:} Assume that the tournament $D$ is sinkless.
Let $u \in V$. Then, we can apply a $2$-path reversal operation to $D$
that decreases $\deg^+ u$ by $1$.
\end{statement}
\begin{proof}[Proof of Observation 1.]
The tournament $D$ is sinkless. In other words, it has no vertex that
has outdegree $0$. In particular, the vertex $u$ does not have
outdegree $0$. Hence, there exists at least one arc $\tup{u, v}$ of
$D$ having source $u$. Consider such an arc. Furthermore, the vertex
$v$ also does not have outdegree $0$ (since $D$ has no vertex that has
outdegree $0$). Thus, there exists at least one arc $\tup{v, w}$ of
$D$ having source $v$. Consider such an arc.

We have $u \neq v$ (since $\tup{u, v}$ is an arc of $D$) and
$v \neq w$ (since $\tup{v, w}$ is an arc of $D$). Also, if we had
$u = w$, then both $\tup{v, u} = \tup{v, w}$ and $\tup{u, v}$ would
be arcs of $D$, and this would contradict the fact that $D$ is a
tournament (indeed, a tournament has only one arc between two distinct
vertices). Hence, we cannot have $u = w$. Thus, $u \neq w$.
Now, the three vertices $u$, $v$ and $w$ of $D$ are distinct (since
$u \neq v$, $v \neq w$ and $u \neq w$) and satisfy
$\tup{u, v} \in A$ and $\tup{v, w} \in A$ (since $\tup{u, v}$ and
$\tup{v, w}$ are arcs of $D$). Hence, a new tournament $D''_{u, v, w}$
is defined. Recall that this new tournament $D''_{u, v, w}$ differs
from $D$ in that the two arcs $\tup{u, v}$ and $\tup{v, w}$ are
replaced by the two new arcs $\tup{v, u}$ and $\tup{w, v}$ (due to the
definition of $D''_{u, v, w}$). In particular, $D''_{u, v, w}$ is
lacking the arc $\tup{u, v}$ that $D$ used to have, but does not have
any new arcs (i.e., arcs that $D$ lacked) with source $u$. Thus,
$\deg^+_{D''_{u, v, w}} u = \deg^+_{D} u - 1$.

But the tournament $D''_{u, v, w}$ clearly is obtained from $D$ by a
$2$-path reversal operation. This $2$-path reversal operation has
decreased $\deg^+ u$ by $1$
(since $\deg^+_{D''_{u, v, w}} u = \deg^+_{D} u - 1$). Hence,
Observation 1 is proven.
\end{proof}

\begin{statement}
\textit{Observation 2:} We can apply a sequence of $2$-path reversal
operations to $D$ that ensures the following: The tournament obtained
at the end of this sequence is not sinkless.
\end{statement}
\begin{proof}[Proof of Observation 2.]
If $D$ already is not sinkless, then Observation 2 obviously holds
(just apply the empty sequence). Otherwise, fix any vertex $u \in V$.
(This is possible, since $\abs{V} = N > 0$.) Observation 1 shows that
we can apply a $2$-path reversal operation to $D$ that decreases
$\deg^+ u$ by $1$. Apply this $2$-path reversal operation, and replace
$D$ by the resulting tournament. Repeat this step as often as
possible (each time applying Observation 2 anew, as long as $D$ is
sinkless). This process must eventually come to an
end\footnote{Indeed, each time we apply the $2$-path reversal
operation, the outdegree $\deg^+ u$ is decreased by $1$; but this
outdegree cannot keep decreasing by $1$ indefinitely.}, and thus we
eventually end up with a tournament that is no longer
sinkless.\footnote{Note that we are not guaranteed to obtain
$\deg^+ u = 0$ in the final tournament. We are only guaranteed that it
will not be sinkless! There may be another vertex $v$ satisfying
$\deg^+ v = 0$ instead.}
This proves Observation 2.
\end{proof}

Now, our goal is to show that there is a sequence of $2$-path reversal
operations that transforms $D$ into a transitive tournament. We
achieve this by performing the following procedure:

\begin{itemize}
\item \textbf{First step:} We first perform a sequence of
      $2$-path reversal operations that transforms $D$ into a
      tournament that is not sinkless. Such a sequence exists because
      of Observation 2. Let $E$ be the tournament obtained at the end
      of this step.
\item \textbf{Second step:} Now, the tournament $E$ is not sinkless.
      In other words, $E$ has a vertex that has outdegree $0$. Fix
      such a vertex, and denote it by $p$. Let $E_1$ be the
      tournament obtained from $E$ by removing the vertex $p$ and the
      arcs whose target or source is $p$.\ \ \ \ \footnote{Formally
      speaking, if we write the tournament $E$ as $\tup{V, A'}$, then
      $E_1$ is the tournament $\tup{V_1, A'_1}$,
      where $V_1 = V \setminus \set{p}$, where
      $A'_1 = \set{ a \in A' \mid
      \text{ neither the source nor the target of } a \text{ is } p}$.
      Of course, due to $p$ having outdegree $0$, the tournament $E$
      has no arcs with source $p$, and so we only need to care about
      arcs with target $p$.}
      Then, the number of the vertices of $E_1$ is
      $\abs{V} - 1 = N - 1$ (since $\abs{V} = N$). Therefore, by the
      induction hypothesis, we can apply
      Exercise~\ref{exe.tourn.2-path-rev} to $E_1$ instead of $D$.
      We thus conclude that there is a sequence of $2$-path reversal
      operations that transforms $E_1$ into a transitive tournament.
      Fix such a sequence, and apply it ``inside $E$'' (i.e., apply
      the same operations to $E$, ignoring the vertex $p$).
      Let $F$ be the tournament obtained at the end of this step.
\end{itemize}

The tournament $F$ is thus obtained from $D$ by a sequence of
$2$-path reversal operations. What do we know about $F$ ?

\begin{itemize}
\item First, we
know that the vertex $p$ has outdegree $0$ in $F$ (because it had
outdegree $0$ in $E$, and because no arcs with source or target $p$
have been modified by the operations that transformed $E$ into $F$).
In other words, the tournament $F$ has no arcs with source $p$.
Therefore, the tournament $F$ has no $3$-cycles that contain the
vertex $p$ (because if a $3$-cycle contains $p$, then there must be
an arc with source $p$).

\item
We furthermore know that the tournament obtained from $F$ by removing
the vertex $p$ is transitive (because the operations that transformed
$E$ into $F$ were chosen in such a way as to transform $E_1$ into a
transitive tournament). In other words, the tournament obtained from
$F$ by removing the vertex $p$ has no $3$-cycles. Equivalently, the
tournament $F$ has no $3$-cycles that do not contain the vertex $p$.
\end{itemize}

We thus have seen that the tournament $F$ has no $3$-cycles that
contain the vertex $p$, but also has no $3$-cycles that do not contain
the vertex $p$. Hence, the tournament $F$ has no $3$-cycles at all.
In other words, the tournament $F$ is transitive.

Hence, there is a sequence of $2$-path reversal operations that
transforms $D$ into a transitive tournament (namely, the sequence of
operations that transformed $D$ into $F$). In other words,
Exercise~\ref{exe.tourn.2-path-rev} is solved in the case when
$\abs{V} = N$. This completes the induction step, and so the solution
of Exercise~\ref{exe.tourn.2-path-rev} is complete.
\end{proof}

\begin{remark}
Exercise~\ref{exe.tourn.2-path-rev} suggests an additional question:
If $E$ and $F$ are two tournaments with the same vertex set, then is
it always possible to transform $E$ into $F$ by a sequence of
$2$-path reversal operations?

The answer to this question is ``no'', and there is a rather neat
reason for this: WLOG assume that the common vertex set of $E$ and $F$
is $\set{1, 2, \ldots, n}$. If $D$ is a tournament with vertex set
$\set{1, 2, \ldots, n}$, then an \textit{inversion} of $D$ will mean
an arc $\tup{i, j}$ of $D$ satisfying $i > j$. Now, it is easy to see
that if we apply a $2$-path reversal operation to a tournament with
vertex set $\set{1, 2, \ldots, n}$, then the number of inversions of
the tournament does not change modulo $2$ (i.e., if this number was
even, then it remains even; and if this number was odd, then it
remains odd). Hence, $E$ cannot be transformed into $F$ by a sequence
of $2$-path reversal operations unless the number of inversions of $E$
is congruent to the number of inversions of $F$ modulo $2$.

But what if they are congruent? I don't know. Feel free to comment!
\end{remark}

\begin{thebibliography}{9999999999}                                                                                       %

\bibitem[ChDiGr92]{ChDiGr92}
Fan Chung, Persi Diaconis, Ron Graham,
\textit{Universal cycles for combinatorial structures},
Discrete Mathematics 110 (1992), pp. 43--59,
\newline\url{http://www.math.ucsd.edu/~fan/wp/universalcycle.pdf}.

\bibitem[Moon13]{Moon13}
John W. Moon,
\textit{Topics on Tournaments},
Project Gutenberg Release \#42833, 2013,
\newline \url{http://onlinebooks.library.upenn.edu/webbin/gutbook/lookup?num=42833}.

\bibitem[Stanle13]{Stanley13}
Richard P. Stanley,
\textit{Algebraic Combinatorics:
Walks, Trees, Tableaux, and More},
Springer 2013.
\newline See \url{http://www-math.mit.edu/~rstan/algcomb/} for errata
and a downloadable draft of the book.

\end{thebibliography}

\end{document}