\documentclass[numbers=enddot,12pt,final,onecolumn,notitlepage]{scrartcl}%
\usepackage[headsepline,footsepline,manualmark]{scrlayer-scrpage}
\usepackage[all,cmtip]{xy}
\usepackage{amssymb}
\usepackage{amsmath}
\usepackage{amsthm}
\usepackage{framed}
\usepackage{comment}
\usepackage{color}
\usepackage{hyperref}
\usepackage{ifthen}
\usepackage[sc]{mathpazo}
\usepackage[T1]{fontenc}
\usepackage{needspace}
\usepackage{tabls}
%TCIDATA{OutputFilter=latex2.dll}
%TCIDATA{Version=5.50.0.2960}
%TCIDATA{LastRevised=Friday, September 16, 2016 20:39:00}
%TCIDATA{SuppressPackageManagement}
%TCIDATA{<META NAME="GraphicsSave" CONTENT="32">}
%TCIDATA{<META NAME="SaveForMode" CONTENT="1">}
%TCIDATA{BibliographyScheme=Manual}
%TCIDATA{Language=American English}
%BeginMSIPreambleData
\providecommand{\U}[1]{\protect\rule{.1in}{.1in}}
%EndMSIPreambleData
\newcounter{exer}
\theoremstyle{definition}
\newtheorem{theo}{Theorem}[section]
\newenvironment{theorem}[1][]
{\begin{theo}[#1]\begin{leftbar}}
{\end{leftbar}\end{theo}}
\newtheorem{lem}[theo]{Lemma}
\newenvironment{lemma}[1][]
{\begin{lem}[#1]\begin{leftbar}}
{\end{leftbar}\end{lem}}
\newtheorem{prop}[theo]{Proposition}
\newenvironment{proposition}[1][]
{\begin{prop}[#1]\begin{leftbar}}
{\end{leftbar}\end{prop}}
\newtheorem{defi}[theo]{Definition}
\newenvironment{definition}[1][]
{\begin{defi}[#1]\begin{leftbar}}
{\end{leftbar}\end{defi}}
\newtheorem{remk}[theo]{Remark}
\newenvironment{remark}[1][]
{\begin{remk}[#1]\begin{leftbar}}
{\end{leftbar}\end{remk}}
\newtheorem{coro}[theo]{Corollary}
\newenvironment{corollary}[1][]
{\begin{coro}[#1]\begin{leftbar}}
{\end{leftbar}\end{coro}}
\newtheorem{conv}[theo]{Convention}
\newenvironment{condition}[1][]
{\begin{conv}[#1]\begin{leftbar}}
{\end{leftbar}\end{conv}}
\newtheorem{quest}[theo]{Question}
\newenvironment{algorithm}[1][]
{\begin{quest}[#1]\begin{leftbar}}
{\end{leftbar}\end{quest}}
\newtheorem{warn}[theo]{Warning}
\newenvironment{conclusion}[1][]
{\begin{warn}[#1]\begin{leftbar}}
{\end{leftbar}\end{warn}}
\newtheorem{conj}[theo]{Conjecture}
\newenvironment{conjecture}[1][]
{\begin{conj}[#1]\begin{leftbar}}
{\end{leftbar}\end{conj}}
\newtheorem{exam}[theo]{Example}
\newenvironment{example}[1][]
{\begin{exam}[#1]\begin{leftbar}}
{\end{leftbar}\end{exam}}
\newtheorem{exmp}[exer]{Exercise}
\newenvironment{exercise}[1][]
{\begin{exmp}[#1]\begin{leftbar}}
{\end{leftbar}\end{exmp}}
\newenvironment{statement}{\begin{quote}}{\end{quote}}
\iffalse
\newenvironment{proof}[1][Proof]{\noindent\textbf{#1.} }{\ \rule{0.5em}{0.5em}}
\fi
\let\sumnonlimits\sum
\let\prodnonlimits\prod
\let\cupnonlimits\bigcup
\let\capnonlimits\bigcap
\renewcommand{\sum}{\sumnonlimits\limits}
\renewcommand{\prod}{\prodnonlimits\limits}
\renewcommand{\bigcup}{\cupnonlimits\limits}
\renewcommand{\bigcap}{\capnonlimits\limits}
\setlength\tablinesep{3pt}
\setlength\arraylinesep{3pt}
\setlength\extrarulesep{3pt}
\voffset=0cm
\hoffset=-0.7cm
\setlength\textheight{22.5cm}
\setlength\textwidth{15.5cm}
\newenvironment{verlong}{}{}
\newenvironment{vershort}{}{}
\newenvironment{noncompile}{}{}
\excludecomment{verlong}
\includecomment{vershort}
\excludecomment{noncompile}
\newcommand{\id}{\operatorname{id}}
\newcommand{\rev}{\operatorname{rev}}
\newcommand{\conncomp}{\operatorname{conncomp}}
\newcommand{\conn}{\operatorname{conn}}
\newcommand{\NN}{\mathbb{N}}
\newcommand{\ZZ}{\mathbb{Z}}
\newcommand{\QQ}{\mathbb{Q}}
\newcommand{\RR}{\mathbb{R}}
\newcommand{\powset}[2][]{\ifthenelse{\equal{#2}{}}{\mathcal{P}\left(#1\right)}{\mathcal{P}_{#1}\left(#2\right)}}
% $\powset[k]{S}$ stands for the set of all $k$-element subsets of
% $S$. The argument $k$ is optional, and if not provided, the result
% is the whole powerset of $S$.
\newcommand{\set}[1]{\left\{ #1 \right\}}
% $\set{...}$ yields $\left\{ ... \right\}$.
\newcommand{\abs}[1]{\left| #1 \right|}
% $\abs{...}$ yields $\left| ... \right|$.
\newcommand{\tup}[1]{\left( #1 \right)}
% $\tup{...}$ yields $\left( ... \right)$.
\newcommand{\ive}[1]{\left[ #1 \right]}
% $\ive{...}$ yields $\left[ ... \right]$.
\newcommand{\verts}[1]{\operatorname{V}\left( #1 \right)}
% $\verts{...}$ yields $\operatorname{V}\left( ... \right)$.
\newcommand{\edges}[1]{\operatorname{E}\left( #1 \right)}
% $\edges{...}$ yields $\operatorname{E}\left( ... \right)$.
\newcommand{\arcs}[1]{\operatorname{A}\left( #1 \right)}
% $\arcs{...}$ yields $\operatorname{A}\left( ... \right)$.
\newcommand{\underbrack}[2]{\underbrace{#1}_{\substack{#2}}}
% $\underbrack{...1}{...2}$ yields
% $\underbrace{...1}_{\substack{...2}}$. This is useful for doing
% local rewriting transformations on mathematical expressions with
% justifications.
\ihead{Math 5707 Spring 2017 (Darij Grinberg): homework set 4}
\ohead{page \thepage}
\cfoot{}
\begin{document}

\begin{center}
\textbf{Math 5707 Spring 2017 (Darij Grinberg): homework set 4}

\textbf{due: Wed, 29 Mar 2017, in class} or by email
(\texttt{dgrinber@umn.edu}) before class

\textbf{Please hand in solutions to FIVE of the 7 problems.}
\end{center}

%\tableofcontents

\subsection{Reminders}

See the
\href{http://www-users.math.umn.edu/~dgrinber/5707s17/nogra.pdf}{lecture notes}
and also the
\href{http://www-users.math.umn.edu/~dgrinber/5707s17/}{handwritten notes}
for relevant material.
See also
\href{http://www-users.math.umn.edu/~dgrinber/5707s17/hw2s.pdf}{the solutions to homework set 2}
for various conventions and notations that are in use here.

\subsection{Exercise \ref{exe.hw4.menger.DA}: Directed arc-disjoint
version of Menger's theorem}

In class, you have seen the following two theorems:

\begin{theorem}[Menger's theorem, DVE (directed vertex-disjoint
version)] \label{thm.menger.DVE}
Let $D = \tup{V, A}$ be a digraph.
Let $S$ and $T$ be two subsets of $V$.

An \textit{$S$-$T$-path} in $D$ means a path in $D$ whose starting
point belongs to $S$ and whose ending point belongs to $T$.

Several paths are said to be \textit{vertex-disjoint} if no two
have a vertex in common.

A subset $C$ of $V$ is said to be \textit{$S$-$T$-disconnecting} if
each $S$-$T$-path contains at least one vertex in $C$.

The maximum number of vertex-disjoint $S$-$T$-paths equals the
minimum size of an $S$-$T$-disconnecting subset of $V$.
\end{theorem}

\begin{theorem}[Menger's theorem, DVI (directed internally
vertex-disjoint version)] \label{thm.menger.DVI}
Let $D = \tup{V, A}$ be a digraph.
Let $s$ and $t$ be two vertices of $D$ such that
$\tup{s, t} \notin A$.

An \textit{$s$-$t$-path} in $D$ means a path from $s$ to $t$ in $D$.

Several $s$-$t$-paths are said to be
\textit{internally vertex-disjoint} if no two have a vertex in common
except for the vertices $s$ and $t$.

A subset $C$ of $V$ is said to be an \textit{$s$-$t$-vertex-cut} if it
contains neither $s$ nor $t$, and if
each $s$-$t$-path contains at least one vertex in $C$.

The maximum number of internally vertex-disjoint $s$-$t$-paths equals
the minimum size of an $s$-$t$-vertex-cut.
\end{theorem}

Here are two other versions of Menger's theorem:

\begin{theorem}[Menger's theorem, DA (directed arc-disjoint
version)] \label{thm.menger.DA}
Let $D = \tup{V, A, \phi}$ be a multidigraph.
Let $s$ and $t$ be two distinct vertices of $D$.

An \textit{$s$-$t$-path} in $D$ means a path from $s$ to $t$ in $D$.

Several paths in $D$ are said to be
\textit{arc-disjoint} if no two have an arc in common.

A subset $C$ of $A$ is said to be an \textit{$s$-$t$-cut} if it has
the form
\[
C = \set{ a \in A \mid \text{the source of } a \text{ belongs to } U
                        \text{, but the target of } a \text{ does not}
        }
\]
for some subset $U$ of $V$ satisfying $s \in U$ and $t \notin U$.

The maximum number of arc-disjoint $s$-$t$-paths equals
the minimum size of an $s$-$t$-cut.
\end{theorem}

\begin{remark}
The assumption in Theorem~\ref{thm.menger.DA} that $s$ and $t$ be
distinct is not really necessary -- you can omit it if you are willing
to put up with the idea that the maximum number of arc-disjoint
$s$-$t$-paths is $\infty$ (because you can count the trivial
path $\tup{s}$ infinitely often, thanks to it being arc-disjoint from
itself), and that the minimum size of an $s$-$t$-cut is $\infty$ as
well (since there exists no $s$-$t$-cut, and thus you are taking the
minimum of an empty set of integers, which according to one possible
convention is $\infty$).
Feel free to ignore these kinds of hairsplitting.
\end{remark}

\begin{theorem}[Menger's theorem, DAS (directed arc-disjoint set
version)] \label{thm.menger.DAS}
Let $D = \tup{V, A, \phi}$ be a multidigraph.
Let $S$ and $T$ be two disjoint subsets of $V$.

An \textit{$S$-$T$-path} in $D$ means a path in $D$ whose starting
point belongs to $S$ and whose ending point belongs to $T$.

Several paths in $D$ are said to be
\textit{arc-disjoint} if no two have an arc in common.

A subset $C$ of $A$ is said to be an \textit{$S$-$T$-cut} if it has
the form
\[
C = \set{ a \in A \mid \text{the source of } a \text{ belongs to } U
                        \text{, but the target of } a \text{ does not}
        }
\]
for some subset $U$ of $V$ satisfying $S \subseteq U$ and
$T \cap U = \varnothing$.

The maximum number of arc-disjoint $S$-$T$-paths equals
the minimum size of an $S$-$T$-cut.
\end{theorem}

\begin{exercise} \label{exe.hw4.menger.DA}
Prove Theorem~\ref{thm.menger.DA} and
Theorem~\ref{thm.menger.DAS}.
(You are allowed to use Theorem~\ref{thm.menger.DVE} and
Theorem~\ref{thm.menger.DVI}.)
\end{exercise}

\subsection{Exercise \ref{exe.hw4.menger.undir}: Undirected Menger's
theorems}

\begin{exercise} \label{exe.hw4.menger.undir}
State and prove analogues of Theorem~\ref{thm.menger.DVE},
Theorem~\ref{thm.menger.DVI}, Theorem~\ref{thm.menger.DA} and
Theorem~\ref{thm.menger.DAS} for undirected graphs (simple graphs in
the case of the first two theorems; multigraphs for the last two).
(You can use the directed-graph versions in the proofs.)
\end{exercise}

\subsection{Exercise \ref{exe.hw4.matching-product}: matchings in a
Cartesian product}

Recall that each simple graph $G = \tup{V, E}$ can be viewed as a
multigraph in a natural way (namely, as the multigraph
$\tup{V, E, \iota}$, where $\iota$ is the map from $E$ to
$\powset[2]{V}$ sending each edge $e$ to $e$).
Thus, everything we say about multigraphs can be applied to simple
graphs.

Let me recall a definition from class in slightly greater generality:

\begin{definition}
Let $G = \tup{V, E, \phi}$ be a multigraph.
A \textit{matching} in $G$ means a subset $M$ of $E$ such that no two
distinct edges in $M$ have a vertex in common.
\end{definition}

In class, we have been discussing matchings in simple graphs; of
course, this is a particular case of matchings in multigraphs.
The difference between simple graphs and multigraphs does not really
matter for the purpose of the existence or nonexistence of matchings
(after all, a matching cannot use more than one edge through each
vertex, so it cannot include two parallel edges);
but it matters if you want to, e.g., count matchings.

\begin{definition}
A \textit{bipartite graph} is a triple $\tup{G, X, Y}$,
% (this is just a
% slightly suggestive notation for $\tup{G, X, Y}$, where the semicolon
% is supposed to separate the multigraph $G$ from the two sets $X$ and
% $Y$),
where $G$ is a multigraph, and
where $X$ and $Y$ are two subsets of $\verts{G}$
satisfying the following conditions:

\begin{itemize}
\item We have $X \cap Y = \varnothing$ and
      $X \cup Y = \verts{G}$.
      (In other words, each vertex of $G$ lies in exactly one of the
      two sets $X$ and $Y$.)
\item Each edge of $G$ has exactly one endpoint in $X$ and exactly one
      endpoint in $Y$.
\end{itemize}

We shall usually write $\tup{G; X, Y}$ instead of $\tup{G, X, Y}$ for
a bipartite graph, putting a semicolon between the $G$ and the $X$ in
order to stress the different roles that $G$ plays and that $X$ and
$Y$ play.
\end{definition}

In class, we used simple graphs instead of multigraphs in the above
definition (if I remember correctly).
Again, the difference is not important, as far as the things done in
class are concerned (i.e., comparing the sizes of
matchings and vertex covers).
Again, the difference starts creeping up when you start counting
(matchings, cycles, paths, etc.), but this is not a class about
counting.

\begin{definition}
Let $M$ be a matching in a multigraph $G$.

\textbf{(a)} A vertex $v$ of $G$ is said to be \textit{matched in $M$}
if there exists an edge $e \in M$ such that $v$ is an endpoint of $e$.
In this case, this edge is unique (since $M$ is a matching), and the
other endpoint of this edge (i.e., the one distinct from $v$) is
called the \textit{$M$-partner} of $v$.

\textbf{(b)} Let $S$ be a subset of $\verts{G}$.
The matching $M$ is said to be \textit{$S$-complete} if each vertex
$v \in S$ is matched in $M$.
\end{definition}

\begin{exercise} \label{exe.hw4.matching-product}
Let $\tup{G; X, Y}$ and $\tup{H; U, V}$ be bipartite graphs.

Assume that $G$ is a simple graph and has an $X$-complete matching.

Assume that $H$ is a simple graph and has a $U$-complete matching.

Consider the Cartesian product $G \times H$ of $G$ and $H$
defined in Exercise 1 of
\href{http://www-users.math.umn.edu/~dgrinber/5707s17/hw2s.pdf}{homework set 2}.

\textbf{(a)} Show that
$\tup{G \times H; \tup{X \times V} \cup \tup{Y \times U},
                  \tup{X \times U} \cup \tup{Y \times V}}$
is a bipartite graph.

\textbf{(b)} Prove that the graph $G \times H$ has an
$\tup{X \times V} \cup \tup{Y \times U}$-complete matching.
\end{exercise}

(We required that $G$ and $H$ be simple graphs just in order to not
have to define $G \times H$ for multigraphs.)

\subsection{Exercise \ref{exe.hw4.subsets}: extending subsets}

\begin{exercise} \label{exe.hw4.subsets}
Let $S$ be a finite set.
Let $k \in \NN$ be such that $\abs{S} \geq 2k+1$.
Prove that there exists an injective map
$f : \powset[k]{S} \to \powset[k+1]{S}$ such that
each $X \in \powset[k]{S}$ satisfies
$f \tup{X} \supseteq X$.

(In other words, prove that we can add to each $k$-element subset
$X$ of $S$ an additional element from $S \setminus X$ such that the
resulting $\tup{k+1}$-element subsets are distinct.)

[\textbf{Hint:} First, reduce the problem to the case when
$\abs{S} = 2k+1$.
Then, in that case, restate it as a claim about matchings in a
certain bipartite graph.]
\end{exercise}

\subsection{Exercise \ref{exe.hw4.parabolic-derangements}:
the self-grading problem}

\begin{exercise} \label{exe.hw4.parabolic-derangements}
Let $S$ be a finite set, and let $k \in \NN$.
Let $A_1, A_2, \ldots, A_k$ be $k$ subsets of $S$
such that each element of $S$ lies in exactly one of these $k$
subsets.
Prove that the following statements are equivalent:

\begin{itemize}
\item \textit{Statement 1:} There exists a bijection
      $\sigma : S \to S$ such that each $i \in \set{1, 2, \ldots, k}$
      satisfies $\sigma \tup{A_i} \cap A_i = \varnothing$.
\item \textit{Statement 2:} Each $i \in \set{1, 2, \ldots, k}$
      satisfies $\abs{A_i} \leq \abs{S} / 2$.
\end{itemize}
\end{exercise}

(A restatement of Exercise~\ref{exe.hw4.parabolic-derangements}:
Let $S$ be a finite set of students who have submitted homework.
The students have been collaborating on the homework, forming $k$
disjoint collaboration groups.
A lazy professor wants the students to grade each other's homework,
but he wants to avoid having a student grading the homework of a
student from the same collaboration group.
Prove that he can organize this (i.e., find a grader for each student)
if and only if each collaboration group has size $\leq \abs{S}/2$.)

\subsection{Exercise \ref{exe.hw4.lattice}: unions and intersections
of neighbor-critical sets}

\begin{definition}
Let $G$ be a multigraph.
If $S$ is a subset of $\verts{G}$, then
$N\tup{S}$ (or, to be more precise, $N_G\tup{S}$) shall denote the
subset
$\set{ u \in \verts{G} \mid \text{at least one neighbor of } u
                            \text{ belongs to } S }$
of $\verts{G}$.
\end{definition}

\begin{exercise} \label{exe.hw4.lattice}
Let $\tup{G; X, Y}$ be a bipartite graph.
Assume that each $S \subseteq X$ satisfies
$\abs{N\tup{S}} \geq \abs{S}$.
(Thus, Hall's theorem shows that $G$ has an $X$-complete matching.)

A subset $S$ of $X$ will be called \textit{neighbor-critical} if
$\abs{N\tup{S}} = \abs{S}$.

Let $A$ and $B$ be two neighbor-critical subsets of $X$.
Prove that the subsets $A \cup B$ and $A \cap B$ are also
neighbor-critical.
\end{exercise}

\subsection{Exercise \ref{exe.hw4.syscom}: systems of common
representatives}

\begin{exercise} \label{exe.hw4.syscom}
Let $S$ be a finite set. Let $k \in \NN$.

Let $A_1, A_2, \ldots, A_k$ be $k$ subsets of $S$.

Let $B_1, B_2, \ldots, B_k$ be $k$ subsets of $S$.

A \textit{system of common representatives} shall mean a
choice of $k$ distinct elements $t_1, t_2, \ldots, t_k$ of
$S$ as well as two bijections
$f : \set{1, 2, \ldots, k} \to \set{1, 2, \ldots, k}$
and
$g : \set{1, 2, \ldots, k} \to \set{1, 2, \ldots, k}$
such that each $i \in \set{1, 2, \ldots, k}$ satisfies
\[
 t_i \in A_{f\tup{i}}
 \qquad \text{ and } \qquad
 t_i \in B_{g\tup{i}} .
\]

Prove that a system of common representatives exists if and only
if each two subsets $I$ and $J$ of $\set{1, 2, \ldots, k}$
satisfy
\[
 \abs{\tup{\bigcup_{i \in I} A_i} \cap \tup{\bigcup_{j \in J} B_j}}
 \geq \abs{I} + \abs{J} - k .
\]

[You are free to use any of
Theorem~\ref{thm.menger.DVE},
Theorem~\ref{thm.menger.DVI}, Theorem~\ref{thm.menger.DA} and
Theorem~\ref{thm.menger.DAS} here.]
\end{exercise}


\end{document}